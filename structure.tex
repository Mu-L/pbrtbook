%%%%%%%%%%%%%%%%%%%%%%%%%%%%%%%%%%%%%%%%%
% The Legrand Orange Book
% Structural Definitions File
% Version 2.1 (26/09/2018)
%
% Original author:
% Mathias Legrand (legrand.mathias@gmail.com) with modifications by:
% Vel (vel@latextemplates.com)
%
% This file was downloaded from:
% http://www.LaTeXTemplates.com
%
% License:
% CC BY-NC-SA 3.0 (http://creativecommons.org/licenses/by-nc-sa/3.0/)
%
%%%%%%%%%%%%%%%%%%%%%%%%%%%%%%%%%%%%%%%%%

%----------------------------------------------------------------------------------------
%	VARIOUS REQUIRED PACKAGES AND CONFIGURATIONS
%----------------------------------------------------------------------------------------

\usepackage{graphicx} % Required for including pictures
\graphicspath{{Pictures/}} % Specifies the directory where pictures are stored
\usepackage{subfig}
\usepackage{tikz} % Required for drawing custom shapes
\usepackage{pstricks}
\usepackage[english]{babel} % English language/hyphenation

\usepackage{enumitem} % Customize lists
\setlist{nolistsep} % Reduce spacing between bullet points and numbered lists

\usepackage{booktabs} % Required for nicer horizontal rules in tables
\usepackage{bm}
\usepackage{xcolor} % Required for specifying colors by name
\definecolor{ocre}{RGB}{243,102,25} % Define the orange color used for highlighting throughout the book

%----------------------------------------------------------------------------------------
%	MARGINS
%----------------------------------------------------------------------------------------

\usepackage{geometry} % Required for adjusting page dimensions and margins
\geometry{
    paper=a4paper, % Paper size, change to letterpaper for US letter size
    top=3cm, % Top margin
    bottom=3cm, % Bottom margin
    inner=3.2cm, % Left margin
    outer=4.8cm, % Right margin
    headheight=14pt, % Header height
    footskip=1.4cm, % Space from the bottom margin to the baseline of the footer
    headsep=10pt, % Space from the top margin to the baseline of the header
    %showframe, % Uncomment to show how the type block is set on the page
}

%----------------------------------------------------------------------------------------
%	FONTS
%----------------------------------------------------------------------------------------
\usepackage[UTF8]{ctex}
\setCJKmainfont[Path=fonts/,BoldFont={SourceHanSerifSC-Bold.otf},ItalicFont={simkai.ttf}]{SourceHanSerifSC-Regular.otf}
\setCJKsansfont[Path=fonts/,BoldFont={SourceHanSansSC-Bold.otf}]{SourceHanSansSC-Regular.otf}
\usepackage{avant} % Use the Avantgarde font for headings
%\usepackage{times} % Use the Times font for headings
\usepackage{mathptmx} % Use the Adobe Times Roman as the default text font together with math symbols from the Sym­bol, Chancery and Com­puter Modern fonts

\usepackage{microtype} % Slightly tweak font spacing for aesthetics
\usepackage[utf8]{inputenc} % Required for including letters with accents
\usepackage[T1]{fontenc} % Use 8-bit encoding that has 256 glyphs

%---------------------------------------------------------------------------------------
% use sidenotes but fix its bug to change font size
% see: https://tex.stackexchange.com/questions/532245/how-to-modify-fonts-in-sidenotes
%---------------------------------------------------------------------------------------
\usepackage{sidenotes}
\usepackage{xparse}
\let\oldmarginpar\marginpar
\RenewDocumentCommand{\marginpar}{om}{%
    \IfNoValueTF{#1}
    {\oldmarginpar{\mymparsetup #2}}
    {\oldmarginpar[\mymparsetup #1]{\mymparsetup #2}}}
\newcommand{\mymparsetup}{\scriptsize\itshape}

%---------------------------------------------------------------------------------------
% use \eg ... see: https://stackoverflow.com/a/39363004/12128185
%---------------------------------------------------------------------------------------
\usepackage{xspace}
\makeatletter
\DeclareRobustCommand\onedot{\futurelet\@let@token\@onedot}
\def\@onedot{\ifx\@let@token.\else.\null\fi\xspace}
\def\eg{\emph{e.g}\onedot} \def\Eg{\emph{E.g}\onedot}
\def\ie{\emph{i.e}\onedot} \def\Ie{\emph{I.e}\onedot}
\def\cf{\emph{c.f}\onedot} \def\Cf{\emph{C.f}\onedot}
\def\etc{\emph{etc}\onedot} \def\vs{\emph{vs}\onedot}
\def\wrt{w.r.t\onedot} \def\dof{d.o.f\onedot}
\def\etal{\emph{et al}\onedot}
\makeatother

%----------------------------------------------------------------------------------------
%	BIBLIOGRAPHY AND INDEX
%----------------------------------------------------------------------------------------

\usepackage[style=authoryear,citestyle=authoryear-comp,maxbibnames=99,maxcitenames=3,mincitenames=1,uniquename=false,sorting=nyt,sortcites=true,autopunct=true,babel=hyphen,hyperref=true,abbreviate=false,backref=true,backend=biber,natbib=true]{biblatex}
\addbibresource{bibliography.bib} % BibTeX bibliography file
\defbibheading{bibempty}{}
\renewcommand*{\finalnamedelim}{{\ifcitation{和}{, and }}}
\renewcommand*{\andothersdelim}{\ifcitation{}{, }}
\DefineBibliographyStrings{english}{andothers={\ifcitation{等}{ et al.}}}
\usepackage{calc} % For simpler calculation - used for spacing the index letter headings correctly
\usepackage{makeidx} % Required to make an index
\makeindex % Tells LaTeX to create the files required for indexing
\usepackage{xifthen} % provides \isempty test
\newcommand\keyindex[3]{\ifthenelse{\isempty{#1}}{}{\ifthenelse{\isempty{#2}}{\ifthenelse{\isempty{#3}}{{\sffamily#1}}{{\sffamily#1}\index{#3!#1}}}{\ifthenelse{\isempty{#3}}{{\sffamily#1}(#2)\index{#2#1}}{{\sffamily#1}(#2)\index{#3!#2#1}}}}}

%----------------------------------------------------------------------------------------
%	MAIN TABLE OF CONTENTS
%----------------------------------------------------------------------------------------

\usepackage{titletoc} % Required for manipulating the table of contents

\contentsmargin{0cm} % Removes the default margin

% Part text styling (this is mostly taken care of in the PART HEADINGS section of this file)
\titlecontents{part}
[0cm] % Left indentation
{\addvspace{20pt}\bfseries} % Spacing and font options for parts
{}
{}
{}

% Chapter text styling
\titlecontents{chapter}
[1.25cm] % Left indentation
{\addvspace{12pt}\large\sffamily\bfseries} % Spacing and font options for chapters
{\color{ocre!60}\contentslabel[\Large\thecontentslabel]{1.25cm}\color{ocre}} % Formatting of numbered sections of this type
{\color{ocre}} % Formatting of numberless sections of this type
{\color{ocre!60}\normalsize\;\titlerule*[.5pc]{.}\;\thecontentspage} % Formatting of the filler to the right of the heading and the page number

% Section text styling
\titlecontents{section}
[1.25cm] % Left indentation
{\addvspace{3pt}\sffamily\bfseries} % Spacing and font options for sections
{\contentslabel[\thecontentslabel]{1.25cm}} % Formatting of numbered sections of this type
{} % Formatting of numberless sections of this type
{\hfill\color{black}\thecontentspage} % Formatting of the filler to the right of the heading and the page number

% Subsection text styling
\titlecontents{subsection}
[1.25cm] % Left indentation
{\addvspace{1pt}\sffamily\small} % Spacing and font options for subsections
{\contentslabel[\thecontentslabel]{1.25cm}} % Formatting of numbered sections of this type
{} % Formatting of numberless sections of this type
{\ \titlerule*[.5pc]{.}\;\thecontentspage} % Formatting of the filler to the right of the heading and the page number

% Figure text styling
\titlecontents{figure}
[1.25cm] % Left indentation
{\addvspace{1pt}\sffamily\small} % Spacing and font options for figures
{\thecontentslabel\hspace*{1em}} % Formatting of numbered sections of this type
{} % Formatting of numberless sections of this type
{\ \titlerule*[.5pc]{.}\;\thecontentspage} % Formatting of the filler to the right of the heading and the page number

% Table text styling
\titlecontents{table}
[1.25cm] % Left indentation
{\addvspace{1pt}\sffamily\small} % Spacing and font options for tables
{\thecontentslabel\hspace*{1em}} % Formatting of numbered sections of this type
{} % Formatting of numberless sections of this type
{\ \titlerule*[.5pc]{.}\;\thecontentspage} % Formatting of the filler to the right of the heading and the page number

%----------------------------------------------------------------------------------------
%	MINI TABLE OF CONTENTS IN PART HEADS
%----------------------------------------------------------------------------------------

% Chapter text styling
\titlecontents{lchapter}
[0em] % Left indentation
{\addvspace{15pt}\large\sffamily\bfseries} % Spacing and font options for chapters
{\color{ocre}\contentslabel[\Large\thecontentslabel]{1.25cm}\color{ocre}} % Chapter number
{}
{\color{ocre}\normalsize\sffamily\bfseries\;\titlerule*[.5pc]{.}\;\thecontentspage} % Page number

% Section text styling
\titlecontents{lsection}
[0em] % Left indentation
{\sffamily\small} % Spacing and font options for sections
{\contentslabel[\thecontentslabel]{1.25cm}} % Section number
{}
{}

% Subsection text styling (note these aren't shown by default, display them by searchings this file for tocdepth and reading the commented text)
\titlecontents{lsubsection}
[.5em] % Left indentation
{\sffamily\footnotesize} % Spacing and font options for subsections
{\contentslabel[\thecontentslabel]{1.25cm}}
{}
{}

%----------------------------------------------------------------------------------------
%	HEADERS AND FOOTERS
%----------------------------------------------------------------------------------------

\usepackage{fancyhdr} % Required for header and footer configuration

\pagestyle{fancy} % Enable the custom headers and footers
% \ifusechapterimage ... \fi 在main.tex中用\usechapterimagefalse或\usechapterimagetrue控制,该句影响页眉
\renewcommand{\chaptermark}[1]{\markboth{\sffamily\normalsize\bfseries \ifusechapterimage 第\thechapter 章\fi \ #1}{}} % Styling for the current chapter in the header
\renewcommand{\sectionmark}[1]{\markright{\sffamily\normalsize\thesection\hspace{5pt}#1}{}} % Styling for the current section in the header

\fancyhf{} % Clear default headers and footers
\fancyhead[LE,RO]{\sffamily\normalsize\thepage} % Styling for the page number in the header
\fancyhead[LO]{\rightmark} % Print the nearest section name on the left side of odd pages
\fancyhead[RE]{\leftmark} % Print the current chapter name on the right side of even pages
%\fancyfoot[C]{\thepage} % Uncomment to include a footer

\renewcommand{\headrulewidth}{0.5pt} % Thickness of the rule under the header

\fancypagestyle{plain}{% Style for when a plain pagestyle is specified
    \fancyhead{}\renewcommand{\headrulewidth}{0pt}%
}

% Removes the header from odd empty pages at the end of chapters
\makeatletter
\renewcommand{\cleardoublepage}{
    \clearpage\ifodd\c@page\else
        \hbox{}
        \vspace*{\fill}
        \thispagestyle{empty}
        \newpage
    \fi}

%----------------------------------------------------------------------------------------
%	THEOREM STYLES
%----------------------------------------------------------------------------------------

\usepackage{amsmath,amsfonts,amssymb,amsthm} % For math equations, theorems, symbols, etc
\usepackage{mathrsfs}
\usepackage{yhmath}
\usepackage{boondox-calo}
\usepackage{extarrows} %长等号
\newcommand{\intoo}[2]{\mathopen{]}#1\,;#2\mathclose{[}}
\newcommand{\ud}{\mathop{\mathrm{{}d}}\mathopen{}}
\newcommand{\intff}[2]{\mathopen{[}#1\,;#2\mathclose{]}}
\renewcommand{\qedsymbol}{$\blacksquare$}
\newtheorem{notation}{记号}[section]
\allowdisplaybreaks[4] % 公式跨页断行最大强度

% Boxed/framed environments
\newtheoremstyle{ocrenumbox}% Theorem style name
{0pt}% Space above
{0pt}% Space below
{\normalfont}% Body font
{}% Indent amount
{\small\bf\sffamily\color{ocre}}% Theorem head font
{\;}% Punctuation after theorem head
{0.25em}% Space after theorem head
{\small\sffamily\color{ocre}\thmname{#1}\nobreakspace\thmnumber{\@ifnotempty{#1}{}\@upn{#2}}% Theorem text (e.g. Theorem 2.1)
    \thmnote{\nobreakspace\the\thm@notefont\sffamily\bfseries\color{black}---\nobreakspace#3.}} % Optional theorem note

\newtheoremstyle{blacknumex}% Theorem style name
{5pt}% Space above
{5pt}% Space below
{\normalfont}% Body font
{} % Indent amount
{\small\bf\sffamily}% Theorem head font
{\;}% Punctuation after theorem head
{0.25em}% Space after theorem head
{\small\sffamily{\tiny\ensuremath{\blacksquare}}\nobreakspace\thmname{#1}\nobreakspace\thmnumber{\@ifnotempty{#1}{}\@upn{#2}}% Theorem text (e.g. Theorem 2.1)
    \thmnote{\nobreakspace\the\thm@notefont\sffamily\bfseries---\nobreakspace#3.}}% Optional theorem note

\newtheoremstyle{blacknumbox} % Theorem style name
{0pt}% Space above
{0pt}% Space below
{\normalfont}% Body font
{}% Indent amount
{\small\bf\sffamily}% Theorem head font
{\;}% Punctuation after theorem head
{0.25em}% Space after theorem head
{\small\sffamily\thmname{#1}\nobreakspace\thmnumber{\@ifnotempty{#1}{}\@upn{#2}}% Theorem text (e.g. Theorem 2.1)
    \thmnote{\nobreakspace\the\thm@notefont\sffamily\bfseries---\nobreakspace#3.}}% Optional theorem note

% Non-boxed/non-framed environments
\newtheoremstyle{ocrenum}% Theorem style name
{5pt}% Space above
{5pt}% Space below
{\normalfont}% Body font
{}% Indent amount
{\small\bf\sffamily\color{ocre}}% Theorem head font
{\;}% Punctuation after theorem head
{0.25em}% Space after theorem head
{\small\sffamily\color{ocre}\thmname{#1}\nobreakspace\thmnumber{\@ifnotempty{#1}{}\@upn{#2}}% Theorem text (e.g. Theorem 2.1)
    \thmnote{\nobreakspace\the\thm@notefont\sffamily\bfseries\color{black}---\nobreakspace#3.}} % Optional theorem note
\makeatother

% Defines the theorem text style for each type of theorem to one of the three styles above
\newcounter{dummy}
\numberwithin{dummy}{section}
\theoremstyle{ocrenumbox}
\newtheorem{theoremeT}[dummy]{定理}
\newtheorem{problem}{Problem}[chapter]
\newtheorem{exerciseT}{Exercise}[chapter]
\theoremstyle{blacknumex}
\newtheorem{exampleT}{例}[section]
\newtheorem{proveT}{证明}[section]
\newenvironment{prove}{\begin{proveT}}{\hfill{\tiny\ensuremath{\qedhere\blacksquare}}\end{proveT}}
\theoremstyle{blacknumbox}
\newtheorem{vocabulary}{Vocabulary}[chapter]
\newtheorem{definitionT}{定义}[section]
\newtheorem{corollaryT}[dummy]{推论}
\theoremstyle{ocrenum}
\newtheorem{proposition}[dummy]{定理}

%----------------------------------------------------------------------------------------
%	DEFINITION OF COLORED BOXES
%----------------------------------------------------------------------------------------

\RequirePackage[framemethod=default]{mdframed} % Required for creating the theorem, definition, exercise and corollary boxes

% Theorem box
\newmdenv[skipabove=7pt,
    skipbelow=7pt,
    backgroundcolor=black!5,
    linecolor=ocre,
    innerleftmargin=5pt,
    innerrightmargin=5pt,
    innertopmargin=5pt,
    leftmargin=0cm,
    rightmargin=0cm,
    innerbottommargin=5pt]{tBox}

% Exercise box
\newmdenv[skipabove=7pt,
    skipbelow=7pt,
    rightline=false,
    leftline=true,
    topline=false,
    bottomline=false,
    backgroundcolor=ocre!10,
    linecolor=ocre,
    innerleftmargin=5pt,
    innerrightmargin=5pt,
    innertopmargin=5pt,
    innerbottommargin=5pt,
    leftmargin=0cm,
    rightmargin=0cm,
    linewidth=4pt]{eBox}

% Definition box
\newmdenv[skipabove=7pt,
    skipbelow=7pt,
    rightline=false,
    leftline=true,
    topline=false,
    bottomline=false,
    linecolor=ocre,
    innerleftmargin=5pt,
    innerrightmargin=5pt,
    innertopmargin=0pt,
    leftmargin=0cm,
    rightmargin=0cm,
    linewidth=4pt,
    innerbottommargin=0pt]{dBox}

% Corollary box
\newmdenv[skipabove=7pt,
    skipbelow=7pt,
    rightline=false,
    leftline=true,
    topline=false,
    bottomline=false,
    linecolor=gray,
    backgroundcolor=black!5,
    innerleftmargin=5pt,
    innerrightmargin=5pt,
    innertopmargin=5pt,
    leftmargin=0cm,
    rightmargin=0cm,
    linewidth=4pt,
    innerbottommargin=5pt]{cBox}

% Creates an environment for each type of theorem and assigns it a theorem text style from the "Theorem Styles" section above and a colored box from above
\newenvironment{theorem}{\begin{tBox}\begin{theoremeT}}{\end{theoremeT}\end{tBox}}
\newenvironment{exercise}{\begin{eBox}\begin{exerciseT}}{\hfill{\color{ocre}\tiny\ensuremath{\blacksquare}}\end{exerciseT}\end{eBox}}
\newenvironment{definition}{\begin{dBox}\begin{definitionT}}{\end{definitionT}\end{dBox}}
\newenvironment{example}{\begin{exampleT}}{\hfill{\tiny\ensuremath{\blacksquare}}\end{exampleT}}
\newenvironment{corollary}{\begin{cBox}\begin{corollaryT}}{\end{corollaryT}\end{cBox}}

%----------------------------------------------------------------------------------------
%	REMARK ENVIRONMENT
%----------------------------------------------------------------------------------------

\newenvironment{remark}{\par\vspace{10pt}\small % Vertical white space above the remark and smaller font size
    \begin{list}{}{
            \leftmargin=35pt % Indentation on the left
            \rightmargin=25pt}\item\ignorespaces % Indentation on the right
              \makebox[-2.5pt]{\begin{tikzpicture}[overlay]
                      \node[draw=ocre!60,line width=1pt,circle,fill=ocre!25,font=\sffamily\bfseries,inner sep=2pt,outer sep=0pt] at (-15pt,0pt){\textcolor{ocre}{R}};\end{tikzpicture}} % Orange R in a circle
              \advance\baselineskip -1pt}{\end{list}\vskip5pt} % Tighter line spacing and white space after remark

%----------------------------------------------------------------------------------------
%	SECTION NUMBERING IN THE MARGIN
%----------------------------------------------------------------------------------------

\makeatletter
\renewcommand{\@seccntformat}[1]{\llap{\textcolor{ocre}{\csname the#1\endcsname}\hspace{1em}}}
\renewcommand{\section}{\@startsection{section}{1}{\z@}
    {-4ex \@plus -1ex \@minus -.4ex}
    {1ex \@plus.2ex }
    {\normalfont\large\sffamily\bfseries}}
\renewcommand{\subsection}{\@startsection {subsection}{2}{\z@}
    {-3ex \@plus -0.1ex \@minus -.4ex}
    {0.5ex \@plus.2ex }
    {\normalfont\sffamily\bfseries}}
\renewcommand{\subsubsection}{\@startsection {subsubsection}{3}{\z@}
    {-2ex \@plus -0.1ex \@minus -.2ex}
    {.2ex \@plus.2ex }
    {\normalfont\small\sffamily\bfseries}}
\renewcommand\paragraph{\@startsection{paragraph}{4}{\z@}
    {-2ex \@plus-.2ex \@minus .2ex}
    {.1ex}
    {\normalfont\small\sffamily\bfseries}}

%----------------------------------------------------------------------------------------
%	PART HEADINGS
%----------------------------------------------------------------------------------------

% Numbered part in the table of contents
\newcommand{\@mypartnumtocformat}[2]{%
    \setlength\fboxsep{0pt}%
    \noindent\colorbox{ocre!20}{\strut\parbox[c][.7cm]{\ecart}{\color{ocre!70}\Large\sffamily\bfseries\centering#1}}\hskip\esp\colorbox{ocre!40}{\strut\parbox[c][.7cm]{\linewidth-\ecart-\esp}{\Large\sffamily\centering#2}}%
}

% Unnumbered part in the table of contents
\newcommand{\@myparttocformat}[1]{%
    \setlength\fboxsep{0pt}%
    \noindent\colorbox{ocre!40}{\strut\parbox[c][.7cm]{\linewidth}{\Large\sffamily\centering#1}}%
}

\newlength\esp
\setlength\esp{4pt}
\newlength\ecart
\setlength\ecart{1.2cm-\esp}
\newcommand{\thepartimage}{}%
\newcommand{\partimage}[1]{\renewcommand{\thepartimage}{#1}}%
\def\@part[#1]#2{%
    \ifnum \c@secnumdepth >-2\relax%
        \refstepcounter{part}%
        \addcontentsline{toc}{part}{\texorpdfstring{\protect\@mypartnumtocformat{\thepart}{#1}}{\partname~\thepart\ ---\ #1}}
    \else%
        \addcontentsline{toc}{part}{\texorpdfstring{\protect\@myparttocformat{#1}}{#1}}%
    \fi%
    \startcontents%
    \markboth{}{}%
    {\thispagestyle{empty}%
        \begin{tikzpicture}[remember picture,overlay]%
            \node at (current page.north west){\begin{tikzpicture}[remember picture,overlay]%
                    \fill[ocre!20](0cm,0cm) rectangle (\paperwidth,-\paperheight);
                    \node[anchor=north] at (4cm,-3.25cm){\color{ocre!40}\fontsize{220}{100}\sffamily\bfseries\thepart};
                    \node[anchor=south east] at (\paperwidth-1cm,-\paperheight+1cm){\parbox[t][][t]{8.5cm}{
                            \printcontents{l}{0}{\setcounter{tocdepth}{1}}% The depth to which the Part mini table of contents displays headings; 0 for chapters only, 1 for chapters and sections and 2 for chapters, sections and subsections
                        }};
                    \node[anchor=north east] at (\paperwidth-1.5cm,-3.25cm){\parbox[t][][t]{15cm}{\strut\raggedleft\color{white}\fontsize{30}{30}\sffamily\bfseries#2}};
                \end{tikzpicture}};
        \end{tikzpicture}}%
    \@endpart}
\def\@spart#1{%
    \startcontents%
    \phantomsection
    {\thispagestyle{empty}%
        \begin{tikzpicture}[remember picture,overlay]%
            \node at (current page.north west){\begin{tikzpicture}[remember picture,overlay]%
                    \fill[ocre!20](0cm,0cm) rectangle (\paperwidth,-\paperheight);
                    \node[anchor=north east] at (\paperwidth-1.5cm,-3.25cm){\parbox[t][][t]{15cm}{\strut\raggedleft\color{white}\fontsize{30}{30}\sffamily\bfseries#1}};
                \end{tikzpicture}};
        \end{tikzpicture}}
    \addcontentsline{toc}{part}{\texorpdfstring{%
            \setlength\fboxsep{0pt}%
            \noindent\protect\colorbox{ocre!40}{\strut\protect\parbox[c][.7cm]{\linewidth}{\Large\sffamily\protect\centering #1\quad\mbox{}}}}{#1}}%
    \@endpart}
\def\@endpart{\vfil\newpage
    \if@twoside
        \if@openright
            \null
            \thispagestyle{empty}%
            \newpage
        \fi
    \fi
    \if@tempswa
        \twocolumn
    \fi}

%----------------------------------------------------------------------------------------
%	CHAPTER HEADINGS
%----------------------------------------------------------------------------------------

% A switch to conditionally include a picture, implemented by Christian Hupfer
% \ifusechapterimage ... \fi 在main.tex中用\usechapterimagefalse或\usechapterimagetrue控制
\newif\ifusechapterimage
\usechapterimagetrue
\newcommand{\thechapterimage}{}%
\newcommand{\chapterimage}[1]{\ifusechapterimage\renewcommand{\thechapterimage}{#1}\fi}%
\newcommand{\autodot}{.}
\def\@makechapterhead#1{ % 影响\chapter命令
    {\parindent \z@ \raggedright \normalfont
            \ifnum \c@secnumdepth >\m@ne
                \if@mainmatter % 影响正文部分,在main.tex中以\mainmatter控制,另有\frontmatter控制序文部分
                    \begin{tikzpicture}[remember picture,overlay]
                        \node at (current page.north west)
                        {\begin{tikzpicture}[remember picture,overlay]
                                \node[anchor=north west,inner sep=0pt] at (0,0) {\ifusechapterimage\includegraphics[width=\paperwidth]{\thechapterimage}\fi};
                                \draw[anchor=west] (\Gm@lmargin,-9cm) node [line width=2pt,rounded corners=15pt,draw=ocre,fill=white,fill opacity=0.5,inner sep=15pt]{\strut\makebox[22cm]{}};
                                \draw[anchor=west] (\Gm@lmargin+1.3cm,-9cm) node {\huge\sffamily\bfseries\color{black}\thechapter\autodot~#1\strut};
                                % 标题和红框分别距离页顶距离为下移9cm
                            \end{tikzpicture}};
                    \end{tikzpicture}
                    \par\vspace*{270\p@} % 标题下正文间距
                \else % 影响非正文部分
                    \begin{tikzpicture}[remember picture,overlay]
                        \node at (current page.north west)
                        {\begin{tikzpicture}[remember picture,overlay]
                                \node[anchor=north west,inner sep=0pt] at (0,0) {\ifusechapterimage\includegraphics[width=\paperwidth]{\thechapterimage}\fi};
                                \draw[anchor=west] (\Gm@lmargin,-3.5cm) node [line width=2pt,rounded corners=15pt,draw=ocre,fill=white,fill opacity=0.5,inner sep=15pt]{\strut\makebox[22cm]{}};
                                \draw[anchor=west] (\Gm@lmargin+1.3cm,-3.5cm) node [align=left] {\huge\sffamily\bfseries\color{black}#1\strut};
                                % 删去了\thechapter\autodot所以可关掉自动编号
                            \end{tikzpicture}};
                    \end{tikzpicture}
                    \par\vspace*{60\p@}
                \fi\fi}}

%-------------------------------------------

\def\@makeschapterhead#1{ % 影响\chapter*命令
    \begin{tikzpicture}[remember picture,overlay]
        \node at (current page.north west)
        {\begin{tikzpicture}[remember picture,overlay]
                \node[anchor=north west,inner sep=0pt] at (0,0) {\ifusechapterimage\includegraphics[width=\paperwidth]{\thechapterimage}\fi};
                \draw[anchor=west] (\Gm@lmargin,-9cm) node [line width=2pt,rounded corners=15pt,draw=ocre,fill=white,fill opacity=0.5,inner sep=15pt]{\strut\makebox[22cm]{}};
                \draw[anchor=west] (\Gm@lmargin+1.3cm,-9cm) node {\huge\sffamily\bfseries\color{black}#1\strut};
            \end{tikzpicture}};
    \end{tikzpicture}
    \par\vspace*{270\p@}}
\makeatother

%----------------------------------------------------------------------------------------
%	LINKS
%----------------------------------------------------------------------------------------

\usepackage[pdfauthor={Kanition},pdftitle={基于物理的渲染:从理论到实现},pdflang={CN},pdfsubject={Physically Based Rendering: From Theory To Implementation},pdfkeywords={graphics,rendering,physically-based-rendering},baseurl={https://github.com/kanition/pbrtbook}]{hyperref}
% \hypersetup{hidelinks,backref=true,pagebackref=true,hyperindex=true,colorlinks=false,breaklinks=true,urlcolor=ocre,bookmarks=true,bookmarksopen=false}
\hypersetup{hidelinks,colorlinks,linkcolor=.,citecolor=blue,breaklinks=true,urlcolor=blue,bookmarksopen=false}

\usepackage{bookmark}
\bookmarksetup{
    open,
    numbered,
    addtohook={%
            \ifnum\bookmarkget{level}=0 % chapter
                \bookmarksetup{bold}%
            \fi
            \ifnum\bookmarkget{level}=-1 % part
                \bookmarksetup{color=ocre,bold}%
            \fi
        }
}


%----------------------------------------------------------------------------------------
% code show
%----------------------------------------------------------------------------------------
\usepackage[ruled,linesnumbered]{algorithm2e}
\usepackage{listings}
\lstset{%
    language={C++}, %language为,还有{[Visual]C++}{[ISO]C++}
    alsolanguage=[ANSI]C, %可以添加很多个alsolanguage,如alsolanguage=matlab,alsolanguage=VHDL等
    tabsize=4, %
    basicstyle=\ttfamily\footnotesize, % 设置代码的大小
    keywordstyle=\color[RGB]{0,84,166}\bfseries, %代码关键字
    stringstyle=\ttfamily\color[RGB]{33,166,86}, % 代码字符串的特殊格式
    commentstyle=\color[RGB]{115,48,11}\scriptsize\rmfamily, %注释
    rulecolor=\color[RGB]{243,102,25},%代码边框
    frame=leftline, %代码框
    framerule=2pt,
    showstringspaces=false,%不显示代码字符串中间的空格标记
    keepspaces=true,
    breakindent=10pt,
    numbers=left,%左侧显示行号 往左靠,还可以为right,或none,即不加行号
    stepnumber=1,%若设置为2,则显示行号为1,3,5,即stepnumber为公差,默认stepnumber=1
    numberstyle={\color[RGB]{33,166,86}\scriptsize} ,%设置行号的大小,大小有tiny,scriptsize,footnotesize,small,normalsize,large等
    numbersep=8pt, %设置行号与代码的距离,默认是5pt
    showspaces=false, %
    flexiblecolumns=true, %
    breaklines=true, %对过长的代码自动换行
    breakautoindent=true,
    aboveskip=1em, %代码块边框
    tabsize=2,
    showstringspaces=false, %不显示字符串中的空格
    backgroundcolor=\color{black!5}, %代码背景色,或\color[rgb]{0.91,0.91,0.91}
    escapeinside=``, %在``里显示中文 %% added by http://bbs.ctex.org/viewthread.php?tid=53451
    fontadjust,
    captionpos=t,
    framextopmargin=2pt,framexbottommargin=2pt,abovecaptionskip=-3pt,belowcaptionskip=3pt,
    xleftmargin=0em,xrightmargin=0em, % 设定listing左右的空白
    texcl=true, % 设定中文冲突,断行,listing数字的样式
    extendedchars=false,% 设定中文冲突
    columns=flexible, % 列模式
    mathescape=true % 设定数学环境输入
}
\newcommand\codecolor[0]{\color[RGB]{142,12,242}} %设置格式

% https://tex.stackexchange.com/questions/17057/hypertarget-seems-to-aim-a-line-too-low
\makeatletter
\newcommand{\htarget}[2]{\Hy@raisedlink{\hypertarget{#1}{}}#2}
\makeatother

\newcommand\initcode[2]{\htarget{code:#1}{\codecolor\itshape{<<{#1}>>#2}}} %代码段名称定义
\newcommand\refcode[2]{\hyperlink{code:#1}{\codecolor\itshape{<<{#1}>>#2}}} %代码段引用跳转
\newcommand\initcnt[1]{\newcounter{#1}\newcounter{#1last}\newcounter{#1next}\setcounter{#1}{0}\setcounter{#1last}{-1}\setcounter{#1next}{1}} %设置三个计数器,记住当前以及前后的编号
\newcommand\addcnt[1]{\stepcounter{#1last}\stepcounter{#1next}\stepcounter{#1}} %三个各自计数器加1
\newcommand\nextcode[1]{\hyperlink{code:#1:\arabic{#1next}}{\htarget{code:#1:\arabic{#1}}{\codecolor $\downarrow$}}} %引用下一段代码
\newcommand\lastcode[1]{\hyperlink{code:#1:\arabic{#1last}}{\htarget{code:#1:\arabic{#1}}{\codecolor $\uparrow$}}} %引用上一段代码
\newcommand\initnext[1]{\initcnt{#1}\nextcode{#1}\addcnt{#1}} %初始化且引用下一段代码
\newcommand\lastnext[1]{\lastcode{#1}\nextcode{#1}\addcnt{#1}} %引用前后代码

% 代码变量定义、引用跳转
\newcommand\refvar[3][]{\ifthenelse{\isempty{#1}}{\hyperlink{codevar:#2}{\ttfamily #2#3}}{\hyperlink{codevar:#1}{\ttfamily #2#3}}}
\newcommand\initvar[3][]{\ifthenelse{\isempty{#1}}{\htarget{codevar:#2}{\ttfamily #2#3}}{\htarget{codevar:#1}{\ttfamily #2#3}}}

\newcommand\reffig[1]{图\ref{fig:#1}}
\newcommand\reftab[1]{表\ref{tab:#1}}
\newcommand\refeq[1]{式\ref{eq:#1}}
\newcommand\refsec[1]{\ref{sec:#1}节}
\newcommand\refchap[1]{\ref{chap:#1}章}
\newcommand\refsub[1]{\ref{sub:#1}节}

\usepackage{pifont}
\newcommand\circleone{\ding{172}}
\newcommand\circletwo{\ding{173}}
\newcommand\circlethree{\ding{174}}

\newcommand\compcolor[1]{\text{\itshape\sffamily\bfseries #1}}
\newcommand{\Equiv}{\ \tikz[baseline=-0.5ex]{\foreach \y in {0.45,0.15,-0.15,-0.45} \draw[yshift=\y ex] (0,0)--(1.5ex,0);}\ }
\usepackage{siunitx}
\usepackage{multirow}