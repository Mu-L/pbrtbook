\begin{center}
      {\LARGE\bfseries 关于本仓库拒绝刊载于GitCode\\及其关联网站的声明}
\end{center}\vspace{30pt}

鉴于GitCode及其关联网站的过往劣迹,
尤其是近来未经授权大量镜像GitHub高流量账号和仓库,
故意伪造出原作者在其网站上开设账号和发布内容的假象,
并使用AI手段批量生成质量堪忧的引流推广软文,以此为网站牟取商业利益,
本仓库作者(署名Kanition)在此严正声明:

\begin{enumerate}
      \item 强烈谴责这些网站违背开源精神、污染开源社区环境、冒充开源作者身份、
            窃取开源作品成果、误导广大用户的无耻行径。
            强调任何举措都应遵守相应的许可协议、通行的开源社区准则与基本的社会公共道德。
            敦促有关实体和个人悬崖勒马,立即停止侵权行为。
      \item {\bfseries 未经本人许可,本仓库拒绝以任何形式被网站官方刊载在GitCode及其关联网站上。
            对于已经刊载的内容,本人要求立即无条件进行下架与彻底删除处理。}
      \item 本声明中的“刊载”形式包括但不限于由网站官方
            (包括但不限于其操控的公共账号、下属员工个人账号和伪装成普通用户的机器账号)
            镜像发布本仓库的源码、编译产物、提交记录、问题与计划列表、讨论区记录等的全部或部分内容,
            以及发布任何提及本仓库的内容。
      \item 本声明中的“GitCode及其关联网站”包括:
            \begin{itemize}
                  \item GitCode:gitcode.com
                  \item 与GitCode实际上存在或疑似存在用户数据共享的网站,
                        或由GitCode的实控组织机构或个人开办的其他网站,包括但不限于:
                        \begin{itemize}
                              \item CSDN:csdn.net
                              \item OSCHINA:oschina.net
                              \item Gitee:gitee.com
                              \item 华为云:huaweicloud.com
                              \item InsCode:inscode.csdn.net
                        \end{itemize}
                  \item 以上述网站名义在其他网络平台上开办的公共账号空间,
                        包括但不限于微博、微信公众号、知乎、抖音、bilibili、
                        百度百家号、今日头条、简书、YouTube、X(原Twitter)、GitLab、极狐等。
            \end{itemize}
      \item 普通个人用户(上述网站的实控组织机构成员除外)可在遵守许可协议的前提下
            自由地于上述网站发布本仓库相关内容而不受前述条款限制。
      \item {\bfseries 任何用户发布与本仓库相关的内容都不得冒用本人署名“Kanition”,也不得使用极其相似的署名误导用户。}
      \item 呼吁上述实体和个人发扬开源共享精神,维护开源社区良好秩序,共同推动开源事业高质量发展。
      \item 本人保留依据事态后续发展作出进一步反应并调整本声明内容的权利。
\end{enumerate}
