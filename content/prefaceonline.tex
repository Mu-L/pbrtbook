{\Huge\bfseries 在线版序言}\vspace{30pt}\\

2004年发行的第一版《基于物理的渲染》只有纸质书。
2010年第二版新增了Kindle版,但不幸的是
所有交叉引用和索引都在转换中丢失了。
终于,2016年发布的第三版转换出了良好的Kindle版和PDF版。
尽管电子版有所改进,但我们觉得它还远称不上完美。

文学编程是《基于物理的渲染》的核心。
它是Donald Knuth提出的一种软件编写方法,
比起在计算机上把源码转换为可执行指令,
它更重视人类阅读源码时的直观性。
文学编程将复杂程序分解为便于理解的片段,
并提供多种方式对其交叉引用,以帮助理解每个片段的内容。

对于纸上的文学编程,每页都有丰富的辅助信息并编有定向页码。
边栏有索引指向当前页所用标识符对应的代码定义所在的页码,
并且每个代码段都有表示其它部分定义所在页码和被引用的页码。

这种格式很有效,但页码绕得烦人。
而且翻书找页也很麻烦。
我们想到,如果把电子设备——台式机、笔记本、平板甚至手机——
作为本书内容的主要交付工具,我们会获得怎样的阅读体验?
没有页码了,取而代之的是超链接,直接把读者引向目标且容易返回原处。

改善的不仅只有导航:
现代显示器比打印纸有好得多的色彩保真度和动态范围,
结合计算机使用还能与书中图示交互。
对于通篇在讲图像与三维世界的书籍而言这是极大的优点。

2018年夏季,我们获取到出版商的授权;
我们非常感谢他们慷慨归还版权。
这样我们就能自由决定是否尝试以这种新形式呈现本书内容。
我们的答案是肯定的。
废寝忘食一个月后,
我们实现了一个系统,
把本书从以前编写时所用的标记语言转换为HTML。
你现在读的就是它\sidenote{译者注:原作者可能没料到有人又把它翻译回PDF了。}。

本书在线版与第三版《基于物理的渲染》很接近。
我们只作了以下修改:
\begin{itemize}
    \item 更新一些插图所用的渲染图像,
    \item 为图像查看增加交互,
    \item 重画所有插图,
    \item 把比较同一场景的不同渲染结果的多幅图像合并为一幅图像,
    \item 合并读者反馈的勘误。
\end{itemize}

前两点还需要说明一下。关于更新图像:
纸质书的一大挑战是确保诸如蒙特卡罗噪声等图像痕迹在页面上可见。
我们担心印刷会引入多余的模糊,
也担心成书过程中有好心人帮倒忙给图像降噪使其看起来更清晰。
因此我们用最近邻滤波器放大了本应展现差错的图像,
使这些差错能在印刷过程中保留下来。

现在这个担心是多余的了。
很高兴能重新渲染这些图像,
连一个像素大小的差错都能保留了。

第二点修改标志着我们朝交互式内容探究迈出了第一步:
在网页浏览器中,可以细究渲染图像的细节和区别,
这是纸质书做不到的。
在线版大多数渲染图像都能放大、全屏查看以及和其他图像比较。
它们都有此图标
\sidenote{译者注:在线版是一片雪花图案。
    不过PDF没法实现网页端那么强大的交互功能,
    所以阅读本书时就无视它吧。}:*。
将鼠标悬停在该图标上可获取相关操作的详细信息。\\

\noindent{\LARGE\bfseries 路线图}

本书计划大致每年发布一次更新
\sidenote{译者注:本书在翻译时已经发布第四版书籍与代码了。}。
尽管在线版比纸质版更容易快速更新,
我们还是认为适当的更新速度有利于对下次发布做严谨的检查和编辑。

除了扩展pbrt功能跟进最新研究,
我们还计划为新版增加更多交互元素。
\citet{4b212a02-105c-42a2-ad5c-91c16a06e815}
编写的《\citetitle{4b212a02-105c-42a2-ad5c-91c16a06e815}》\footnote{\citeurl{4b212a02-105c-42a2-ad5c-91c16a06e815}}
一书展现了这种媒体的无限可能。

我们会在线保留本书的早前版本,URL均和首发时保持一致;
新版会放在单独的目录中。
因此链接到此处的内容是安全的,不必担心未来断链
\sidenote{译者注:本中译版不作此承诺。}。\\

\noindent{\LARGE\bfseries 报告错误}

{\itshape 详见英文原版}\\

\noindent{\LARGE\bfseries 致谢}

{\itshape 详见英文原版}\\

\noindent{\LARGE\bfseries 许可证}

{\itshape 详见英文原版}