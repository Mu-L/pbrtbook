\chapterimage{Pictures/chap07/checkerboard-ref-465x930.png}
\chapter{采样与重构}\label{chap:采样与重构}
\setcounter{sidenote}{1}
尽管像pbrt那样的渲染器最终输出的是彩色像素的2D网格,
但实际上入射辐射是定义在胶片平面上的连续函数。
从该连续函数计算出离散像素值的方法会显著影响渲染器生成的最终图像的质量;
如果没有仔细执行该过程,则会出现伪影\sidenote{译者注:原文artifact。}。
反之,如果执行得很好,则为此进行相对少量的额外计算就能极大提升渲染图像的质量。

本章从介绍\emph{采样理论}开始——即从定义在连续域上的函数
取出离散样本值并用它们重建与原本类似的新函数的理论。
在采样理论和低偏差点集(一种均匀分布的样本点类型)思想的基础上,
本章定义的\refvar{Sampler}{}以不同方式生成$n$维样本向量
\footnote{回想上一章中\refvar{Camera}{}用\refvar{CameraSample}{}
    在胶片平面、透镜上以及时间域中取点——通过取用这些样本向量前几维来设定\refvar{CameraSample}{}值。}。
本章将介绍五种\refvar{Sampler}{}实现,涵盖了采样问题的各种方法。

本章以类\refvar{Filter}{}和\refvar{Film}{}作结。
\refvar{Filter}{}用于确定每个像素周围要融合多少倍样本量来计算最终像素值,
类\refvar{Film}{}则积累图像样本对图中像素的贡献量。

\section{采样理论}\label{sec:采样理论}
数字图像表示为一组像素值,通常对齐到矩形网格。
当在物理设备上展示数字图像时,这些值用于确定显示器上像素发射的光谱功率。
当考虑数字图像时,区分图像像素与显示器像素很重要,
前者表示一个函数在特定样本位置的值,后者是具有某个发光分布的物理对象
(例如对于LCD显示器,当以倾斜角度观察它时,颜色和亮度可能会极大变化)。
显示器用图像像素值在显示器表面构造新的图像函数。
该函数定义在显示器所有点位上,而不只是数字图像像素的无穷小点上。
这样取一组样本值并将其转换回连续函数的过程称为\keyindex{重建}{reconstruction}{}。

为了计算数字图像中的离散像素值,必须采样原始连续定义的图像函数。
在pbrt中,像大多数其他光线追踪渲染器那样,
获取图像函数有关信息的唯一方法就是通过追踪光线来对其采样。
例如,能计算胶片平面上两点间的图像函数变化边界的通用方法是不存在的。
尽管可以通过在像素位置上精确采样该函数来生成图像,
但通过在不同位置上取用更多样本并将这些关于图像函数的
额外信息融合到最终的像素值中能得到更好的结果。
实际上,为了有最佳质量的结果,计算像素值时应使得
在显示设备上重建的图像尽可能与虚拟相机胶片平面上的场景原始图像逼近。
注意这和希望显示器像素在其位置上取用图像函数实际值的目标有些微妙区别。
处理这一区别是本章实现的算法的主要目标
\footnote{本书中我们将忽略物理显示器像素特性相关问题并
    在显示器执行本节后面所述理想重建过程的假设下处理。
    该假设显然与真实显示器的工作方式不符,但这里它避免了不必要的复杂分析。
    \citet{GLASSNER1995}的第3章很好地处理了非理想显示设备
    及其对图像采样和重建过程的影响。}。

因为采样和重建过程涉及估值,所以它引入了称作\keyindex{混叠}{aliasing}{alias混叠}的误差,
并会以许多方式表现出来,包括锯齿状边缘或动画中的闪烁。
产生这些误差的原因是采样过程不能捕获来自连续定义的图像函数的全部信息。

作为这些思想的一个例子,考虑一个1D函数(我们也会称之为信号)即$f(x)$,
我们可以求函数定义域中任意期望位置$x'$处的值$f(x')$.
每个这样的$x'$称为\keyindex{样本位置}{sample position}{},
$f(x')$的值称为\keyindex{样本值}{sample value}{}。
\reffig{7.1}展示了光滑1D函数的样本集,以及逼近原始函数$f$的重建信号$\tilde{f}$.
本例中,$\tilde{f}$是分段线性函数,通过线性插值相邻样本值来逼近$f$
(已经熟悉采样理论的读者会认出这是用帽函数\sidenote{译者注:原文hat function。}重建的)。
因为关于$f$唯一可用的信息是来自在位置$x'$处的采样值,
且没有关于$f$在样本间特性的信息,所以$\tilde{f}$不能完全匹配$f$.
\begin{figure}[htbp]
    \centering
    \subfloat[]{\includegraphics[width=0.4\linewidth]{chap07/point-sampling.eps}}\,\,
    \subfloat[]{\includegraphics[width=0.4\linewidth]{chap07/linear-reconstruction.eps}}
    \caption{(a)通过取$f(x)$的\emph{样本点}集(标实心记),我们确定了函数在这些位置处的值。
        (b)样本值可用于\emph{重建}逼近$f(x)$的函数$\tilde{f}(x)$.
        \refsub{混叠}介绍的采样原理准确描述了关于$f(x)$的条件、
        所需样本的数目,以及使得$\tilde{f}(x)$和$f(x)$一模一样的重建技术。
        原始函数有时能只从样本点中完全重建的事实令人瞩目。}
    \label{fig:7.1}
\end{figure}

\keyindex{傅里叶分析}{Fourier analysis}{}可用于评估重建函数与原始函数间的匹配质量。
本节将用丰富细节来介绍一部分采样和重建过程中涉及的傅里叶分析主要思想,
但略去了许多性质的证明并跳过了与pbrt所用的采样算法没有直接关系的细节。
本章“扩展阅读”一节有关于这些话题详细信息的指引。

\subsection{频域与傅里叶变换}\label{sub:频域与傅里叶变换}
傅里叶分析的基础之一是\keyindex{傅里叶变换}{Fourier transform}{},
它在\keyindex{频域}{frequency domain}{}中来表示函数(我们称
通常的函数是在\keyindex{空域}{spatial domain}{}中表示的)。
考虑\reffig{7.2}中的两个函数。\reffig{7.2.1}中$x$的函数变化得相对较慢,
而\reffig{7.2.2}中的函数变化得迅速得多。称变化越慢的函数有越低频的分量。
\begin{figure}[htbp]
    \centering
    \subfloat[]{\includegraphics[width=0.32\linewidth]{chap07/func-lowfreq.eps}\label{fig:7.2.1}}\,\,\,\,
    \subfloat[]{\includegraphics[width=0.32\linewidth]{chap07/func-highfreq.eps}\label{fig:7.2.2}}
    \caption{(a)低频函数和(b)高频函数。粗略地说,函数频率越高,在给定区域内变化得越快。}
    \label{fig:7.2}
\end{figure}

\reffig{7.3}展示了这两个函数在频率空间的表示;低频函数的表示比高频函数更快变为0.
\begin{figure}[htbp]
    \centering
    \subfloat[]{\includegraphics[width=0.32\linewidth]{chap07/fourier-lowfreq.eps}}\,\,\,\,
    \subfloat[]{\includegraphics[width=0.32\linewidth]{chap07/fourier-highfreq.eps}}
    \caption{\reffig{7.2}中的函数的频率空间表示。本图展示了每个频率$\omega$对空域中每个函数的贡献。}
    \label{fig:7.3}
\end{figure}

许多函数可以分解为平移过的正弦曲线的加权和。
约瑟夫·傅里叶\sidenote{译者注:Jean-Baptiste Joseph Fourier,
    18至19世纪法国著名数学家和物理学家。其中文译名还常作“傅立叶”。}
首先描述了这一奇特事实,傅里叶变换即将函数转换为该表示。
函数的频率空间表示便于深入了解其一些特点——正弦函数的频率分布对应于原函数的频率分布。
使用该形式后,就能用傅里叶分析深入了解采样和重建过程引入的误差以及如何降低该误差带来的感知影响。

1D函数$f(x)$的傅里叶变换为\footnote{要告知读者的是,
    在不同领域中该积分前的常数并不总是一样的。例如一些作者(包括许多物理界的)
    更喜欢在两个积分前乘上$\frac{1}{\sqrt{2\pi}}$.}
\begin{align}\label{eq:7.1}
    F(\omega)=\int_{-\infty}^{\infty}f(x)\mathrm{e}^{-\mathrm{i}2\pi\omega x}\mathrm{d}x\, .
\end{align}
(回想$\mathrm{e}^{-\mathrm{i}x}=\cos x+\mathrm{i}\sin x$,其中$\mathrm{i}=\sqrt{-1}$.)
为了简便,这里我们将只考虑\keyindex{偶函数}{even function}{},
即$f(-x)=f(x)$,这种情况下$f$的傅里叶变换没有虚数项。
新函数$F$是\keyindex{频率}{frequency}{}$\omega$的函数
\footnote{本章中,我们将用符号$\omega$表示频率。在本书剩下部分中,$\omega$表示规范化的方向向量。
    这种记号的重复在使用它们的给定上下文中不应混淆。简单来说,当我们说函数的“频谱”(spectrum)时,
    我们是在说它在其频率空间表示中的频率分布,而不是和颜色相关的东西。}。
我们将用$\mathcal{F}$表示傅里叶变换运算
\sidenote{译者注:原文使用的符号是$\mathrm{F}$,这里译者换用更常用的花体$\mathcal{F}$,
    也更利于阅读时与其他符号区分。},即$\mathcal{F}\{f(x)\}=F(\omega)$.
$\mathcal{F}$显然是线性运算——即对任意标量$a$都有$\mathcal{F}\{af(x)\}=a\mathcal{F}\{f(x)\}$,
且$\mathcal{F}\{f(x)+g(x)\}=\mathcal{F}\{f(x)\}+\mathcal{F}\{g(x)\}$.

\refeq{7.1}称为傅里叶分析方程,
有时简称\keyindex{傅里叶变换}{Fourier transform}{}
\sidenote{译者注:一般要求函数$f(x)$是实数域上
    的\keyindex{可积函数}{integrable function}{}。}。
我们也可用\keyindex{傅里叶合成}{Fourier synthesis}{}方程从频域变换回空域,
也称作\keyindex{傅里叶逆变换}{inverse Fourier transform}{}
\sidenote{译者注:本书定义中$\omega$是频率,此外用角频率的定义也很常见。}:
\begin{align}\label{eq:7.2}
    f(x)=\int_{-\infty}^{\infty}F(\omega)\mathrm{e}^{\mathrm{i}2\pi\omega x}\mathrm{d}\omega\, .
\end{align}

\reftab{7.1}展示了许多重要函数及其频率空间表示
\sidenote{译者注:表中原文将频域函数写作$f(\omega)$,译者改为了$F(\omega)$.
    译者还把原文余弦函数的频域表示改为等价但更常见的写法;
    此外,原文表中shah函数一行系数与后文矛盾,已对其作了修正。
    具体推导过程可参考译者补充的\refsec{译者补充:傅里叶变换}。}。
这些函数中许多都基于\keyindex{狄拉克$\delta$分布}{Dirac delta distribution}{},
该空间函数的定义使得$\displaystyle\int_{-\infty}^{\infty}\delta(x)\mathrm{d}x=1$,且对任意$x\neq0$,都有$\delta(x)=0$.
这些性质的一个重要结论是\sidenote{译者注:我为该式加上了积分上下限以表明它是定积分。}
\begin{align*}
    \int_{-\infty}^{\infty} f(x)\delta(x)\mathrm{d}x=f(0)\, .
\end{align*}

$\delta$分布不能表示为标准数学函数,但通常可以视作以原点为中心且宽度逼近0的
单位面积矩形函数\sidenote{译者注:原文box function。}的极限。
\begin{table}[htbp]
    \centering\begin{tabular}{l p{170pt}}
        \toprule
        {\bfseries 空域}                                         & {\bfseries 频率空间表示}                                                                                       \\
        \midrule
        矩形函数:$f(x)=\left\{\begin{array}{ll}
                1, & \text{若}|x|<\frac{1}{2}, \\
                0, & \text{其他}.
            \end{array}\right.$ & sinc函数:$\displaystyle F(\omega)=\mathrm{sinc}(\omega)=\frac{\sin(\pi\omega)}{\pi\omega}$                    \\
        \hline
        高斯函数:$f(x)=\mathrm{e}^{-\pi x^2}$                   & 高斯函数:$F(\omega)=\mathrm{e}^{-\pi \omega^2}$                                                               \\
        \hline
        常函数:$f(x)=1$                                         & $\delta$函数:$F(\omega)=\delta(\omega)$                                                                       \\
        \hline
        余弦函数:$f(x)=\cos x$                                  & 平移的$\delta$函数:
        $F(\omega)=\frac{1}{2}\left(\delta\left(\omega-\frac{1}{2\pi}\right)+\delta\left(\omega+\frac{1}{2\pi}\right)\right)$                                                     \\
        \hline
        shah函数:$f(x)=III_T(x)=\sum\limits_k\delta(x-kT)$      & $F(\omega)=\frac{1}{T}III_{\frac{1}{T}}(\omega)=\frac{1}{T}\sum\limits_k\delta\left(\omega-\frac{k}{T}\right)$ \\
        \bottomrule
    \end{tabular}
    \caption{傅里叶变换对。空域中的函数及其频率空间表示。
        因为傅里叶变换的对称性,如果左边一列被当作频率空间,
        则右边一列是这些函数的空间等价。}
    \label{tab:7.1}
\end{table}

\subsection{理想采样与重建}\label{sub:理想采样与重建}
利用频率空间分析,我们现在能正式研究采样的性质了。
回想采样过程要求我们选择一组等间隔样本位置并计算这些位置的函数值。
形式上,这对应于让该函数乘以“shah”——或称“冲激串”函数
\sidenote{译者注:也称\keyindex{狄拉克梳状函数}{Dirac comb function}{}、
    \keyindex{shah函数}{shah function}{}或\keyindex{采样函数}{sampling function}{}。},
即无数等间隔的$\delta$函数之和。
shah函数$III_T(x)$定义为
\sidenote{译者注:为了和虚数单位$\mathrm{i}$更好区分,
    译者将式子中的下标$i$改为$k$,下同;
    此外,译者去掉了该式系数$T$.}
\begin{align*}
    III_T(x)=\sum\limits_{k=-\infty}^{\infty}\delta(x-kT)\, ,
\end{align*}
其中$T$定义了\keyindex{周期}{period}{},也称\keyindex{采样率}{sampling rate}{}。
\reffig{7.4}展示了采样的正式定义。相乘后得到函数在等间隔点处取值的无限序列
\sidenote{译者注:我去掉了该式系数$T$.}:
\begin{align*}
    III_T(x)f(x)=\sum\limits_k\delta(x-kT)f(kT)\, .
\end{align*}
\begin{figure}[htbp]
    \centering
    \subfloat[]{\includegraphics[width=0.32\linewidth]{chap07/func-to-sample.eps}}\,
    \subfloat[]{\includegraphics[width=0.32\linewidth]{chap07/shah-samples.eps}}\,
    \subfloat[]{\includegraphics[width=0.32\linewidth]{chap07/shah-sampled-function.eps}}
    \caption{形式化的采样过程。(a)函数$f(x)$乘以(b)shah函数$III_T(x)$,
        得到(c)表示其在每个样本点处取值的缩放后的$\delta$函数的无限序列。}
    \label{fig:7.4}
\end{figure}

通过选择重建\keyindex{滤波}{filter}{}函数$r(x)$并计算\keyindex{卷积}{convolution}{},
这些样本值可用于定义重建的函数$\tilde{f}$,即
\begin{align*}
    (III_T(x)f(x))\otimes r(x)\, ,
\end{align*}
其中卷积运算$\otimes$定义为
\begin{align*}
    f(x)\otimes g(x)=\int_{-\infty}^{\infty}f(x')g(x-x')\mathrm{d}x'\, .
\end{align*}

对于重建,卷积给出以样本点为中心缩放后的重建滤波器实例加权和
\sidenote{译者注:我去掉了该式系数$T$.}:
\begin{align*}
    \tilde{f}(x)=\sum\limits_{k=-\infty}^{\infty}f(kT)r(x-kT)\, .
\end{align*}

例如\reffig{7.1}中使用了三角形重建\keyindex{滤波器}{filter}{}$r(x)=\max(0,1-|x|)$.
\reffig{7.5}展示了为该例所用的缩放后的三角形函数。
\begin{figure}[htbp]
    \centering\includegraphics[width=0.6\linewidth]{chap07/func-tri-reconstruction.eps}
    \caption{虚线表示的三角形重建滤波器实例的和给出了实线表示的对原始函数的重建逼近。}
    \label{fig:7.5}
\end{figure}

为了得到直观的结果,我们经历了看似不用这么复杂的过程:
用一些方法在样本间插值也能得到重建函数$\tilde{f}(x)$.
然而通过仔细构建这些背景,傅里叶分析现在能更简单地用于该过程。

通过在频域分析采样函数,我们能更深入理解采样过程。
特别地,我们将能确定原始函数能从其在样本位置的取值中完全恢复的条件——一个很强的结论。
对于此处的讨论,我们现在假设函数$f(x)$是\keyindex{带限}{band limited}{}的——
存在某个频率$\omega_0$使得$f(x)$在大于$\omega_0$处不再包含任何频率。
根据定义,带限函数具有紧支撑\sidenote{译者注:compact support。}的频率空间表示,
即对于所有$|\omega|>\omega_0$都有$F(\omega)=0$.\reffig{7.3}中的两个频谱都是带限的。

傅里叶分析所用的一个重要思想是两个函数之积的傅里叶变换$\mathcal{F}\{f(x)g(x)\}$可
表示为它们各自傅里叶变换$F(\omega)$和$G(\omega)$的卷积:
\begin{align*}
    \mathcal{F}\{f(x)g(x)\}=F(\omega)\otimes G(\omega)\, .
\end{align*}

类似地,空域卷积等价于频域乘积:
\begin{align}\label{eq:7.3}
    \mathcal{F}\{f(x)\otimes g(x)\}=F(\omega)G(\omega)\, .
\end{align}

这些性质是傅里叶分析的标准参考文献中得来的。
利用这些思想可以发现,空域中原始的采样步骤,即shah函数与原始函数$f(x)$相乘,
可等价描述为频域中$F(\omega)$与另一shah函数的卷积。

从\reftab{7.1}中我们还知道shah函数$III_T(x)$的频谱;
周期为$T$的shah函数的傅里叶变换是另一个周期为$\displaystyle\frac{1}{T}$的shah函数。
牢记周期间的倒数关系很重要:它意味着如果样本在空域中隔得较远,
则它们在频域中离得更近。

因此采样信号的频域表示通过$F(\omega)$和新的shah函数的卷积给出。
让$\delta$函数与一个函数卷积得到该函数副本,故用shah函数卷积
得到原始函数副本的无限序列,间隔等于该shah的周期(\reffig{7.6})。
它是样本序列的频率空间表示。
\begin{figure}[htbp]
    \centering\includegraphics[width=0.45\linewidth]{chap07/func-convolve-shah.eps}
    \caption{$F(\omega)$与shah函数的卷积。结果是$F$的无数个副本。}
    \label{fig:7.6}
\end{figure}

现在我们有了该函数频谱副本的无限集,我们该怎样重建原始函数呢?
观察\reffig{7.6},答案很明显:只需要抹除除了以原点为中心外的所有频谱副本,就能得到原始的$F(\omega)$.
\begin{figure}[htbp]
    \centering
    \subfloat[]{\includegraphics[width=0.32\linewidth]{chap07/func-convolve-shah-a.eps}}\,
    \subfloat[]{\includegraphics[width=0.32\linewidth]{chap07/unit-box-filter.eps}}\,
    \subfloat[]{\includegraphics[width=0.32\linewidth]{chap07/single-func-after-box.eps}}
    \caption{(a)$F(\omega)$的副本序列和(b)合适的矩形函数相乘得到(c)原始频谱。}
    \label{fig:7.7}
\end{figure}

为了丢弃除了中间外的所有频谱副本,我们乘以具有合适宽度的矩形函数(\reffig{7.7})。
宽度为$T$的矩形函数$\textstyle\prod_T(x)$定义为
\begin{align*}
    {\textstyle\prod_T}(x)=\left\{\begin{array}{ll}
        \displaystyle\frac{1}{T}, & \text{若}\displaystyle|x|<\frac{T}{2}, \\
        0,                        & \text{其他}.
    \end{array}\right.
\end{align*}

该相乘步骤对应了用重建滤波器在空域做卷积。这是理想采样与重建过程。总结为
\sidenote{译者注:我增加了系数$\frac{1}{T}$.}:
\begin{align*}
    \tilde{F}=(F(\omega)\otimes \frac{1}{T}III_{\frac{1}{T}}(\omega))\textstyle\prod_{\frac{1}{T}}(\omega)\, .
\end{align*}

这是个重要结论:仅仅通过采样一组均匀间隔的点,我们就能确定$f(x)$的精准频率空间表示。
除了知道该函数是带限的外,没有使用关于函数成分的额外信息。

在空域运用等价过程同样能完全恢复$f(x)$.因为矩形函数的傅里叶逆变换是sinc函数,
所以空域中的理想重建是
\begin{align*}
    \tilde{f}=(f(x)III_T(x))\otimes \mathrm{sinc}_T(x)\, ,
\end{align*}
其中$\mathrm{sinc}_T(x)=\mathrm{sinc}(Tx)$,因此\sidenote{译者注:原文该式
    将$\mathrm{sinc}_T(x-kT)$误写为$\mathrm{sinc}(x-kT)$,已修正。}
\begin{align*}
    \tilde{f}(x)=\sum\limits_{k=-\infty}^{\infty}\mathrm{sinc}_T(x-kT)f(kT)\, .
\end{align*}

不幸的是,因为sinc函数有无限定义域,所以必须用所有采样值$f(kT)$来计算空域中$\tilde{f}(x)$的任一特定值。
实际实现中更爱用空间范围有限的滤波器,即使它们不能完美重建原始函数。

图形学常用的可选方法是用矩形函数做重建,即高效地对$x$附近区域内的全部样本值做平均。
通过考虑矩形滤波器的频域表现可以看到这是非常糟糕的选择:
该技术试图通过\emph{乘以sinc}来分离出函数频谱中间的副本,
这不仅在选出函数频谱中央副本方面做得很差,
还包含了无限序列中其他副本的高频贡献。

\subsection{混叠}\label{sub:混叠}
除了sinc函数无限定义域的问题外,理想采样与重建方法一个最严重的实际问题是它假设信号是带限的。
对于非带限信号,或者没能以足够高的采样率采样其频率成分的信号,
之前描述的过程会重建出与原始信号不同的函数。

成功重建的关键是用宽度合适的矩形函数相乘以完全恢复原始频谱$F(\omega)$的能力。
注意在\reffig{7.6}中,信号频谱的副本被空白空间分隔,所以能够被完美重建。
然而如果以更低的采样率采样原始函数,考虑一下会发生什么。
回想周期为$T$的shah函数$III_T$的傅里叶变换是周期为$\displaystyle\frac{1}{T}$的新shah函数。
这意味着如果空域中样本间的距离增大,频域的样本间隔会变小,
将频谱$F(\omega)$的副本挤在一起。如果副本挨得太近,它们就开始重叠。

因为副本被加在一起,所以得到的频谱看起来不再像许多原始的副本(\reffig{7.8})。
当该新频谱乘以矩形函数后,结果是相似但不等于原始$F(\omega)$的频谱:
原始信号的高频细节渗入到重建信号频谱的低频区域。
这些新的低频伪影称作\keyindex{混叠}{alias}{}(因为高频“伪装”成低频),
得到的信号被称是\keyindex{混叠的}{aliased}{alias混叠}。
\begin{figure}
    \centering
    \subfloat[]{\includegraphics[width=0.4\linewidth]{chap07/freq-space-overlap.eps}}\,
    \subfloat[]{\includegraphics[width=0.4\linewidth]{chap07/freq-space-aliasing.eps}}
    \caption{(a)采样率过低时,函数频谱副本会重叠,当执行重建时会导致(b)混叠。}
    \label{fig:7.8}
\end{figure}

\reffig{7.9}\sidenote{译者注:原文该图题注函数表达式有笔误,已修正。}
展示了欠采样并重建1D函数$f(x)=1+\cos(4\pi x^2)$时的混叠效应。
\begin{figure}[htbp]
    \centering
    \subfloat[]{\includegraphics[width=0.4\linewidth]{chap07/freq-increasing-func.eps}}\,
    \subfloat[]{\includegraphics[width=0.4\linewidth]{chap07/freq-increasing-aliased.eps}}
    \caption{函数$f(x)=1+\cos(4\pi x^2)$采样点的混叠。(a)该函数。
        (b)以0.125单位为样本间隔采样并用sinc滤波器执行完美重建后所重建出的函数。
        混叠造成原始函数的高频信息被丢失了并作为低频误差重新出现。}
    \label{fig:7.9}
\end{figure}

一种可能解决重叠频谱问题的办法是简单地增加采样率
直到频谱的副本隔得足够远而不再重叠,进而完全消除混叠。
事实上,\keyindex{采样定理}{sampling theorem}{}准确告诉我们所需的采样率。
该定理说只要均匀样本点的频率$\omega_s$大于信号中出现的最大频率$\omega_0$的两倍,
就能从样本中完美重建原始信号。该最小采样频率称为\keyindex{奈奎斯特频率}{Nyquist frequency}{frequency频率}
\sidenote{译者注:哈里·奈奎斯特(Harry Nyquist),19至20世纪瑞典裔美国著名物理学家,通讯理论的奠基者之一。}。

对于非带限信号($\omega_0=\infty$),不可能以足够高的采样率执行完美重建。
非带限信号有无限支撑的频谱,所以无论其频谱副本隔得多远(即无论我们用多高的采样率),
都总会有重叠。不幸的是,计算机图形学中要处理的函数很少是带限的。
特别地,任何不连续的函数都不是带限的,因此我们不能完美采样和重建它。
这是有道理的,因为两个样本间的函数连续性是不清楚的,样本没有提供不连续处的信息。
因此除了提高采样率外还必须用不同方法来消除混叠可能引入到渲染器结果中的误差。

\subsection{抗锯齿技术}\label{sub:抗锯齿技术}
如果不仔细对待采样和重建,则最终图像中可能出现大量伪影。
有时区分采样伪影与重建伪影很有用;确切地说,我们会称采样伪影为\keyindex{预混叠}{prealiasing}{alias混叠},
称重建伪影为\keyindex{后混叠}{postaliasing}{alias混叠}。任意想要修正这些误差的尝试都
大体划分为\keyindex{抗锯齿}{antialiasing}{}\sidenote{译者注:也称反混叠。}。
本节将回顾除了只增加采样率外的大量抗锯齿技术。

\subsubsection*{非均匀采样}
尽管知道我们要采样的图像函数有无穷的频率成分因而不能从样本点中完美重建,
但以非均匀的方式改变样本间隔有可能降低混叠的视觉影响。
如果$\xi$表示在0到1间的随机数,则基于冲激串的非均匀样本集为
\begin{align*}
    \sum\limits_{k=-\infty}^{\infty}\delta\left(x-(k+\frac{1}{2}-\xi)T\right)\, .
\end{align*}

对于不足以刻画该函数的固定采样率,均匀和非均匀采样都会得到不正确的重建信号。
然而非均匀采样倾向于将规则的混叠伪影转化为不容易引起人类视觉系统注意的噪声。
在频率空间,采样信号的副本最终也被随机平移,
所以当执行重建时结果是随机误差而不是有条理的混叠。

\subsubsection*{自适应采样}
另一个对抗混叠的建议方法是\keyindex{自适应超采样}{adaptive supersampling}{}:
如果我们能辨别出频率高于奈奎斯特上限的信号区域,
则我们可以在这些区域再取额外样本而不用承担在每处都增加采样频率所致的计算开销。
在实际中让该方法奏效是很困难的,因为寻找所有需要超采样的地方会很难。
大多数这样做的技术都基于测试相邻样本值并找到两值间有明显变化的地方;
然后假设该区域信号有较高频率。

通常相邻样本值不能确定地告诉我们它们之间到底发生了什么:
即使它们值相同,函数也可能在它们间有巨大变化。
或者相邻样本可能有相差很大的值但实际上并没有出现任何混叠。
例如,第\refchap{纹理}的纹理滤波算法全力消除场景中图像贴图和表面过程纹理造成的混叠;
我们不想让自适应采样例程在纹理值迅速变化但实际上没有出现过高频率的区域不必要地采额外样本。

\subsubsection*{预滤波}
采样理论提供的另一个消除混叠的方法是对原始函数滤波(即模糊)
使得所用采样率不能精确捕获的高频率不再保留下来。
该方法应用于第\refchap{纹理}的纹理函数。
该技术通过从被采样函数中移除信息来改变其特性,模糊一般不如混叠令人讨厌。

回想我们要用选定宽度的矩形滤波器与原始函数频谱相乘使得
奈奎斯特上限之上的频率被移除。在空域,这对应于原始函数与sinc滤波器做卷积,
\begin{align*}
    f(x)\otimes \mathrm{sinc}(2\omega_sx)\, .
\end{align*}

在实际中,我们也可以用范围有限的滤波器。该滤波器的频率空间表示能
帮助弄清它能有多逼近理想sinc滤波器的表现。

\reffig{7.10}展示了函数$1+\cos(4\pi x^2)$与\refsec{图像重建}
介绍的范围有限的sinc变种的卷积\sidenote{译者注:原文正文与插图
    题注的函数表达式均有笔误,与图示不符,译者已修改。}。
注意高频细节被消除了;该函数可用\reffig{7.9}的采样率采样和重建而无混叠。
\begin{figure}[htbp]
    \centering\includegraphics[width=0.4\linewidth]{chap07/highfreq-prefiltered.eps}
    \caption{函数$1+\cos(4\pi x^2)$与移除采样率$T=0.125$对应的
        奈奎斯特上限之外频率的滤波器的卷积。高频细节已从该函数移除掉,
        使得新函数至少能无混叠地被采样与重建。}
    \label{fig:7.10}
\end{figure}

\subsection{应用到图像合成}\label{sub:应用到图像合成}
这些思想应用到2D情况的渲染场景的图像采样和重建很简单:
我们有一张图像可视作2D图像位置$(x,y)$到辐亮度值$L$的函数:
\begin{align*}
    f(x,y)\rightarrow L\, .
\end{align*}

好消息是,有了我们的光线追踪器,我们能在我们所选的任意点$(x,y)$处求该函数的值。
坏消息是,一般不太可能在采样前对$f$预滤波来从中移除高频。
因此本章采样器使用两种策略,既将采样率提升至超过最终图像基础像素间隔,
也有非均匀分布样本以将混叠转化为噪声。

将场景函数的定义推广为也依赖于时间$t$以及采样处的透镜位置$(u,v)$的更高维函数很有用。
因为来自相机的光线基于这五个量,它们中任意一个变化都会得到不同的光线,
进而可能是不同的$f$值。对于特定的图像位置,该点的辐亮度一般
随时间(如果场景中有运动物体)和透镜上的位置(如果相机有光圈有限的透镜)变化。

更一般地,因为第\refchap{光传输I:表面反射}到第\refchap{光传输III:双向方法}
定义的许多积分器都用统计技术来估计沿给定光线的辐亮度,
所以当重复给定相同光线时它们可能返回不同的辐亮度值。
如果我们进一步将场景辐亮度函数扩展至包含积分器所用的样本值
(例如,为了照明计算而用于在面光源上选点的值),
我们就有甚至更高维的图像函数
\begin{align*}
    f(x,y,t,u,v,i_1,i_2,\ldots)\rightarrow L\, .
\end{align*}

采样好所有这些维度是高效生成高质量图像的重要一部分。
例如如果我们保证图像上位置$(x,y)$附近倾向于在透镜上有不同的$(u,v)$,
则得到的渲染图像会有更小的误差,因为每个样本更有可能考虑了其相邻样本没有考虑的场景信息。
后面几节的类\refvar{Sampler}{}会解决高效采样所有这些维度的问题。

\subsection{渲染中的混叠来源}\label{sub:渲染中的混叠来源}
几何体是在渲染图像中造成混叠的最常见因素之一。
当投影到图像平面时,物体的边界引入了\keyindex{阶跃函数}{step function}{}——
图像函数的值突然从一个值跳到另一个值。
不仅阶跃函数像前面所述那样有无穷的频率成分,
而且更糟糕的是,完美重建滤波器在运用于混叠样本时也会造成伪影:
重建函数中出现\keyindex{振铃}{ringing}{}伪影,
即称作\keyindex{吉布斯现象}{Gibbs phenomenon}{}的效应。
\reffig{7.11}为1D函数展示了该效应的例子。
正如我们将在本章后续所看到的,选择有效的重建滤波器来面对混叠需要科学、艺术以及个人品味。
\begin{figure}[htbp]
    \centering\includegraphics[width=0.5\linewidth]{chap07/gibbs-phenomenon.eps}
    \caption{吉布斯现象的图示。当没有以奈奎斯特采样率采样函数却又用
        sinc滤波器重建一组混叠的样本时,重建的函数会有“振铃”伪影,它在真实函数附近振荡。
        这里用0.125的样本间隔采样1D阶跃函数(虚线)。当用sinc重建时,振铃出现了(实线)。}
    \label{fig:7.11}
\end{figure}

场景中非常小的物体也会造成几何混叠。如果几何体小到落入图像平面样本之间,
则它会在一个动画的若干帧中不可预测地消失和重现。

混叠的另一个来源可能来自物体上的纹理和材质。没有被正确滤波的纹理贴图
(解决该问题是第\refchap{纹理}的主要内容)或光泽表面的小高光
可能造成\keyindex{着色混叠}{shading aliasing}{alias混叠}。
如果采样率没有高到足以充分采样这些特征,则会导致混叠。
此外,一个物体投射的尖锐阴影会在最终图像中引入另一个阶跃函数。
尽管有可能从图像平面上的几何边来辨别阶跃函数的位置,
但从阴影边界中检测阶跃函数则更加困难。

对于渲染图像中混叠的关键认识是,我们永远不可能移除所有这些来源,
所有我们必须开发技术来减轻其对最终图像质量的影响。

\subsection{理解像素}\label{sub:理解像素}
在本章剩余内容中牢记两个关于像素的观点很重要。
第一,一定记住构成图像的像素是图像函数在图像平面上离散位置的样本点;这样的像素没有相应的“面积”。
正如\citet{Smith95apixel}着重指出的,将像素视作具有有限面积的小方形是错误的认知模型,
会导致一系列问题。通过用信号处理的方法介绍本章话题,我们尝试为更准确的认知模型奠定基础。

第二个问题是最终图像中的像素是在像素网格上的离散整数坐标$(x,y)$处自然定义的,
但本章的\refvar{Sampler}{}是在连续的浮点位置$(x,y)$处生成图像样本的。
映射这两个域的自然方法是将连续坐标舍入到最近的离散坐标;
既然它把刚好和离散坐标有相同值的连续坐标就映射为那个离散坐标,该方法看起来不错。
然而结果是,给定覆盖范围$[x_0,x_1]$的离散坐标集,则连续坐标集覆盖范围为$\displaystyle\left[x_0-\frac{1}{2},x_1+\frac{1}{2}\right)$.
因此任何为给定离散像素范围生成连续样本位置的代码都被$\displaystyle\frac{1}{2}$的偏移量扰乱。
它们很容易被忘记并导致隐晦的错误。

如果我们改用
\begin{align*}
    d=\lfloor c\rfloor\, ,
\end{align*}
将连续坐标$c$截断为离散坐标$d$,并通过
\begin{align*}
    c=d+\frac{1}{2}\, ,
\end{align*}
将离散转换为连续,则离散范围$[x_0,x_1]$对应的连续坐标范围
自然是$[x_0,x_1+1)$且所得代码会简单得多\citep{HECKBERT1990246}。
\reffig{7.12}展示了我们将在pbrt中采用的这一转化。
\begin{figure}[htbp]
    \centering\includegraphics[width=0.5\linewidth]{chap07/Pixelsdiscretecontinuous.eps}
    \caption{图像中的像素可以解释为\emph{离散}或\emph{连续}坐标。
    离散图像五个像素的宽度覆盖了连续像素范围$[0,5)$.
    特定离散像素$d$的坐标在连续表示中为$\displaystyle d+\frac{1}{2}$.}
    \label{fig:7.12}
\end{figure}

\section{采样接口}\label{sec:采样接口}
正如先在\refsub{应用到图像合成}介绍的,
pbrt中实现的渲染方法包含了在图像平面的2D点之外的额外维度上选择样本点。
各种算法将用于生成这些点,但它们的所有实现都继承自定义其接口的抽象类\refvar{Sampler}{}。
核心采样声明和函数在文件\href{https://github.com/mmp/pbrt-v3/blob/master/src/core/sampler.h}{\ttfamily core/sampler.h}
和\href{https://github.com/mmp/pbrt-v3/blob/master/src/core/sampler.cpp}{\ttfamily core/sampler.cpp}中。
样本生成的每种实现都在目录{\ttfamily samplers/}下其自己的源文件内。

\refvar{Sampler}{}的任务是生成$[0,1)^n$中$n$维样本的序列,
其中每个图像样本都要为其生成这样的样本向量,且每个样本中的维数$n$可能会变,
这取决于光传输算法执行的计算(见\reffig{7.13})。
\begin{figure}[htbp]
    \centering\includegraphics[width=0.9\linewidth]{chap07/Samplerndimensional.eps}
    \caption{采样器为每个图像样本生成用来合成最终图像的$n$维样本向量。
        这里,像素$(3,8)$正被采样,且在该像素区域内有两个图像样本。
        样本向量的前两维给出样本在该像素内的偏移量$(x,y)$,
        接下来三维决定相应相机光线的时间和透镜位置。后续维度用于
        第\refchap{光传输I:表面反射}、\refchap{光传输II:体积渲染}和\refchap{光传输III:双向方法}中
        的蒙特卡罗光传输算法。这里,光传输算法已经请求了样本向量中的四个2D数组样本;
        例如,这些值可能用于选择面光源上的四个点来为图像样本计算辐亮度。}
    \label{fig:7.13}
\end{figure}

因为样本值必须严格小于1,所以定义一个常数\refvar{OneMinusEpsilon}{}很有用,
它表示小于1的最大可表示浮点常数。然后,我们会截断样本向量值使之不大于该值。
\begin{lstlisting}
`\initcode{Random Number Declarations}{=}`
#ifdef PBRT_FLOAT_IS_DOUBLE
static const `\refvar{Float}{}` `\initvar{OneMinusEpsilon}{}` = 0x1.fffffffffffffp-1;
#else
static const `\refvar{Float}{}` OneMinusEpsilon = 0x1.fffffep-1;
#endif
\end{lstlisting}

可能最简单的\refvar{Sampler}{}实现是当每次需要样本向量的额外分量时直接返回$[0,1)$内的均匀随机值。
这样的采样器可产生正确的图像但会需要非常多的样本(以及更多要追踪的光线与更多的时间)来
创建用更先进采样器所能取得的相同质量的图像。
使用更佳采样模式的运行时间开销大致和用诸如均匀随机数的低质量模式相同;
因为为每个图像样本计算辐亮度比计算样本的分量值会有大得多的开销,
所以这样做是有回报的(\reffig{7.14})。
\begin{figure}[htbp]
    \centering
    \subfloat[差的采样]{\includegraphics[width=\linewidth]{chap07/spheres-bad-sampler.png}\label{fig:7.14.1}}\\
    \subfloat[更好的采样]{\includegraphics[width=\linewidth]{chap07/spheres-better-sampler.png}\label{fig:7.14.2}}
    \caption{用(a)相对低效的采样器和(b)精心设计的采样器渲染的场景,
        它们用了同样多的样本。从高光边缘到光泽反射,图像质量的提升是明显的。}
    \label{fig:7.14}
\end{figure}

下面假设这些样本向量的一些特性:
\begin{itemize}
    \item \refvar{Sampler}{}生成的前五维通常由\refvar{Camera}{}使用。
          这种情况下,前两个专门用于选择图像上当前像素区域内的点;
          第三个用于计算应该取用该样本的时间;第四和五维为景深给出透镜位置$(u,v)$.
    \item 一些采样算法在样本向量的某些维度中生成的样本比其他维度更好。
          在系统其他地方,我们假设一般前面的维度具有放置得最好的样本值。
\end{itemize}

还要注意\refvar{Sampler}{}生成的$n$维样本通常不会整个显式表示或存储,
而是常常按照光传输算法的需要逐步生成。(然而,存储整个样本向量并对其分量逐渐作出调整
是\refsub{基本样本空间采样器}中\refvar{MLTSampler}{}的基础,
它用于\refsub{MLT积分器}的\refvar{MLTIntegrator}{}。)

\subsection{评估样本模式:偏差}\label{sub:评价样本模式:偏差}
\begin{remark}
    本节含有高级内容,第一次阅读时可以跳过。
\end{remark}

傅里叶分析给了我们一种方法来评估2D采样模式的质量,
但它只是让我们能够在可表示的带限频率方面量化增加更均匀间隔的样本所带来的提升。
考虑到图像中出现了来自边缘的无穷频率成分以及蒙特卡罗光传输算法对$(n>2)$维样本向量的需求,
傅里叶分析对于我们的需求而言是不够的。

给定一个渲染器和放置样本的候选算法,一种评估该算法效果的方法是用其
采样模式来渲染图像并计算它和用大量样本渲染的参考图像相比的误差。
本章后面我们将用该方法比较采样算法,不过它只告诉了我们该算法对于特定场景的表现如何,
且若没有经过渲染过程它将不能让我们感觉出样本点的质量。

除了傅里叶分析,数学家还发明了一个称作\keyindex{偏差}{discrepancy}{}的概念
用于评估$n$维样本位置模式的质量。分布良好(稍后形式化说明)的模式有低的偏差值,
且因此该样本模式生成问题可以考虑成寻找点的合适的\emph{低偏差}模式
\footnote{当然,这样使用偏差隐含假设了用于计算偏差的度量
    对于图像采样而言是与模式的质量有良好关联性的,这可能会有所区别,
    尤其是当人类视觉系统参与该过程时。}。
大量确定性技术已经被开发出来,甚至能在高维空间中生成低偏差点集
(本章后面所用的大多数采样算法都使用这些技术)。

偏差的基本思想是$n$维空间$[0,1)^n$中点集的质量可通过查看域$[0,1)^n$中的各区域、
数出每个区域内的点数并拿每个区域的体积和其内的样本点数作比较来评估。
通常,给定的某一占比体积内应该大致含有样本点总数的相同比例。
尽管不可能总是这种情况,但我们仍可尽量使用让实际体积与点估计的体积间的最大差异(即偏差)最小化的模式。
\reffig{7.15}展示了该思想在二维下的例子。
\begin{figure}[htbp]
    \centering\includegraphics[width=0.5\linewidth]{chap07/Boxdiscrepancy.eps}
    \caption{给定$[0,1)^2$中2D样本点集后矩形(阴影)的偏差。
    四个样本点中的一个在矩形内,所以该点集将把矩形的面积估为$\frac{1}{4}$.
    该矩形是真实面积是$0.3\times0.3=0.09$,所以该特定矩形的偏差
    为$0.25-0.09=0.16$.通常,我们关心的是找出所有可能的矩形(或某些其他形状)中的最大偏差。}
    \label{fig:7.15}
\end{figure}

为了计算点集的偏差,我们首先取作为$[0,1)^n$子集的一簇形状$B$.
例如常用一个角位于原点的方盒。其对应于
\begin{align*}
    B=\{[0,v_1]\times[0,v_2]\times\cdots\times[0,v_n]\}\, ,
\end{align*}
其中$0\le v_i<1$.给定样本点序列$P=x_1,\ldots,x_N$,
$P$关于$B$的偏差为\footnote{算符$\sup$,也称作\emph{上确界},给出了定义域内函数值的最紧上界。}
\begin{align}\label{eq:7.4}
    D_N(B,P)=\sup\limits_{b\in B}\left|\frac{\#\{x_i\in b\}}{N}-V(b)\right|\, ,
\end{align}
其中$\#\{x_i\in b\}$是$b$中的点数,$V(b)$是$b$的体积。

对于为什么\refeq{7.4}是合理的质量度量的直观解释是,
值$\displaystyle\frac{\#\{x_i\in b\}}{N}$是
由特定点集$P$给出的方盒$b$体积的近似。
因此,偏差是所有可能的方盒用这种办法逼近其体积时的最差误差。
当形状集$B$是一个角在原点的方盒集时,
该值称为\keyindex{星偏差}{star discrepancy}{discrepancy偏差}
\sidenote{译者注:也称“均匀性偏差”。},$D^*_N(P)$.
对于$B$的另一个流行选择是全体轴对齐框的集合,即去掉了一个角在原点的限制。

对于一些特定点集,可以解析计算偏差。例如考虑一维中的点集
\begin{align*}
    x_i=\frac{i}{N}\, .
\end{align*}
我们可以看到$x_i$的星偏差为\sidenote{译者注:原文将$x_N$误写为$x_n$,已修正。}
\begin{align*}
    D^*_N(x_1,\ldots,x_N)=\frac{1}{N}\, .
\end{align*}
例如,取区间$b=\displaystyle\left[0,\frac{1}{N}\right)$.则$V(b)=\displaystyle\frac{1}{N}$,
但$\#\{x_i\in b\}=0$.该区间(以及区间$\displaystyle\left[0,\frac{2}{N}\right)$等)
的体积与体积内所见点的比例有最大的差异。

该序列的星偏差可通过稍微对其改动来改进:
\begin{align}\label{eq:7.5}
    x_i=\frac{i-\frac{1}{2}}{N}\, .
\end{align}
则
\begin{align*}
    D^*_N(x_i)=\frac{1}{2N}\, .
\end{align*}
说明一维点序列的星偏差边界为
\sidenote{译者注:这里我简单推导了该式:假设$P=\{x_i\}_{i=1}^{N}$已经按照
升序排列。构造一个新序列$Q=\left\{\frac{j}{N}\right\}_{j=0}^{N}$,然后让$P$和$Q$的元素交错排列构造
新序列$U=\{u_k\}_{k=1}^{2N+1}=\left\{0,x_1,\frac{1}{N},\ldots,x_i,\frac{i}{N},\ldots,x_N,1\right\}$.
则依据偏差的定义可以证明,$D^*_N(x_i)=\max\limits_{1\le k\le 2N}{|u_{k+1}-u_k|}=\max\limits_{1\le i\le N}{d_i}$,
其中$d_i=\max{\left(\left|x_i-\frac{i-1}{N}\right|,\left|x_i-\frac{i}{N}\right|\right)}$.
又注意到$d_i=\frac{1}{2N}+\left|x_i-\frac{2i-1}{2N}\right|$,于是得到文中该式。}
\begin{align*}
    D^*_N(x_i)=\frac{1}{2N}+\max\limits_{1\le i\le N}\left|x_i-\frac{2i-1}{2N}\right|\, .
\end{align*}
因此,之前\refeq{7.5}中的序列具有1D序列中所能取到的最低偏差。
通常,分析和计算1D序列的偏差边界比高维简单得多。
对于构造更复杂的点序列、高维序列以及比方盒更不规则的形状,
通常必须通过构造大量形状$b$、计算其偏差并报告找到的最大值来数值地估计偏差。

聪明的读者会注意到根据低偏差度量,1D中该均匀序列是最优的,
但本章前面我们说过,对于2D中的图像采样,不规则的抖动模式优于均匀模式,
因为它们将混叠替换为噪声。在这一框架下,均匀样本显然不是最好的。
幸运的是,更高维的低偏差模式比在一维中更不均匀得多,
因此实际中通常能作为样本模式工作得很好。
然而,其根本的均匀性意味着低偏差模式比伪随机变化的模式更可能倾向于视觉上令人讨厌的混叠。

只靠偏差并不一定是好的度量:一些低偏差点集表现出样本的聚集性,
其中两个或以上样本可能靠得很近。\refsec{Sobol采样器}的Sobol采样器
\sidenote{译者注:得名于立陶宛裔俄罗斯著名数学家Ilya Meyerovich Sobol。
原文对“Sobol”这一名称均加上了撇号,译文将其略去了,下同。}
尤其困扰于该问题——见\reffig{7.36},它展示了其前两维的图示。
直觉上,靠得太近的样本不能很好地利用采样资源:
一个样本离另一个越近,它就越不可能给出关于被采样函数的新信息。
因此,计算点集中任意两个样本间的最小距离也已被证明是一种有用的样本模式质量度量;最小距离越大越好。

有各种算法用来生成在该度量下得分不错的\keyindex{泊松圆盘}{Poisson disk}{}采样模式。
通过构造,泊松圆盘模式内没有两个点比某一距离$d$更近。
研究已表明眼睛的视杆细胞和视锥细胞也按该方式分布,
这进一步验证了该分布适合用来成像的观点。
实际中,我们发现泊松圆盘模式对于采样2D图像工作得很好,
但对于更复杂的渲染情形中的更高维采样会比更好的低偏差模式更低效;
见“扩展阅读”一节了解更多信息。

\subsection{基本采样器接口}\label{sub:基本采样器接口}
基类\refvar{Sampler}{}不仅定义了采样器的接口,
还提供了一些通用功能供\refvar{Sampler}{}
的实现使用。
\begin{lstlisting}
`\initcode{Sampler Declarations}{=}\initnext{SamplerDeclarations}`
class `\initvar{Sampler}{}` {
public:
    `\refcode{Sampler Interface}{}`
    `\refcode{Sampler Public Data}{}`
protected:
    `\refcode{Sampler Protected Data}{}`
private:
    `\refcode{Sampler Private Data}{}`
};
\end{lstlisting}

所有\refvar{Sampler}{}实现必须提供指定了要为最终图像中每个像素生成的样本数量的构造函数。
在罕见情况下,将胶片建模为只有单个覆盖整个可视区域的“像素”可能对系统有用
(这种过载的像素定义有点夸张,但我们允许它简化某些实现方面)。
既然该“像素”可能有数十亿个样本,我们就用64位精度的变量来存储样本数量。
\begin{lstlisting}
`\initcode{Sampler Method Definitions}{=}\initnext{SamplerMethodDefinitions}`
`\refvar{Sampler}{}`::`\refvar{Sampler}{}`(int64_t samplesPerPixel)
: `\refvar{samplesPerPixel}{}`(samplesPerPixel) { }
\end{lstlisting}
\begin{lstlisting}
`\initcode{Sampler Public Data}{=}`
const int64_t `\initvar{samplesPerPixel}{}`;
\end{lstlisting}

当渲染算法准备好在给定像素上工作时,它通过提供
该像素在图像中的坐标并调用\refvar{StartPixel}{()}来开始。
一些\refvar{Sampler}{}实现利用哪个像素正被采样的知识来提升
其为该像素生成的样本整体分布,而其他的则忽略该信息。
\begin{lstlisting}
`\initcode{Sampler Interface}{=}\initnext{SamplerInterface}`
virtual void `\initvar{StartPixel}{}`(const `\refvar{Point2i}{}` &p);
\end{lstlisting}

方法\refvar{Get1D}{()}为当前样本向量的下一维返回样本值,
\refvar{Get2D}{()}则为下两维返回样本值。
尽管能通过使用调取一对\refvar{Get1D}{()}返回的值来构造2D样本值,
但一些采样器如果知道两维会一起用时能够生成更好的点分布。
\begin{lstlisting}
`\refcode{Sampler Interface}{+=}\lastnext{SamplerInterface}`
virtual `\refvar{Float}{}` `\initvar{Get1D}{}`() = 0;
virtual `\refvar{Point2f}{}` `\initvar{Get2D}{}`() = 0;
\end{lstlisting}

在pbrt中,我们不支持从采样器中获取3D或更高维度的样本值,
因为它们一般对于这里实现的渲染算法类型而言是非必需的。
如果需要,可用来自低维分量的多个值来构造高维样本点。

这些接口的一个显著特点是必须仔细编写使用样本值的代码使其
总是以同样的顺序获取样本维度。考虑下列代码:\\
{\ttfamily
sampler->StartPixel(p);\\
do \{\\
\indent Float v = a(sampler->Get1D());\\
\indent if (v > 0)\\
\indent \indent v += b(sampler->Get1D());\\
\indent v += c(sampler->Get1D());\\
\} while (sampler->StartNextSample());
}

情况下,样本向量的第一维总会传给函数{\ttfamily a()};
当执行调用{\ttfamily b()}的代码路径时,{\ttfamily b()}会收到第二维。
然而,若{\ttfamily if}测试并不总是为真或假,则{\ttfamily c()}有时
会从样本向量的第二维收到样本,否则从第三维收到样本。
因此,采样器为了提供在每个维度评估都分布良好的样本点所作的努力就白费了。
故需要仔细编写使用\refvar{Sampler}{}的代码使得它始终如一地用掉样本向量维度以避免该问题。

为了方便,基类\refvar{Sampler}{}提供了为给定像素初始化\refvar{CameraSample}{}的方法。
\begin{lstlisting}
`\refcode{Sampler Method Definitions}{+=}\lastnext{SamplerMethodDefinitions}`
`\refvar{CameraSample}{}` `\refvar{Sampler}{}`::`\initvar{GetCameraSample}{}`(const `\refvar{Point2i}{}` &pRaster) {
    `\refvar{CameraSample}{}` cs;
    cs.`\refvar{pFilm}{}` = (`\refvar{Point2f}{}`)pRaster + `\refvar{Get2D}{}`();
    cs.`\refvar[CameraSample::time]{time}{}` = `\refvar{Get1D}{}`();
    cs.`\refvar{pLens}{}` = `\refvar{Get2D}{}`();
    return cs;
}
\end{lstlisting}

一些渲染算法为其采样的某些维度利用了样本值数组;
比起生成一系列单独的样本,大多数样本生成算法通过考虑
数组所有元素上以及一个像素内所有样本上的样本值分布可生成更高质量的样本数组。

如果需要样本数组,则必须在渲染开始前请求之。
在渲染开始前——例如在重写了方法\refvar{SamplerIntegrator::Preprocess}{()}的方法中,
应该为每个这样维度的数组调用方法\refvar[Request1DArray]{Request[12]DArray}{()}。
例如,在具有两个面光源的场景中,当积分器追踪了四条阴影射线到第一个光源,
八条到第二个光源时,积分器会为每个图像样本请求两个2D样本数组,
每个分别有四个和八个样本(需要2D数组是因为需要两个维度来参数化光源表面)。
在\refsec{俄罗斯轮盘赌与划分}中,我们将看到使用样本数组会怎样对应于
用“划分”\sidenote{译者注:原文splitting。}的蒙特卡罗技术更密集地采样光传输积分的某些维度。
\begin{lstlisting}
`\refcode{Sampler Interface}{+=}\lastnext{SamplerInterface}`
void `\initvar{Request1DArray}{}`(int n);
void `\initvar{Request2DArray}{}`(int n);
\end{lstlisting}

大多数\refvar{Sampler}{}能更好地生成某些特定大小的数组。
例如,在数量为2的幂时,来自\refvar{ZeroTwoSequenceSampler}{}的样本
则分布得好得多。方法\linebreak
\refvar[RoundCount]{Sampler::RoundCount}{()}
帮助传达该信息。需要样本数组的代码应该用想要取用的样本数目调用该方法,
以给\refvar{Sampler}{}机会把样本数目调整到更好的值。
然后应该用返回的值作为从\refvar{Sampler}{}实际请求样本的数目。
默认实现返回不变的给定数目。
\begin{lstlisting}
`\refcode{Sampler Interface}{+=}\lastnext{SamplerInterface}`
virtual int `\initvar{RoundCount}{}`(int n) const {
    return n;
}
\end{lstlisting}

在渲染时,可调用方法\refvar[Get1DArray]{Get[12]DArray}{()}获取
指向之前请求的当前维度下样本数组起点的指针。
在\refvar{Get1DArray}{()}和\refvar{Get2DArray}{()}的代码行中
\sidenote{译者注:原文疑似笔误写成了\refvar{Get1D}{()}和\refvar{Get2D}{()},已修正。},
它们返回指向样本数组的指针,数组大小由参数{\ttfamily n}传给
初始化时对\refvar[Request1DArray]{Request[12]DArray}{()}
的相应调用。调用者也必须提供数组大小以“获取”方法用于验证返回的缓冲区具有期望大小。
\begin{lstlisting}
`\refcode{Sampler Interface}{+=}\lastnext{SamplerInterface}`  
const `\refvar{Float}{}` *`\initvar{Get1DArray}{}`(int n);
const `\refvar{Point2f}{}` *`\initvar{Get2DArray}{}`(int n);
\end{lstlisting}

为一个样本完成这些工作后,积分器调用\refvar{StartNextSample}{()}。
该调用为当前像素通知\refvar{Sampler}{}针对
样本分量的后续请求应该返回下一个样本从第一维起的值。
该方法返回{\ttfamily true},直到已经为每个像素生成了请求的原始数目样本
(此时调用者应该要么在另一个像素上开始工作要么停止试图使用更多样本)。
\begin{lstlisting}
`\refcode{Sampler Interface}{+=}\lastnext{SamplerInterface}`
virtual bool `\initvar{StartNextSample}{}`();
\end{lstlisting}

\refvar{Sampler}{}的实现存储了关于当前样本的各种状态:
正在采样哪个像素,用了该样本的多少维度等等。
因此对于多个线程同时使用的单个\refvar{Sampler}{}而言自然是不安全的。
方法\refvar{Clone}{()}生成一个初始\refvar{Sampler}{}的新实例给渲染线程使用;
它为采样器的随机数生成器(如果有)接收一个种子值,这样不同线程会有不同的随机数序列。
在多个图块间复用相同伪随机数序列可能导致微妙的图像伪影,例如重复的噪声模式。

方法\refvar{Clone}{()}的各种实现一般并不有趣,所以这里文中没有包含它们。
\begin{lstlisting}
`\refcode{Sampler Interface}{+=}\lastnext{SamplerInterface}`
virtual std::unique_ptr<`\refvar{Sampler}{}`> `\initvar{Clone}{}`(int seed) = 0;
\end{lstlisting}

一些光传输算法(特别是\refsec{随机渐进光子映射}的随机渐进光子映射)
在进行到下一像素前并不使用当前像素内的所有样本,
而是跳跃到周围的像素,每个里面每次取一个样本。
方法\refvar{SetSampleNumber}{()}允许积分器在当前像素内设置样本的索引以生成下一个。
一旦{\ttfamily sampleNum}大于或等于每个像素请求的原始样本数目该方法就返回{\ttfamily false}。
\begin{lstlisting}
`\refcode{Sampler Interface}{+=}\lastcode{SamplerInterface}`
virtual bool `\initvar{SetSampleNumber}{}`(int64_t sampleNum);
\end{lstlisting}

\subsection{采样器实现}\label{sub:采样器实现}
基类\refvar{Sampler}{}在其接口内提供了一些方法的实现。
首先,方法\refvar[Sampler::StartPixel]{StartPixel}{()}的实现记录当前正被采样的像素坐标
并置零\refvar{currentPixelSampleIndex}{}即像素中当前正被生成的样本数量。
注意这是有一个实现的虚方法;重载该方法的子类需要显式调用\refvar{Sampler::StartPixel}{()}。
\begin{lstlisting}
`\refcode{Sampler Method Definitions}{+=}\lastnext{SamplerMethodDefinitions}`
void `\initvar[Sampler::StartPixel]{\refvar{Sampler}{}::\refvar{StartPixel}{}}{}`(const `\refvar{Point2i}{}` &p) {
    `\refvar{currentPixel}{}` = p;
    `\refvar{currentPixelSampleIndex}{}` = 0;
    `\refcode{Reset array offsets for next pixel sample}{}`
}
\end{lstlisting}

\refvar{Sampler}{}子类可获取当前像素坐标和像素内的样本数量,
但它们应当将其作为只读值对待。
\begin{lstlisting}
`\initcode{Sampler Protected Data}{=}\initnext{SamplerProtectedData}`
`\refvar{Point2i}{}` `\initvar{currentPixel}{}`;
int64_t `\initvar{currentPixelSampleIndex}{}`;
\end{lstlisting}

当像素样本被更新或显式设置时,\refvar{currentPixelSampleIndex}{}也随之更新。
像\refvar[Sampler::StartPixel]{StartPixel}{()}那样,
方法\refvar[Sampler::StartNextSample]{StartNextSample}{()}和\refvar[Sampler::SetSampleNumber]{SetSampleNumber}{()}也
都是虚实现;这些实现也必须由\refvar{Sampler}{}子类中重载它们的实现来显式调用。
\begin{lstlisting}
`\refcode{Sampler Method Definitions}{+=}\lastnext{SamplerMethodDefinitions}`
bool `\initvar[Sampler::StartNextSample]{\refvar{Sampler}{}::\refvar{StartNextSample}{}}{}`() {
    `\refcode{Reset array offsets for next pixel sample}{}`
    return ++`\refvar{currentPixelSampleIndex}{}` < `\refvar{samplesPerPixel}{}`;
}
\end{lstlisting}
\begin{lstlisting}
`\refcode{Sampler Method Definitions}{+=}\lastnext{SamplerMethodDefinitions}`
bool `\initvar[Sampler::SetSampleNumber]{\refvar{Sampler}{}::\refvar{SetSampleNumber}{}}{}`(int64_t sampleNum) {
    `\refcode{Reset array offsets for next pixel sample}{}`
    `\refvar{currentPixelSampleIndex}{}` = sampleNum;
    return `\refvar{currentPixelSampleIndex}{}` < `\refvar{samplesPerPixel}{}`;
}
\end{lstlisting}

基类\refvar{Sampler}{}的实现也仔细记录
对样本分量数组的请求并为这些值分配存储空间。
所需的样本数组大小存于\refvar{samples1DArraySizes}{}和\refvar{samples2DArraySizes}{},
整个像素的样本数组值的内存分配于\refvar{sampleArray1D}{}和\refvar{sampleArray2D}{}。
每份分配中前{\ttfamily n}个值用于像素中首个样本的相应数组,以此类推。
\begin{lstlisting}
`\refcode{Sampler Method Definitions}{+=}\lastnext{SamplerMethodDefinitions}`
void `\initvar[Sampler::Request1DArray]{\refvar{Sampler}{}::\refvar{Request1DArray}{}}{}`(int n) {
    `\refvar{samples1DArraySizes}{}`.push_back(n);
    `\refvar{sampleArray1D}{}`.push_back(std::vector<`\refvar{Float}{}`>(n * `\refvar{samplesPerPixel}{}`));
}
\end{lstlisting}
\begin{lstlisting}
`\refcode{Sampler Method Definitions}{+=}\lastnext{SamplerMethodDefinitions}`
void `\initvar[Sampler::Request2DArray]{\refvar{Sampler}{}::\refvar{Request2DArray}{}}{}`(int n) {
    `\refvar{samples2DArraySizes}{}`.push_back(n);
    `\refvar{sampleArray2D}{}`.push_back(std::vector<`\refvar{Point2f}{}`>(n * `\refvar{samplesPerPixel}{}`));
}
\end{lstlisting}
\begin{lstlisting}
`\refcode{Sampler Protected Data}{+=}\lastcode{SamplerProtectedData}`
std::vector<int> `\initvar{samples1DArraySizes}{}`, `\initvar{samples2DArraySizes}{}`;
std::vector<std::vector<`\refvar{Float}{}`>> `\initvar{sampleArray1D}{}`;
std::vector<std::vector<`\refvar{Point2f}{}`>> `\initvar{sampleArray2D}{}`;
\end{lstlisting}

像方法\refvar[Get1DArray]{Get[12]DArray}{()}获取当前样本内的数组那样,\refvar{array1DOffset}{}和
\refvar{array2DOffset}{}被更新成将为样本向量返回的下一数组的索引。
\begin{lstlisting}
`\initcode{Sampler Private Data}{=}`
size_t `\initvar{array1DOffset}{}`, `\initvar{array2DOffset}{}`;
\end{lstlisting}
当处理新像素或当前像素中样本数量改变时,这些数组偏移量必须重置为0.
\begin{lstlisting}
`\initcode{Reset array offsets for next pixel sample}{=}`
`\refvar{array1DOffset}{}` = `\refvar{array2DOffset}{}` = 0;
\end{lstlisting}

要返回合适的数组指针,首先要基于当前样本向量内已经消耗了多少来选择合适的数组,
然后基于当前像素样本索引返回其合适的实例。
\begin{lstlisting}
`\refcode{Sampler Method Definitions}{+=}\lastnext{SamplerMethodDefinitions}`
const `\refvar{Float}{}` *`\initvar[Sampler::Get1DArray]{\refvar{Sampler}{}::\refvar{Get1DArray}{}}{}`(int n) {
    if (`\refvar{array1DOffset}{}` == `\refvar{sampleArray1D}{}`.size())
        return nullptr;
    return &`\refvar{sampleArray1D}{}`[`\refvar{array1DOffset}{}`++][`\refvar{currentPixelSampleIndex}{}` * n];
}
\end{lstlisting}
\begin{lstlisting}
`\refcode{Sampler Method Definitions}{+=}\lastnext{SamplerMethodDefinitions}`
const `\refvar{Point2f}{}` *`\initvar[Sampler::Get2DArray]{\refvar{Sampler}{}::\refvar{Get2DArray}{}}{}`(int n) {
    if (`\refvar{array2DOffset}{}` == `\refvar{sampleArray2D}{}`.size())
        return nullptr;
    return &`\refvar{sampleArray2D}{}`[`\refvar{array2DOffset}{}`++][`\refvar{currentPixelSampleIndex}{}` * n];
}
\end{lstlisting}

\subsection{像素采样器}\label{sub:像素采样器}
尽管一些采样算法很容易递进生成每个样本向量的元素,但其他算法会更自然地为一个像素同时生成
所有样本向量所有维度上的样本值。类\refvar{PixelSampler}{}
实现了一些对该类采样器的实现有用的功能。
\begin{lstlisting}
`\refcode{Sampler Declarations}{+=}\lastnext{SamplerDeclarations}`
class `\initvar{PixelSampler}{}` : public `\refvar{Sampler}{}` {
public:
    `\refcode{PixelSampler Public Methods}{}`
protected:
    `\refcode{PixelSampler Protected Data}{}`
};
\end{lstlisting}
\begin{lstlisting}
`\initcode{PixelSampler Public Methods}{=}`
`\refvar{PixelSampler}{}`(int64_t samplesPerPixel, int nSampledDimensions);
bool `\refvar[PixelSampler::StartNextSample]{StartNextSample}{}`();
bool `\refvar[PixelSampler::SetSampleNumber]{SetSampleNumber}{}`(int64_t);
`\refvar{Float}{}` `\refvar[PixelSampler::Get1D]{Get1D}{}`();
`\refvar{Point2f}{}` `\refvar[PixelSampler::Get2D]{Get2D}{}`();
\end{lstlisting}

渲染算法要用的样本向量维数是不能提前知道的
(确实,它只隐式取决于调用\refvar{Get1D}{()}和\refvar{Get2D}{()}的次数
以及请求的数组)。因此,\refvar{PixelSampler}{}构造函数
接收\refvar{Sampler}{}要计算的非数组样本值的最大维数。
如果所有这些分量维度都用掉了,则\refvar{PixelSampler}{}直接为额外维度返回均匀随机值。

对于每个预先计算的维度,构造函数都分配一个{\ttfamily vector}来存储样本值,
像素内的每个样本对应一个值。这些向量按{\ttfamily\refvar{samples1D}{}[dim][pixelSample]}来索引
\sidenote{译者注:原文将\refvar{samples1D}{}误写为{\ttfamily sample1D},已修正。};
尽管交换这些索引的顺序可能看起来更合理,但现在这样的内存排布——
对于给定维度,所有样本的所有样本分量值在内存中是连续的
\sidenote{译者注:指这些值的内存地址是连续的。}——
对于生成这些值的代码而言变得更方便了。
\begin{lstlisting}
`\refcode{Sampler Method Definitions}{+=}\lastnext{SamplerMethodDefinitions}`
`\refvar{PixelSampler}{}`::`\refvar{PixelSampler}{}`(int64_t samplesPerPixel,
        int nSampledDimensions)
    : `\refvar{Sampler}{}`(samplesPerPixel) {
    for (int i = 0; i < nSampledDimensions; ++i) {
        `\refvar{samples1D}{}`.push_back(std::vector<`\refvar{Float}{}`>(samplesPerPixel));
        `\refvar{samples2D}{}`.push_back(std::vector<`\refvar{Point2f}{}`>(samplesPerPixel));
    }
}
\end{lstlisting}

继承自\refvar{PixelSampler}{}的\refvar{Sampler}{}实现的
关键责任接着是在其方法\refvar{StartPixel}{()}
中填充数组\refvar{samples1D}{}和\refvar{samples2D}{}
(以及\refvar{sampleArray1D}{}和\refvar{sampleArray2D}{})。

\refvar{current1DDimension}{}和\refvar{current2DDimension}{}保存了
当前像素样本针对对应数组的偏移量。在开始处理每个新样本前必须将它们重置为0.
\begin{lstlisting}
`\initcode{PixelSampler Protected Data}{=}\initnext{PixelSamplerProtectedData}`
std::vector<std::vector<`\refvar{Float}{}`>> `\initvar{samples1D}{}`;
std::vector<std::vector<`\refvar{Point2f}{}`>> `\initvar{samples2D}{}`;
int `\initvar{current1DDimension}{}` = 0, `\initvar{current2DDimension}{}` = 0;
\end{lstlisting}
\begin{lstlisting}
`\refcode{Sampler Method Definitions}{+=}\lastnext{SamplerMethodDefinitions}`
bool `\initvar[PixelSampler::StartNextSample]{\refvar{PixelSampler}{}::\refvar{StartNextSample}{}}`() {
    `\refvar{current1DDimension}{}` = `\refvar{current2DDimension}{}` = 0;
    return `\refvar{Sampler}{}::\refvar[Sampler::StartNextSample]{StartNextSample}{}`();
}
\end{lstlisting}
\begin{lstlisting}
`\refcode{Sampler Method Definitions}{+=}\lastnext{SamplerMethodDefinitions}`
bool `\initvar[PixelSampler::SetSampleNumber]{\refvar{PixelSampler}{}::\refvar{SetSampleNumber}{}}{}`(int64_t sampleNum) {
    `\refvar{current1DDimension}{}` = `\refvar{current2DDimension}{}` = 0;
    return `\refvar{Sampler}{}::\refvar[Sampler::SetSampleNumber]{SetSampleNumber}{}`(sampleNum);
}
\end{lstlisting}

有了子类\refvar{PixelSampler}{}计算的数组中的样本值,
实现\refvar{Get1D}{()}只需依维度返回值直到算出的
所有维度都已被用掉,此时返回均匀随机值。
\begin{lstlisting}
`\refcode{Sampler Method Definitions}{+=}\lastnext{SamplerMethodDefinitions}`
`\refvar{Float}{}` `\initvar[PixelSampler::Get1D]{\refvar{PixelSampler}{}::\refvar{Get1D}{}}{}`() {
    if (`\refvar{current1DDimension}{}` < `\refvar{samples1D}{}`.size())
        return `\refvar{samples1D}{}`[`\refvar{current1DDimension}{}`++][`\refvar{currentPixelSampleIndex}{}`];
    else
        return `\refvar[PixelSampler::rng]{rng}{}`.`\refvar{UniformFloat}{}`();
}
\end{lstlisting}

{\initvar{PixelSampler::Get2D}{()}}同理,所以这里不再介绍。

\refvar{PixelSampler}{}用的随机数生成器是{\ttfamily protected}的
而不是{\ttfamily private}的。这对于其一些也需要随机数
来初始化\refvar{samples1D}{}和\refvar{samples2D}{}的子类会很方便。
\begin{lstlisting}
`\refcode{PixelSampler Protected Data}{+=}\lastcode{PixelSamplerProtectedData}`
`\refvar{RNG}{}` `\initvar[PixelSampler::rng]{rng}{}`;
\end{lstlisting}

\subsection{全局采样器}\label{sub:全局采样器}
其他生成样本的算法很少基于像素而是自然地生成分布于整幅图像的连续样本,
连续访问完全不同的像素(许多这样的采样器会高效地放置每个追加的样本
使其填充$n$维样本空间中的最大空洞,这自然导致后续样本在不同像素内)。
这些采样算法对于目前描述的\refvar{Sampler}{}接口有点问题:
例如考虑一个为前两维生成如\reftab{7.2}中间一列所示的一系列样本值的采样器。
这些样本值乘以图像每维分辨率得到图像平面中的样本位置
(这里我们为了简化考虑一幅$2\times3$的图像)。
注意对于这里的采样器(其实是\refvar{HaltonSampler}{}),
每六个样本就访问每个像素。若我们正渲染的图像每个像素用三个样本,
则为了给像素$(0,0)$生成所有的样本,我们需要生成索引为0、6和12的样本,以此类推。
\begin{table}[htb]
    \centering
    \begin{tabular}{lll}
        \toprule
        样本索引 & $[0,1)^2$的样本坐标   & 像素样本坐标          \\
        \midrule
        0        & $(0.000000,0.000000)$ & $(0.000000,0.000000)$ \\
        1        & $(0.500000,0.333333)$ & $(1.000000,1.000000)$ \\
        2        & $(0.250000,0.666667)$ & $(0.500000,2.000000)$ \\
        3        & $(0.750000,0.111111)$ & $(1.500000,0.333333)$ \\
        4        & $(0.125000,0.444444)$ & $(0.250000,1.333333)$ \\
        5        & $(0.625000,0.777778)$ & $(1.250000,2.333333)$ \\
        6        & $(0.375000,0.222222)$ & $(0.750000,0.666667)$ \\
        7        & $(0.875000,0.555556)$ & $(1.750000,1.666667)$ \\
        8        & $(0.062500,0.888889)$ & $(0.125000,2.666667)$ \\
        9        & $(0.562500,0.037037)$ & $(1.125000,0.111111)$ \\
        10       & $(0.312500,0.370370)$ & $(0.625000,1.111111)$ \\
        11       & $(0.812500,0.703704)$ & $(1.625000,2.111111)$ \\
        12       & $(0.187500,0.148148)$ & $(0.375000,0.444444)$ \\
        $\vdots$ &                       &                       \\
        \bottomrule
    \end{tabular}
    \caption{\refvar{HaltonSampler}{}生成中间一列坐标的前两维。
        因为它是个\refvar{GlobalSampler}{},所以它必须定义从像素坐标到样本索引的逆映射;
        这里,它通过将第一维坐标放大2倍、第二维坐标放大3倍
        以在$2\times3$像素的图像上放置样本,得到右边一列的像素样本坐标。}
    \label{tab:7.2}
\end{table}

若有了这样的采样器,我们就能定义\refvar{Sampler}{}接口使得
它为每个样本指定正在渲染的像素而不是相反(即告诉\refvar{Sampler}{}要渲染哪个像素)。

然而,采用目前的设计也有很好的理由:该方法更易把胶片分解为
小的图块以供多线程渲染,每个线程计算一个可高效并入最终图像的局部区域内的像素。
因此,我们必须要求这样的采样器能无序生成样本,使得每个像素的全部样本是连续生成的。

\refvar{GlobalSampler}{}帮助沟通\refvar{Sampler}{}接口的要求
与这类采样器的合理操作。它提供了\refvar{Sampler}{}所有纯虚方法的实现,
即代之以其子类必须实现的两个新的纯虚方法。
\begin{lstlisting}
`\refcode{Sampler Declarations}{+=}\lastcode{SamplerDeclarations}`
class `\initvar{GlobalSampler}{}` : public `\refvar{Sampler}{}` {
public:
    `\refcode{GlobalSampler Public Methods}{}`
private:
    `\refcode{GlobalSampler Private Data}{}`
};
\end{lstlisting}
\begin{lstlisting}
`\initcode{GlobalSampler Public Methods}{=}\initnext{GlobalSamplerPublicMethods}`
`\refvar{GlobalSampler}{}`(int64_t samplesPerPixel) : `\refvar{Sampler}{}`(samplesPerPixel) { }
\end{lstlisting}

有两个方法是实现必须提供的。第一个是\refvar{GetIndexForSample}{()},
它执行从当前像素和给定样本索引到样本向量全集中全局索引的逆映射。
例如,对于生成\reftab{7.2}中值的\refvar{Sampler}{},
如果\refvar{currentPixel}{}是$(0,2)$,则\refvar{GetIndexForSample}{(0)}会返回2,
因为样本索引2相应的像素样本坐标$(0.5,2)$对应着该像素区域中的首个样本
\sidenote{译者注:原文写的坐标值是$(0.25,0.666667)$,疑是笔误,已修改。}。
\begin{lstlisting}
`\refcode{GlobalSampler Public Methods}{+=}\lastnext{GlobalSamplerPublicMethods}`
virtual int64_t `\initvar{GetIndexForSample}{}`(int64_t sampleNum) const = 0;
\end{lstlisting}

紧密相关的\refvar{SampleDimension}{()}为
序列中第{\ttfamily index}个样本向量的给定维度返回样本值。
因为前两维用于偏移到当前像素,所以它们要做特殊处理:
该方法的实现返回的值应该是当前像素内的样本偏移量,
而不是原始的$[0,1)^2$样本值。例如\reftab{7.2}中,
\refvar{SampleDimension}{(4,1)}中会返回0.333333,
因为索引为4的样本的第二维相对于像素$(0,1)$偏移了这么多。
\begin{lstlisting}
`\refcode{GlobalSampler Public Methods}{+=}\lastcode{GlobalSamplerPublicMethods}`
virtual `\refvar{Float}{}` `\initvar{SampleDimension}{}`(int64_t index, int dimension) const = 0;
\end{lstlisting}

当开始为一个像素生成样本时,必须重置样本的维度并找到像素内首个样本的索引。
像所有采样器那样,接下来生成样本数组的所有值。
\begin{lstlisting}
`\refcode{Sampler Method Definitions}{+=}\lastnext{SamplerMethodDefinitions}`
void `\initvar[GlobalSampler::StartPixel]{\refvar{GlobalSampler}{}::\refvar{StartPixel}{}}{}`(const `\refvar{Point2i}{}` &p) {
    `\refvar{Sampler}{}`::`\refvar[Sampler::StartPixel]{StartPixel}{}`(p);
    `\refvar[GlobalSampler::dimension]{dimension}{}` = 0;
    `\refvar{intervalSampleIndex}{}` = `\refvar{GetIndexForSample}{}`(0);
    `\refcode{Compute arrayEndDim for dimensions used for array samples}{}`
    `\refcode{Compute 1D array samples for GlobalSampler}{}`
    `\refcode{Compute 2D array samples for GlobalSampler}{}`
}
\end{lstlisting}

成员变量\refvar[GlobalSampler::dimension]{dimension}{}跟踪
采样器实现将被要求生成的样本值的下一维;
当调用\refvar[GlobalSampler::Get1D]{Get1D}{()}和
\refvar[GlobalSampler::Get2D]{Get2D}{()}时它是递增的。
\refvar{intervalSampleIndex}{}记录当前像素内当前样本$s_i$对应的样本索引。
\begin{lstlisting}
`\initcode{GlobalSampler Private Data}{=}\initnext{GlobalSamplerPrivateData}`
int `\initvar[GlobalSampler::dimension]{dimension}{}`;
int64_t `\initvar{intervalSampleIndex}{}`;
\end{lstlisting}

必须决定为数组样本使用样本向量的哪些维度。
在靠前的维度比后面的维度质量更好的假设下,
为\refvar{CameraSample}{}留出前几个维度很重要,
因为这些样本值的质量经常对最终图像质量有很大影响。

因此,\refvar{arrayStartDim}{}前的维度用于常规的1D和2D样本,
而后续维度用于先1D再2D的数组样本。最后,起始于\refvar{arrayEndDim}{}的更高维
进一步用于非数组的1D和2D样本。当\refvar{GlobalSampler}{}构造函数运行时
不可能计算\refvar{arrayEndDim}{},因为目前还没有积分器请求数组样本。
因此,该值在方法\refvar[GlobalSampler::StartPixel]{StartPixel}{()}中
(重复且冗余地)计算。
\begin{lstlisting}
`\refcode{GlobalSampler Private Data}{+=}\lastcode{GlobalSamplerPrivateData}`
static const int `\initvar{arrayStartDim}{}` = 5;
int `\initvar{arrayEndDim}{}`;
\end{lstlisting}

所有像素样本的数组样本总数由像素样本数量与请求的样本数组尺寸的乘积给出。
\begin{lstlisting}
`\initcode{Compute arrayEndDim for dimensions used for array samples}{=}`
`\refvar{arrayEndDim}{}` = `\refvar{arrayStartDim}{}` +
              `\refvar{sampleArray1D}{}`.size() + 2 * `\refvar{sampleArray2D}{}`.size();
\end{lstlisting}

实际生成数组样本只需计算当前样本维度内所需值的数量。
\begin{lstlisting}
`\initcode{Compute 1D array samples for GlobalSampler}{=}`
for (size_t i = 0; i < `\refvar{samples1DArraySizes}{}`.size(); ++i) {
    int nSamples = `\refvar{samples1DArraySizes}{}`[i] * `\refvar{samplesPerPixel}{}`;
    for (int j = 0; j < nSamples; ++j) {
        int64_t index = `\refvar{GetIndexForSample}{}`(j);
        `\refvar{sampleArray1D}{}`[i][j] =
            `\refvar{SampleDimension}{}`(index, `\refvar{arrayStartDim}{}` + i);
    }
}
\end{lstlisting}

2D样本数组的生成类似;这里不再介绍
代码片\refcode{Compute 2D array samples for GlobalSampler}{}
\sidenote{译者注:我补充回来了。}。
\begin{lstlisting}
`\initcode{Compute 2D array samples for GlobalSampler}{=}`
int dim = `\refvar{arrayStartDim}{}` + `\refvar{samples1DArraySizes}{}`.size();
for (size_t i = 0; i < `\refvar{samples2DArraySizes}{}`.size(); ++i) {
    int nSamples = `\refvar{samples2DArraySizes}{}`[i] * `\refvar{samplesPerPixel}{}`;
    for (int j = 0; j < nSamples; ++j) {
        int64_t idx = `\refvar{GetIndexForSample}{}`(j);
        `\refvar{sampleArray2D}{}`[i][j].x = `\refvar{SampleDimension}{}`(idx, dim);
        `\refvar{sampleArray2D}{}`[i][j].y = `\refvar{SampleDimension}{}`(idx, dim+1);
    }
    dim += 2;
}
`\refvar{Assert}{}`(dim == `\refvar{arrayEndDim}{}`);
\end{lstlisting}

当像素样本变化时,必须重置当前样本维度计数器并计算像素内下一样本的样本索引。
\begin{lstlisting}
`\refcode{Sampler Method Definitions}{+=}\lastnext{SamplerMethodDefinitions}`
bool `\initvar[GlobalSampler::StartNextSample]{\refvar{GlobalSampler}{}::\refvar{StartNextSample}{}}{}`() {
    `\refvar[GlobalSampler::dimension]{dimension}{}` = 0;
    `\refvar{intervalSampleIndex}{}` = `\refvar{GetIndexForSample}{}`(`\refvar{currentPixelSampleIndex}{}` + 1);
    return `\refvar{Sampler}{}`::`\refvar[Sampler::StartNextSample]{StartNextSample}{}`();
}
\end{lstlisting}
\begin{lstlisting}
`\refcode{Sampler Method Definitions}{+=}\lastnext{SamplerMethodDefinitions}`
bool `\initvar[GlobalSampler::SetSampleNumber]{\refvar{GlobalSampler}{}::\refvar{SetSampleNumber}{}}{}`(int64_t sampleNum) {
    `\refvar[GlobalSampler::dimension]{dimension}{}` = 0;
    `\refvar{intervalSampleIndex}{}` = `\refvar{GetIndexForSample}{}`(sampleNum);
    return `\refvar{Sampler}{}`::`\refvar[Sampler::SetSampleNumber]{SetSampleNumber}{}`(sampleNum);
}
\end{lstlisting}

有了该机制,获取常规1D样本值只需跳过分配给数组样本的维度
并把当前样本索引和维度传给实现的方法\refvar{SampleDimension}{()}。
\begin{lstlisting}
`\refcode{Sampler Method Definitions}{+=}\lastnext{SamplerMethodDefinitions}`
`\refvar{Float}{}` `\initvar[GlobalSampler::Get1D]{\refvar{GlobalSampler}{}::\refvar{Get1D}{}}{}`() {
    if (`\refvar[GlobalSampler::dimension]{dimension}{}` >= `\refvar{arrayStartDim}{}` && `\refvar[GlobalSampler::dimension]{dimension}{}` < `\refvar{arrayEndDim}{}`)
        `\refvar[GlobalSampler::dimension]{dimension}{}` = `\refvar{arrayEndDim}{}`;
    return `\refvar{SampleDimension}{}`(`\refvar{intervalSampleIndex}{}`, `\refvar[GlobalSampler::dimension]{dimension}{}`++);
}
\end{lstlisting}

2D样本样本同理。
\begin{lstlisting}
`\refcode{Sampler Method Definitions}{+=}\lastcode{SamplerMethodDefinitions}`
`\refvar{Point2f}{}` `\initvar[GlobalSampler::Get2D]{\refvar{GlobalSampler}{}::\refvar{Get2D}{}}{}`() {
    if (`\refvar[GlobalSampler::dimension]{dimension}{}` + 1 >= `\refvar{arrayStartDim}{}` && `\refvar[GlobalSampler::dimension]{dimension}{}` < `\refvar{arrayEndDim}{}`)
        `\refvar[GlobalSampler::dimension]{dimension}{}` = `\refvar{arrayEndDim}{}`;
    `\refvar{Point2f}{}` p(`\refvar{SampleDimension}{}`(`\refvar{intervalSampleIndex}{}`, `\refvar[GlobalSampler::dimension]{dimension}{}`),
              `\refvar{SampleDimension}{}`(`\refvar{intervalSampleIndex}{}`, `\refvar[GlobalSampler::dimension]{dimension}{}` + 1));
    `\refvar[GlobalSampler::dimension]{dimension}{}` += 2;
    return p;
}
\end{lstlisting}

\input{content/chap0703.tex}

\input{content/chap0704.tex}

\input{content/chap0705.tex}

\input{content/chap0706.tex}

\input{content/chap0707.tex}

\input{content/chap0708.tex}

\input{content/chap0709.tex}

\input{content/chap0710.tex}

\section{习题}\label{sec:习题07}
\begin{enumerate}
      \item \circletwo 可以根据\refvar{RadicalInverse}{()}中的实现代码,
            为基2实现一个专门版本的\refvar{ScrambledRadicalInverse}{()}。
            确定怎样将随机数字排列映射为单个数位运算并实现该方法。
            比较算出的值和当前实现生成的值以确保你的方法是对的
            并通过编写一个小巧的基准程序来度量你的方法有多快。
      \item \circletwo 当前,每个样本的第三到五维是被时间和镜头样本用掉的,
            即便并非所有场景都需要这些样本值。因为样本向量中的更低维常比
            后面的分布得更好,所以这会造成不必要的图像质量下降。
            修改pbrt使得相机能表明其样本需求然后在需要样本来
            初始化\refvar{CameraSample}{}时利用该信息。
            别忘了更新\refvar[arrayStartDim]{GlobalSampler::arrayStartDim}{}的值。
            用\refvar{DirectLightingIntegrator}{}
            渲染图像并和当前实现的结果比较。你看到有改进吗?
            用不同采样器时结果有何区别?你怎样解释你所见到的各采样器间的任何区别?
      \item \label{sub:7.11.3}\circletwo 把\citet{Kensler2013Pixar}介绍
            改进的多重扰动采样方法实现为pbrt中的新
            \refvar{Sampler}{}。比较它和用\refvar{StratifiedSampler}{}、
            \refvar{HaltonSampler}{}以及\refvar{SobolSampler}{}渲染时
            的图像质量和渲染时间。
      \item \circletwo\citet{10.1007/3-540-31186-6_14}\sidenote{译者注:
                  在Springer获取的引用条目标注为2006年,
                  属于2004年的会议;原文则均标为2004年。}
            和\citet{10.1007/978-3-540-74496-2_12}\sidenote{译者注:
                  在Springer获取的引用条目标注为2008年,属于2006年的会议;
                  原文则均标为2006年。}描述了图像合成中\keyindex{一阶点阵}{rank-1 lattices}{}的应用。
            一阶点阵是另一种高效生成高质量低偏差样本点序列的方式。
            阅读他们的论文并基于该方法实现一个\refvar{Sampler}{}。
            比较它和pbrt中其他采样器的结果。
      \item \circletwo 用pbrt当前的\refvar{FilmTile}{}实现时,
            若重新渲染一幅图像,由于线程在后续运行中以不同顺序完成图块,
            图像中的像素值可能有轻微变化。例如一个像素最终值取自三个不同
            图像采样块中的样本,$v_1+v_2+v_3$,其值可能有时算作$(v_1+v_2)+v_3$
            而有时为$v_1+(v_2+v_3)$.由于浮点舍入,这两个值可能不同。
            尽管这些区别通常不成问题,但当想用自动化测试脚本验证对系统
            作出的无伤大雅的更改不会在渲染图像中实际引发任何区别时,
            它们就会造成灾难。修改\refvar[MergeFilmTile]{Film::MergeFilmTile}{()}使其
            以一致的顺序合并图块,从而让最终像素值不再被该不一致性干扰
            (例如你的实现可能缓存\refvar{FilmTile}{}并只在一个图块的
            上方和左侧相邻图块都已被合并时才合并它)。确保你的实现不引入
            任何意义上的性能倒退。度量因\refvar{FilmTile}{}生命期更长
            而新增的内存使用量;它和总内存使用量有什么关系?
      \item \circletwo 如\refsec{胶片与成像管道}中所述,
            方法\refvar[AddSplat]{Film::AddSplat}{()}没用滤波函数
            而是代之高效地用矩形滤波器把样本溅射到其最靠近的单个像素上。
            为了应用任意滤波器,必须规范化滤波器使得它在定义域上的积分为一;
            pbrt目前并不要求\refvar{Filter}{}满足该约束。
            修改\refvar{Film}{}构造函数中\refvar[Film::filterTable]{filterTable}{}的计算,
            使得制表函数规范化(别忘了在计算规范化因子时,
            表格只保存函数四分之一的范围)。然后修改方法\refvar{AddSplat}{()}的
            实现以使用该滤波器。研究其导致的执行时间和图像质量的区别。
      \item \circleone 修改pbrt以创建为每条相机光线存于\refvar{Film}{}的值
            都正比于计算该光线辐亮度所花时长的图像(一个1像素宽的矩形滤波器
            可能是对该习题最有用的滤波器)。用该技术渲染各种场景。
            得到的图像对于系统性能带来了怎样的启发?当你查看它们时
            你可能需要缩放像素值或取其对数来看到有意义的变化。
      \item \circletwo 辐射度量学中线性假设的一个优点是场景的最终图像和
            分别考虑每个光源分布的图像之和是一样的(假设使用不会
            截断像素辐亮度值的浮点图像块格式)。该性质意味着如果渲染器为
            每个光源创建单独的图像,可以写个交互式灯光设计工具让
            快速查看缩放场景中单个光源作用的影响而无需重新渲染成为可能。
            可代之以缩放一个光源的单独图像然后再对所有光源图像求和
            重新生成最终图像(该技术首先应用于\citet{10.1145/122718.122723}的
            歌剧灯光设计)。修改pbrt来为场景中的每个光源输出单独的图像,
            并写一个按该方式利用它们的交互式灯光设计工具。
      \item \circlethree \citet{10.1145/54852.378514}注意到
            有一簇重建滤波器同时用了函数值和它在该点的导数来进行
            比只知道函数值好得多的重建。此外,他们报告他们已为朗伯和
            冯氏反射模型\sidenote{译者注:原文Lambertian and Phong reflection models。}的
            屏幕空间导数推导出解析解,然而他们没有在其论文中包含这些表达式。
            研究基于导数的重建,扩展pbrt以支持该技术。
            因为给一般形状和BSDF模型的屏幕空间导数推导表达式可能很难,
            研究基于有限差分的近似即可。\refsec{采样与抗锯齿}射线差分背后
            基于该思想的技术可能对该尝试有成效。
      \item \circlethree \keyindex{基于图像的渲染}{image-based rendering}{render渲染}是
            使用一个场景一幅或多幅图像合成不同于原始视角的新视角图像的一组技术的总称。
            其中一种方法是\keyindex{光场渲染}{light field rendering}{render渲染},
            即用一组来自密集间隔位置的图像\citep{10.1145/237170.237199,10.1145/237170.237200}。
            阅读这两篇关于光场的论文,并修改pbrt以直接生成场景的光场,
            而不需要渲染器运行多次,每次只针对一个相机位置。
            为此可能有必要编写专门的\refvar{Camera}{}、\refvar{Sampler}{}和\refvar{Film}{}。
            此外,编写一个交互式光场查看器来加载你的实现生成的光场并生成场景的新视角。
      \item \circlethree 比起只保存图像中的光谱值,常常更有用的是
            保存场景中在每个像素处可见的物体的额外信息。
            例如见\citet{10.1145/325334.325247}和\citet{10.1145/97879.97901}的SIGGRAPH论文。
            例如,如果保存每个像素处物体的3D位置、曲面法线以及BRDF,
            则移动光源后可高效地重新渲染场景\citep{10.1145/344779.344938}。
            或者,如果每个样本保存沿其相机光线可见的所有物体信息而不是只存第一个,
            则可以重新渲染移动视点后的新图像\citep{10.1145/280814.280882}。
            研究深度帧缓冲区\sidenote{译者注:原文deep frame buffer,不确定该词翻译。}的表示
            和利用它的算法;扩展pbrt以支持创建像这样的图像,并开发对它们进行操作的工具。
      \item \circletwo 为图像重建实现中值滤波器:对于每个像素,保存
            滤波器范围内其周围所有样本的中值。该任务很复杂,因为事实上
            当前\refvar{Film}{}实现中的滤波器必须是\keyindex{线性的}{linear}{}——
            滤波函数值只取决于样本相对于像素位置的位置,样本值对滤波函数值没有影响。
            因为实现假设滤波器是线性的,且因为它在把样本值的贡献加到图像中后就不再保存了,
            所以实现中值滤波器要求一般化\refvar{Film}{}或开发新的\refvar{Film}{}实现。
            用像\refvar{PathIntegrator}{}那样搭配常规图像滤波器会有讨厌的图像噪声的积分器渲染图像。
            中值滤波器在减少噪声上有多成功?用中值滤波器有视觉缺陷吗?
            你能实现该方法而无需在计算最终像素值前保存所有图像样本值吗?
      \item \circletwo 中值滤波器的一种替代是丢弃像素滤波器区域中
            具有最小贡献的样本和具有最大贡献的样本。该方法更多使用采样期间收集的信息。
            实现该方法并比较它和中值滤波器的结果。
      \item \circlethree 实现\citeauthor{keller1998quasi}及其合作者
            介绍的非连续缓冲区\citep{keller1998quasi,10.2312:EGWR:EGWR02:015-024}。
            你可能需要修改\refvar{Integrator}{}的接口使得它们可以独自返回
            直接和间接照明贡献然后独立将其传给\refvar{Film}{}。
            渲染图像以展示其在用间接照明渲染图像时的高效性。
      \item \circlethree 实现近年一种自适应采样和重建技术,
            例如\citet{10.1145/1360612.1360632}、\citet{10.1145/1531326.1531399}、
            \citet{10.1145/1618452.1618486}或者\citet{10.1145/2641762}介绍的。
            比起只用高采样率的均匀采样它们生成同等质量图像要高效多少?
            对于无需自适应采样的简单场景它们如何影响运行时间?
      \item \circlethree 调研色调重建算法的当前研究
            (例如见\citet{reinhard2010high}、\citet{10.1145/2366145.2366220}),
            并实现其中一个或多个算法。对pbrt渲染的大量场景使用你的实现,
            并讨论比起查看无色调重建的图像你所见到的改进。
\end{enumerate}

\section{译者补充:傅里叶变换}\label{sec:译者补充:傅里叶变换}
\begin{remark}
    本节内容不是原书内容,而是译者根据\citet{enwiki:1115652231,enwiki:1115414995,
        enwiki:1098200554,enwiki:1114206769}、\citet{DigitalSignalProcessing}补充的,
    请酌情参考和斧正。
\end{remark}
\begin{notation}
    本节所指的时域和原书前文中的空域是等价的概念,不影响本质。
\end{notation}
\subsection{单位冲激函数}\label{sub:单位冲激函数}
\begin{definition}
    数学中,\keyindex{狄拉克$\delta$分布}{Dirac delta distribution}{}是定义在实数域上的广义分布。
    它在除零以外的点上都取零,且在整个实数域上的积分等于一。通常记作$\delta(\cdot)$.
\end{definition}

狄拉克$\delta$分布也称\keyindex{狄拉克$\delta$函数}{Dirac delta function}{},
简称\keyindex{$\delta$分布}{delta distribution}{}或
\keyindex{$\delta$函数}{delta function}{},
它最早由英国理论物理学家保罗·狄拉克(Paul Adrien Maurice Dirac)提出,
在物理和工程界有广泛应用,也称作\keyindex{单位冲激函数}{unit impulse function}{}。

单位冲激函数不是严格意义上的函数,但形式上遵守微积分运算法则。
可以将其视作在非零处取零,在零处取无穷大,即
\begin{align}
    \delta(t)\approx\left\{
    \begin{array}{ll}
        +\infty, & \text{当}t=0,     \\
        0,       & \text{当}t\neq 0,
    \end{array}
    \right.
\end{align}
且满足如下积分约束的函数:
\begin{align}
    \int_{-\infty}^{\infty}\delta(t)\mathrm{d}t=1\, .
\end{align}

依据单位冲激函数的定义,可推导出以下性质:
\begin{theorem}\label{theorem:7.ex01.symmetry}
    单位冲激函数具有缩放性质:对任意实数$\alpha\neq0$,有
    \begin{align}
        \delta(\alpha t)=\frac{\delta(t)}{|\alpha|}\, .
    \end{align}
    特别地,单位冲激函数具有对称性,即
    \begin{align}
        \delta(t)=\delta(-t)\, .
    \end{align}
\end{theorem}
\begin{definition}
    称实数域上满足$\displaystyle\int_{-\infty}^{\infty}|f(x)|\mathrm{d}x<\infty$的
    函数$f$为\keyindex{可积函数}{integrable function}{}。
\end{definition}
\begin{theorem}\label{theorem:7.ex01.1}
    单位冲激函数具有时延性质,也称平移性质或筛选性质,
    即对于可积函数$f$,它可以采样出$t=\tau$处的值:
    \begin{align}
        \int_{-\infty}^{\infty}\delta(t-\tau)f(t)\mathrm{d}t=f(\tau)\, .
    \end{align}
\end{theorem}
\subsection{傅里叶变换的定义}\label{sub:傅里叶变换的定义}
\begin{definition}
    对于可积函数$f(t)$,其(一元)\keyindex{傅里叶变换}{Fourier transform}{}为
    \begin{align}
        F(\omega)=\mathcal{F}\{f(t)\}=\int_{-\infty}^{\infty}f(t)\mathrm{e}^{-\mathrm{i}2\pi\omega t}\mathrm{d}t\, ,
    \end{align}
    其中$\mathrm{i}$为虚数单位,$\mathrm{e}$为自然对数的底;
    称$F(\omega)$为$f(t)$的频域表示,也有文献记作$\mathcal{F}\{f\}(\omega)$,
    其中$\omega$表示\keyindex{频率}{frequency}{};
    也称$f(t)$和$F(\omega)$构成一个傅里叶变换对,记作$f(t)\leftrightarrow F(\omega)$;
    同时,相应的(一元)\keyindex{傅里叶逆变换}{inverse Fourier transform}{}为
    \begin{align}\label{eq:7.ex01.inverseFourier}
        f(t)=\mathcal{F}^{-1}\{F(\omega)\}=\int_{-\infty}^{\infty}F(\omega)\mathrm{e}^{\mathrm{i}2\pi\omega t}\mathrm{d}\omega\, .
    \end{align}
\end{definition}

\begin{theorem}\label{theorem:7.ex01.2}
    对于单位冲激函数$\delta(t)$,其频率表示为$F(\omega)=1$.
\end{theorem}
\begin{prove}
    由傅里叶变换定义,
    \begin{align}
        F(\omega)=\int_{-\infty}^{\infty}\delta(t)\mathrm{e}^{-\mathrm{i}2\pi\omega t}\mathrm{d}t
        =\mathrm{e}^{-\mathrm{i}2\pi\omega\cdot0}
        =1\, .
    \end{align}
\end{prove}

\begin{theorem}\label{theorem:7.ex01.3}
    对于单位常函数$f(t)=1$,其频率表示为$\delta(\omega)$.
\end{theorem}
\begin{prove}
    定义\keyindex{双边指数衰减函数}{two-sided decaying exponential function}{}为
    \sidenote{属于\keyindex{拉普拉斯分布}{Laplace distribution}{distribution分布}。}
    \begin{align}
        f_a(t)=\mathrm{e}^{-a|t|},\quad (a>0)\, ,
    \end{align}
    则单位常函数可视作该函数的极限,即
    \begin{align}
        f(t)=\lim\limits_{a\rightarrow0^+}f_a(t)=1\, .
    \end{align}
    于是常函数的频率表示满足
    \begin{align}
        F(\omega) & =\int_{-\infty}^{\infty}f(t)\mathrm{e}^{-\mathrm{i}2\pi\omega t}\mathrm{d}t
        =\int_{-\infty}^{\infty}\lim\limits_{a\rightarrow0^+}\mathrm{e}^{-a|t|}\mathrm{e}^{-\mathrm{i}2\pi\omega t}\mathrm{d}t
        =\lim\limits_{a\rightarrow0^+}\int_{-\infty}^{\infty}\mathrm{e}^{-a|t|-\mathrm{i}2\pi\omega t}\mathrm{d}t\nonumber                                                                                  \\
                  & =\lim\limits_{a\rightarrow0^+}\left(\int_{-\infty}^0\mathrm{e}^{(a-\mathrm{i}2\pi\omega)t}\mathrm{d}t+\int_0^{\infty}\mathrm{e}^{-(a+\mathrm{i}2\pi\omega)t}\mathrm{d}t\right)\nonumber \\
                  & =\lim\limits_{a\rightarrow0^+}\left(\frac{\mathrm{e}^{(a-\mathrm{i}2\pi\omega)t}}{a-\mathrm{i}2\pi\omega}\bigg|_{t=-\infty}^0
        +\frac{\mathrm{e}^{-(a+\mathrm{i}2\pi\omega)t}}{-(a+\mathrm{i}2\pi\omega)}\bigg|_{t=0}^{\infty}\right)\nonumber                                                                                     \\
                  & =\lim\limits_{a\rightarrow0^+}\left(\frac{1}{a-\mathrm{i}2\pi\omega}+\frac{1}{a+\mathrm{i}2\pi\omega}\right)=\lim\limits_{a\rightarrow0^+}\frac{2a}{a^2+4\pi^2\omega^2}\nonumber        \\
                  & =\left\{\begin{array}{ll}
            0,      & \text{若}\omega\neq0, \\
            \infty, & \text{若}\omega=0.
        \end{array}\right.
    \end{align}
    注意到上式取极限的部分
    \sidenote{属于\keyindex{柯西分布}{Cauchy distribution}{distribution分布}。}
    在实数域上积分与$a$无关且为
    \begin{align}
        \int_{-\infty}^{\infty}\frac{2a}{a^2+4\pi^2\omega^2}\mathrm{d}\omega
        =\frac{1}{\pi}\int_{-\infty}^{\infty}\frac{1}{1+\left(\frac{2\pi\omega}{a}\right)^2}\mathrm{d}\frac{2\pi\omega}{a}
        =\frac{1}{\pi}\arctan\frac{2\pi\omega}{a}\bigg|_{\omega=-\infty}^{\infty}=1\, .
    \end{align}
    因此它实际上就是单位冲激函数,即
    \begin{align}
        F(\omega)=\delta(\omega)\, .
    \end{align}
\end{prove}
\subsection{傅里叶变换的性质}\label{sub:傅里叶变换的性质}
\begin{theorem}
    傅里叶变换具有线性性质:对于傅里叶变换对$g(t)\leftrightarrow G(\omega)$
    与$h(t)\leftrightarrow H(\omega)$,给定任意实数$\alpha,\beta$,则
    \begin{align}
        \alpha g(t)+\beta h(t)\leftrightarrow \alpha G(\omega)+\beta H(\omega)\, .
    \end{align}
\end{theorem}

\begin{theorem}
    傅里叶变换具有缩放性质:对于傅里叶变换对$f(t)\leftrightarrow F(\omega)$,
    给定任意实数$\alpha\neq0$,则
    \begin{align}
        f(\alpha t)\leftrightarrow\frac{1}{|\alpha|} F\left(\frac{\omega}{\alpha}\right)\, .
    \end{align}
\end{theorem}
\begin{prove}
    依照傅里叶变换定义,
    \begin{align}\label{eq:7.ex01.scale}
        \mathcal{F}\{f(\alpha t)\}=\int_{-\infty}^{\infty}f(\alpha t)\mathrm{e}^{-\mathrm{i}2\pi\omega t}\mathrm{d}t
        =\frac{1}{\alpha}\int_{-\infty}^{\infty}f(\alpha t)\mathrm{e}^{-\mathrm{i}2\pi\frac{\omega}{\alpha}\alpha t}\mathrm{d}(\alpha t)\, .
    \end{align}
    当$\alpha>0$时,\refeq{7.ex01.scale}化为
    \begin{align}
        \mathcal{F}\{f(\alpha t)\}=\frac{1}{\alpha}\int_{-\infty}^{\infty}f(t)\mathrm{e}^{-\mathrm{i}2\pi\frac{\omega}{\alpha}t}\mathrm{d}t
        =\frac{1}{\alpha}F\left(\frac{\omega}{\alpha}\right)\, .
    \end{align}
    当$\alpha<0$时,\refeq{7.ex01.scale}化为
    \begin{align}
        \mathcal{F}\{f(\alpha t)\}=\frac{1}{\alpha}\int_{\infty}^{-\infty}f(t)\mathrm{e}^{-\mathrm{i}2\pi\frac{\omega}{\alpha}t}\mathrm{d}t
        =-\frac{1}{\alpha}F\left(\frac{\omega}{\alpha}\right)\, .
    \end{align}
    于是综合起来表示有
    \begin{align}
        \mathcal{F}\{f(\alpha t)\}=\frac{1}{|\alpha|}F\left(\frac{\omega}{\alpha}\right)\, .
    \end{align}
\end{prove}

\begin{theorem}\label{theorem:7.ex01.4}
    傅里叶变换具有频移与时移性质,即对于傅里叶变换对$f(t)\leftrightarrow F(\omega)$,
    给定任意常数$\omega_0$和$\tau$,则有相应的变换对
    \begin{align}
        f(t)\mathrm{e}^{\mathrm{i}2\pi\omega_0 t} & \leftrightarrow F(\omega-\omega_0)\, ,                              \\
        f(t-\tau)                                 & \leftrightarrow F(\omega)\mathrm{e}^{-\mathrm{i}2\pi\omega\tau}\, .
    \end{align}
\end{theorem}
\begin{prove}
    对于时域表示$f(t)\mathrm{e}^{\mathrm{i}2\pi\omega_0 t}$,其傅里叶变换为
    \begin{align}
        \int_{-\infty}^{\infty}f(t)\mathrm{e}^{\mathrm{i}2\pi\omega_0 t}\mathrm{e}^{-\mathrm{i}2\pi\omega t}\mathrm{d}t
        =\int_{-\infty}^{\infty}f(t)\mathrm{e}^{-\mathrm{i}2\pi(\omega-\omega_0) t}\mathrm{d}t=F(\omega-\omega_0)\, .
    \end{align}
    对于频率表示$F(\omega)\mathrm{e}^{-\mathrm{i}2\pi\omega\tau}$,其傅里叶逆变换为
    \begin{align}
        \int_{-\infty}^{\infty}F(\omega)\mathrm{e}^{-\mathrm{i}2\pi\omega\tau}\mathrm{e}^{\mathrm{i}2\pi\omega t}\mathrm{d}\omega
        =\int_{-\infty}^{\infty}F(\omega)\mathrm{e}^{\mathrm{i}2\pi\omega(t-\tau)}\mathrm{d}\omega
        =f(t-\tau)\, .
    \end{align}
\end{prove}

\begin{theorem}
    傅里叶变换和逆变换互为逆运算,即
    \begin{align}
        \mathcal{F}^{-1}\{\mathcal{F}\{f(t)\}\}      & =f(t)\, ,      \\
        \mathcal{F}\{\mathcal{F}^{-1}\{F(\omega)\}\} & =F(\omega)\, .
    \end{align}
\end{theorem}
\begin{prove}
    利用定理\ref{theorem:7.ex01.symmetry}、\ref{theorem:7.ex01.1}、\ref{theorem:7.ex01.2}以及\ref{theorem:7.ex01.4}可得
    \begin{align}
        \mathcal{F}^{-1}\{\mathcal{F}\{f(t)\}\}= & \int_{-\infty}^{\infty}\left(\int_{-\infty}^{\infty}f(\tau)\mathrm{e}^{-\mathrm{i}2\pi\omega\tau}\mathrm{d}\tau\right)\mathrm{e}^{\mathrm{i}2\pi\omega t}\mathrm{d}\omega\nonumber \\
        =                                        & \int_{-\infty}^{\infty}f(\tau)\left(\int_{-\infty}^{\infty}\mathrm{e}^{\mathrm{i}2\pi\omega(t-\tau)}\mathrm{d}\omega\right)\mathrm{d}\tau\nonumber                                 \\
        =                                        & \int_{-\infty}^{\infty}f(\tau)\delta(t-\tau)\mathrm{d}\tau\nonumber                                                                                                                \\
        =                                        & f(t)\, .
    \end{align}
    第二个式子同理可证。
\end{prove}

\begin{theorem}
    傅里叶变换具有微分性质:对于绝对连续可微函数$f$及其傅里叶变换$F(\omega)$,有
    \begin{align}
        \frac{\mathrm{d}f(t)}{\mathrm{d}t}\leftrightarrow\mathrm{i}2\pi\omega F(\omega)\, .
    \end{align}
\end{theorem}
\begin{prove}
    对\refeq{7.ex01.inverseFourier}两边求导即可得证明:
    \begin{align}
        \frac{\mathrm{d}f(t)}{\mathrm{d}t} & =\frac{\mathrm{d}}{\mathrm{d}t}\int_{-\infty}^{\infty}F(\omega)\mathrm{e}^{\mathrm{i}2\pi\omega t}\mathrm{d}\omega\nonumber              \\
                                           & =\int_{-\infty}^{\infty}\frac{\mathrm{d}}{\mathrm{d}t}\left(F(\omega)\mathrm{e}^{\mathrm{i}2\pi\omega t}\right)\mathrm{d}\omega\nonumber \\
                                           & =\int_{-\infty}^{\infty}(\mathrm{i}2\pi\omega F(\omega))\mathrm{e}^{\mathrm{i}2\pi\omega t}\mathrm{d}\omega\, .
    \end{align}
\end{prove}

\begin{theorem}
    当有傅里叶变换对$f(t)\leftrightarrow F(\omega)$,
    则$f(t)$的\keyindex{直流分量}{DC component}{}为
    \begin{align}
        \int_{-\infty}^{\infty}f(t)\mathrm{d}t=F(0)\, .
    \end{align}
\end{theorem}

\begin{definition}
    称满足$\displaystyle\int_{-\infty}^{\infty}|f(x)|^2\mathrm{d}x<\infty$的
    函数$f$为\keyindex{平方可积函数}{square-integrable function}{}。
\end{definition}
\begin{theorem}[\keyindex{普朗歇尔定理}{Plancherel theorem}{}]
    对于平方可积函数$f(t)$及其傅里叶变换$F(\omega)$,有等式
    \begin{align}
        \int_{-\infty}^{\infty}|f(t)|^2\mathrm{d}t=\int_{-\infty}^{\infty}|F(\omega)|^2\mathrm{d}\omega\, .
    \end{align}
\end{theorem}
\begin{prove}
    依照傅里叶变换定义,有\sidenote{$\overline{f(t)}$表示$f(t)$的共轭。}
    \begin{align}
        \int_{-\infty}^{\infty}|f(t)|^2\mathrm{d}t & =\int_{-\infty}^{\infty}f(t)\overline{f(t)}\mathrm{d}t\nonumber                                                                                                                                                                                \\
                                                   & =\int_{-\infty}^{\infty}\left(\int_{-\infty}^{\infty}F(\xi)\mathrm{e}^{\mathrm{i}2\pi\xi t}\mathrm{d}\xi\right)\left(\overline{\int_{-\infty}^{\infty}F(\omega)\mathrm{e}^{\mathrm{i}2\pi\omega t}\mathrm{d}\omega}\right)\mathrm{d}t\nonumber \\
                                                   & =\int_{-\infty}^{\infty}\int_{-\infty}^{\infty}\int_{-\infty}^{\infty}F(\xi)\overline{F(\omega)}\mathrm{e}^{\mathrm{i}2\pi(\xi-\omega) t}\mathrm{d}\xi\mathrm{d}\omega\mathrm{d}t\nonumber                                                     \\
                                                   & =\int_{-\infty}^{\infty}\int_{-\infty}^{\infty}F(\xi)\overline{F(\omega)}\left(\int_{-\infty}^{\infty}\mathrm{e}^{\mathrm{i}2\pi(\xi-\omega) t}\mathrm{d}t\right)\mathrm{d}\xi\mathrm{d}\omega\nonumber                                        \\
                                                   & =\int_{-\infty}^{\infty}\int_{-\infty}^{\infty}F(\xi)\overline{F(\omega)}\delta(\xi-\omega)\mathrm{d}\xi\mathrm{d}\omega\nonumber                                                                                                              \\
                                                   & =\int_{-\infty}^{\infty}\left(\int_{-\infty}^{\infty}F(\xi)\delta(\xi-\omega)\mathrm{d}\xi\right)\overline{F(\omega)}\mathrm{d}\omega\nonumber                                                                                                 \\
                                                   & =\int_{-\infty}^{\infty}F(\omega)\overline{F(\omega)}\mathrm{d}\omega=\int_{-\infty}^{\infty}|F(\omega)|^2\mathrm{d}\omega\, .
    \end{align}
\end{prove}

\begin{definition}
    对于可积函数$g(t)$与$h(t)$,称
    \begin{align}
        g(t)\otimes h(t)=\int_{-\infty}^{\infty}g(\tau)h(t-\tau)\mathrm{d}\tau
    \end{align}
    为$g(t)$与$h(t)$的\keyindex{卷积}{convolution}{},更多文献记作$g\ast h$.
\end{definition}
\begin{theorem}[\keyindex{卷积定理}{convolution theorem}{}]
    函数在时域上的卷积和在频域上的乘积等价;在时域上的乘积和在频域上的卷积等价。
    具体地,对于傅里叶变换对$g(t)\leftrightarrow G(\omega)$与$h(t)\leftrightarrow H(\omega)$,有
    \begin{align}
        \mathcal{F}\{g(t)\otimes h(t)\} & =G(\omega)H(\omega)\, ,         \\
        \mathcal{F}\{g(t)h(t)\}         & =G(\omega)\otimes H(\omega)\, .
    \end{align}
\end{theorem}
\begin{prove}
    由傅里叶变换定义,
    \begin{align}
        \mathcal{F}\{g(t)\otimes h(t)\} & =\int_{-\infty}^{\infty}(g(t)\otimes h(t))\mathrm{e}^{-\mathrm{i}2\pi\omega t}\mathrm{d}t\nonumber                                                                                              \\
                                        & =\int_{-\infty}^{\infty}\left(\int_{-\infty}^{\infty}g(\tau)h(t-\tau)\mathrm{d}\tau\right)\mathrm{e}^{-\mathrm{i}2\pi\omega t}\mathrm{d}t\nonumber                                              \\
                                        & =\int_{-\infty}^{\infty}g(\tau)\left(\int_{-\infty}^{\infty}h(t-\tau)\mathrm{e}^{-\mathrm{i}2\pi\omega t}\mathrm{d}t\right)\mathrm{d}\tau\nonumber                                              \\
                                        & =\int_{-\infty}^{\infty}g(\tau)\mathrm{e}^{-\mathrm{i}2\pi\omega\tau}\left(\int_{-\infty}^{\infty}h(t-\tau)\mathrm{e}^{-\mathrm{i}2\pi\omega (t-\tau)}\mathrm{d}t\right)\mathrm{d}\tau\nonumber \\
                                        & =\int_{-\infty}^{\infty}g(\tau)\mathrm{e}^{-\mathrm{i}2\pi\omega\tau}H(\omega)\mathrm{d}\tau\nonumber                                                                                           \\
                                        & =H(\omega)\int_{-\infty}^{\infty}g(\tau)\mathrm{e}^{-\mathrm{i}2\pi\omega\tau}\mathrm{d}\tau\nonumber                                                                                           \\
                                        & =G(\omega)H(\omega)\, .
    \end{align}
    \begin{align}
        \mathcal{F}\{g(t)h(t)\} & =\int_{-\infty}^{\infty}g(t)h(t)\mathrm{e}^{-\mathrm{i}2\pi\omega t}\mathrm{d}t\nonumber                                                                                    \\
                                & =\int_{-\infty}^{\infty}\left(\int_{-\infty}^{\infty}G(\xi)\mathrm{e}^{\mathrm{i}2\pi\xi t}\mathrm{d}\xi\right)h(t)\mathrm{e}^{-\mathrm{i}2\pi\omega t}\mathrm{d}t\nonumber \\
                                & =\int_{-\infty}^{\infty}G(\xi)\left(\int_{-\infty}^{\infty}h(t)\mathrm{e}^{-\mathrm{i}2\pi(\omega-\xi)t}\mathrm{d}t\right)\mathrm{d}\xi\nonumber                            \\
                                & =\int_{-\infty}^{\infty}G(\xi)H(\omega-\xi)\mathrm{d}\xi\nonumber                                                                                                           \\
                                & =G(\omega)\otimes H(\omega)\, .
    \end{align}
\end{prove}

\subsection{常见傅里叶变换对}\label{sub:常见傅里叶变换对}
\begin{theorem}
    对于\keyindex{矩形函数}{rectangular function}{}
    \begin{align}
        f(t)=\left\{\begin{array}{ll}
            1,                        & \displaystyle\text{若}|t|<\frac{1}{2}, \\
            \displaystyle\frac{1}{2}, & \displaystyle\text{若}|t|=\frac{1}{2}, \\
            0,                        & \displaystyle\text{若}|t|>\frac{1}{2},
        \end{array}\right.
    \end{align}
    其频率表示为
    \begin{align}
        F(\omega)=\frac{\sin(\pi\omega)}{\pi\omega}\, .
    \end{align}
\end{theorem}
\begin{prove}
    由傅里叶变换定义,
    \begin{align}
        F(\omega) & =\int_{-\infty}^{\infty}f(t)\mathrm{e}^{-\mathrm{i}2\pi\omega t}\mathrm{d}t
        =\int_{-\frac{1}{2}}^{\frac{1}{2}}\mathrm{e}^{-\mathrm{i}2\pi\omega t}\mathrm{d}t
        =-\frac{\mathrm{e}^{-\mathrm{i}2\pi\omega t}}{\mathrm{i}2\pi\omega}\bigg|_{t=-\frac{1}{2}}^{\frac{1}{2}}\nonumber \\
                  & =-\frac{\mathrm{e}^{-\mathrm{i}\pi\omega}-\mathrm{e}^{\mathrm{i}\pi\omega}}{\mathrm{i}2\pi\omega}
        =\frac{\mathrm{i}2\sin(\pi\omega)}{\mathrm{i}2\pi\omega}
        =\frac{\sin(\pi\omega)}{\pi\omega}\, .
    \end{align}
\end{prove}

\begin{theorem}
    对于\keyindex{高斯函数}{Gaussian function}{}$f(t)=\mathrm{e}^{-\pi t^2}$,
    其频率表示为$F(\omega)=\mathrm{e}^{-\pi\omega^2}$.
\end{theorem}
\begin{prove}
    由傅里叶变换定义,
    \begin{align}
        F(\omega) & =\int_{-\infty}^{\infty}f(t)\mathrm{e}^{-\mathrm{i}2\pi\omega t}\mathrm{d}t
        =\int_{-\infty}^{\infty}\mathrm{e}^{-\pi t^2}\mathrm{e}^{-\mathrm{i}2\pi\omega t}\mathrm{d}t
        =\int_{-\infty}^{\infty}\mathrm{e}^{-\pi((t+\mathrm{i}\omega)^2+\omega^2)}\mathrm{d}t\nonumber                  \\
                  & =\mathrm{e}^{-\pi\omega^2}\int_{-\infty}^{\infty}\mathrm{e}^{-\pi(t+\mathrm{i}\omega)^2}\mathrm{d}t
        =\mathrm{e}^{-\pi\omega^2}\int_{-\infty}^{\infty}\mathrm{e}^{-\pi t^2}\mathrm{d}t
        =\mathrm{e}^{-\pi\omega^2}\, .
    \end{align}
\end{prove}

\subsubsection*{余弦函数}
对于余弦函数
\begin{align}
    f(t)=\cos t\, ,
\end{align}
其频率表示为
\begin{align}
    F(\omega) & =\int_{-\infty}^{\infty}f(t)\mathrm{e}^{-\mathrm{i}2\pi\omega t}\mathrm{d}t\nonumber                                                                                                \\
              & =\int_{-\infty}^{\infty}(\cos t)\mathrm{e}^{-\mathrm{i}2\pi\omega t}\mathrm{d}t\nonumber                                                                                            \\
              & =\int_{-\infty}^{\infty}\frac{1}{2}(\mathrm{e}^{\mathrm{i}t}+\mathrm{e}^{-\mathrm{i}t})\mathrm{e}^{-\mathrm{i}2\pi\omega t}\mathrm{d}t\nonumber                                     \\
              & =\frac{1}{2}\int_{-\infty}^{\infty}(\mathrm{e}^{\mathrm{i}2\pi\frac{1}{2\pi}t}+\mathrm{e}^{\mathrm{i}2\pi\frac{-1}{2\pi}t})\mathrm{e}^{-\mathrm{i}2\pi\omega t}\mathrm{d}t\nonumber \\
              & =\frac{1}{2}\left(\delta\left(\omega-\frac{1}{2\pi}\right)+\delta\left(\omega+\frac{1}{2\pi}\right)\right)\, .
\end{align}

\subsubsection*{shah函数}
\begin{theorem}
    周期为$T$的函数$f(t)$可被展开为唯一的\keyindex{傅里叶级数}{Fourier series}{},其指数形式为
    \begin{align}
        f(t)=\sum\limits_{n=-\infty}^{\infty}c_n\mathrm{e}^{\mathrm{i}2\pi\frac{n}{T}t}\, ,
    \end{align}
    其中系数
    \begin{align}
        c_n=\frac{1}{T}\int\limits_T f(t)\mathrm{e}^{-\mathrm{i}2\pi\frac{n}{T}t}\mathrm{d}t\, .
    \end{align}
\end{theorem}

对于周期为$T$的shah函数
\begin{align}
    f(t)=\sum\limits_{k=-\infty}^{\infty}\delta(t-kT)\, ,
\end{align}
其傅里叶展开中的系数为
\begin{align}
    c_n=\frac{1}{T}\int_{-\frac{T}{2}}^{\frac{T}{2}}f(t)\mathrm{e}^{-\mathrm{i}2\pi\frac{n}{T}t}\mathrm{d}t
    =\frac{1}{T}\int_{-\frac{T}{2}}^{\frac{T}{2}}\delta(t)\mathrm{e}^{-\mathrm{i}2\pi\frac{n}{T}t}\mathrm{d}t
    =\frac{1}{T}\, .
\end{align}
于是shah函数可展开为
\begin{align}
    f(t)=\frac{1}{T}\sum\limits_{n=-\infty}^{\infty}\mathrm{e}^{\mathrm{i}2\pi\frac{n}{T}t}\, .
\end{align}
因此其频域表示为
\begin{align}
    F(\omega) & =\int_{-\infty}^{\infty}f(t)\mathrm{e}^{-\mathrm{i}2\pi\omega t}\mathrm{d}t\nonumber                                                                                            \\
              & =\int_{-\infty}^{\infty}\left(\frac{1}{T}\sum\limits_{n=-\infty}^{\infty}\mathrm{e}^{\mathrm{i}2\pi\frac{n}{T}t}\right)\mathrm{e}^{-\mathrm{i}2\pi\omega t}\mathrm{d}t\nonumber \\
              & =\frac{1}{T}\sum\limits_{n=-\infty}^{\infty}\int_{-\infty}^{\infty}\mathrm{e}^{-\mathrm{i}2\pi(\omega-\frac{n}{T})t}\mathrm{d}t\nonumber                                        \\
              & =\frac{1}{T}\sum\limits_{n=-\infty}^{\infty}\delta\left(\omega-\frac{n}{T}\right)\, .
\end{align}

\section{译者补充:初等数论基础}\label{sec:译者补充:初等数论基础}

\begin{remark}
    本节内容不是原书内容,而是译者根据\citet{ElementaryNumberTheory}
    以及\citet{wiki:ExtendedEuclideanAlgorithm}补充的,请酌情参考和斧正。
\end{remark}

\begin{notation}
    本节我们重申以下记号:
    \begin{itemize}
        \item 用$\mathbb{N}$表示全体正整数构成的集合;$\mathbb{Z}$表示全体整数构成的集合。
        \item 若命题$p$能推出命题$q$,则记为$p\Rightarrow q$;若$p$与$q$等价,则记为$p\Leftrightarrow q$.
    \end{itemize}
\end{notation}


\begin{theorem}[\protect\keyindex{最小自然数原理}{least number principle}{}]\label{theorem:7.ex02.1}
    设$T$是$\mathbb{N}$的一非空子集,则必有$t_0\in T$,
    使对任意的$t\in T$有$t_0\le t$,即$t_0$是$T$中最小的自然数。
\end{theorem}

% \begin{theorem}[最大自然数原理]
%     设$M$是$\mathbb{N}$的一非空子集,若$M$有上界(即存在$a\in \mathbb{N}$使
%     对任意的$m\in M$有$m\le a$),则必有$m_0\in M$,使对任意的$m\in M$有$m\le m_0$,
%     即$m_0$是$M$中最大的自然数。
% \end{theorem}

% \begin{theorem}[\protect\keyindex{归纳原理}{principle of induction}{}]
%     设$S\subseteq \mathbb{N}$,且满足
%     \begin{enumerate}
%         \item 有$1\in S$;
%         \item 对任意$n\in S$都有$n+1\in S$;
%     \end{enumerate}
%     则$S=\mathbb{N}$.
% \end{theorem}

% \begin{theorem}[\protect\keyindex{数学归纳法}{mathematical induction}{}]
%     设$P(n)$是关于自然数$n$的命题,若
%     \begin{enumerate}
%         \item 当$n=1$时,$P(1)$成立;
%         \item $P(n)$成立时必能推出$P(n+1)$成立;
%     \end{enumerate}
%     则$P(n)$对所有自然数$n$均成立。
% \end{theorem}

% \begin{theorem}[\protect 第二种数学归纳法]
%     设$P(n)$是关于自然数$n$的命题,若
%     \begin{enumerate}
%         \item 当$n=1$时,$P(1)$成立;
%         \item 设$n>1$,对所有自然数$m<n$都有$P(m)$成立时必能推出$P(n)$成立;
%     \end{enumerate}
%     则$P(n)$对所有自然数$n$均成立。
% \end{theorem}

\begin{theorem}[\protect\keyindex{鸽巢原理}{pigeonhole principle}{}]\label{theorem:7.ex02.2}
    对于某$n\in\mathbb{N}$,现有$n$个笼子和$n+1$只鸽子,
    所有的鸽子都被关在鸽笼里,那么至少有一个笼子有至少2只鸽子。
    也称\keyindex{狄利克雷抽屉原理}{Dirichlet's drawer principle}{}。
\end{theorem}

\subsection{整除与带余除法}\label{sub:整除与带余除法}
\begin{definition}
    设$a,b\in\mathbb{Z}$且$a\neq0$,若存在$q\in\mathbb{Z}$使得$b=aq$,
    则称$a$\keyindex{整除}{divide evenly}{}$b$,或说$b$能被$a$整除,记作$a|b$,
    并称$a$是$b$的\keyindex{因数}{divisor}{},也称{\sffamily 约数}、{\sffamily 除数},
    $b$是$a$的\keyindex{倍数}{multiple}{}。$a$不能整除$b$时记作$a\nmid b$.
\end{definition}

\begin{example}
    6能整除18,记作$6|18$,6是18的因数,18是6的倍数。
\end{example}

\begin{theorem}\label{theorem:7.ex02.3}
    整除具有以下性质:
    \begin{enumerate}
        \item $a|b\Leftrightarrow -a|b \Leftrightarrow a|-b \Leftrightarrow |a|||b|$;
        \item $a|b$且$b|c \Rightarrow a|c$;
        \item $a|b$且$a|c \Leftrightarrow$对任意的$x,y\in\mathbb{Z}$有$a|bx+cy$;
        \item 设$m\neq0$,则$a|b\Leftrightarrow ma|mb$;
        \item $a|b$且$b|a\Rightarrow b=\pm a$;
        \item 设$b\neq0$,则$a|b\Rightarrow |a|\le|b|$.
    \end{enumerate}
\end{theorem}
% \begin{corollary}
%     非零整数的因数只有有限个。
% \end{corollary}
% \begin{theorem}
%     设整数$b\neq0$,而$d_1,d_2,\ldots,d_k$是$b$的全体因数,
%     则$\displaystyle\frac{b}{d_1},\frac{b}{d_2},\ldots,\frac{b}{d_k}$也是
%     $b$的全体因数。此外,若$b>0$,则当$d$遍历$b$的全体正因数时,
%     $\displaystyle\frac{b}{d}$也遍历$b$的全体正因数。
% \end{theorem}
\begin{definition}
    设整数$p\neq0,\pm1$,若$p$除了$\pm1,\pm p$外没有其他因数,
    则称$p$为\keyindex{质数}{prime number}{},也称{\sffamily 素数}、{\sffamily 不可约数}。
    若$a\neq0,\pm1$且$a$不是质数,则称$a$是\keyindex{合数}{composite number}{}。
\end{definition}
\begin{example}
    3、5、11是质数,4、6、12是合数。0和1既不是质数也不是合数。
\end{example}
\begin{notation}
    下文若无特别说明,所指的质数总是正的。
\end{notation}
% \begin{theorem}
%     \begin{enumerate}
%         \item $a>1$是合数$\Leftrightarrow$$a=de,1<d<a,1<e<a$;
%         \item 若$d>1$,$p$是质数且$d|p$,则$d=p$.
%     \end{enumerate}
% \end{theorem}
% \begin{theorem}
%     若$a$是合数,则必存在质数$p|a$.
% \end{theorem}
% \begin{definition}
%     若一个整数的因数是质数时,称该因数为\keyindex{质因数}{prime factor}{}。
% \end{definition}
% \begin{theorem}
%     设整数$a\ge2$,则$a$一定可表示为质数的乘积(包括$a$本身是质数),即
%     \begin{align}\label{eq:7.ex02.primefactor}
%         a=p_1p_2\cdots p_s\, ,
%     \end{align}
%     其中$p_j(1\le j\le s)$是质数。
% \end{theorem}
% \begin{example}
%     1260共有6个质因数(包括相同的),其中不相同的有4个,即
%     $1260=2\times2\times3\times3\times5\times7=2^2\times3^2\times5\times7$.
% \end{example}
% \begin{corollary}
%     设整数$a\ge2$,
%     \begin{enumerate}
%         \item 若$a$是合数,则必有质数$p|a$且$p\le\sqrt{a}$;
%         \item 若$a$有表示\refeq{7.ex02.primefactor},则必有质数$p|a$且$p\le a^{\frac{1}{s}}$.
%     \end{enumerate}
% \end{corollary}
% \begin{theorem}
%     质数有无穷多个。
% \end{theorem}
% \begin{theorem}
%     设全体质数按大小排序成
%     \begin{align}
%         p_1=2,\quad p2=3,\quad p_3=5,\ldots\, .
%     \end{align}
%     则有
%     \begin{align}
%         p_n\le2^{2^{n-1}},\quad n=1,2,\ldots\, ,
%     \end{align}
%     及
%     \begin{align}
%         \pi(x)>\log_2{\log_2{x}},\quad x\ge2\, ,
%     \end{align}
%     其中$\pi(x)$表示不超过$x$的质数个数。
% \end{theorem}

初等数论的证明中最重要、最基本、最直接的工具是下面的
\keyindex{带余除法}{division with remainder}{},
也称\keyindex{欧几里德除法}{Euclidean division}{}。
\begin{theorem}\label{theorem:7.ex02.4}
    % \label{theorem:7.ex02.EuclideanDivision}
    对于给定的$a,b\in\mathbb{Z}$且$a\neq0$,必存在唯一一对$q,r\in\mathbb{Z}$,满足
    \begin{align}\label{eq:7.ex02.EuclideanDivision}
        b=qa+r,\quad 0\le r<|a|\, .
    \end{align}
    此外,$a|b \Leftrightarrow r=0$.
\end{theorem}
\begin{prove}
    {\sffamily 唯一性}\quad 若还有整数$q'$与$r'$满足
    \begin{align}\label{eq:7.ex02.prove-theorem4-01}
        b=q'a+r',\quad 0\le r'<|a|\, ,
    \end{align}
    不妨设$r'\ge r$.则由\refeq{7.ex02.EuclideanDivision}和\refeq{7.ex02.prove-theorem4-01}得$0\le r'-r<|a|$,及
    \begin{align}
        r'-r=(q-q')a\, .
    \end{align}
    若$r'-r>0$,则由上式及定理\ref{theorem:7.ex02.3}(6)
    推出$|a|\le r'-r$.这和$r'-r<|a|$矛盾。所以必有$r'=r$,进而得$q'=q$.

        {\sffamily 存在性}\quad 当$a|b$时,可取$q=\displaystyle\frac{b}{a}$,$r=0$.
    当$a\nmid b$时。考虑集合
    \begin{align}
        T=\{b-ka:k=0,\pm1,\pm2,\ldots\}\, .
    \end{align}
    容易看出,集合$T$中必有正整数,所以由定理\ref{theorem:7.ex02.1}知,
    $T$中必有一个最小正整数,设为
    \begin{align}
        t_0=b-k_0a>0\, .
    \end{align}
    现在来证明必有$t_0<|a|$.因$a\nmid b$,所以$t_0\neq |a|$.
    若$t_0>|a|$,则$t_1=t_0-|a|>0$,显然$t_1\in T$,$t_1<t_0$.
    这和$t_0$的最小性矛盾。取$q=k_0$,$r=t_0$就满足要求。

    最后,显然当$b=qa+r$时,$a|b \Leftrightarrow a|r$.
    当满足$0\le r<|a|$时,由定理\ref{theorem:7.ex02.3}(6)就
    推出$a|r \Leftrightarrow r=0$.这就证明了定理的最后一部分。
\end{prove}

上述定理还有更灵活的形式。
\begin{theorem}\label{theorem:7.ex02.5}
    对于给定的$a,b,d\in\mathbb{Z}$且$a\neq0$,必存在唯一一对$q_1,r_1\in\mathbb{Z}$,满足
    \begin{align}\label{eq:7.7.ex02.remainder}
        b=q_1a+r_1,\quad d\le r_1<|a|+d\, .
    \end{align}
    此外,$a|b \Leftrightarrow a|r_1$.
\end{theorem}

只要对$a$和$b-d$用定理\ref{theorem:7.ex02.4}即可推出定理\ref{theorem:7.ex02.5}。
适当选取$d$可令\refeq{7.7.ex02.remainder}变形为下面的形式:
\begin{align}
    b & =q_1a+r_1, & -\frac{|a|}{2}< r_1\le\frac{|a|}{2}\, ,\label{eq:7.ex02.remainder02} \\
    b & =q_1a+r_1, & -\frac{|a|}{2}\le r_1<\frac{|a|}{2}\, ,\label{eq:7.ex02.remainder03} \\
    b & =q_1a+r_1, & 1\le r_1\le |a|\, .\label{eq:7.ex02.remainder04}
\end{align}
通常称\refeq{7.ex02.EuclideanDivision}中的$r$为$b$被$a$除后的\keyindex{最小非负余数}{least non-negative remainder}{remainder余数},
\refeq{7.ex02.remainder02}和\refeq{7.ex02.remainder03}中的$r_1$都称为\keyindex{绝对最小余数}{least absolute remainder}{remainder余数},
\refeq{7.ex02.remainder04}中的$r_1$称为\keyindex{最小正余数}{least positive remainder}{remainder余数},
\refeq{7.7.ex02.remainder}中的$r_1$统称为\keyindex{余数}{remainder}{}。

% \begin{corollary}
%     设$a>0$,任意整数被$a$除后所得的最小非负余数是且仅是$0,1,\ldots,a-1$这$a$个数中的一个。
% \end{corollary}
\begin{corollary}
    给定正整数$a\ge2$,则任一正整数$n$必可唯一表示为
    \begin{align}
        n=r_ka^k+r_{k-1}a^{k-1}+\cdots+r_1a+r_0\, ,
    \end{align}
    其中整数$k\ge0,0\le r_j\le a-1(0\le j\le k),r_k\neq0$.
    这即正整数的$a$进制表示。
\end{corollary}
\begin{prove}
    对正整数$n$必有唯一的$k\ge 0$,使得$a^k\le n<a^{k+1}$.
    由带余除法知,必有唯一的$q_0,r_0$满足
    \begin{align}
        n=q_0a+r_0,\quad 0\le r_0<a\, .
    \end{align}
    若$k=0$,则必有$q_0=0$,$1\le r_0<a$,所以结论成立。
    设结论对$k=m\ge0$成立,则当$k=m+1$时,上式中的$q_0$必满足
    \begin{align}
        a^m\le q_0<a^{m+1}\, .
    \end{align}
    由假设知
    \begin{align}
        q_0=s_ma^m+\cdots+s_0\, ,
    \end{align}
    其中$0\le s_j\le a-1(0\le j\le m-1)$,$1\le s_m\le a-1$.因而有
    \begin{align}
        n=s_ma^{m+1}+\cdots+s_0a+r_0\, ,
    \end{align}
    即结论对$m+1$也成立。由数学归纳法,推论得证。
\end{prove}

\subsection{最大公因数与最小公倍数}\label{sub:最大公因数与最小公倍数}
\begin{definition}
    设$a_1,a_2\in\mathbb{Z}$,若$d|a_1$且$d|a_2$,则称$d$是
    $a_1$与$a_2$的\keyindex{公因数}{common divisor}{divisor因数}。
    一般地,设$a_1,\ldots,a_k$是$k$个整数,若$d|a_1,\cdots,d|a_k$,
    则称$d$是$a_1,\ldots,a_k$的公因数。
\end{definition}
\begin{example}
    12和18的公因数是$\pm1,\pm2,\pm3,\pm6$.$n$和$n+1$的公因数是$\pm1$.
    当$a_1,\ldots,a_k$中有一个不为零时,它们的公因数个数有限。
\end{example}
\begin{definition}
    设$a_1,a_2\in\mathbb{Z}$不全为零,称$a_1$和$a_2$的公因数中
    最大的为$a_1$和$a_2$的\keyindex{最大公因数}{greatest common divisor}{divisor因数}(GCD),
    记作$(a_1,a_2)$.一般地,设$a_1,\ldots,a_k$是$k$个不全为零的整数,
    称$a_1,\ldots,a_k$的公因数中最大的为$a_1,\ldots,a_k$的最大公因数,
    记作$(a_1,\ldots,a_k)$.用$\mathcal{D}(a_1,\ldots,a_k)$表示$a_1,\ldots,a_k$的
    所有公因数组成的集合。于是
    \begin{align}
        (a_1,a_2)        & =\max\limits_{d\in\mathcal{D}(a_1,a_2)}{d}\, ,        \\
        (a_1,\ldots,a_k) & =\max\limits_{d\in\mathcal{D}(a_1,\ldots,a_k)}{d}\, .
    \end{align}
\end{definition}
\begin{example}
    $\mathcal{D}(12,16)=\{\pm1,\pm2,\pm3,\pm6\}$,$(12,18)=6$;
    $\mathcal{D}(6,10,-15)=\{\pm1\}$,$(6,10,-15)=1$;
    $(n,n+1)=1$.
\end{example}
\begin{theorem}\label{theorem:7.ex02.6}
    最大公因数满足以下性质:
    \begin{enumerate}
        \item $(a_1,a_2)=(a_2,a_1)=(-a_1,a_2)$;一般地,\\
              $(a_1,a_2,\ldots,a_i,\ldots,a_k)=(a_i,a_2,\ldots,a_1,\ldots,a_k)=(-a_1,a_2,\ldots,a_i,\ldots,a_k)$;
        \item $a_1|a_j(j=2,\ldots,k)\Rightarrow (a_1,a_2)=(a_1,a_2,\ldots,a_k)=|a_1|$;
        \item 对任意整数$x$,$(a_1,a_2)=(a_1,a_2,a_1x)$;$(a_1,\ldots,a_k)=(a_1,\ldots,a_k,a_1x)$;
        \item 对任意整数$x$,$(a_1,a_2)=(a_1,a_2+a_1x)$;\\
              $(a_1,a_2,a_3,\ldots,a_k)=(a_1,a_2+a_1x,a_3,\ldots,a_k)$;
        \item 若$p$是质数,则
              \begin{align}
                  (p,a_1)=\left\{
                  \begin{array}{ll}
                      p, & \text{若}p|a_1\, ,      \\
                      1, & \text{若}p\nmid a_1\, ;
                  \end{array}
                  \right.
              \end{align}
              一般地
              \begin{align}
                  (p,a_1,\ldots,a_k)=\left\{
                  \begin{array}{ll}
                      p, & \text{若}p|a_j,\quad j=1,2,\ldots,k, \\
                      1, & \text{其他。}
                  \end{array}
                  \right.
              \end{align}
    \end{enumerate}
\end{theorem}
\begin{definition}
    若$(a_1,a_2)=1$,则称$a_1$和$a_2$是\keyindex{互质}{coprime}{}
    (或relatively prime、mutually prime)的,也称{\sffamily 互素}、{\sffamily 既约}。
    一般地,若$(a_1,\ldots,a_k)=1$,则称$a_1,\ldots,a_k$是互质的。
\end{definition}
\begin{theorem}\label{theorem:7.ex02.7}
    若存在整数$x_1,\ldots,x_k$使得$a_1x_1+\cdots+a_kx_k=1$,则$a_1,\ldots,a_k$是互质的。
\end{theorem}
\begin{prove}
    因为$a_1,\ldots,a_k$的任意公因数$d$一定要整除1,所以必有$d=\pm1$.定理得证。
\end{prove}
\begin{theorem}\label{theorem:7.ex02.8}
    设正整数$m|(a_1,\ldots,a_k)$,则
    \begin{align}\label{eq:7.ex02.prove-theorem8-01}
        m\left(\frac{a_1}{m},\cdots,\frac{a_k}{m}\right)=(a_1,\ldots,a_k)\, .
    \end{align}
    特别地有
    \begin{align}\label{eq:7.ex02.prove-theorem8-02}
        \left(\frac{a_1}{(a_1,\cdots,a_k)},\ldots,\frac{a_k}{(a_1,\cdots,a_k)}\right)=1\, .
    \end{align}
\end{theorem}
\begin{prove}
    记$D=(a_1,\ldots,a_k)$.由$m|D$,$D|a_j(1\le j \le k)$知
    $m|a_j(1\le j \le k)$,故
    \begin{align}
        \frac{D}{m}\bigg|\frac{a_j}{m},\quad j=1,\ldots,k\, ,
    \end{align}
    即$\displaystyle\frac{D}{m}$是$\displaystyle\frac{a_1}{m},\ldots,\frac{a_k}{m}$的公因数
    且为正,所以由定义知
    \begin{align}\label{eq:7.ex02.prove-theorem8-03}
        \frac{D}{m}\le\left(\frac{a_1}{m},\ldots,\frac{a_k}{m}\right)\, .
    \end{align}
    另一方面,若$\displaystyle d\bigg|\frac{a_j}{m}(1\le j\le k)$,
    则$md|a_j(j=1,\ldots,k)$,由定义知
    \begin{align}
        md\le D,\quad \text{即}d\le\frac{D}{m}\, .
    \end{align}
    取$d=\displaystyle\left(\frac{a_1}{m},\ldots,\frac{a_k}{m}\right)$,
    由此及\refeq{7.ex02.prove-theorem8-03}即得\refeq{7.ex02.prove-theorem8-01}。
    在\refeq{7.ex02.prove-theorem8-01}中取$m=(a_1,\ldots,a_k)$即得\refeq{7.ex02.prove-theorem8-02}。
\end{prove}
\begin{definition}
    设$a_1,a_2\in\mathbb{Z}$均不为零,若$a_1|l$且$a_2|l$,
    则称$l$是$a_1$和$a_2$的\keyindex{公倍数}{common multiple}{multiple倍数}。
    一般地,设$a_1,\ldots,a_k$是$k$个均不为零的整数,
    若$a_1|l,\ldots,a_k|l$,则称$l$是$a_1,\ldots,a_k$的公倍数。
    此外,以$\mathcal{L}(a_1,\ldots,a_k)$表示$a_1,\ldots,a_k$的所有公倍数构成的集合。
\end{definition}
\begin{example}
    $\mathcal{L}(2,3)=\{0,\pm6,\pm12,\ldots,\pm6k,\ldots\}$.
\end{example}
\begin{definition}
    设$a_1,a_2\in\mathbb{Z}$均不为零,我们把$a_1$和$a_2$公倍数中的最小正数
    称为$a_1$和$a_2$的\keyindex{最小公倍数}{least common multiple}{multiple倍数},记作$[a_1,a_2]$,即
    \begin{align}
        [a_1,a_2]=\min\limits_{l\in\mathcal{L}(a_1,a_2),l>0}{l}\, .
    \end{align}
    一般地,设$a_1,\ldots,a_k\in\mathbb{Z}$均不为零,我们把
    $a_1,\ldots,a_k$公倍数中的最小正数称为$a_1,\ldots,a_k$的最小公倍数,
    记作$[a_1,\ldots,a_k]$,即
    \begin{align}
        [a_1,\ldots,a_k]=\min\limits_{l\in\mathcal{L}(a_1,\ldots,a_k),l>0}{l}\, .
    \end{align}
\end{definition}
\begin{example}
    $[2,3]=6$;$[2,3,4]=12$.
\end{example}
\begin{theorem}\label{theorem:7.ex02.9}
    最小公倍数满足以下性质:
    \begin{enumerate}
        \item $[a_1,a_2]=[a_2,a_1]=[-a_1,a_2]$;一般有\\
              $[a_1,a_2,\ldots,a_i,\ldots,a_k]=[a_i,a_2,\ldots,a_1,\ldots,a_k]=[-a_1,a_2,\ldots,a_i,\ldots,a_k]$;
        \item $a_2|a_1\Rightarrow [a_1,a_2]=|a_1|$;\\
              $a_j|a_1(2\le j\le k)\Rightarrow [a_1,\ldots,a_k]=|a_1|$;
        \item 对任意的$d|a_1$,有$[a_1,a_2]=[a_1,a_2,d]$;$[a_1,\ldots,a_k]=[a_1,\ldots,a_k,d]$.
    \end{enumerate}
\end{theorem}
\begin{theorem}\label{theorem:7.ex02.10}
    设$m>0$,则$[ma_1,\ldots,ma_k]=m[a_1,\ldots,a_k]$.
\end{theorem}
\begin{prove}
    设$L=[ma_1,\ldots,ma_k], L'=[a_1,\ldots,a_k]$.
    由$ma_j|L(1\le j\le k)$推出$\displaystyle a_j\bigg|\frac{L}{m}(1\le j\le k)$,
    进而由最小公倍数定义知$L'\le\displaystyle\frac{L}{m}$.
    另一方面,由$a_j|L'(1\le j\le k)$推出$ma_j|mL'(1\le j\le k)$,
    进而由最小公倍数定义得$L\le mL'$.由此定理得证。
\end{prove}
\begin{theorem}\label{theorem:7.ex02.11}
    $a_j|c(1\le j\le k)\Leftrightarrow [a_1,\ldots,a_k]|c$.
\end{theorem}
\begin{prove}
    $[a_1,\ldots,a_k]|c\Rightarrow a_j|c(1\le j\le k)$是显然的。
    下面证$a_j|c(1\le j\le k)\Rightarrow [a_1,\ldots,a_k]|c$.
    设$L=[a_1,\ldots,a_k]$.由定理\ref{theorem:7.ex02.4}知,有$q,r$使得
    \begin{align}
        c=qL+r,\quad 0\le r<L\, .
    \end{align}
    由此及$a_j|c$推出$a_j|r(1\le j\le k)$,所以$r$是公倍数。
    进而由最小公倍数的定义及$0\le r<L$可得$r=0$,即$L|c$.
    结论表明:公倍数一定是最小公倍数的倍数。
\end{prove}
\begin{theorem}\label{theorem:7.ex02.12}
    设$D$为正整数,则$D=(a_1,\ldots,a_k)$的充要条件是
    \begin{enumerate}
        \item $D|a_j(1\le j\le k)$;
        \item 若$d|a_j(1\le j\le k)$,则$d|D$.
    \end{enumerate}
\end{theorem}
\begin{prove}
    {\sffamily 充分性}\quad 由第一个条件知$D$是$a_j(1\le j\le k)$的公因数,
    由第二个条件、定理\ref{theorem:7.ex02.3}(6)及$D\ge1$知,
    $a_j(1\le j\le k)$的任一公因数$d$有$|d|\le D$.
    因而由定义知$D=(a_1,\ldots,a_k)$.

        {\sffamily 必要性}\quad 设$d_1,\ldots,d_s$是$a_1,\ldots,a_k$的
    全体公因数,$L=[d_1,\ldots,d_s]$.由定理\ref{theorem:7.ex02.11}
    知$L|a_j(1\le j\le k)$,因此$L$满足了两个条件。
    由上面充分性的证明知$L=(a_1,\ldots,a_k)=D$.必要性得证。
    结论表明:公因数一定是最大公因数的因数。
\end{prove}
\begin{theorem}\label{theorem:7.ex02.13}
    设$m>0$,则$m(b_1,\ldots,b_k)=(mb_1,\ldots,mb_k)$.
\end{theorem}
\begin{prove}
    在定理\ref{theorem:7.ex02.8}中取$a_j=mb_j(1\le j\le k)$,
    由定理\ref{theorem:7.ex02.12}可得$m|(a_1,\ldots,a_k)$.
    因此\refeq{7.ex02.prove-theorem8-01}成立,即本定理结论成立。
\end{prove}
\begin{theorem}\label{theorem:7.ex02.14}
    \begin{enumerate}
        \item $(a_1,a_2,a_3,\ldots,a_k)=((a_1,a_2),a_3,\ldots,a_k)$;
        \item $(a_1,\ldots,a_{k+r})=((a_1,\ldots,a_k),(a_{k+1},\ldots,a_{k+r}))$.
    \end{enumerate}
\end{theorem}
\begin{prove}
    对于第一个结论:若$d|a_j(1\le j\le k)$,则由定理\ref{theorem:7.ex02.12}知,
    $d|(a_1,a_2)$,$d|a_j(3\le j\le k)$;反过来,若$d|(a_1,a_2)$,$d|a_j(3\le j\le k)$,
    则由定义知,$d|a_j(1\le j\le k)$.这就证明了
    \begin{align}
        \mathcal{D}(a_1,a_2,a_3,\ldots,a_k)=\mathcal{D}((a_1,a_2),a_3,\ldots,a_k)\, .
    \end{align}
    故第一个结论成立。由它可立即推出第二个结论。
\end{prove}
\begin{theorem}\label{theorem:7.ex02.15}
    设$(m,a)=1$,则$(m,ab)=(m,b)$.
\end{theorem}
\begin{prove}
    $m=0$时$a=\pm1$,结论显然成立。当$m\neq0$时,
    由定理\ref{theorem:7.ex02.6}、定理\ref{theorem:7.ex02.13}和定理\ref{theorem:7.ex02.14}可得
    \begin{align}
        (m,b)=(m,b(m,a))=(m,(mb,ab))=(m,mb,ab)=(m,ab)\, .
    \end{align}
    得证。
\end{prove}
\begin{theorem}\label{theorem:7.ex02.16}
    设$(m,a)=1$,那么,若$m|ab$,则$m|b$.
\end{theorem}
\begin{prove}
    由定理\ref{theorem:7.ex02.6}和定理\ref{theorem:7.ex02.15}得
    $|m|=(m,ab)=(m,b)$,于是$m|b$.
\end{prove}
% \begin{theorem}
%     $[a_1,a_2](a_1,a_2)=|a_1a_2|$.
% \end{theorem}
\begin{theorem}\label{theorem:7.ex02.17}
    设$a_1,\ldots,a_k\in\mathbb{Z}$不全为零,则有
    \begin{enumerate}
        \item $(a_1,\ldots,a_k)=\min\{s=a_1x_1+\cdots+a_kx_k:x_j\in\mathbb{Z}(1\le j\le k),s>0\}$,即
              $a_1,\ldots,a_k$的最大公因数等于$a_1,\ldots,a_k$的所有整系数线性组合
              构成的集合$S$中的最小正整数。
        \item 一定存在一组整数$x'_1,\ldots,x'_k$使得
              \begin{align}\label{eq:7.ex02.theorem17-02}
                  (a_1,\ldots,a_k)=a_1x'_1+\cdots+a_kx'_k\, .
              \end{align}
    \end{enumerate}
\end{theorem}
\begin{prove}
    由于$0<a_1^2+\cdots+a_k^2\in S$,所以集合$S$中有正整数,
    由定理\ref{theorem:7.ex02.1}知$S$中必有最小正整数,记为$s_0$.
    显然对任一公因数$d|a_j(1\le j \le k)$必有$d|s_0$,所以$|d|\le s_0$.
    另一方面,对任一$a_j$由定理\ref{theorem:7.ex02.4}知存在$q_j,r_j$满足
    \begin{align}
        a_j=q_js_0+r_j,\quad 0\le r_j<s_0\, .
    \end{align}
    显然$r_j\in S$.若$r_j>0$,则和$s_0$的最小性矛盾,所以$r_j=0$,
    即$s_0|a_j(1\le j \le k)$.所以$s_0$是最大公因数。$s_0$当然是
    \refeq{7.ex02.theorem17-02}右边的形式。
\end{prove}

% \subsection*{算术基本定理}
% \begin{theorem}
%     设$p$是质数,$p|a_1a_2$,则$p|a_1$或$p|a_2$至少有一个成立。
%     一般地,若$p|a_1\cdots a_k$,则$p|a_1,\ldots,p|a_k$至少有一个成立。
% \end{theorem}
% \begin{theorem}[\protect\keyindex{算术基本定理}{fundamental theorem of arithmetic}{}]
%     设$a>1$,则必有
%     \begin{align}\label{eq:7.ex02.arithmeticfundamental}
%         a=p_1p_2\cdots p_s\, ,
%     \end{align}
%     其中$p_j(1\le j\le s)$是质数,且在不计次序的意义下,
%     表示\refeq{7.ex02.arithmeticfundamental}是唯一的。
% \end{theorem}

\subsection{辗转相除法}\label{sub:辗转相除法}
\keyindex{辗转相除法}{Euclidean algorithm}{},
也称{\sffamily 欧几里得算法},是指下面求取最大公因数的算法。
它最早出现于欧几里得的《几何原本》中,我国则可追溯至约东汉出现的《九章算术》。
\begin{theorem}\label{theorem:7.ex02.18}
    给定$u_0,u_1\in\mathbb{Z}$,且$u_1\neq0,u_1\nmid u_0$.
    我们一定可以反复应用定理\ref{theorem:7.ex02.4}得到下面$k+1$个等式:
    \begin{align}\label{eq:7.ex02.EuclideanAlgorithm}
        u_0     & =q_0u_1+u_2,             &  & 0<u_2<|u_1|,\nonumber    \\
        u_1     & =q_1u_2+u_3,             &  & 0<u_3<u_2,\nonumber      \\
        u_2     & =q_2u_3+u_4,             &  & 0<u_4<u_3,\nonumber      \\
        \cdots  & \cdots\cdots\cdots\cdots &  & \cdots\cdots\cdots\cdots \\
        u_{k-2} & =q_{k-2}u_{k-1}+u_k,     &  & 0<u_k<u_{k-1},\nonumber  \\
        u_{k-1} & =q_{k-1}u_k+u_{k+1},     &  & 0<u_{k+1}<u_k,\nonumber  \\
        u_k     & =q_ku_{k+1}.             &  & \nonumber
    \end{align}
\end{theorem}
\begin{prove}
    对$u_0,u_1$应用定理\ref{theorem:7.ex02.4},由$u_1\nmid u_0$知
    必有第一式成立。同样地,如果$u_2\nmid u_1$就得到第二式。
    如果$u_2\nmid u_1$就证明定理对$k=1$成立。以此类推,就得到
    \begin{align}
        |u_1|>u_2>u_3\cdots>u_{j+1}>0
    \end{align}
    以及前面$j$个等式成立。若$u_{j+1}|u_j$,则定理对$k=j$成立;
    若$u_{j+1}\nmid u_j$,则继续对$u_j,u_{j+1}$用定理\ref{theorem:7.ex02.4}。
    由于小于$|u_1|$的正整数只有有限个,而1整除任一整数,
    所以该过程不能无限进行下去,一定会出现某个$k$,要么$1<u_{k+1}|u_k$,
    要么$1=u_{k+1}|u_k$,证毕。
\end{prove}
\begin{theorem}\label{theorem:7.ex02.19}
    在定理\ref{theorem:7.ex02.18}的条件和符号下,我们有
    \begin{enumerate}
        \item $u_{k+1}=(u_0,u_1)$;
        \item $d|u_0$且$d|u_1$的充要条件是$d|u_{k+1}$;
        \item 存在整数$x_0,x_1$,使$u_{k+1}=x_0u_0+x_1u_1$.
    \end{enumerate}
\end{theorem}
\begin{prove}
    利用定理\ref{theorem:7.ex02.6}(1)、(4),从\refeq{7.ex02.EuclideanAlgorithm}的
    最后一式开始依次往上推,可得
    \begin{align}
        u_{k+1} & =(u_{k+1},u_k)=(u_k,u_{k-1})=(u_{k-1},u_{k-2})=\cdots\nonumber \\
                & =(u_4,u_3)=(u_3,u_2)=(u_2,u_1)=(u_1,u_0)\, ,
    \end{align}
    这就得到了第一个结论。利用定理\ref{theorem:7.ex02.3}(2)、(3),
    从\refeq{7.ex02.EuclideanAlgorithm}立即推出第二个结论。
    由\refeq{7.ex02.EuclideanAlgorithm}的第$k$式知$u_{k+1}$可
    表示为$u_{k-1}$和$u_k$的整系数线性组合,
    利用\refeq{7.ex02.EuclideanAlgorithm}的第$k-1$式
    可消去该表示中的$u_k$,将$u_{k+1}$表示为$u_{k-2}$和$u_{k-1}$的
    整系数线性组合。以此类推利用\refeq{7.ex02.EuclideanAlgorithm}的
    第$k-2,k-3,\ldots,2,1$式,就能相应地消去$u_{k-1},u_{k-2},\ldots,u_3,u_2$,
    最后将$u_{k+1}$表示为$u_0$和$u_1$的整系数线性组合,即证明了第三个结论。
\end{prove}
% \begin{example}
%     利用辗转相除法求198和252的最大公因数,并将其表示为198和252的整系数线性组合。因为
%     \begin{align*}
%         252 & =1\times198+54\, , \\
%         198 & =3\times54+36\, ,  \\
%         54  & =1\times36+18\, ,  \\
%         36  & =2\times18\, ,
%     \end{align*}
%     于是$(252,198)=(198,54)=(54,36)=(36,18)=18$,且得
%     \begin{align*}
%         18 & =54-1\times36                 \\
%            & =54-(198-3\times54)           \\
%            & =-198+4\times54               \\
%            & =-198+4\times(252-1\times198) \\
%            & =4\times252-5\times198\, .
%     \end{align*}
% \end{example}

\keyindex{扩展欧几里得算法}{extended Euclidean algorithm}{}是
辗转相除法的扩展。对于给定的不全为零整数$a,b$(不妨设$b\neq0$),
它求解关于整数变量$x,y$的方程
\begin{align}\label{eq:7.ex02.ExtendedEuclideanAlgorithm}
    ax+by=(a,b)\, .
\end{align}
由定理\ref{theorem:7.ex02.17}知该方程一定有解。
如果$a$为负数,则可以转化为
\begin{align}
    |a|(-x)+by=(|a|,b)\, ,
\end{align}
$b$为负数时同理。因此我们只考虑$a,b$不小于零的情况。
在定理\ref{theorem:7.ex02.18}的记号下,原始的
辗转相除法求解$(a,b)$的递推过程是
\begin{align}
    u_0     & =a\, ,\nonumber                  \\
    u_1     & =b\, ,\nonumber                  \\
    u_2     & =u_0-q_0u_1\, ,\nonumber         \\
    u_3     & =u_1-q_1u_2\, ,\nonumber         \\
    \ldots\nonumber                            \\
    u_{k+1} & =u_{k-1}-q_{k-1}u_k\, ,\nonumber \\
    u_{k+2} & =u_k-q_ku_{k+1}=0\, .
\end{align}
最后一步得到$u_{k+2}=0$时算法终止,此时$u_{k+1}|u_k$,即$(a,b)=u_{k+1}$.

扩展欧几里得算法则还利用了以上步骤中的商$q_i$以求解\refeq{7.ex02.ExtendedEuclideanAlgorithm}:
它新引入了两组序列$s_i,t_i$,并初始化$s_0=1,s_1=0,t_0=0,t_1=1$,
在辗转相除法每步计算$u_{i+1}=u_{i-1}-q_{i-1}u_i$后额外计算
\begin{align}
    s_{i+1} & =s_{i-1}-q_{i-1}s_i\, , \\
    t_{i+1} & =t_{i-1}-q_{i-1}t_i\, ,
\end{align}
则当$u_{k+2}=0$算法终止时,求得\refeq{7.ex02.ExtendedEuclideanAlgorithm}的解
即为$x=s_{k+1},y=t_{k+1}$.
\begin{prove}
    当$i=0,1$时,显然可以验证
    \begin{align}\label{eq:7.ex02.ExtendedEuclidean-prove-01}
        as_i+bt_i=u_i
    \end{align}
    成立。若\refeq{7.ex02.ExtendedEuclidean-prove-01}对某个$i$成立,则有
    \begin{align}
        u_{i+1} & =u_{i-1}-q_{i-1}u_i\nonumber                          \\
                & =(as_{i-1}+bt_{i-1})-q_{i-1}(as_i+bt_i)\nonumber      \\
                & =a(s_{i-1}-q_{i-1}s_i)+b(t_{i-1}-q_{i-1}t_i)\nonumber \\
                & =as_{i+1}+bt_{i+1}\, ,
    \end{align}
    即\refeq{7.ex02.ExtendedEuclidean-prove-01}对$i+1$也成立。
    由数学归纳法,\refeq{7.ex02.ExtendedEuclidean-prove-01}对
    算法步骤中的所有$i$都成立,故
    \begin{align}
        as_{k+1}+bt_{k+1}=u_{k+1}=(a,b)\, ,
    \end{align}
    即求得\refeq{7.ex02.ExtendedEuclideanAlgorithm}的解为$x=s_{k+1},y=t_{k+1}$.
\end{prove}
\begin{example}
    以$a=240,b=46$为例演示扩展欧几里得法,具体步骤是
    \begin{align}
        \begin{array}{lrrrr}
            i & q_{i-2}     & u_i              & s_i             & t_i                \\
            0 & -           & 240              & 1               & 0                  \\
            1 & -           & 46               & 0               & 1                  \\
            2 & 240\div46=5 & 240-5\times46=10 & 1-5\times0=1    & 0-5\times1=-5      \\
            3 & 46\div10=4  & 46-4\times10=6   & 0-4\times1=-4   & 1-4\times(-5)=21   \\
            4 & 10\div6=1   & 10-1\times6=4    & 1-1\times(-4)=5 & -5-1\times21=-26   \\
            5 & 6\div4=1    & 6-1\times4=2     & -4-1\times5=-9  & 21-1\times(-26)=47 \\
            6 & 4\div2=2    & 4-2\times2=0     & -               & -
        \end{array}\nonumber
    \end{align}
    算法在$i=6$时终止,由$i=5$时的$u_i,s_i,t_i$可得$-9\times240+47\times46=(240,46)=2$.
\end{example}

\subsection{同余}\label{sub:同余}
\begin{definition}
    设$a,b,m\in\mathbb{Z}$且$m\neq0$,若$m|a-b$,则称$a$与$b$\keyindex{模$m$同余}{congruent modulo $m$}{},
    也称$a$同余于$b$模$m$、$b$是$a$对模$m$的剩余,记作
    \begin{align}\label{eq:7.ex02.congruent}
        a\equiv b\pmod{m}\, ,
    \end{align}
    其中$m$称为\keyindex{模}{modulus}{},称\refeq{7.ex02.congruent}为模$m$的同余式;
    否则称$a$不同余于$b$模$m$、$b$不是$a$对模$m$的剩余,记作
    \begin{align}
        a\not\equiv b\pmod{m}\, .
    \end{align}
\end{definition}

因为$m|a-b\Leftrightarrow -m|a-b$,所以\refeq{7.ex02.congruent}等价于$a\equiv b\pmod{-m}$.
由此,下文均假定模$m\ge1$.\refeq{7.ex02.congruent}中,
若$0\le b<m$,则称$b$是$a$对模$m$的最小非负剩余;
若$1\le b\le m$,则称$b$是$a$对模$m$的最小正剩余;
若$\displaystyle -\frac{m}{2}<b\le\frac{m}{2}$(或$\displaystyle -\frac{m}{2}\le b<\frac{m}{2}$),
则称$b$是$a$对模$m$的绝对最小剩余。
\begin{example}
    $m|a$可记为$a\equiv 0\pmod{m}$;偶数可记为$a\equiv 0\pmod{2}$;
    奇数可记为$a\equiv 1\pmod{2}$.
\end{example}

% \begin{theorem}
%     $a$与$b$模$m$同余的充要条件是$a$和$b$被$m$除后的最小非负余数相等,即若
%     \begin{align}
%         a & =q_1m+r_1, & 0\le r_1<m\, , \\
%         b & =q_2m+r_2, & 0\le r_2<m\, ,
%     \end{align}
%     则$r_1=r_2$.
% \end{theorem}
% \begin{prove}
%     因为$a-b=(q_1-q_2)m+(r_1-r_2)$,所以$m|a-b$的充要条件是$m|r_1-r_2$,
%     由此及$0\le |r_1-r_2|<m$即得$r_1=r_2$.
% \end{prove}

% 容易证明,$a$对模$m$的最小非负剩余、最小正剩余、绝对最小剩余
% 正好分别是$a$被$m$除后的最小非负余数、最小正余数、绝对最小余数。

\begin{theorem}\label{theorem:7.ex02.20}
    同余是一种等价关系,即有
    \begin{enumerate}
        \item $a\equiv a\pmod{m}$;
        \item $a\equiv b\pmod{m} \Leftrightarrow b\equiv a\pmod{m}$;
        \item $a\equiv b\pmod{m}, b\equiv c\pmod{m} \Rightarrow a\equiv c\pmod{m}$.
    \end{enumerate}
\end{theorem}
\begin{prove}
    由$m|a-a=0$,$m|a-b\Leftrightarrow m|b-a$,以及
    $m|a-b,m|b-c\Rightarrow m|(a-b)+(b-c)=a-c$,就推出这三个性质。
\end{prove}
\begin{theorem}\label{theorem:7.ex02.21}
    同余式可以相加,即若
    \begin{align}\label{eq:7.ex02.addcongruent}
        a\equiv b\pmod{m},\qquad c\equiv d\pmod{m}\, ,
    \end{align}
    则
    \begin{align}
        a+c\equiv b+d\pmod{m}\, .
    \end{align}
\end{theorem}
\begin{prove}
    由$m|a-b,m|c-d\Rightarrow m|(a-b)+(c-d)=(a+c)-(b+d)$,就证明该结论。
\end{prove}
\begin{theorem}\label{theorem:7.ex02.22}
    同余式可以相乘,即若\refeq{7.ex02.addcongruent}成立,则有
    \begin{align}
        ac\equiv bd\pmod{m}\, .
    \end{align}
\end{theorem}
\begin{prove}
    由$a=b+k_1m$,$c=d+k_2m$推出$ac=bd+(bk_2+dk_1+k_1k_2m)m$,就证明该结论。
\end{prove}
% \begin{theorem}
%     设$f(x)=a_nx^n+\cdots+a_0$,$g(x)=b_nx^n+\cdots+b_0$是
%     两个整系数多项式,满足
%     \begin{align}\label{eq:7.ex02.polynomialcongruent}
%         a_j\equiv b_j\pmod{m},\quad 0\le j\le n\, .
%     \end{align}
%     那么若$a\equiv b\pmod{m}$,则
%     \begin{align}
%         f(a)\equiv g(b)\pmod{m}\, .
%     \end{align}
% \end{theorem}
% \begin{definition}
%     把满足\refeq{7.ex02.polynomialcongruent}的这两个多项式
%     称作多项式$f(x)$与$g(x)$模$m$同余,记作
%     \begin{align}
%         f(x)\Equiv g(x)\pmod{m}\, .
%     \end{align}
% \end{definition}

% \begin{theorem}
%     设$d\ge1$, $d|m$,则$a\equiv b\pmod{m} \Rightarrow a\equiv b\pmod{d}$.
% \end{theorem}
% \begin{theorem}
%     设$d\neq0$,则$a\equiv b\pmod{m} \Leftrightarrow da\equiv db\pmod{|d|m}$.
% \end{theorem}

注意在模不变的条件下,同余式两边不能相约。
\begin{example}
    $6\times3\equiv6\times8\pmod{10}$,但是$3\not\equiv8\pmod{10}$.
\end{example}

\begin{theorem}\label{theorem:7.ex02.23}
    同余式$\displaystyle ca\equiv cb\pmod{m}\Leftrightarrow a\equiv b\pmod{\frac{m}{(c,m)}}$.
    特别地,当$(c,m)=1$时可得$a\equiv b\pmod{m}$,即此时可两边约去$c$.
\end{theorem}
\begin{prove}
    $ca\equiv cb\pmod{m}$即$m|c(a-b)$,这等价于
    \begin{align}
        \frac{m}{(c,m)}\bigg|\frac{c}{(c,m)}(a-b)\, .
    \end{align}
    由定理\ref{theorem:7.ex02.16}以及$\displaystyle\left(\frac{m}{(c,m)},\frac{c}{(c,m)}\right)=1$知
    这等价于
    \begin{align}
        \frac{m}{(c,m)}\bigg|a-b\, .
    \end{align}
    就证明了该结论。
\end{prove}
\begin{theorem}\label{theorem:7.ex02.24}
    若$m\ge1$,$(a,m)=1$,则存在$c$使得
    \begin{align}\label{eq:7.ex02.modularinverse}
        ca\equiv1\pmod{m}\, .
    \end{align}
    我们把$c$称作$a$对模$m$的逆,或\keyindex{模逆元}{modular multiplicative inverse}{},
    记作$a^{-1}\pmod{m}$或$a^{-1}$.
\end{theorem}
\begin{prove}
    由定理\ref{theorem:7.ex02.17}知,存在$x_0,y_0$,
    使得$ax_0+my_0=1$,取$c=x_0$即满足要求。
\end{prove}
$a$对模$m$的逆不是唯一的。若$c$是$a$对模$m$的逆,
则任一$\bar{c}\equiv c\pmod{m}$也必是$a$对模$m$的逆;
$a$对模$m$的任意两个逆$c_1,c_2$必有$c_1\equiv c_2\pmod{m}$;
若$(a,m)=1$,则$(a^{-1},m)=1$,及$(a^{-1})^{-1}\equiv a\pmod{m}$.
\begin{notation}
    下文中约定$a^{-1}\pmod{m}$或$a^{-1}$指
    任一取定的满足\refeq{7.ex02.modularinverse}的$c$.
\end{notation}

\begin{example}
    $a$对模7的逆(只列出了一个值):
    \begin{table}[htbp]
        \centering
        \begin{tabular}{c|cccccc}
            \toprule
            $a$              & 1 & 2 & 3 & 4 & 5 & 6 \\
            \midrule
            $a^{-1}\pmod{7}$ & 1 & 4 & 5 & 2 & 3 & 6 \\
            \bottomrule
        \end{tabular}
        \caption{$a$对模7的逆(只列出了一个值)。}
        \label{tab:7.ex02.modularinverse}
    \end{table}
\end{example}

\begin{theorem}\label{theorem:7.ex02.25}
    同余式组
    \begin{align}
        a\equiv b\pmod{m_j}\, \quad j=1,2,\ldots,k
    \end{align}
    同时成立的充要条件是
    \begin{align}
        a\equiv b\pmod{[m_1,\ldots,m_k]}\, .
    \end{align}
\end{theorem}
\begin{prove}
    由定理\ref{theorem:7.ex02.11}知,$m_j|a-b(j=1,2,\ldots,k)$同时成立的
    充要条件是$[m_1,\ldots,m_k]|a-b$,得证。
\end{prove}
\begin{definition}[同余类(剩余类)]
    由定理\ref{theorem:7.ex02.20}知,对于给定的模$m$,
    整数的同余关系是一个等价关系,因此全体整数可按对模$m$是否同余
    分为若干个两两不相交的集合,使得在同一个集合中的任意两个数
    对模$m$一定同余,而属于不同集合中的两个数对模$m$一定不同余。
    每一个这样的集合称为是模$m$的\keyindex{同余类}{congruence class}{},
    或模$m$的\keyindex{剩余类}{residue class}{}。
    我们把$r$所属的模$m$的同余类表示为$r\mod{m}$.
\end{definition}

\begin{theorem}\label{theorem:7.ex02.26}
    同余类具有以下性质:
    \begin{enumerate}
        \item $r\mod{m}=\{r+km:k\in\mathbb{Z}\}$;
        \item $r\mod{m}=s\mod{m}\Leftrightarrow r\equiv s\pmod{m}$;
        \item 对任意的$r,s$,要么$r\mod{m}=s\mod{m}$,要么$r\mod{m}$与$s\mod{m}$的交集为空集。
    \end{enumerate}
\end{theorem}
\begin{theorem}\label{theorem:7.ex02.27}
    对于给定的模$m$,有且恰有$m$个不同的模$m$的同余类,即
    \begin{align}\label{eq:7.ex02.theorem27-02}
        0\mod{m},\quad 1\mod{m},\quad \ldots,\quad (m-1)\mod{m}\, .
    \end{align}
\end{theorem}
\begin{prove}
    由定理\ref{theorem:7.ex02.26}(2)知这是$m$个两两不同的同余类。
    对每个整数$a$,由定理\ref{theorem:7.ex02.4}知
    \begin{align}
        a=qm+r,\quad 0\le r<m\, .
    \end{align}
    故由定理\ref{theorem:7.ex02.26}(1)知,$a\in r\mod{m}$,
    即必属于\refeq{7.ex02.theorem27-02}中的某个同余类。
\end{prove}
\begin{theorem}\label{theorem:7.ex02.28}
    同余具有以下性质:
    \begin{enumerate}
        \item 在任意取定的$m+1$个整数中,必有两个数对模$m$同余;
        \item 存在$m$个数两两对模$m$不同余。
    \end{enumerate}
\end{theorem}
\begin{prove}
    由定理\ref{theorem:7.ex02.27},对模$m$共有$m$个由\refeq{7.ex02.theorem27-02}给出的同余类,
    所以根据定理\ref{theorem:7.ex02.2},$m+1$个数中必有两个数属于同一个模$m$的同余类,
    这两个数就对模$m$同余,第一个结论得证。在每个同余类$r\mod{m}(0\le r<m)$中
    取定一个数$x_r$作代表,就得到$m$个两两对模$m$不同余的数$x_0,x_1,\ldots,x_{m-1}$,第二个结论得证。
\end{prove}

由定理\ref{theorem:7.ex02.28}可引进以下概念:
\begin{definition}
    一组数$y_1,\ldots,y_s$称为是模$m$的\keyindex{完全剩余系}{complete residue system}{},
    如果对任意的$a$有且仅有一个$y_j$满足$a\equiv y_j\pmod{m}$.
\end{definition}

\subsection{同余方程}\label{sub:同余方程}
\begin{definition}
    设整系数多项式
    \begin{align}
        f(x)=a_nx^n+\cdots+a_1x+a_0\, ,
    \end{align}
    我们称含有变量$x$的同余式
    \begin{align}\label{eq:7.ex02.congruenceequation}
        f(x)\equiv0\pmod{m}
    \end{align}
    为模$m$的\keyindex{同余方程}{congruence equation}{equation方程}。
    若整数$c$满足
    \begin{align}
        f(c)\equiv0\pmod{m}\, ,
    \end{align}
    则称$c$是同余方程\refeq{7.ex02.congruenceequation}的\keyindex{解}{solution}{}。
\end{definition}

在上述定义中,显然同余类$c\mod{m}$中的任一整数也是
同余方程\refeq{7.ex02.congruenceequation}的解。
我们把这些解都看作是相同的,也常说同余类$c\mod{m}$是该方程的解,
写为$x\equiv c\pmod{m}$.当$c_1,c_2$均为该同余方程的解且对模$m$不同余时
才把它们看作是不同的解。我们把所有对模$m$两两不同余的解的个数
称为是同余方程\refeq{7.ex02.congruenceequation}
的\keyindex{解数}{number of solutions}{}。
因此我们只需要在模$m$的一组完全剩余系中来解模$m$的同余方程。
显然模$m$的同余方程的解数至多为$m$.
\begin{example}
    对于同余方程$4x^2+27x-12\equiv0\pmod{15}$,
    取模15的一个完全剩余系$-7,-6,\ldots,-1,0,1,\ldots,6,7$,
    直接代入验算知$x=-6,3$是解,所以该同余方程的解
    是$x\equiv -6,3\pmod{15}$,解数为2.
\end{example}
\begin{definition}
    设$m\nmid a$,称
    \begin{align}\label{eq:7.ex02.linearcongruence}
        ax\equiv b\pmod{m}
    \end{align}
    为模$m$的{\sffamily 一次同余方程}。
\end{definition}
\begin{example}
    同余方程$6x\equiv2\pmod{8}$的解是$x\equiv-1,3\pmod{8}$,解数为2.
\end{example}
\begin{theorem}\label{theorem:7.ex02.29}
    当$(a,m)=1$时,同余方程\refeq{7.ex02.linearcongruence}必有解,且其解数为1.
\end{theorem}
\begin{prove}
    当$(a,m)=1$时,由定理\ref{theorem:7.ex02.24}知,
    $a$对模$m$有逆$a^{-1}$(任取一个)满足
    \begin{align}
        aa^{-1}\equiv1\pmod{m}\, .
    \end{align}
    容易看出
    \begin{align}
        x_1=a^{-1}b
    \end{align}
    就满足同余方程\refeq{7.ex02.linearcongruence}。若还有解$x_2$,则有
    \begin{align}
        ax_2\equiv ax_1\pmod{m}\, ,
    \end{align}
    由此根据定理\ref{theorem:7.ex02.23}得
    \begin{align}
        x_2\equiv x_1\pmod{m}\, .
    \end{align}
    这就证明了解数为1.
\end{prove}
% \begin{theorem}
%     同余方程\refeq{7.ex02.linearcongruence}有解的充要条件是
%     \begin{align}\label{eq:7.ex02.conditionsolution}
%         (a,m)|b\, .
%     \end{align}
%     在有解时,其解数等于$(a,m)$;若$x_0$是它的解,则它的$(a,m)$个解是
%     \begin{align}
%         x\equiv x_0+\frac{m}{(a,m)}t\pmod{m},\quad t=0,\ldots,(a,m)-1\, .
%     \end{align}
% \end{theorem}
% \begin{theorem}\label{theorem:7.ex02.hassolutionlinear}
%     当$(a,m)=1$时,同余方程\refeq{7.ex02.linearcongruence}必有解,且其解数为1.
% \end{theorem}
% \begin{prove}
%     当$(a,m)=1$时,由定理\ref{theorem:7.ex02.modularinverse}知,
%     $a$对模$m$有逆$a^{-1}$(任取一个)满足
%     \begin{align}
%         aa^{-1}\equiv1\pmod{m}\, .
%     \end{align}
%     容易看出
%     \begin{align}
%         x_1=a^{-1}b
%     \end{align}
%     就满足同余方程\refeq{7.ex02.linearcongruence}。若还有解$x_2$,则有
%     \begin{align}
%         ax_2\equiv ax_1\pmod{m}\, ,
%     \end{align}
%     则从定理\ref{theorem:7.ex02.congruentreduce}推出
%     \begin{align}
%         x_2\equiv x_1\pmod{m}\, .
%     \end{align}
%     这就证明了解数为1.
% \end{prove}
% \begin{theorem}
%     同余方程\refeq{7.ex02.linearcongruence}有解的充要条件
%     是\refeq{7.ex02.conditionsolution}成立。在有解时,
%     它的解数等于$(a,m)$,以及若$x_0$是\refeq{7.ex02.linearcongruence}的解,
%     则它的$(a,m)$个解是
%     \begin{align}
%         x\equiv x_0+\frac{m}{(a,m)}t\pmod{m},\quad t=0,1,\ldots,(a,m)-1\, .
%     \end{align}
% \end{theorem}

\begin{definition}
    设$f_j(x), j=1,2,\ldots,k$是整系数多项式,我们把含有变量$x$的一组同余式
    \begin{align}\label{eq:7.ex02.congruencegroup}
        f_j(x)\equiv0\pmod{m_j},\quad 1\le j\le k\, ,
    \end{align}
    称为{\sffamily 同余方程组}。若整数$c$同时满足
    \begin{align}
        f_j(c)\equiv0\pmod{m_j},\quad 1\le j\le k\, ,
    \end{align}
    则称$c$是同余方程组\refeq{7.ex02.congruencegroup}的\keyindex{解}{solution}{}。
\end{definition}

显然在上述定义中,同余类
\begin{align}\label{eq:7.ex02.groupsolution}
    c\mod{m},\quad m=[m_1,\ldots,m_k]
\end{align}
中任一整数也是同余方程组\refeq{7.ex02.congruencegroup}的解,
我们把它们看作是相同的,也常说同余类\refeq{7.ex02.groupsolution}是
该同余方程组的一个解,写作$x\equiv c\pmod{m}$.
当$c_1,c_2$均为该同余方程组的解且对模$m$不同余时
才把它们看作是不同的解。我们把所有对模$m$两两不同余的解的个数
称为是同余方程组\refeq{7.ex02.congruencegroup}的\keyindex{解数}{number of solutions}{}。
因此我们只需要在模$m$的一组完全剩余系中来解该同余方程组,
它的解数至多为$m$.此外,只要同余方程组中任一一个方程无解,
则\refeq{7.ex02.congruencegroup}一定无解。
\begin{theorem}[\protect\keyindex{中国剩余定理}{Chinese remainder theorem}{}(CRT)]\label{theorem:7.ex02.30}
    也称{\sffamily 孙子定理}:设$m_1,\ldots,m_k$是两两互质的正整数,
    则对任意整数$a_1,\ldots,a_k$,一次同余方程组
    \begin{align}\label{eq:7.ex02.CRT}
        x\equiv a_j\pmod{m_j},\quad 1\le j\le k\, ,
    \end{align}
    必有解,且解数为1.事实上,该同余方程组的解是
    \begin{align}
        x\equiv M_1M_1^{-1}a_1+\ldots+M_kM_k^{-1}a_k\pmod{m}\, ,
    \end{align}
    这里$m=m_1m_2\cdots m_k$,$m=m_jM_j(1\le j\le k)$,以及$M_j^{-1}$是满足
    \begin{align}
        M_jM_j^{-1}\equiv1\pmod{m_j},\quad 1\le j\le k
    \end{align}
    的一个整数(即是$M_j$对模$m_j$的逆)。
\end{theorem}
\begin{prove}
    首先指出一个事实:若$x_0$满足同余方程组\refeq{7.ex02.CRT},
    且$x_0'$满足下面的另一同余方程组
    \begin{align}
        x\equiv a_j'\pmod{m_j},\quad 1\le j\le k\, ,
    \end{align}
    则$x_0+x_0'$一定是同余方程组
    \begin{align}
        x\equiv a_j+a_j'\pmod{m_j},\quad 1\le j\le k
    \end{align}
    的解。因此,我们可用下面的叠加方法来求同余方程组\refeq{7.ex02.CRT}的解。设
    \begin{align}\label{eq:7.ex02.proveCRT03}
        a_j^{(i)}=\left\{\begin{array}{ll}
            a_j, & \text{若}i=j\, ,     \\
            0,   & \text{若}i\neq j\, .
        \end{array}\right.
    \end{align}
    对每个固定的$i(1\le j\le k)$考虑同余方程组
    \begin{align}\label{eq:7.ex02.proveCRT01}
        x\equiv a_j^{(i)}\pmod{m_j},\quad 1\le j\le k\, .
    \end{align}
    注意到$j\neq i$时$a_j^{(i)}=0$,结合$m_j$两两互质,
    由这个方程组的第$1,\ldots,i-1,i+1,\ldots,k$个方程知
    \begin{align}
        x\equiv0\pmod{M_i}\, ,
    \end{align}
    即存在整数$y$使得
    \begin{align}\label{eq:7.ex02.proveCRT02}
        x=M_iy\, .
    \end{align}
    代入第$i$个方程得
    \begin{align}
        M_iy\equiv a_i\pmod{m_i}\, .
    \end{align}
    由定理\ref{theorem:7.ex02.29}的证明知
    \begin{align}
        y\equiv M_i^{-1}a_i\pmod{m_i}\, ,
    \end{align}
    即
    \begin{align}
        M_iy\equiv M_iM_i^{-1}a_i\pmod{m}\, .
    \end{align}
    由此及\refeq{7.ex02.proveCRT02}得
    \begin{align}
        x\equiv M_iM_i^{-1}a_i\pmod{m}\, .
    \end{align}
    容易验证,$M_iM_i^{-1}a_i$确是同余方程组\refeq{7.ex02.proveCRT01}的解,
    且由定理\ref{theorem:7.ex02.29}知解数为1.
    注意到由\refeq{7.ex02.proveCRT03}可得
    \begin{align}
        a_j^{(1)}+a_j^{(2)}+\cdots+a_j^{(k)}=a_j\, ,
    \end{align}
    所以$M_1M_1^{-1}a_1+\cdots+M_kM_k^{-1}a_k$一定是同余方程组\refeq{7.ex02.CRT}的解。
    若$c_1,c_2$均是同余方程组\refeq{7.ex02.CRT}的解,
    则必有
    \begin{align}
        c_1\equiv c_2\pmod{m_j},\quad 1\le j\le k\, .
    \end{align}
    又因为$m_1,\ldots,m_k$两两互质,所以
    \begin{align}
        m=m_1m_2\cdots m_k=[m_1,\ldots,m_k]\, .
    \end{align}
    利用定理\ref{theorem:7.ex02.25}结合上两式可得
    \begin{align}
        c_1\equiv c_2\pmod{m}\, ,
    \end{align}
    即同余方程组\refeq{7.ex02.CRT}的解数必为1.
\end{prove}
\begin{example}
    解同余方程组
    \begin{align}
        \left\{
        \begin{array}{l}
            x\equiv1\pmod{3}\, ,  \\
            x\equiv-1\pmod{5}\, , \\
            x\equiv2\pmod{7}\, ,  \\
            x\equiv-2\pmod{11}\, .
        \end{array}
        \right.
    \end{align}
    {\sffamily 解}\quad 取$m_1=3,m_2=5,m_3=7,m_4=11$,
    满足定理\ref{theorem:7.ex02.30}的条件。这时,
    $M_1=5\times7\times11,M_2=3\times7\times11,M_3=3\times5\times11,M_4=3\times5\times7$.
    由于$M_1\equiv(-1)\times1\times(-1)\equiv1\pmod{3}$,所以令
    \begin{align}
        1\equiv M_1M_1^{-1}\equiv M_1^{-1}\pmod{3}\, ,
    \end{align}
    因此可取$M_1^{-1}=1$.由于$M_2\equiv(-2)\times2\times1\equiv1\pmod{5}$,令
    \begin{align}
        1\equiv M_2M_2^{-1}\equiv M_2^{-1}\pmod{5}\, ,
    \end{align}
    因此可取$M_2^{-1}=1$.由于$M_3\equiv3\times5\times4\equiv4\pmod{7}$,令
    \begin{align}
        1\equiv M_3M_3^{-1}\equiv4M_3^{-1}\pmod{7}\, ,
    \end{align}
    因此可取$M_3^{-1}=2$.由$M_4\equiv3\times5\times7\equiv4\times7\equiv6\pmod{11}$,令
    \begin{align}
        1\equiv M_4M_4^{-1}\equiv6M_4^{-1}\pmod{11}\, ,
    \end{align}
    因此可取$M_4^{-1}=2$.进而由定理\ref{theorem:7.ex02.30}知
    同余方程组的解为
    \begin{align}
        x & \equiv(5\times7\times11)\times1\times1+(3\times7\times11)\times1\times(-1)\nonumber                               \\
          & \qquad+(3\times5\times11)\times2\times2+(3\times5\times7)\times2\times(-2)\pmod{3\times5\times7\times11}\nonumber \\
          & \equiv385-231+660-420\equiv394\pmod{1155}\, .
    \end{align}
\end{example}

至此我们介绍了Halton样本生成中求解一次同余方程组所需的全部背景知识。
代码片{\refcode{Compute Halton sample offset for currentPixel}{}}中
即采用了定理\ref{theorem:7.ex02.30}:其中{\ttfamily dimOffset}对应了$a_j$,
{\ttfamily\refvar{sampleStride}{} / \refvar{baseScales}{}}对应了$M_j$,
{\refvar{multInverse}{}}对应了$M_j^{-1}$,它由扩展欧几里得法求得,代码为
\begin{lstlisting}
static uint64_t `\initvar{multiplicativeInverse}{}`(int64_t a, int64_t n) {
    int64_t x, y;
    `\refvar{extendedGCD}{}`(a, n, &x, &y);
    return `\refvar{Mod}{}`(x, n);
}
\end{lstlisting}
\begin{lstlisting}
static void `\initvar{extendedGCD}{}`(uint64_t a, uint64_t b, int64_t *x, int64_t *y) {
    if (b == 0) {
        *x = 1;
        *y = 0;
        return;
    }
    int64_t d = a / b, xp, yp;
    `\refvar{extendedGCD}{}`(b, a % b, &xp, &yp);
    *x = yp;
    *y = xp - (d * yp);
}
\end{lstlisting}

