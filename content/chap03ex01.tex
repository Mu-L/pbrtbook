\section{译者补充:微分几何基础}\label{sec:译者补充:微分几何基础}
\begin{remark}
    本节内容不是原书内容,而是译者参照教材补充的,请酌情参考和斧正。
\end{remark}

\subsection{曲线的概念}\label{sub:曲线的概念}
\begin{definition}
    给出两个集合$E$和$E'$,如果集合$E$中的每一个点(或称元素)$x$,
    有$E'$中的点$x'$和它对应,
    则我们说给定了$E$到$E'$的一个\keyindex{映射}{mapping}{}$f$.
    $x'$称为点$x$的\keyindex{象}{image}{},
    $x$称为$x'$的\keyindex{原象}{inverse image}{}。
\end{definition}
\begin{definition}
    对于任取集合$E$中的点$x_1$和$x_2$,若$x_1\neq x_2$时必有$f(x_1)\neq f(x_2)$,
    则称映射$f$是\keyindex{一一映射}{one-one mapping}{mapping映射}或\keyindex{单射}{injection}{}。
\end{definition}
\begin{definition}
    在欧氏空间中给出两个集合$E,E'$,
    对于$E$中任一个点$x_0$和任一个数$\varepsilon>0$,
    存在数$\delta>0$,使得对于$E$中与$x_0$的距离小于$\delta$的任意一点$x$来说,
    点$f(x)$与$f(x_0)$间的距离小于$\varepsilon$,
    则称映射$f$是\keyindex{连续的}{continuous}{}。
\end{definition}
\begin{definition}
    如果$f(E)=E'$,则称$f$是从$E$到$E'$的\keyindex{到上映射}{onto mapping}{mapping映射}
    或\keyindex{满射}{surjection}{mapping映射}。
\end{definition}
\begin{definition}
    如果一个开的直线段到三维欧氏空间内建立的对应$f$是一一的、双方连续的到上映射
    (这种映射称为\keyindex{拓扑映射}{topological mapping}{mapping映射}或\keyindex{同态映射}{homeomorphic mapping}{mapping映射}),
    则我们把三维欧氏空间中的映射的象称为\keyindex{简单曲线段}{%simple curve segment
    }{curve曲线}。
\end{definition}

我们以后所讨论的曲线都是简单曲线段,不另做声明。

我们可以确立曲线的方程。在直线段上引入坐标$t(a<t<b)$,
在空间中引入笛卡尔直角坐标$(x,y,z)$,
则上述映射的解析表达式是
\begin{align}\label{eq:03ex01.1}
    \left\{\begin{array}{c}
        x=f(t), \\
        y=g(t), \\
        z=h(t),
    \end{array}
    \right.\quad a<t<b\, .
\end{align}
习惯上常把\refeq{03ex01.1}中的函数关系符号$f,g,h$分别
写作$x,y,z$,于是\refeq{03ex01.1}可写为
\begin{align}\label{eq:03ex01.2}
    \left\{\begin{array}{c}
        x=x(t), \\
        y=y(t), \\
        z=z(t),
    \end{array}
    \right.\quad a<t<b\, .
\end{align}
\refeq{03ex01.2}称为曲线的\keyindex{参数表示}{parametric representation}{}或\keyindex{参数方程}{parametric equation}{},
$t$称为曲线的\keyindex{参数}{parameter}{}。

由于向量函数$\bm r(t)$可表示为$\bm r(t)=x(t)\mathbf{i}+y(t)\mathbf{j}+z(t)\mathbf{k}$,
因而曲线的参数方程\refeq{03ex01.2}也可以写成向量函数的形式:
\begin{align}\label{eq:03ex01.3}
    \bm r=\bm r(t),\quad a<t<b\, .
\end{align}
即在空间中给定一点$O$,以该点作为始点放上对于$t$的所有值的向量$\bm r(t)$.
于是对于$t$的每个值,我们得到确定的向量$\overrightarrow{OM}=\bm r(t)$,
它的始点是点$O$,而终点$M$则与$t$值有关,当$t$在$(a,b)$内变化时,
点$M$在空间中画出一条轨迹,这就是由参数$t$所给定的曲线。
点$M$的向量表达式称为曲线的\keyindex{向量参数表示}{}{parametric representation参数表示}。

\begin{definition}
    如果曲线的参数表示式中的函数
    是\keyindex{$k$阶连续可微}{$k$-times continuously differentiable}{}的函数,
    则把该曲线称为\keyindex{$C^k$阶曲线}{$C^k$-curve}{curve曲线}。
    当$k=1$时,也就是$C^1$阶曲线
    又称为\keyindex{光滑曲线}{smooth curve}{curve曲线}。
\end{definition}

\begin{definition}
    给出$C^1$类的曲线$\bm r=\bm r(t)$,假设对于该曲线上一点($t=t_0$)有
    \begin{align}\label{eq:03ex01.4}
        \bm r'(t_0)\neq \bm 0\, ,
    \end{align}
    则这一点称为曲线的\keyindex{正则点}{regular point}{point点}。
    注意\refeq{03ex01.4}表示$x'(t_0),y'(t_0),z'(t_0)$中至少有一个不等于零。
\end{definition}

以后我们只考虑曲线的正则点。实际上$\bm r'(t_0)=\bm 0$是很特殊的。
如果在一段曲线上$\bm r'(t_0)\equiv\bm 0$,则$\bm r(t)$变成常向量,
这时这段曲线缩成一点,所以一段曲线上$\bm r'(t_0)=\bm 0$的点一般是孤立点。
\begin{definition}
    曲线上所有点都是正则点时,称该曲线为\keyindex{正则曲线}{regular curve}{curve曲线}。
\end{definition}

\begin{definition}
    给出曲线上一点$P$,点$Q$是曲线上$P$的邻近一点,
    使点$Q$沿曲线趋近于点$P$,若\keyindex{割线}{secant line}{}$PQ$趋近于特定位置,
    则把$PQ$的该极限位置称为曲线在$P$点的\keyindex{切线}{tangent line}{},
    定点$P$称为\keyindex{切点}{tangent point}{}。
\end{definition}

\begin{definition}
    若曲线$\bm r=\bm r(t)$上的点$P$对应参数$t_0$且$\bm r(t)$在$t_0$处可微,则
    \begin{align}\label{eq:03ex01.5}
        \bm r'(t_0)=\lim\limits_{\Delta t\rightarrow0}{\frac{\bm r(t_0+\Delta t)-\bm r(t_0)}{\Delta t}}
    \end{align}
    称为曲线在点$P$的\keyindex{切向量}{tangent vector}{vector向量}。
\end{definition}

由于我们已经规定只研究曲线的正则点,所以曲线上一点的切向量是存在的,
它就是切线上的一个非零向量,其正向和曲线参数$t$的增量方向一致。

\begin{corollary}
    曲线在参数$t_0$处的切线方程是
    \begin{align}\label{eq:03ex01.6}
        \frac{X-x(t_0)}{x'(t_0)}=\frac{Y-y(t_0)}{y'(t_0)}=\frac{Z-z(t_0)}{z'(t_0)}\, .
    \end{align}
\end{corollary}

\begin{definition}
    过切点垂直于切线的平面称为
    曲线的\keyindex{法平面}{normal plane}{}。
\end{definition}

\begin{corollary}
    曲线在参数$t_0$处的法面方程是
    \begin{align}\label{eq:03ex01.7}
        x'(t_0)(X-x(t_0))+y'(t_0)(Y-y(t_0))+z'(t_0)(Z-z(t_0))=0\, .
    \end{align}
\end{corollary}

\begin{corollary}
    曲线$\bm r=\bm r(t)$中从$\bm r(a)$到$\bm r(t)$的有向弧长是
    \begin{align}\label{eq:03ex01.8}
        \sigma(t)=\int_a^t{|\bm r'(t)|\mathrm{d}t}\, .
    \end{align}
\end{corollary}

\subsection{曲面的概念}\label{sub:曲面的概念}
\begin{definition}
    平面上不自交的简单\keyindex{闭曲线}{closed curve}{curve曲线}称为\keyindex{Jordan曲线}{Jordan curve}{curve曲线}。
    它将平面分为两个都以该曲线为边界的部分,其中一个是有限的,另一个是无限的;
    有限的区域称为\keyindex{初等区域}{}{curve曲线},即Jordan曲线的内部。
\end{definition}
\begin{example}
    正方形或矩形内部,圆或椭圆内部都是初等区域。
\end{example}

\begin{definition}
    如果平面上初等区域到三维欧氏空间内建立的映射是一一的、双方连续的到上映射,
    则把三维欧氏空间中的象称为\keyindex{简单曲面}{%simple surface
    }{surface曲面}。
\end{definition}
\begin{example}
    矩形纸片(初等区域)卷成的带有裂缝的圆柱面是简单曲面。
\end{example}

我们假定以后所讨论的曲面都是简单曲面,不另作说明。

给出平面上一初等区域$\mathscr{D}$,$\mathscr{D}$中的点的笛卡尔坐标是$(u,v)$,
$\mathscr{D}$经过上述映射$f$后的象是曲面$S$.
对于空间的笛卡尔坐标系来说,$S$上的点的坐标是$(x,y,z)$,
则可以写出$f$的解析表达式:
\begin{align}\label{eq:03ex01.9}
    \begin{array}{l}
        x=f_1(u,v)\, , \\
        y=f_2(u,v)\, , \\
        z=f_3(u,v)\, ,
    \end{array}\quad (u,v)\in \mathscr{D}\, .
\end{align}
称\refeq{03ex01.9}为曲面$S$的\keyindex{参数表示}{parametric representation}{}或\keyindex{参数方程}{parametric equation}{},
$u$和$v$称为曲面的\keyindex{参数}{parameter}{}或\keyindex{曲纹坐标}{curve coordinate}{coordinate坐标}。

习惯上常把\refeq{03ex01.9}中的函数关系符号$f_1,f_2$和$f_3$分别写成$x,y$和$z$,即
\begin{align}\label{eq:03ex01.10}
    \begin{array}{l}
        x=x(u,v)\, , \\
        y=y(u,v)\, , \\
        z=z(u,v)\, ,
    \end{array}\quad (u,v)\in \mathscr{D}\, .
\end{align}
有时也将其简写称向量函数的形式:
\begin{align}\label{eq:03ex01.11}
    \bm r=\bm r(u,v),\quad (u,v)\in \mathscr{D}\, .
\end{align}

\begin{definition}
    初等区域$\mathscr{D}$所在平面上的坐标直线$v=$常数或$u=$常数
    在曲面上的象称为曲面的\keyindex{坐标曲线}{coordinate curve}{}。
    使$v$等于常数$v_0$而$u$变动时的曲线$\bm r=\bm r(u,v_0)$叫$u$-曲线;
    使$u$等于常数$u_0$而$v$变动时的曲线$\bm r=\bm r(u_0,v)$叫$v$-曲线。
    这两族坐标曲线在曲面上构成的坐标网称为曲面上的\keyindex{曲纹坐标网}{curve coordinate net}{}。
\end{definition}

\begin{definition}
    若曲面方程中的函数有直到$k$阶的连续\keyindex{偏微商}{partial derivative}{},
    则该曲面称为\keyindex{$k$阶正则曲面}{}{surface曲面}或\keyindex{$C^k$阶曲面}{$C^k$-surface}{surface曲面}。
    特别地,$C^1$阶曲面又称为\keyindex{光滑曲面}{}{surface曲面}。
\end{definition}

以后我们假定所讨论的曲面都是光滑的。

\begin{definition}
    曲面$\bm r=\bm r(u,v)$上$(u_0,v_0)$点处两条坐标曲线的切向量分别为
    \begin{align}\label{eq:03ex01.12}
        \bm r_u(u_0,v_0) & =\frac{\partial \bm r}{\partial u}(u_0,v_0)\, , \\
        \bm r_v(u_0,v_0) & =\frac{\partial \bm r}{\partial v}(u_0,v_0)\, .
    \end{align}
    若它们不平行,即$\bm r_u\times\bm r_v$在$(u_0,v_0)$点不等于$\bm 0$,
    则称该点为曲面的\keyindex{正则点}{regular point}{point点}。
\end{definition}

以后我们只讨论曲面的正则点。

若曲面上点的曲纹坐标由下列方程确定:
\begin{align}\label{eq:03ex01.13}
    u & =u(t)\, , \\
    v & =v(t)\, ,
\end{align}
其中$t$是自变量,代入曲面的参数方程可得
该点的\keyindex{向径}{radius vector}{vector向量}为
\begin{align}\label{eq:03ex01.14}
    \bm r=\bm r\left(u(t),v(t)\right)=\bm r(t)\, .
\end{align}
当$t$在某区间上变动时,
关于$t$的函数$\bm r$相应的终点在曲面上确定了某一曲线,
该曲线在曲面上$(u_0,v_0)$点处的切方向称为
曲面在该点的\keyindex{切方向}{tangent direction}{direction方向}或\keyindex{方向}{direction}{}。
它平行于
\begin{align}\label{eq:03ex01.15}
    \bm r'(t)=\bm r_u\frac{\mathrm{d}u}{\mathrm{d}t}+\bm r_v\frac{\mathrm{d}v}{\mathrm{d}t}\, ,
\end{align}
其中$\bm r_u$和$\bm r_v$分别是在该点的两条坐标曲线的切向量。
$\bm r'(t),\bm r_u$和$\bm r_v$共面。
\begin{definition}
    曲面上正则点的所有切方向都在
    过该点的坐标曲线的切向量$\bm r_u$和$\bm r_v$所决定的平面上,
    该平面称为曲面在该点的\keyindex{切平面}{tangent plane}{}。
\end{definition}
\begin{corollary}
    曲面$\bm r=\bm r(u,v)$在点$P_0(u_0,v_0)$处的切平面方程为
    \begin{align}\label{eq:03ex01.16}
        \left|
        \begin{array}{ccc}
            X-x(u_0,v_0) & Y-y(u_0,v_0) & Z-z(u_0,v_0) \\
            x_u(u_0,v_0) & y_u(u_0,v_0) & z_u(u_0,v_0) \\
            x_v(u_0,v_0) & y_v(u_0,v_0) & z_v(u_0,v_0)
        \end{array}\right|=0\, .
    \end{align}
\end{corollary}

\begin{definition}
    曲线在正则点处垂直于切平面的方向称为曲面的\keyindex{法方向}{normal direction}{direction方向}。
    过该点平行于法方向的直线称为曲面在该点的\keyindex{法线}{normal}{}。
\end{definition}
\begin{corollary}
    曲面$\bm r=\bm r(u,v)$的法向量为$\displaystyle\bm N=\bm r_u\times\bm r_v$,
    单位法向量$\bm n=\displaystyle\frac{\bm r_u\times\bm r_v}{|\bm r_u\times\bm r_v|}$.
\end{corollary}
\begin{corollary}
    曲面$\bm r=\bm r(u,v)$在点$P_0(u_0,v_0)$处的法线方程为
    \begin{align}\label{eq:03ex01.17}
        \frac{X-x(u_0,v_0)}{\left|
            \begin{array}{cc}
                y_u(u_0,v_0) & z_u(u_0,v_0) \\
                y_v(u_0,v_0) & z_v(u_0,v_0)
            \end{array}
            \right|}=\frac{Y-y(u_0,v_0)}{\left|
            \begin{array}{cc}
                z_u(u_0,v_0) & x_u(u_0,v_0) \\
                z_v(u_0,v_0) & x_v(u_0,v_0)
            \end{array}
            \right|}=\frac{Z-z(u_0,v_0)}{\left|
            \begin{array}{cc}
                x_u(u_0,v_0) & y_u(u_0,v_0) \\
                x_v(u_0,v_0) & y_v(u_0,v_0)
            \end{array}
            \right|}\, .
    \end{align}
\end{corollary}

\begin{definition}
    \keyindex{曲面的第一基本形式}{first fundamental form of a surface}{}为
    \begin{align}\label{eq:03ex01.18}
        \uppercase\expandafter{\romannumeral1}=\mathrm{d}\bm r^2=E\mathrm{d}u^2+2F\mathrm{d}u\mathrm{d}v+G\mathrm{d}v^2\, .
    \end{align}
    其中系数
    \begin{align}\label{eq:03ex01.19}
        E=\bm r_u\cdot\bm r_u,\quad F=\bm r_u\cdot\bm r_v,\quad G=\bm r_v\cdot\bm r_v
    \end{align}
    称为\keyindex{曲面的第一类基本量}{fundamental quantities of first kind for surfaces}{}。
\end{definition}

\begin{corollary}
    曲面的第一基本形式是正定的,即
    \begin{align}\label{eq:03ex01.20}
        E>0,\quad G>0,\quad EG-F^2>0\, .
    \end{align}
\end{corollary}

\begin{corollary}
    曲面的第一基本形式决定曲面上曲线的弧长。
    设曲面上某曲线$C$为$u=u(t),v=v(t)$,
    则其上两点$A(t_0),B(t_1)$沿$C$的有向弧长为
    \begin{align}\label{eq:03ex01.21}
        s=\int_{t_0}^{t_1}{\frac{\mathrm{d}s}{\mathrm{d}t}\mathrm{d}t}=\int_{t_0}^{t_1}
        {\sqrt{E\left(\frac{\mathrm{d}u}{\mathrm{d}t}\right)^2+
            2F\frac{\mathrm{d}u}{\mathrm{d}t}\frac{\mathrm{d}v}{\mathrm{d}t}+
            G\left(\frac{\mathrm{d}v}{\mathrm{d}t}\right)^2}
        \mathrm{d}t}\, .
    \end{align}
\end{corollary}

\begin{corollary}
    曲面的第一基本形式决定曲面的面积。
    曲面$\bm r=\bm r(u,v)$上的曲面域$D$对应的$(u,v)$平面区域为$\mathscr{D}$,
    则$D$是面积为
    \begin{align}\label{eq:03ex01.22}
        \sigma=\iint\limits_{\mathscr{D}}{|\bm r_u\times\bm r_v|\mathrm{d}u\mathrm{d}v}=\iint\limits_{\mathscr{D}}{\sqrt{EG-F^2}\mathrm{d}u\mathrm{d}v}\, .
    \end{align}
\end{corollary}

\begin{definition}
    $C^2$阶曲面中$\bm r(u,v)$有连续的二阶导函数$\bm r_{uu},\bm r_{uv},\bm r_{vv}$,
    单位法向量为$\bm n$,则该\keyindex{曲面的第二基本形式}{second fundamental form of a surface}{}为
    \begin{align}\label{eq:03ex01.23}
        \uppercase\expandafter{\romannumeral2}=\bm n\cdot\mathrm{d}^2\bm r=-\mathrm{d}\bm n\cdot\mathrm{d}\bm r=L\mathrm{d}u^2+2M\mathrm{d}u\mathrm{d}v+N\mathrm{d}v^2\, ,
    \end{align}
    其中系数
    \begin{align}\label{eq:03ex01.24}
        L & =\bm r_{uu}\cdot\bm n=-\bm r_u\cdot\bm n_u,                      \\
        M & =\bm r_{uv}\cdot\bm n=-\bm r_u\cdot\bm n_v=-\bm r_v\cdot\bm n_u, \\
        N & =\bm r_{vv}\cdot\bm n=-\bm r_v\cdot\bm n_v\,
    \end{align}
    称为\keyindex{曲面的第二类基本量}{fundamental quantities of second kind for surfaces}{}。
\end{definition}

曲面的第二基本形式近似等于曲面与切平面的有向距离的两倍,刻画了曲面在空间中的弯曲性。
当曲面在给定点向$\bm n$的正侧弯曲时为正,反之为负。

\subsection{曲面的基本公式}\label{sub:曲面的基本公式}
\begin{notation}
    为了让式子更简洁规律,本小节采用新的符号来表示之前的量:
    \begin{align*}%\label{eq:03ex01.25}
        \bm u^1    & =\bm u,      & \bm u^2    & =v,          & \bm r_1    & =\bm r_u,    & \bm r_2         & =\bm r_v,                                                                                       \\
        \bm r_{11} & =\bm r_{uu}, & \bm r_{22} & =\bm r_{vv}, & \bm r_{12} & =\bm r_{uv}, & \bm r_{21}      & =\bm r_{vu},                                                                                    \\
        g_{11}     & =E,          & g_{22}     & =G,          & g_{12}     & =g_{21}=F,   & g               & =EG-F^2=\left|\begin{array}{cc}g_{11}&g_{12}\\ g_{21}&g_{22}\end{array}\right|,                                                \\
        L_{11}     & =L,          & L_{22}     & =N,          & L_{12}     & =L_{21}=M,   & \sum\limits_{i} & =\sum\limits_{i=1}^{2},\quad \sum\limits_{i,j}=\sum\limits_{i=1}^{2}{\sum\limits_{j=1}^{2}}\, .
    \end{align*}
\end{notation}

对于一个$C^3$阶曲面$S$即$\bm r=\bm r(u^1,u^2)$(这里$u^2$中2是上标,不是平方),它确定了向量
\begin{align}\label{eq:03ex01.26}
    \bm r_1=\frac{\partial \bm r}{\partial u^1}, \quad \bm r_2=\frac{\partial \bm r}{\partial u^2}, \quad \bm n=\frac{\bm r_1\times\bm r_2}{\sqrt{g}}\, .
\end{align}
对这三个向量再求导,并且命
\begin{align}\label{eq:03ex01.27}
    \left\{\begin{array}{l}
        \displaystyle\bm r_{ij}=\lambda_{ij}\bm n+\sum\limits_{k}{\varGamma_{ij}^k\bm r_k}, \\
        \displaystyle\bm n_i=\sum\limits_{j}{\mu_i^j\bm r_j},
    \end{array}\right.\quad i,j=1,2\, .
\end{align}
其中$\varGamma_{ij}^k,\lambda_{ij},\mu_i^j$为待定系数,$i,j,k=1,2$为上下标。

注意到$\bm r_k\cdot\bm n=0$且$\bm r_{ij}\cdot\bm n=L_{ij}$,
于是\refeq{03ex01.27}第一式两边点乘$\bm n$得
\begin{align}\label{eq:03ex01.28}
    \lambda_{ij}=\bm r_{ij}\cdot\bm n=L_{ij}\, .
\end{align}
注意到$g_{ij}=\bm r_i\cdot\bm r_j$,对它们求导得
\begin{align}\label{eq:03ex01.29}
    \frac{\partial g_{ij}}{\partial u^l}=\bm r_{il}\cdot\bm r_{j}+\bm r_{i}\cdot\bm r_{jl}\, .
\end{align}
由于$\bm r_{ij}=\bm r_{ji}$且$g_{ij}=g_{ji}$,所以有
\begin{align}\label{eq:03ex01.30}
    \frac{1}{2}\left(\frac{\partial g_{il}}{\partial u^j}+\frac{\partial g_{jl}}{\partial u^i}-\frac{\partial g_{ij}}{\partial u^l}\right)=\bm r_{ij}\cdot\bm r_l=\sum\limits_{k}{\varGamma_{ij}^k g_{kl}}, \quad i,j,l=1,2\, .
\end{align}
记$[g_{ij}]_{2\times2}$的逆矩阵为$[g^{ij}]_{2\times2}$,即
\begin{align}\label{eq:03ex01.31}
    \sum\limits_{k}{g^{ik}g_{kj}}=\left\{\begin{array}{ll}
        1, & \text{if}\quad i=j,     \\
        0, & \text{if}\quad i\neq j,
    \end{array}\right.\quad i,j=2\, .
\end{align}
于是从\refeq{03ex01.30}可解得系数$\varGamma_{ij}^k$为
\begin{align}\label{eq:03ex01.32}
    \varGamma_{ij}^k=\frac{1}{2}\sum\limits_{l}{g^{lk}\left(\frac{\partial g_{il}}{\partial u^j}+\frac{\partial g_{jl}}{\partial u^i}-\frac{\partial g_{ij}}{\partial u^l}\right)},\quad i,j,k=1,2\, .
\end{align}
对于\refeq{03ex01.27}第二式,两边点乘$\bm r_k$得
\begin{align}\label{eq:03ex01.33}
    \bm n_i\cdot\bm r_k=-\bm r_{ik}\cdot\bm n=-L_{ik}=\sum\limits_{j}{\mu_i^j g_{jk}}, \quad i,k=1,2\, .
\end{align}
因此
\begin{align}\label{eq:03ex01.34}
    \mu_i^j=-\sum\limits_{k}{L_{ik}g^{kj}},\quad i,j=1,2\, .
\end{align}
于是我们得到了$\bm r_1,\bm r_2$和$\bm n$的导向量。
\begin{theorem}
    $C^3$阶曲面$S$有
    \begin{align}\label{eq:03ex01.35}
        \left\{\begin{array}{l}
            \displaystyle\bm r_{ij}=L_{ij}\bm n+\sum\limits_{k}{\varGamma_{ij}^k\bm r_k}, \\
            \displaystyle\bm n_i=-\sum\limits_{j,k}{L_{ik}g^{kj}\bm r_j},
        \end{array}\right.\quad i,j=1,2\, ,
    \end{align}
    上式称为\keyindex{曲面的基本公式}{fundamental formulas for surfaces}{surface曲面},
    其中第一式称为\keyindex{曲面的高斯公式}{Gauss formula of surfaces}{surface曲面},
    第二式称为\keyindex{曲面的外恩加滕公式}{Weingarten formula of surfaces}{surface曲面},
    系数
    \begin{align}\label{eq:03ex01.36}
        \varGamma_{ij}^k=\frac{1}{2}\sum\limits_{l}{g^{lk}\left(\frac{\partial g_{il}}{\partial u^j}+\frac{\partial g_{jl}}{\partial u^i}-\frac{\partial g_{ij}}{\partial u^l}\right)},\quad i,j,k=1,2
    \end{align}
    称为\keyindex{第二类Christoffel符号}{Christoffel symbol of the second kind}{}或\keyindex{联络系数}{coefficient of connection}{}。
\end{theorem}
\begin{notation}
    将曲面基本方程中的新记号用旧记号表示:
    \begin{align*}
        g^{11}                               & =\frac{G}{EG-F^2},                                                       &
        g^{22}                               & =\frac{E}{EG-F^2},                                                         \\
        g^{12}                               & =\frac{-F}{EG-F^2},                                                      &
        g^{21}                               & =\frac{-F}{EG-F^2},                                                        \\
        \frac{\partial g_{11}}{\partial u^1} & =\frac{\partial E}{\partial u}=E_u,                                      &
        \frac{\partial g_{11}}{\partial u^2} & =\frac{\partial E}{\partial v}=E_v,                                        \\
        \frac{\partial g_{22}}{\partial u^1} & =\frac{\partial G}{\partial u}=G_u,                                      &
        \frac{\partial g_{22}}{\partial u^2} & =\frac{\partial G}{\partial v}=G_v,                                        \\
        \frac{\partial g_{12}}{\partial u^1} & =\frac{\partial g_{21}}{\partial u^1}=\frac{\partial F}{\partial u}=F_u, &
        \frac{\partial g_{12}}{\partial u^2} & =\frac{\partial g_{21}}{\partial u^2}=\frac{\partial F}{\partial v}=F_v,   \\
        \varGamma_{11}^1                     & =\frac{GE_u-F(2F_u-E_v)}{2(EG-F^2)},                                     &
        \varGamma_{11}^2                     & =\frac{E(2F_u-E_v)-FE_u}{2(EG-F^2)},                                       \\
        \varGamma_{12}^1                     & =\frac{GE_v-FG_u}{2(EG-F^2)},                                            &
        \varGamma_{12}^2                     & =\frac{EG_u-FE_v}{2(EG-F^2)},                                              \\
        \varGamma_{22}^1                     & =\frac{G(2F_v-G_u)-FG_v}{2(EG-F^2)},                                     &
        \varGamma_{22}^2                     & =\frac{EG_v-F(2F_v-G_u)}{2(EG-F^2)},                                       \\
        \mu_1^1                              & =\frac{-LG+MF}{EG-F^2},                                                  &
        \mu_1^2                              & =\frac{LF-ME}{EG-F^2},                                                     \\
        \mu_2^1                              & =\frac{NF-MG}{EG-F^2},                                                   &
        \mu_2^2                              & =\frac{-NE+MF}{EG-F^2}\, .
    \end{align*}
\end{notation}