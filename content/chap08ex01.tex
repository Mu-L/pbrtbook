\section{译者补充:微面模型相关推导}\label{sec:译者补充:微面模型相关推导}
\begin{remark}
    本节内容不是原书内容,而是译者根据\citet{heitz:hal-01024289}并自行推导补充的,请酌情参考和斧正。
\end{remark}
\subsection{微面分布函数的定义与性质}\label{sub:微面分布函数的定义与性质}
如图,我们考虑一个足够小的宏曲面$\mathcal{G}$,设它是个绝对光滑的平面,具有法线$\bm n$.
微面模型中,真正粗糙起伏的曲面,即微曲面$\mathcal{M}$,是由许多偏离宏曲面的微面构成的。
或者说,微曲面$\mathcal{M}$上所有的点在方向$\bm n$上投影即得$\mathcal{G}$.
将宏曲面上的点记为${\bm p}_{\mathrm{g}}$,其周围的微分面元为$\mathrm{d}{\bm p}_{\mathrm{g}}$,
则宏曲面的面积为
\begin{align}
    \label{eq:08ex01-macrosurfaceArea}
    S=\int\limits_{\mathcal{G}}\mathrm{d}{\bm p}_{\mathrm{g}}\, .
\end{align}

将$\mathcal{M}$上的点记作${\bm p}_{\mathrm{m}}$,该点处的法线记作${\bm\omega}_{\mathrm{m}}({\bm p}_{\mathrm{m}})$.
引入三维意义下的狄拉克$\delta$分布,
它满足$\displaystyle\int\limits_{\varOmega}\delta({\bm\omega})\mathrm{d}{\bm\omega}=1$.
并记$\delta_{\bm\omega}({\bm\omega}')=\delta({\bm\omega}'-{\bm\omega})$.
由此定义微面分布函数为
\begin{align}
    \label{eq:08ex01-distributionOfNormals}
    D({\bm\omega})=\displaystyle\frac{1}{S}\int\limits_{\mathcal{M}}\delta_{\bm\omega}({\bm\omega}_{\mathrm{m}}({\bm p}_{\mathrm{m}}))\mathrm{d}{\bm p}_{\mathrm{m}}\, ,
\end{align}
其中狄拉克$\delta$分布和$D({\bm\omega})$的量纲均为
球面度的倒数即$\displaystyle\frac{1}{\text{sr}}$,也相当于无量纲。

我们可以把${\bm\omega}_{\mathrm{m}}$视作从$\mathcal{M}$到整个方向空间$\varOmega$的映射。
注意$\varOmega$包含了球心到完整球面上任意一点的所有可能方向(总立体角为$4\pi$,尽管该映射不一定能覆盖全)。
现在考虑$\mathcal{M}$和$\varOmega$各自的子集$\mathcal{M'}$和$\varOmega'$,
并设它们满足以下条件:点${\bm p}_{\mathrm{m}}$属于$\mathcal{M'}$当且仅当
该点处的微面法线${\bm\omega}_{\mathrm{m}}({\bm p}_{\mathrm{m}})$属于$\varOmega'$,即
\begin{align}
    \label{eq:08ex01-spatialToStatistical}
    {\bm p}_{\mathrm{m}}\in\mathcal{M'}\Leftrightarrow{\bm\omega}_{\mathrm{m}}({\bm p}_{\mathrm{m}})\in\varOmega'\, .
\end{align}
由此利用积分换元可得微面分布函数具有计算指定微曲面面积的能力:
\begin{align}
    \label{eq:08ex01-microsurfaceArea}
    \int\limits_{\mathcal{M}'}\mathrm{d}{\bm p}_{\mathrm{m}}=S\int\limits_{\varOmega'}D({\bm\omega}_{\mathrm{m}})\mathrm{d}{\bm\omega}_{\mathrm{m}}\, .
\end{align}
而整个微曲面面积就是
\begin{align}
    \label{eq:08ex01-microsurfaceTotalArea}
    \int\limits_{\mathcal{M}}\mathrm{d}{\bm p}_{\mathrm{m}}=S\int\limits_{\varOmega}D({\bm\omega}_{\mathrm{m}})\mathrm{d}{\bm\omega}_{\mathrm{m}}\, .
\end{align}

进一步地,对于任意关于微面法线的函数$f({\bm\omega}_{\mathrm{m}})$,
利用$D({\bm\omega})$可将空间积分与统计积分相互转化:
\begin{align}
    \label{eq:08ex01-spatialStatisticalIntegral}
    \int\limits_{\mathcal{M}}f({\bm\omega}_{\mathrm{m}}({\bm p}_{\mathrm{m}}))\mathrm{d}{\bm p}_{\mathrm{m}}=S\int\limits_{\varOmega}f({\bm\omega}_{\mathrm{m}})D({\bm\omega}_{\mathrm{m}})\mathrm{d}{\bm\omega}_{\mathrm{m}}\, .
\end{align}

反之,对于任意定义在$\mathcal{M}$上的函数$g({\bm p}_{\mathrm{m}})$,
我们可以定义对应的统计函数$g({\bm\omega}_{\mathrm{m}})$为:
\begin{align}
    \label{eq:08ex01-statisticalFunction}
    g({\bm\omega})=\frac{\displaystyle\int\limits_{\mathcal{M}}\delta_{\bm\omega}({\bm\omega}_{\mathrm{m}}({\bm p}_{\mathrm{m}}))g({\bm p}_{\mathrm{m}})\mathrm{d}{\bm p}_{\mathrm{m}}}{\displaystyle\int\limits_{\mathcal{M}}\delta_{\bm\omega}({\bm\omega}_{\mathrm{m}}({\bm p}_{\mathrm{m}}))\mathrm{d}{\bm p}_{\mathrm{m}}}\, .
\end{align}
该函数也可以实现空间积分与统计积分的相互转化:
\begin{align}
    \label{eq:08ex01-spatialStatisticalIntegral02}
    \int\limits_{\mathcal{M}}g({\bm p}_{\mathrm{m}})\mathrm{d}{\bm p}_{\mathrm{m}}=S\int\limits_{\varOmega}g({\bm\omega}_{\mathrm{m}})D({\bm\omega}_{\mathrm{m}})\mathrm{d}{\bm\omega}_{\mathrm{m}}\, .
\end{align}

\subsection{典型微面分布函数的规范性证明}\label{sub:典型微面分布函数的规范性证明}
本节补充了\refeq{8.10}和\refeq{8.11}所给的
微面分布函数$D({\bm\omega}_{\mathrm{h}})$满足规范性要求的证明,即证明
\begin{align}\label{eq:8.ex-01}
    \int\limits_{H^2({\bm n})}D({\bm\omega}_{\mathrm{h}})\cos\theta_{\mathrm{h}}\mathrm{d}{\bm\omega}_{\mathrm{h}}=1\, .
\end{align}

为了简化证明过程,我们先证明以下积分式(其中$\alpha_x,\alpha_y>0$):
\begin{align}\label{eq:8.ex-02}
    \int_{\varphi_{\mathrm{h}}=0}^{2\pi}\frac{1}{2\pi\alpha_x\alpha_y\left(\frac{\cos^2\varphi_{\mathrm{h}}}{\alpha_x^2}+\frac{\sin^2\varphi_{\mathrm{h}}}{\alpha_y^2}\right)}\mathrm{d}\varphi_{\mathrm{h}}=1\, .
\end{align}
\begin{prove}
    \begin{align}
                                                         & \int_{\varphi_{\mathrm{h}}=0}^{2\pi}\frac{1}{2\pi\alpha_x\alpha_y\left(\frac{\cos^2\varphi_{\mathrm{h}}}{\alpha_x^2}+\frac{\sin^2\varphi_{\mathrm{h}}}{\alpha_y^2}\right)}\mathrm{d}\varphi_{\mathrm{h}}\nonumber \\
        =                                                & \int_{\varphi_{\mathrm{h}}=0}^{2\pi}\frac{\alpha_x\alpha_y}{2\pi(\alpha_x^2\sin^2\varphi_{\mathrm{h}}+\alpha_y^2\cos^2\varphi_{\mathrm{h}})}\mathrm{d}\varphi_{\mathrm{h}}\nonumber                               \\
        =                                                & \frac{\alpha_x\alpha_y}{2\pi}\int_{\varphi_{\mathrm{h}}=0}^{2\pi}\frac{1}{(\alpha_x^2\tan^2\varphi_{\mathrm{h}}+\alpha_y^2)\cos^2\varphi_{\mathrm{h}}}\mathrm{d}\varphi_{\mathrm{h}}\nonumber                     \\
        =                                                & \frac{\alpha_x\alpha_y}{\pi}\int_{\varphi_{\mathrm{h}}=0}^{\pi}\frac{1}{\alpha_x^2\tan^2\varphi_{\mathrm{h}}+\alpha_y^2}\mathrm{d}\tan\varphi_{\mathrm{h}}\nonumber                                               \\
        =                                                & \frac{\alpha_x\alpha_y}{\pi}\int_{\varphi_{\mathrm{h}}=-\frac{\pi}{2}}^{\frac{\pi}{2}}\frac{1}{\alpha_x^2\tan^2\varphi_{\mathrm{h}}+\alpha_y^2}\mathrm{d}\tan\varphi_{\mathrm{h}}\nonumber                        \\
        \xlongequal{\text{令}t=\tan\varphi_{\mathrm{h}}} & \frac{\alpha_x\alpha_y}{\pi}\int_{t=-\infty}^{+\infty}\frac{1}{\alpha_x^2t^2+\alpha_y^2}\mathrm{d}t\nonumber                                                                                                      \\
        =                                                & \frac{1}{\pi}\int_{t=-\infty}^{+\infty}\frac{1}{\left(\frac{\alpha_xt}{\alpha_y}\right)^2+1}\mathrm{d}\frac{\alpha_xt}{\alpha_y}\nonumber                                                                         \\
        =                                                & \frac{1}{\pi}\arctan\frac{\alpha_xt}{\alpha_y}\bigg|_{t=-\infty}^{+\infty}\nonumber                                                                                                                               \\
        =                                                & 1\, .
    \end{align}
\end{prove}

接下来证明各向异性的Beckmann-Spizzichino模型即\refeq{8.10}满足规范性。
\begin{prove}
    设
    \begin{align}\label{eq:8.ex-03}
        \beta=\frac{\cos^2\varphi_{\mathrm{h}}}{\alpha_x^2}+\frac{\sin^2\varphi_{\mathrm{h}}}{\alpha_y^2}>0\quad(\alpha_x,\alpha_y>0)\, .
    \end{align}
    利用上述变量简化积分并结合\refeq{8.ex-02},可得
    \begin{align}
          & \int\limits_{H^2({\bm n})}D({\bm\omega}_{\mathrm{h}})\cos\theta_{\mathrm{h}}\mathrm{d}{\bm\omega}_{\mathrm{h}}\nonumber                                                                                                                                                                                                                                                         \\
        = & \int\limits_{H^2({\bm n})}\frac{\mathrm{e}^{-\left(\frac{\cos^2\varphi_{\mathrm{h}}}{\alpha_x^2}+\frac{\sin^2\varphi_{\mathrm{h}}}{\alpha_y^2}\right)\tan^2\theta_{\mathrm{h}}}}{\pi\alpha_x\alpha_y\cos^4\theta_{\mathrm{h}}}\cos\theta_{\mathrm{h}}\mathrm{d}{\bm\omega}_{\mathrm{h}}\nonumber                                                                                \\
        = & \int_{\varphi_{\mathrm{h}}=0}^{2\pi}\int_{\theta_{\mathrm{h}}=0}^{\frac{\pi}{2}}\frac{\mathrm{e}^{-\left(\frac{\cos^2\varphi_{\mathrm{h}}}{\alpha_x^2}+\frac{\sin^2\varphi_{\mathrm{h}}}{\alpha_y^2}\right)\tan^2\theta_{\mathrm{h}}}}{\pi\alpha_x\alpha_y\cos^3\theta_{\mathrm{h}}}\sin\theta_{\mathrm{h}}\mathrm{d}\theta_{\mathrm{h}}\mathrm{d}\varphi_{\mathrm{h}}\nonumber \\
        = & \int_{\varphi_{\mathrm{h}}=0}^{2\pi}\int_{\theta_{\mathrm{h}}=0}^{\frac{\pi}{2}}\frac{\tan\theta_{\mathrm{h}}}{\pi\alpha_x\alpha_y\cos^2\theta_{\mathrm{h}}}\mathrm{e}^{-\beta \tan^2\theta_{\mathrm{h}}}\mathrm{d}\theta_{\mathrm{h}}\mathrm{d}\varphi_{\mathrm{h}}\nonumber                                                                                                   \\
        = & \int_{\varphi_{\mathrm{h}}=0}^{2\pi}\int_{\theta_{\mathrm{h}}=0}^{\frac{\pi}{2}}\frac{-1}{2\pi\alpha_x\alpha_y\beta}\mathrm{d}\mathrm{e}^{-\beta \tan^2\theta_{\mathrm{h}}}\mathrm{d}\varphi_{\mathrm{h}}\nonumber                                                                                                                                                              \\
        = & \int_{\varphi_{\mathrm{h}}=0}^{2\pi}\frac{-1}{2\pi\alpha_x\alpha_y\beta}\left(\mathrm{e}^{-\beta \tan^2\theta_{\mathrm{h}}}\bigg|_{\theta_{\mathrm{h}}=0}^{\frac{\pi}{2}}\right)\mathrm{d}\varphi_{\mathrm{h}}\nonumber                                                                                                                                                         \\
        = & \int_{\varphi_{\mathrm{h}}=0}^{2\pi}\frac{1}{2\pi\alpha_x\alpha_y\beta}\mathrm{d}\varphi_{\mathrm{h}}\nonumber                                                                                                                                                                                                                                                                  \\
        = & 1\, .
    \end{align}
\end{prove}

Trowbridge-Reitz模型即\refeq{8.11}的证明是类似的。
\begin{prove}
    同样按\refeq{8.ex-03}设好$\beta$,结合\refeq{8.ex-02},可得
    \begin{align}
                                                                 & \int\limits_{H^2({\bm n})}D({\bm\omega}_{\mathrm{h}})\cos\theta_{\mathrm{h}}\mathrm{d}{\bm\omega}_{\mathrm{h}}\nonumber                                                                                                                                                                                                                                                            \\
        =                                                        & \int\limits_{H^2({\bm n})}\frac{1}{\pi\alpha_x\alpha_y\left(1+\left(\frac{\cos^2\varphi_{\mathrm{h}}}{\alpha_x^2}+\frac{\sin^2\varphi_{\mathrm{h}}}{\alpha_y^2}\right)\tan^2\theta_{\mathrm{h}}\right)^2\cos^4\theta_{\mathrm{h}}}\cos\theta_{\mathrm{h}}\mathrm{d}{\bm\omega}_{\mathrm{h}}\nonumber                                                                               \\
        =                                                        & \int_{\varphi_{\mathrm{h}}=0}^{2\pi}\int_{\theta_{\mathrm{h}}=0}^{\frac{\pi}{2}}\frac{\sin\theta_{\mathrm{h}}}{\pi\alpha_x\alpha_y\left(1+\left(\frac{\cos^2\varphi_{\mathrm{h}}}{\alpha_x^2}+\frac{\sin^2\varphi_{\mathrm{h}}}{\alpha_y^2}\right)\tan^2\theta_{\mathrm{h}}\right)^2\cos^3\theta_{\mathrm{h}}}\mathrm{d}\theta_{\mathrm{h}}\mathrm{d}\varphi_{\mathrm{h}}\nonumber \\
        =                                                        & \int_{\varphi_{\mathrm{h}}=0}^{2\pi}\int_{\theta_{\mathrm{h}}=0}^{\frac{\pi}{2}}\frac{\tan\theta_{\mathrm{h}}}{\pi\alpha_x\alpha_y(1+\beta\tan^2\theta_{\mathrm{h}})^2\cos^2\theta_{\mathrm{h}}}\mathrm{d}\theta_{\mathrm{h}}\mathrm{d}\varphi_{\mathrm{h}}\nonumber                                                                                                               \\
        =                                                        & \int_{\varphi_{\mathrm{h}}=0}^{2\pi}\int_{\theta_{\mathrm{h}}=0}^{\frac{\pi}{2}}\frac{1}{2\pi\alpha_x\alpha_y\beta(1+\beta\tan^2\theta_{\mathrm{h}})^2}\mathrm{d}(\beta\tan^2\theta_{\mathrm{h}})\mathrm{d}\varphi_{\mathrm{h}}\nonumber                                                                                                                                           \\
        \xlongequal{\text{令}u=1+\beta\tan^2\theta_{\mathrm{h}}} & \int_{\varphi_{\mathrm{h}}=0}^{2\pi}\int_{u=1}^{+\infty}\frac{1}{2\pi\alpha_x\alpha_y\beta u^2}\mathrm{d}u\mathrm{d}\varphi_{\mathrm{h}}\nonumber                                                                                                                                                                                                                                  \\
        =                                                        & \int_{\varphi_{\mathrm{h}}=0}^{2\pi}\int_{u=1}^{+\infty}\frac{-1}{2\pi\alpha_x\alpha_y\beta}\mathrm{d}u^{-1}\mathrm{d}\varphi_{\mathrm{h}}\nonumber                                                                                                                                                                                                                                \\
        =                                                        & \int_{\varphi_{\mathrm{h}}=0}^{2\pi}\frac{-1}{2\pi\alpha_x\alpha_y\beta}\left(\frac{1}{u}\bigg|_{u=1}^{+\infty}\right)\mathrm{d}\varphi_{\mathrm{h}}\nonumber                                                                                                                                                                                                                      \\
        =                                                        & \int_{\varphi_{\mathrm{h}}=0}^{2\pi}\frac{1}{2\pi\alpha_x\alpha_y\beta}\mathrm{d}\varphi_{\mathrm{h}}\nonumber                                                                                                                                                                                                                                                                     \\
        =                                                        & 1\, .
    \end{align}
\end{prove}