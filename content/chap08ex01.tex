\section{译者补充:两个微面分布函数的规范性证明}\label{sec:译者补充:两个微面分布函数的规范性证明}
本节中译者补充了\refeq{8.10}和\refeq{8.11}所给的
微面分布函数$D({\bm\omega}_{\mathrm{h}})$满足规范性要求的证明,即证明
\begin{align}\label{eq:8.ex-01}
    \int\limits_{H^2({\bm n})}D({\bm\omega}_{\mathrm{h}})\cos\theta_{\mathrm{h}}\mathrm{d}{\bm\omega}_{\mathrm{h}}=1\, .
\end{align}

为了简化证明过程,我们先证明以下积分式(其中$\alpha_x,\alpha_y>0$):
\begin{align}\label{eq:8.ex-02}
    \int_{\varphi_{\mathrm{h}}=0}^{2\pi}\frac{1}{2\pi\alpha_x\alpha_y\left(\frac{\cos^2\varphi_{\mathrm{h}}}{\alpha_x^2}+\frac{\sin^2\varphi_{\mathrm{h}}}{\alpha_y^2}\right)}\mathrm{d}\varphi_{\mathrm{h}}=1\, .
\end{align}
\begin{prove}
    \begin{align}
                                                         & \int_{\varphi_{\mathrm{h}}=0}^{2\pi}\frac{1}{2\pi\alpha_x\alpha_y\left(\frac{\cos^2\varphi_{\mathrm{h}}}{\alpha_x^2}+\frac{\sin^2\varphi_{\mathrm{h}}}{\alpha_y^2}\right)}\mathrm{d}\varphi_{\mathrm{h}}\nonumber \\
        =                                                & \int_{\varphi_{\mathrm{h}}=0}^{2\pi}\frac{\alpha_x\alpha_y}{2\pi(\alpha_x^2\sin^2\varphi_{\mathrm{h}}+\alpha_y^2\cos^2\varphi_{\mathrm{h}})}\mathrm{d}\varphi_{\mathrm{h}}\nonumber                               \\
        =                                                & \frac{\alpha_x\alpha_y}{2\pi}\int_{\varphi_{\mathrm{h}}=0}^{2\pi}\frac{1}{(\alpha_x^2\tan^2\varphi_{\mathrm{h}}+\alpha_y^2)\cos^2\varphi_{\mathrm{h}}}\mathrm{d}\varphi_{\mathrm{h}}\nonumber                     \\
        =                                                & \frac{\alpha_x\alpha_y}{\pi}\int_{\varphi_{\mathrm{h}}=0}^{\pi}\frac{1}{\alpha_x^2\tan^2\varphi_{\mathrm{h}}+\alpha_y^2}\mathrm{d}\tan\varphi_{\mathrm{h}}\nonumber                                               \\
        =                                                & \frac{\alpha_x\alpha_y}{\pi}\int_{\varphi_{\mathrm{h}}=-\frac{\pi}{2}}^{\frac{\pi}{2}}\frac{1}{\alpha_x^2\tan^2\varphi_{\mathrm{h}}+\alpha_y^2}\mathrm{d}\tan\varphi_{\mathrm{h}}\nonumber                        \\
        \xlongequal{\text{令}t=\tan\varphi_{\mathrm{h}}} & \frac{\alpha_x\alpha_y}{\pi}\int_{t=-\infty}^{+\infty}\frac{1}{\alpha_x^2t^2+\alpha_y^2}\mathrm{d}t\nonumber                                                                                                      \\
        =                                                & \frac{1}{\pi}\int_{t=-\infty}^{+\infty}\frac{1}{\left(\frac{\alpha_xt}{\alpha_y}\right)^2+1}\mathrm{d}\frac{\alpha_xt}{\alpha_y}\nonumber                                                                         \\
        =                                                & \frac{1}{\pi}\arctan\frac{\alpha_xt}{\alpha_y}\bigg|_{t=-\infty}^{+\infty}\nonumber                                                                                                                               \\
        =                                                & 1\, .
    \end{align}
\end{prove}

接下来证明各向异性的Beckmann-Spizzichino模型即\refeq{8.10}满足规范性。
\begin{prove}
    设
    \begin{align}\label{eq:8.ex-03}
        \beta=\frac{\cos^2\varphi_{\mathrm{h}}}{\alpha_x^2}+\frac{\sin^2\varphi_{\mathrm{h}}}{\alpha_y^2}>0\quad(\alpha_x,\alpha_y>0)\, .
    \end{align}
    利用上述变量简化积分并结合\refeq{8.ex-02},可得
    \begin{align}
          & \int\limits_{H^2({\bm n})}D({\bm\omega}_{\mathrm{h}})\cos\theta_{\mathrm{h}}\mathrm{d}{\bm\omega}_{\mathrm{h}}\nonumber                                                                                                                                                                                                                                                         \\
        = & \int\limits_{H^2({\bm n})}\frac{\mathrm{e}^{-\left(\frac{\cos^2\varphi_{\mathrm{h}}}{\alpha_x^2}+\frac{\sin^2\varphi_{\mathrm{h}}}{\alpha_y^2}\right)\tan^2\theta_{\mathrm{h}}}}{\pi\alpha_x\alpha_y\cos^4\theta_{\mathrm{h}}}\cos\theta_{\mathrm{h}}\mathrm{d}{\bm\omega}_{\mathrm{h}}\nonumber                                                                                \\
        = & \int_{\varphi_{\mathrm{h}}=0}^{2\pi}\int_{\theta_{\mathrm{h}}=0}^{\frac{\pi}{2}}\frac{\mathrm{e}^{-\left(\frac{\cos^2\varphi_{\mathrm{h}}}{\alpha_x^2}+\frac{\sin^2\varphi_{\mathrm{h}}}{\alpha_y^2}\right)\tan^2\theta_{\mathrm{h}}}}{\pi\alpha_x\alpha_y\cos^3\theta_{\mathrm{h}}}\sin\theta_{\mathrm{h}}\mathrm{d}\theta_{\mathrm{h}}\mathrm{d}\varphi_{\mathrm{h}}\nonumber \\
        = & \int_{\varphi_{\mathrm{h}}=0}^{2\pi}\int_{\theta_{\mathrm{h}}=0}^{\frac{\pi}{2}}\frac{\tan\theta_{\mathrm{h}}}{\pi\alpha_x\alpha_y\cos^2\theta_{\mathrm{h}}}\mathrm{e}^{-\beta \tan^2\theta_{\mathrm{h}}}\mathrm{d}\theta_{\mathrm{h}}\mathrm{d}\varphi_{\mathrm{h}}\nonumber                                                                                                   \\
        = & \int_{\varphi_{\mathrm{h}}=0}^{2\pi}\int_{\theta_{\mathrm{h}}=0}^{\frac{\pi}{2}}\frac{-1}{2\pi\alpha_x\alpha_y\beta}\mathrm{d}\mathrm{e}^{-\beta \tan^2\theta_{\mathrm{h}}}\mathrm{d}\varphi_{\mathrm{h}}\nonumber                                                                                                                                                              \\
        = & \int_{\varphi_{\mathrm{h}}=0}^{2\pi}\frac{-1}{2\pi\alpha_x\alpha_y\beta}\left(\mathrm{e}^{-\beta \tan^2\theta_{\mathrm{h}}}\bigg|_{\theta_{\mathrm{h}}=0}^{\frac{\pi}{2}}\right)\mathrm{d}\varphi_{\mathrm{h}}\nonumber                                                                                                                                                         \\
        = & \int_{\varphi_{\mathrm{h}}=0}^{2\pi}\frac{1}{2\pi\alpha_x\alpha_y\beta}\mathrm{d}\varphi_{\mathrm{h}}\nonumber                                                                                                                                                                                                                                                                  \\
        = & 1\, .
    \end{align}
\end{prove}

Trowbridge-Reitz模型即\refeq{8.11}的证明是类似的。
\begin{prove}
    同样按\refeq{8.ex-03}设好$\beta$,结合\refeq{8.ex-02},可得
    \begin{align}
                                                                 & \int\limits_{H^2({\bm n})}D({\bm\omega}_{\mathrm{h}})\cos\theta_{\mathrm{h}}\mathrm{d}{\bm\omega}_{\mathrm{h}}\nonumber                                                                                                                                                                                                                                                            \\
        =                                                        & \int\limits_{H^2({\bm n})}\frac{1}{\pi\alpha_x\alpha_y\left(1+\left(\frac{\cos^2\varphi_{\mathrm{h}}}{\alpha_x^2}+\frac{\sin^2\varphi_{\mathrm{h}}}{\alpha_y^2}\right)\tan^2\theta_{\mathrm{h}}\right)^2\cos^4\theta_{\mathrm{h}}}\cos\theta_{\mathrm{h}}\mathrm{d}{\bm\omega}_{\mathrm{h}}\nonumber                                                                               \\
        =                                                        & \int_{\varphi_{\mathrm{h}}=0}^{2\pi}\int_{\theta_{\mathrm{h}}=0}^{\frac{\pi}{2}}\frac{\sin\theta_{\mathrm{h}}}{\pi\alpha_x\alpha_y\left(1+\left(\frac{\cos^2\varphi_{\mathrm{h}}}{\alpha_x^2}+\frac{\sin^2\varphi_{\mathrm{h}}}{\alpha_y^2}\right)\tan^2\theta_{\mathrm{h}}\right)^2\cos^3\theta_{\mathrm{h}}}\mathrm{d}\theta_{\mathrm{h}}\mathrm{d}\varphi_{\mathrm{h}}\nonumber \\
        =                                                        & \int_{\varphi_{\mathrm{h}}=0}^{2\pi}\int_{\theta_{\mathrm{h}}=0}^{\frac{\pi}{2}}\frac{\tan\theta_{\mathrm{h}}}{\pi\alpha_x\alpha_y(1+\beta\tan^2\theta_{\mathrm{h}})^2\cos^2\theta_{\mathrm{h}}}\mathrm{d}\theta_{\mathrm{h}}\mathrm{d}\varphi_{\mathrm{h}}\nonumber                                                                                                               \\
        =                                                        & \int_{\varphi_{\mathrm{h}}=0}^{2\pi}\int_{\theta_{\mathrm{h}}=0}^{\frac{\pi}{2}}\frac{1}{2\pi\alpha_x\alpha_y\beta(1+\beta\tan^2\theta_{\mathrm{h}})^2}\mathrm{d}(\beta\tan^2\theta_{\mathrm{h}})\mathrm{d}\varphi_{\mathrm{h}}\nonumber                                                                                                                                           \\
        \xlongequal{\text{令}u=1+\beta\tan^2\theta_{\mathrm{h}}} & \int_{\varphi_{\mathrm{h}}=0}^{2\pi}\int_{u=1}^{+\infty}\frac{1}{2\pi\alpha_x\alpha_y\beta u^2}\mathrm{d}u\mathrm{d}\varphi_{\mathrm{h}}\nonumber                                                                                                                                                                                                                                  \\
        =                                                        & \int_{\varphi_{\mathrm{h}}=0}^{2\pi}\int_{u=1}^{+\infty}\frac{-1}{2\pi\alpha_x\alpha_y\beta}\mathrm{d}u^{-1}\mathrm{d}\varphi_{\mathrm{h}}\nonumber                                                                                                                                                                                                                                \\
        =                                                        & \int_{\varphi_{\mathrm{h}}=0}^{2\pi}\frac{-1}{2\pi\alpha_x\alpha_y\beta}\left(\frac{1}{u}\bigg|_{u=1}^{+\infty}\right)\mathrm{d}\varphi_{\mathrm{h}}\nonumber                                                                                                                                                                                                                      \\
        =                                                        & \int_{\varphi_{\mathrm{h}}=0}^{2\pi}\frac{1}{2\pi\alpha_x\alpha_y\beta}\mathrm{d}\varphi_{\mathrm{h}}\nonumber                                                                                                                                                                                                                                                                     \\
        =                                                        & 1\, .
    \end{align}
\end{prove}