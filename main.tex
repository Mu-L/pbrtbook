%%%%%%%%%%%%%%%%%%%%%%%%%%%%%%%%%%%%%%%%%
% The Legrand Orange Book
% LaTeX Template
% Version 2.4 (26/09/2018)
%
% This template was downloaded from:
% http://www.LaTeXTemplates.com
%
% Original author:
% Mathias Legrand (legrand.mathias@gmail.com) with modifications by:
% Vel (vel@latextemplates.com)
%
% License:
% CC BY-NC-SA 3.0 (http://creativecommons.org/licenses/by-nc-sa/3.0/)
%
% Compiling this template:
% This template uses biber for its bibliography and makeindex for its index.
% When you first open the template, compile it from the command line with the 
% commands below to make sure your LaTeX distribution is configured correctly:
%
% 1) pdflatex main
% 2) makeindex main.idx -s StyleInd.ist
% 3) biber main
% 4) pdflatex main x 2
%
% After this, when you wish to update the bibliography/index use the appropriate
% command above and make sure to compile with pdflatex several times 
% afterwards to propagate your changes to the document.
%
% This template also uses a number of packages which may need to be
% updated to the newest versions for the template to compile. It is strongly
% recommended you update your LaTeX distribution if you have any
% compilation errors.
%
% Important note:
% Chapter heading images should have a 2:1 width:height ratio,
% e.g. 920px width and 460px height.
%
%%%%%%%%%%%%%%%%%%%%%%%%%%%%%%%%%%%%%%%%%

%----------------------------------------------------------------------------------------
%	PACKAGES AND OTHER DOCUMENT CONFIGURATIONS
%----------------------------------------------------------------------------------------

\documentclass[11pt,twoside]{book} % Default font size and left-justified equations

%%%%%%%%%%%%%%%%%%%%%%%%%%%%%%%%%%%%%%%%%
% The Legrand Orange Book
% Structural Definitions File
% Version 2.1 (26/09/2018)
%
% Original author:
% Mathias Legrand (legrand.mathias@gmail.com) with modifications by:
% Vel (vel@latextemplates.com)
% 
% This file was downloaded from:
% http://www.LaTeXTemplates.com
%
% License:
% CC BY-NC-SA 3.0 (http://creativecommons.org/licenses/by-nc-sa/3.0/)
%
%%%%%%%%%%%%%%%%%%%%%%%%%%%%%%%%%%%%%%%%%

%----------------------------------------------------------------------------------------
%	VARIOUS REQUIRED PACKAGES AND CONFIGURATIONS
%----------------------------------------------------------------------------------------

\usepackage{graphicx} % Required for including pictures
\graphicspath{{Pictures/}} % Specifies the directory where pictures are stored
\usepackage{subfig}
\usepackage{tikz} % Required for drawing custom shapes
\usepackage{pstricks}
\usepackage[english]{babel} % English language/hyphenation

\usepackage{enumitem} % Customize lists
\setlist{nolistsep} % Reduce spacing between bullet points and numbered lists

\usepackage{booktabs} % Required for nicer horizontal rules in tables
\usepackage{bm}
\usepackage{xcolor} % Required for specifying colors by name
\definecolor{ocre}{RGB}{243,102,25} % Define the orange color used for highlighting throughout the book

%----------------------------------------------------------------------------------------
%	MARGINS
%----------------------------------------------------------------------------------------

\usepackage{geometry} % Required for adjusting page dimensions and margins
\geometry{
	paper=a4paper, % Paper size, change to letterpaper for US letter size
	top=3cm, % Top margin
	bottom=3cm, % Bottom margin
	inner=3.2cm, % Left margin
	outer=4.8cm, % Right margin
	headheight=14pt, % Header height
	footskip=1.4cm, % Space from the bottom margin to the baseline of the footer
	headsep=10pt, % Space from the top margin to the baseline of the header
	%showframe, % Uncomment to show how the type block is set on the page
}

%----------------------------------------------------------------------------------------
%	FONTS
%----------------------------------------------------------------------------------------
\usepackage[UTF8]{ctex}
\setCJKmainfont[Path=fonts/,BoldFont={SourceHanSerifSC-Bold.otf},ItalicFont={simkai.ttf}]{SourceHanSerifSC-Regular.otf}
\setCJKsansfont[Path=fonts/,BoldFont={SourceHanSansSC-Bold.otf}]{SourceHanSansSC-Regular.otf}
\usepackage{avant} % Use the Avantgarde font for headings
%\usepackage{times} % Use the Times font for headings
\usepackage{mathptmx} % Use the Adobe Times Roman as the default text font together with math symbols from the Sym­bol, Chancery and Com­puter Modern fonts

\usepackage{microtype} % Slightly tweak font spacing for aesthetics
\usepackage[utf8]{inputenc} % Required for including letters with accents
\usepackage[T1]{fontenc} % Use 8-bit encoding that has 256 glyphs

%---------------------------------------------------------------------------------------
% use sidenotes but fix its bug to change font size
% see: https://tex.stackexchange.com/questions/532245/how-to-modify-fonts-in-sidenotes
%---------------------------------------------------------------------------------------
\usepackage{sidenotes}
\usepackage{xparse}
\let\oldmarginpar\marginpar
\RenewDocumentCommand{\marginpar}{om}{%
	\IfNoValueTF{#1}
	{\oldmarginpar{\mymparsetup #2}}
	{\oldmarginpar[\mymparsetup #1]{\mymparsetup #2}}}
\newcommand{\mymparsetup}{\scriptsize\itshape}

%---------------------------------------------------------------------------------------
% use \eg ... see: https://stackoverflow.com/a/39363004/12128185
%---------------------------------------------------------------------------------------
\usepackage{xspace}
\makeatletter
\DeclareRobustCommand\onedot{\futurelet\@let@token\@onedot}
\def\@onedot{\ifx\@let@token.\else.\null\fi\xspace}
\def\eg{\emph{e.g}\onedot} \def\Eg{\emph{E.g}\onedot}
\def\ie{\emph{i.e}\onedot} \def\Ie{\emph{I.e}\onedot}
\def\cf{\emph{c.f}\onedot} \def\Cf{\emph{C.f}\onedot}
\def\etc{\emph{etc}\onedot} \def\vs{\emph{vs}\onedot}
\def\wrt{w.r.t\onedot} \def\dof{d.o.f\onedot}
\def\etal{\emph{et al}\onedot}
\makeatother

%----------------------------------------------------------------------------------------
%	BIBLIOGRAPHY AND INDEX
%----------------------------------------------------------------------------------------

\usepackage[style=authoryear,citestyle=authoryear-comp,maxbibnames=99,maxcitenames=3,mincitenames=1,uniquename=false,sorting=nyt,sortcites=true,autopunct=true,babel=hyphen,hyperref=true,abbreviate=false,backref=true,backend=biber,natbib=true]{biblatex}
\addbibresource{bibliography.bib} % BibTeX bibliography file
\defbibheading{bibempty}{}
\renewcommand*{\finalnamedelim}{{\ifcitation{和}{, and }}}
\renewcommand*{\andothersdelim}{\ifcitation{}{, }}
\DefineBibliographyStrings{english}{andothers={\ifcitation{等}{ et al.}}}
\usepackage{calc} % For simpler calculation - used for spacing the index letter headings correctly
\usepackage{makeidx} % Required to make an index
\makeindex % Tells LaTeX to create the files required for indexing
\usepackage{xifthen} % provides \isempty test
\newcommand\keyindex[3]{\ifthenelse{\isempty{#1}}{}{\ifthenelse{\isempty{#2}}{\ifthenelse{\isempty{#3}}{{\sffamily#1}}{{\sffamily#1}\index{#3!#1}}}{\ifthenelse{\isempty{#3}}{{\sffamily#1}(#2)\index{#2#1}}{{\sffamily#1}(#2)\index{#3!#2#1}}}}}

%----------------------------------------------------------------------------------------
%	MAIN TABLE OF CONTENTS
%----------------------------------------------------------------------------------------

\usepackage{titletoc} % Required for manipulating the table of contents

\contentsmargin{0cm} % Removes the default margin

% Part text styling (this is mostly taken care of in the PART HEADINGS section of this file)
\titlecontents{part}
[0cm] % Left indentation
{\addvspace{20pt}\bfseries} % Spacing and font options for parts
{}
{}
{}

% Chapter text styling
\titlecontents{chapter}
[1.25cm] % Left indentation
{\addvspace{12pt}\large\sffamily\bfseries} % Spacing and font options for chapters
{\color{ocre!60}\contentslabel[\Large\thecontentslabel]{1.25cm}\color{ocre}} % Formatting of numbered sections of this type
{\color{ocre}} % Formatting of numberless sections of this type
{\color{ocre!60}\normalsize\;\titlerule*[.5pc]{.}\;\thecontentspage} % Formatting of the filler to the right of the heading and the page number

% Section text styling
\titlecontents{section}
[1.25cm] % Left indentation
{\addvspace{3pt}\sffamily\bfseries} % Spacing and font options for sections
{\contentslabel[\thecontentslabel]{1.25cm}} % Formatting of numbered sections of this type
{} % Formatting of numberless sections of this type
{\hfill\color{black}\thecontentspage} % Formatting of the filler to the right of the heading and the page number

% Subsection text styling
\titlecontents{subsection}
[1.25cm] % Left indentation
{\addvspace{1pt}\sffamily\small} % Spacing and font options for subsections
{\contentslabel[\thecontentslabel]{1.25cm}} % Formatting of numbered sections of this type
{} % Formatting of numberless sections of this type
{\ \titlerule*[.5pc]{.}\;\thecontentspage} % Formatting of the filler to the right of the heading and the page number

% Figure text styling
\titlecontents{figure}
[1.25cm] % Left indentation
{\addvspace{1pt}\sffamily\small} % Spacing and font options for figures
{\thecontentslabel\hspace*{1em}} % Formatting of numbered sections of this type
{} % Formatting of numberless sections of this type
{\ \titlerule*[.5pc]{.}\;\thecontentspage} % Formatting of the filler to the right of the heading and the page number

% Table text styling
\titlecontents{table}
[1.25cm] % Left indentation
{\addvspace{1pt}\sffamily\small} % Spacing and font options for tables
{\thecontentslabel\hspace*{1em}} % Formatting of numbered sections of this type
{} % Formatting of numberless sections of this type
{\ \titlerule*[.5pc]{.}\;\thecontentspage} % Formatting of the filler to the right of the heading and the page number

%----------------------------------------------------------------------------------------
%	MINI TABLE OF CONTENTS IN PART HEADS
%----------------------------------------------------------------------------------------

% Chapter text styling
\titlecontents{lchapter}
[0em] % Left indentation
{\addvspace{15pt}\large\sffamily\bfseries} % Spacing and font options for chapters
{\color{ocre}\contentslabel[\Large\thecontentslabel]{1.25cm}\color{ocre}} % Chapter number
{}
{\color{ocre}\normalsize\sffamily\bfseries\;\titlerule*[.5pc]{.}\;\thecontentspage} % Page number

% Section text styling
\titlecontents{lsection}
[0em] % Left indentation
{\sffamily\small} % Spacing and font options for sections
{\contentslabel[\thecontentslabel]{1.25cm}} % Section number
{}
{}

% Subsection text styling (note these aren't shown by default, display them by searchings this file for tocdepth and reading the commented text)
\titlecontents{lsubsection}
[.5em] % Left indentation
{\sffamily\footnotesize} % Spacing and font options for subsections
{\contentslabel[\thecontentslabel]{1.25cm}}
{}
{}

%----------------------------------------------------------------------------------------
%	HEADERS AND FOOTERS
%----------------------------------------------------------------------------------------

\usepackage{fancyhdr} % Required for header and footer configuration

\pagestyle{fancy} % Enable the custom headers and footers

\renewcommand{\chaptermark}[1]{\markboth{\sffamily\normalsize\bfseries 第\thechapter 章\ #1}{}} % Styling for the current chapter in the header
\renewcommand{\sectionmark}[1]{\markright{\sffamily\normalsize\thesection\hspace{5pt}#1}{}} % Styling for the current section in the header

\fancyhf{} % Clear default headers and footers
\fancyhead[LE,RO]{\sffamily\normalsize\thepage} % Styling for the page number in the header
\fancyhead[LO]{\rightmark} % Print the nearest section name on the left side of odd pages
\fancyhead[RE]{\leftmark} % Print the current chapter name on the right side of even pages
%\fancyfoot[C]{\thepage} % Uncomment to include a footer

\renewcommand{\headrulewidth}{0.5pt} % Thickness of the rule under the header

\fancypagestyle{plain}{% Style for when a plain pagestyle is specified
	\fancyhead{}\renewcommand{\headrulewidth}{0pt}%
}

% Removes the header from odd empty pages at the end of chapters
\makeatletter
\renewcommand{\cleardoublepage}{
	\clearpage\ifodd\c@page\else
		\hbox{}
		\vspace*{\fill}
		\thispagestyle{empty}
		\newpage
	\fi}

%----------------------------------------------------------------------------------------
%	THEOREM STYLES
%----------------------------------------------------------------------------------------

\usepackage{amsmath,amsfonts,amssymb,amsthm} % For math equations, theorems, symbols, etc
\usepackage{mathrsfs}
\usepackage{yhmath}
\usepackage{boondox-calo}
\usepackage{extarrows} %长等号
\newcommand{\intoo}[2]{\mathopen{]}#1\,;#2\mathclose{[}}
\newcommand{\ud}{\mathop{\mathrm{{}d}}\mathopen{}}
\newcommand{\intff}[2]{\mathopen{[}#1\,;#2\mathclose{]}}
\renewcommand{\qedsymbol}{$\blacksquare$}
\newtheorem{notation}{记号}[section]
\allowdisplaybreaks[4] % 公式跨页断行最大强度

% Boxed/framed environments
\newtheoremstyle{ocrenumbox}% Theorem style name
{0pt}% Space above
{0pt}% Space below
{\normalfont}% Body font
{}% Indent amount
{\small\bf\sffamily\color{ocre}}% Theorem head font
{\;}% Punctuation after theorem head
{0.25em}% Space after theorem head
{\small\sffamily\color{ocre}\thmname{#1}\nobreakspace\thmnumber{\@ifnotempty{#1}{}\@upn{#2}}% Theorem text (e.g. Theorem 2.1)
	\thmnote{\nobreakspace\the\thm@notefont\sffamily\bfseries\color{black}---\nobreakspace#3.}} % Optional theorem note

\newtheoremstyle{blacknumex}% Theorem style name
{5pt}% Space above
{5pt}% Space below
{\normalfont}% Body font
{} % Indent amount
{\small\bf\sffamily}% Theorem head font
{\;}% Punctuation after theorem head
{0.25em}% Space after theorem head
{\small\sffamily{\tiny\ensuremath{\blacksquare}}\nobreakspace\thmname{#1}\nobreakspace\thmnumber{\@ifnotempty{#1}{}\@upn{#2}}% Theorem text (e.g. Theorem 2.1)
	\thmnote{\nobreakspace\the\thm@notefont\sffamily\bfseries---\nobreakspace#3.}}% Optional theorem note

\newtheoremstyle{blacknumbox} % Theorem style name
{0pt}% Space above
{0pt}% Space below
{\normalfont}% Body font
{}% Indent amount
{\small\bf\sffamily}% Theorem head font
{\;}% Punctuation after theorem head
{0.25em}% Space after theorem head
{\small\sffamily\thmname{#1}\nobreakspace\thmnumber{\@ifnotempty{#1}{}\@upn{#2}}% Theorem text (e.g. Theorem 2.1)
	\thmnote{\nobreakspace\the\thm@notefont\sffamily\bfseries---\nobreakspace#3.}}% Optional theorem note

% Non-boxed/non-framed environments
\newtheoremstyle{ocrenum}% Theorem style name
{5pt}% Space above
{5pt}% Space below
{\normalfont}% Body font
{}% Indent amount
{\small\bf\sffamily\color{ocre}}% Theorem head font
{\;}% Punctuation after theorem head
{0.25em}% Space after theorem head
{\small\sffamily\color{ocre}\thmname{#1}\nobreakspace\thmnumber{\@ifnotempty{#1}{}\@upn{#2}}% Theorem text (e.g. Theorem 2.1)
	\thmnote{\nobreakspace\the\thm@notefont\sffamily\bfseries\color{black}---\nobreakspace#3.}} % Optional theorem note
\makeatother

% Defines the theorem text style for each type of theorem to one of the three styles above
\newcounter{dummy}
\numberwithin{dummy}{section}
\theoremstyle{ocrenumbox}
\newtheorem{theoremeT}[dummy]{定理}
\newtheorem{problem}{Problem}[chapter]
\newtheorem{exerciseT}{Exercise}[chapter]
\theoremstyle{blacknumex}
\newtheorem{exampleT}{例}[section]
\newtheorem{proveT}{证明}[section]
\newenvironment{prove}{\begin{proveT}}{\hfill{\tiny\ensuremath{\qedhere\blacksquare}}\end{proveT}}
\theoremstyle{blacknumbox}
\newtheorem{vocabulary}{Vocabulary}[chapter]
\newtheorem{definitionT}{定义}[section]
\newtheorem{corollaryT}[dummy]{推论}
\theoremstyle{ocrenum}
\newtheorem{proposition}[dummy]{定理}

%----------------------------------------------------------------------------------------
%	DEFINITION OF COLORED BOXES
%----------------------------------------------------------------------------------------

\RequirePackage[framemethod=default]{mdframed} % Required for creating the theorem, definition, exercise and corollary boxes

% Theorem box
\newmdenv[skipabove=7pt,
	skipbelow=7pt,
	backgroundcolor=black!5,
	linecolor=ocre,
	innerleftmargin=5pt,
	innerrightmargin=5pt,
	innertopmargin=5pt,
	leftmargin=0cm,
	rightmargin=0cm,
	innerbottommargin=5pt]{tBox}

% Exercise box	  
\newmdenv[skipabove=7pt,
	skipbelow=7pt,
	rightline=false,
	leftline=true,
	topline=false,
	bottomline=false,
	backgroundcolor=ocre!10,
	linecolor=ocre,
	innerleftmargin=5pt,
	innerrightmargin=5pt,
	innertopmargin=5pt,
	innerbottommargin=5pt,
	leftmargin=0cm,
	rightmargin=0cm,
	linewidth=4pt]{eBox}

% Definition box
\newmdenv[skipabove=7pt,
	skipbelow=7pt,
	rightline=false,
	leftline=true,
	topline=false,
	bottomline=false,
	linecolor=ocre,
	innerleftmargin=5pt,
	innerrightmargin=5pt,
	innertopmargin=0pt,
	leftmargin=0cm,
	rightmargin=0cm,
	linewidth=4pt,
	innerbottommargin=0pt]{dBox}

% Corollary box
\newmdenv[skipabove=7pt,
	skipbelow=7pt,
	rightline=false,
	leftline=true,
	topline=false,
	bottomline=false,
	linecolor=gray,
	backgroundcolor=black!5,
	innerleftmargin=5pt,
	innerrightmargin=5pt,
	innertopmargin=5pt,
	leftmargin=0cm,
	rightmargin=0cm,
	linewidth=4pt,
	innerbottommargin=5pt]{cBox}

% Creates an environment for each type of theorem and assigns it a theorem text style from the "Theorem Styles" section above and a colored box from above
\newenvironment{theorem}{\begin{tBox}\begin{theoremeT}}{\end{theoremeT}\end{tBox}}
\newenvironment{exercise}{\begin{eBox}\begin{exerciseT}}{\hfill{\color{ocre}\tiny\ensuremath{\blacksquare}}\end{exerciseT}\end{eBox}}
\newenvironment{definition}{\begin{dBox}\begin{definitionT}}{\end{definitionT}\end{dBox}}
\newenvironment{example}{\begin{exampleT}}{\hfill{\tiny\ensuremath{\blacksquare}}\end{exampleT}}
\newenvironment{corollary}{\begin{cBox}\begin{corollaryT}}{\end{corollaryT}\end{cBox}}

%----------------------------------------------------------------------------------------
%	REMARK ENVIRONMENT
%----------------------------------------------------------------------------------------

\newenvironment{remark}{\par\vspace{10pt}\small % Vertical white space above the remark and smaller font size
	\begin{list}{}{
			\leftmargin=35pt % Indentation on the left
			\rightmargin=25pt}\item\ignorespaces % Indentation on the right
		      \makebox[-2.5pt]{\begin{tikzpicture}[overlay]
				      \node[draw=ocre!60,line width=1pt,circle,fill=ocre!25,font=\sffamily\bfseries,inner sep=2pt,outer sep=0pt] at (-15pt,0pt){\textcolor{ocre}{R}};\end{tikzpicture}} % Orange R in a circle
		      \advance\baselineskip -1pt}{\end{list}\vskip5pt} % Tighter line spacing and white space after remark

%----------------------------------------------------------------------------------------
%	SECTION NUMBERING IN THE MARGIN
%----------------------------------------------------------------------------------------

\makeatletter
\renewcommand{\@seccntformat}[1]{\llap{\textcolor{ocre}{\csname the#1\endcsname}\hspace{1em}}}
\renewcommand{\section}{\@startsection{section}{1}{\z@}
	{-4ex \@plus -1ex \@minus -.4ex}
	{1ex \@plus.2ex }
	{\normalfont\large\sffamily\bfseries}}
\renewcommand{\subsection}{\@startsection {subsection}{2}{\z@}
	{-3ex \@plus -0.1ex \@minus -.4ex}
	{0.5ex \@plus.2ex }
	{\normalfont\sffamily\bfseries}}
\renewcommand{\subsubsection}{\@startsection {subsubsection}{3}{\z@}
	{-2ex \@plus -0.1ex \@minus -.2ex}
	{.2ex \@plus.2ex }
	{\normalfont\small\sffamily\bfseries}}
\renewcommand\paragraph{\@startsection{paragraph}{4}{\z@}
	{-2ex \@plus-.2ex \@minus .2ex}
	{.1ex}
	{\normalfont\small\sffamily\bfseries}}

%----------------------------------------------------------------------------------------
%	PART HEADINGS
%----------------------------------------------------------------------------------------

% Numbered part in the table of contents
\newcommand{\@mypartnumtocformat}[2]{%
	\setlength\fboxsep{0pt}%
	\noindent\colorbox{ocre!20}{\strut\parbox[c][.7cm]{\ecart}{\color{ocre!70}\Large\sffamily\bfseries\centering#1}}\hskip\esp\colorbox{ocre!40}{\strut\parbox[c][.7cm]{\linewidth-\ecart-\esp}{\Large\sffamily\centering#2}}%
}

% Unnumbered part in the table of contents
\newcommand{\@myparttocformat}[1]{%
	\setlength\fboxsep{0pt}%
	\noindent\colorbox{ocre!40}{\strut\parbox[c][.7cm]{\linewidth}{\Large\sffamily\centering#1}}%
}

\newlength\esp
\setlength\esp{4pt}
\newlength\ecart
\setlength\ecart{1.2cm-\esp}
\newcommand{\thepartimage}{}%
\newcommand{\partimage}[1]{\renewcommand{\thepartimage}{#1}}%
\def\@part[#1]#2{%
	\ifnum \c@secnumdepth >-2\relax%
		\refstepcounter{part}%
		\addcontentsline{toc}{part}{\texorpdfstring{\protect\@mypartnumtocformat{\thepart}{#1}}{\partname~\thepart\ ---\ #1}}
	\else%
		\addcontentsline{toc}{part}{\texorpdfstring{\protect\@myparttocformat{#1}}{#1}}%
	\fi%
	\startcontents%
	\markboth{}{}%
	{\thispagestyle{empty}%
		\begin{tikzpicture}[remember picture,overlay]%
			\node at (current page.north west){\begin{tikzpicture}[remember picture,overlay]%	
					\fill[ocre!20](0cm,0cm) rectangle (\paperwidth,-\paperheight);
					\node[anchor=north] at (4cm,-3.25cm){\color{ocre!40}\fontsize{220}{100}\sffamily\bfseries\thepart};
					\node[anchor=south east] at (\paperwidth-1cm,-\paperheight+1cm){\parbox[t][][t]{8.5cm}{
							\printcontents{l}{0}{\setcounter{tocdepth}{1}}% The depth to which the Part mini table of contents displays headings; 0 for chapters only, 1 for chapters and sections and 2 for chapters, sections and subsections
						}};
					\node[anchor=north east] at (\paperwidth-1.5cm,-3.25cm){\parbox[t][][t]{15cm}{\strut\raggedleft\color{white}\fontsize{30}{30}\sffamily\bfseries#2}};
				\end{tikzpicture}};
		\end{tikzpicture}}%
	\@endpart}
\def\@spart#1{%
	\startcontents%
	\phantomsection
	{\thispagestyle{empty}%
		\begin{tikzpicture}[remember picture,overlay]%
			\node at (current page.north west){\begin{tikzpicture}[remember picture,overlay]%	
					\fill[ocre!20](0cm,0cm) rectangle (\paperwidth,-\paperheight);
					\node[anchor=north east] at (\paperwidth-1.5cm,-3.25cm){\parbox[t][][t]{15cm}{\strut\raggedleft\color{white}\fontsize{30}{30}\sffamily\bfseries#1}};
				\end{tikzpicture}};
		\end{tikzpicture}}
	\addcontentsline{toc}{part}{\texorpdfstring{%
			\setlength\fboxsep{0pt}%
			\noindent\protect\colorbox{ocre!40}{\strut\protect\parbox[c][.7cm]{\linewidth}{\Large\sffamily\protect\centering #1\quad\mbox{}}}}{#1}}%
	\@endpart}
\def\@endpart{\vfil\newpage
	\if@twoside
		\if@openright
			\null
			\thispagestyle{empty}%
			\newpage
		\fi
	\fi
	\if@tempswa
		\twocolumn
	\fi}

%----------------------------------------------------------------------------------------
%	CHAPTER HEADINGS
%----------------------------------------------------------------------------------------

% A switch to conditionally include a picture, implemented by Christian Hupfer
\newif\ifusechapterimage
\usechapterimagetrue
\newcommand{\thechapterimage}{}%
\newcommand{\chapterimage}[1]{\ifusechapterimage\renewcommand{\thechapterimage}{#1}\fi}%
\newcommand{\autodot}{.}
\def\@makechapterhead#1{%
	{\parindent \z@ \raggedright \normalfont
			\ifnum \c@secnumdepth >\m@ne
				\if@mainmatter
					\begin{tikzpicture}[remember picture,overlay]
						\node at (current page.north west)
						{\begin{tikzpicture}[remember picture,overlay]
								\node[anchor=north west,inner sep=0pt] at (0,0) {\ifusechapterimage\includegraphics[width=\paperwidth]{\thechapterimage}\fi};
								\draw[anchor=west] (\Gm@lmargin,-9cm) node [line width=2pt,rounded corners=15pt,draw=ocre,fill=white,fill opacity=0.5,inner sep=15pt]{\strut\makebox[22cm]{}};
								\draw[anchor=west] (\Gm@lmargin+1.3cm,-9cm) node {\huge\sffamily\bfseries\color{black}\thechapter\autodot~#1\strut};
							\end{tikzpicture}};
					\end{tikzpicture}
				\else
					\begin{tikzpicture}[remember picture,overlay]
						\node at (current page.north west)
						{\begin{tikzpicture}[remember picture,overlay]
								\node[anchor=north west,inner sep=0pt] at (0,0) {\ifusechapterimage\includegraphics[width=\paperwidth]{\thechapterimage}\fi};
								\draw[anchor=west] (\Gm@lmargin,-9cm) node [line width=2pt,rounded corners=15pt,draw=ocre,fill=white,fill opacity=0.5,inner sep=15pt]{\strut\makebox[22cm]{}};
								\draw[anchor=west] (\Gm@lmargin+1.3cm,-9cm) node {\huge\sffamily\bfseries\color{black}#1\strut};
							\end{tikzpicture}};
					\end{tikzpicture}
				\fi\fi\par\vspace*{270\p@}}}

%-------------------------------------------

\def\@makeschapterhead#1{%
	\begin{tikzpicture}[remember picture,overlay]
		\node at (current page.north west)
		{\begin{tikzpicture}[remember picture,overlay]
				\node[anchor=north west,inner sep=0pt] at (0,0) {\ifusechapterimage\includegraphics[width=\paperwidth]{\thechapterimage}\fi};
				\draw[anchor=west] (\Gm@lmargin,-9cm) node [line width=2pt,rounded corners=15pt,draw=ocre,fill=white,fill opacity=0.5,inner sep=15pt]{\strut\makebox[22cm]{}};
				\draw[anchor=west] (\Gm@lmargin+1.3cm,-9cm) node {\huge\sffamily\bfseries\color{black}#1\strut};
			\end{tikzpicture}};
	\end{tikzpicture}
	\par\vspace*{270\p@}}
\makeatother

%----------------------------------------------------------------------------------------
%	LINKS
%----------------------------------------------------------------------------------------

\usepackage{hyperref}
% \hypersetup{hidelinks,backref=true,pagebackref=true,hyperindex=true,colorlinks=false,breaklinks=true,urlcolor=ocre,bookmarks=true,bookmarksopen=false}
\hypersetup{hidelinks,colorlinks,linkcolor=.,citecolor=blue,breaklinks=true,urlcolor=blue,bookmarksopen=false}

\usepackage{bookmark}
\bookmarksetup{
	open,
	numbered,
	addtohook={%
			\ifnum\bookmarkget{level}=0 % chapter
				\bookmarksetup{bold}%
			\fi
			\ifnum\bookmarkget{level}=-1 % part
				\bookmarksetup{color=ocre,bold}%
			\fi
		}
}


%----------------------------------------------------------------------------------------
% code show
%----------------------------------------------------------------------------------------
\usepackage[ruled,linesnumbered]{algorithm2e}
\usepackage{listings}
\lstset{% 
	language={C++}, %language为,还有{[Visual]C++}{[ISO]C++}
	alsolanguage=[ANSI]C, %可以添加很多个alsolanguage,如alsolanguage=matlab,alsolanguage=VHDL等
	tabsize=4, %
	basicstyle=\ttfamily\footnotesize, % 设置代码的大小
	keywordstyle=\color[RGB]{0,84,166}\bfseries, %代码关键字
	stringstyle=\ttfamily\color[RGB]{33,166,86}, % 代码字符串的特殊格式
	commentstyle=\color[RGB]{115,48,11}\scriptsize\rmfamily, %注释
	rulecolor=\color[RGB]{243,102,25},%代码边框
	frame=leftline, %代码框
	framerule=2pt,
	showstringspaces=false,%不显示代码字符串中间的空格标记
	keepspaces=true,
	breakindent=10pt,
	numbers=left,%左侧显示行号 往左靠,还可以为right,或none,即不加行号
	stepnumber=1,%若设置为2,则显示行号为1,3,5,即stepnumber为公差,默认stepnumber=1
	numberstyle={\color[RGB]{33,166,86}\scriptsize} ,%设置行号的大小,大小有tiny,scriptsize,footnotesize,small,normalsize,large等
	numbersep=8pt, %设置行号与代码的距离,默认是5pt
	showspaces=false, %
	flexiblecolumns=true, %
	breaklines=true, %对过长的代码自动换行
	breakautoindent=true,
	aboveskip=1em, %代码块边框
	tabsize=2,
	showstringspaces=false, %不显示字符串中的空格
	backgroundcolor=\color{black!5}, %代码背景色,或\color[rgb]{0.91,0.91,0.91}
	escapeinside=``, %在``里显示中文 %% added by http://bbs.ctex.org/viewthread.php?tid=53451
	fontadjust,
	captionpos=t,
	framextopmargin=2pt,framexbottommargin=2pt,abovecaptionskip=-3pt,belowcaptionskip=3pt,
	xleftmargin=0em,xrightmargin=0em, % 设定listing左右的空白
	texcl=true, % 设定中文冲突,断行,listing数字的样式
	extendedchars=false,% 设定中文冲突
	columns=flexible, % 列模式
	mathescape=true % 设定数学环境输入
}
\newcommand\codecolor[0]{\color[RGB]{142,12,242}} %设置格式

% https://tex.stackexchange.com/questions/17057/hypertarget-seems-to-aim-a-line-too-low
\makeatletter
\newcommand{\htarget}[2]{\Hy@raisedlink{\hypertarget{#1}{}}#2}
\makeatother

\newcommand\initcode[2]{\htarget{code:#1}{\codecolor\itshape{<<{#1}>>#2}}} %代码段名称定义
\newcommand\refcode[2]{\hyperlink{code:#1}{\codecolor\itshape{<<{#1}>>#2}}} %代码段引用跳转
\newcommand\initcnt[1]{\newcounter{#1}\newcounter{#1last}\newcounter{#1next}\setcounter{#1}{0}\setcounter{#1last}{-1}\setcounter{#1next}{1}} %设置三个计数器,记住当前以及前后的编号
\newcommand\addcnt[1]{\stepcounter{#1last}\stepcounter{#1next}\stepcounter{#1}} %三个各自计数器加1
\newcommand\nextcode[1]{\hyperlink{code:#1:\arabic{#1next}}{\htarget{code:#1:\arabic{#1}}{\codecolor $\downarrow$}}} %引用下一段代码
\newcommand\lastcode[1]{\hyperlink{code:#1:\arabic{#1last}}{\htarget{code:#1:\arabic{#1}}{\codecolor $\uparrow$}}} %引用上一段代码
\newcommand\initnext[1]{\initcnt{#1}\nextcode{#1}\addcnt{#1}} %初始化且引用下一段代码
\newcommand\lastnext[1]{\lastcode{#1}\nextcode{#1}\addcnt{#1}} %引用前后代码

% 代码变量定义、引用跳转
\newcommand\refvar[3][]{\ifthenelse{\isempty{#1}}{\hyperlink{codevar:#2}{\ttfamily #2#3}}{\hyperlink{codevar:#1}{\ttfamily #2#3}}}
\newcommand\initvar[3][]{\ifthenelse{\isempty{#1}}{\htarget{codevar:#2}{\ttfamily #2#3}}{\htarget{codevar:#1}{\ttfamily #2#3}}}

\newcommand\reffig[1]{图\ref{fig:#1}}
\newcommand\reftab[1]{表\ref{tab:#1}}
\newcommand\refeq[1]{式\ref{eq:#1}}
\newcommand\refsec[1]{\ref{sec:#1}节}
\newcommand\refchap[1]{\ref{chap:#1}章}
\newcommand\refsub[1]{\ref{sub:#1}节}

\usepackage{pifont}
\newcommand\circleone{\ding{172}}
\newcommand\circletwo{\ding{173}}
\newcommand\circlethree{\ding{174}}

\newcommand\compcolor[1]{\text{\itshape\sffamily\bfseries #1}}
\newcommand{\Equiv}{\ \tikz[baseline=-0.5ex]{\foreach \y in {0.45,0.15,-0.15,-0.45} \draw[yshift=\y ex] (0,0)--(1.5ex,0);}\ }
\usepackage{siunitx}
\usepackage{multirow} % Insert the commands.tex file which contains the majority of the structure behind the template

%\hypersetup{pdftitle={Title},pdfauthor={Author}} % Uncomment and fill out to include PDF metadata for the author and title of the book

%----------------------------------------------------------------------------------------

\begin{document}
\frontmatter % 控制序文部分
%----------------------------------------------------------------------------------------
%	TITLE PAGE
%----------------------------------------------------------------------------------------

\begingroup
\thispagestyle{empty} % Suppress headers and footers on the title page
\begin{tikzpicture}[remember picture,overlay]
    \node[inner sep=0pt] (background) at (current page.center) {\includegraphics[height=\paperheight]{view-3.png}};
    \draw (current page.center) node [fill=yellow!80!green!40,fill opacity=0.1,text opacity=1,inner sep=1cm]
    {\centering\bfseries\sffamily\parbox[c][][t]{\paperwidth}{\color{white}\centering
    {\Large Physically Based Rendering: From Theory To Implementation}\\[25pt]
    {\fontsize{30pt}{15pt}\textrm{从理论到实现}}\\[15pt]
    {\fontsize{60pt}{15pt}\textrm{基于物理的渲染}}\\[25pt]
    {\fontsize{21pt}{15pt}第三版}\\[20pt]
    {\Large\begin{tabular}{rcl}原著 && Matt Pharr\\ && Wenzel Jakob\\ && Greg Humphreys\\ 翻译 && Kanition\end{tabular}}}};
\end{tikzpicture}
\vfill
\endgroup

%----------------------------------------------------------------------------------------
%	COPYRIGHT PAGE
%----------------------------------------------------------------------------------------

\newpage
~\vfill
\thispagestyle{empty}
\usechapterimagefalse
\noindent \textbf{\LARGE 从理论到实现}\vspace{8pt}\\
\noindent \textbf{\Huge 基于物理的渲染}\vspace{8pt}\\
\noindent \textbf{\large 第三版}\vspace{8pt}\\
\noindent \textbf{\large 原著 \quad Matt Pharr, Wenzel Jakob \& Greg Humphreys}\vspace{5pt}\\
\noindent \textbf{\large 翻译 \quad Kanition}\vspace{16pt}\\

\noindent {\bfseries 英文原版}

\noindent Copyright \copyright\ 2004-2024 Matt Pharr, Wenzel Jakob, and Greg Humphreys

\noindent 官方网址:\url{https://www.pbr-book.org}

\noindent 许可证:CC BY-NC-SA 4.0\\

\noindent {\bfseries 本中译版}

\noindent Copyright \copyright\ 2021-2024 Kanition

\noindent 更新网址:\url{https://github.com/kanition/pbrtbook}

\noindent 许可证:CC BY-NC-SA 4.0

    {\small(详见:\url{https://creativecommons.org/licenses/by-nc-sa/4.0})}

    {\ttfamily\small\input{ver_info.txt}}

{\itshape
本中译版(以下简称“本书”)系译者(笔名 Kanition)自学英文经典书籍
《Physically Based Rendering: From Theory To Implementation》第三版时自行翻译而成。
使用本书及其源码须遵循相关许可证协议。

本书在翻译时遵照原书编排,译文尽力保留了原文词句,但因笔者水平有限,
而原文长句极多,故可能会存在病句甚至误翻,请读者见谅并指正。

此外,笔者根据自己的学习情况对内容作了补充,
例如自行编写补充章节、在边栏进行注释解说、修正一些笔误等。
除补充章节外,行文中凡是笔者自行变动或增添过的地方都有“译者注”的标记。

原书在线版本以网页形式呈现,可以方便地展开、折叠示例代码。
本书虽受到PDF格式限制,但依旧精心保留了代码链接跳转功能,方便读者查阅。
若读者在实践中还有更多需求,建议参考原书所附代码库。

本书由{\scshape \LaTeX} 编写而成,源码已经发布在上述网址,欢迎访问获取最新版。

{\color{red}\sffamily{欢迎提出宝贵意见和建议。如果你发现本书存在错误,请一定要告诉我们!
讨论区:{\normalfont\url{https://github.com/kanition/pbrtbook/discussions}}}}
}
\pagestyle{fancy} % 该风格启用了页眉

\chapter{前言}\label{chap:前言}
\setcounter{page}{1}
渲染是计算机图形学的基础组成部分。
最抽象地说,渲染是把三维场景描述转换为图像的过程。
动画算法、几何建模、材质贴图和计算机图形学其他领域
都须经某些渲染过程来可视化其结果。
从电影到游戏等,渲染无处不在,它为创作、娱乐和可视化开辟了新的领域。

早期的渲染研究重点解决基本问题,例如从给定视点确定哪些物体是可见的。
随着这些问题找到高效解法以及图形学其他领域的持续发展使得场景描述更加丰富逼真,
现代渲染已涵盖了广泛内容,包括物理学、天体物理学、天文学、生物学、心理学、感知研究、理论和应用数学。
渲染的跨学科性是它如此引人的原因之一。

本书以文档化代码的形式提供了构建一个完整的渲染系统所需的一批现代渲染算法。
包括封面在内\sidenote{译者注:本书封面作了更换,但和原书封面是同一组渲染结果。},
本书几乎所有图像都由该软件渲染得到。
且生成这些图像的全部算法均有描述。
该pbrt系统按{\itshape 文学编程}的程序设计方法编写,
即把对系统的描述和实现源码结合在一起。
我们认为,用文学编程法介绍计算机图形学和计算机科学是非常合适的。
算法的一些微妙细节在实现之前往往很难弄清楚,
因此读实际代码更有利于充分理解它们。
我们相信,深入理解哪怕少量的算法也比跑马观花更能打牢进一步研究计算机图形学的基础。

除了阐明实践中如何实现算法外,交代其在完整简单软件系统中的上下文
同样有助于解决中型渲染系统的设计和实现问题。
渲染系统的基本抽象和接口设计对实现的优雅性和可扩展性有实质影响,
但本书不会讨论这类设计取舍。

pbrt和本书内容仅关注{\itshape 逼真渲染},
它可定义为这样的图像生成任务:和相机拍摄的照片难以区分,
或者人类看后被激发的响应与看到实际场景时一致。
我们有许多理由关注逼真感。
逼真图像对电影特效工业至关重要,
因为计算机生成的图像经常必须和真实世界的镜头无缝结合。
娱乐应用中所有图像都是合成的,
逼真感是让观察者忘记所见场景并不实际存在的有效手段。
最后,逼真感为衡量渲染系统输出质量提供了定义合理的指标。

\section*{读者}\label{sec:读者}

本书主要面向三类读者。
第一类是学习计算机图形学课程的研究生或高年级本科生。
本书假定读者拥有大学入门级计算机图形学知识,
只会回顾一些关键概念,例如基本向量几何和变换。
对于没有编写过上万行源码程序的学生,
文学编程风格更能降低学习难度。
为了让读者领会为何要这样构建系统,
我们特别注意解释关键接口和抽象背后的设计考量。

第二类读者是计算机图形学研究人员。
本书为研究人员全面介绍了该领域,
pbrt源码提供了可用的构建基础(至少可使用一部分源码)。
对于其他领域的读者,
我们认为对透彻理解渲染也有助于了解相关背景。

最后一类读者是工业界软件开发者。
尽管这些读者可能很熟悉本书许多内容,
但阅读文学风格的算法解释也许能获取新的角度。
pbrt涵盖了大量高级或艰深算法的实现和技术,
例如细分曲面、蒙特卡罗采样算法、双向路径追踪、Metropolis采样和次表面散射;
经验丰富的渲染从业者应该会很感兴趣。
我们希望能激发这些读者去钻研一个完整而典型的渲染系统的兴趣。

\section*{概述和目标}\label{sec:概述和目标}

pbrt基于{\itshape 光线追踪}算法。
光线追踪是一项优雅的技术,起源于镜片制造。
19世纪Carl Friedrich Gau{\ss}就用透镜手动追踪光线。
计算机上的光线追踪算法跟随无穷小的光线穿过场景直到与曲面相交。
它给出了从特定位置和方向寻找第一个可见物体的简单方法,
这是许多渲染算法的基础。

pbrt的设计和实现贯彻了三个目标:{\itshape 完整性}、{\itshape 解说性}和基于{\itshape 物理性}。

完整性指系统不应缺少高质量商业渲染系统的关键功能。
这意味着要彻底解决重要的实际问题,
例如抗锯齿、稳定性、数值精度以及高效渲染复杂场景的能力。
在设计系统时一开始就应考虑到这些,
因为它们会对系统所有组件产生微妙影响,
且在实现后期阶段很难再改装到系统中。

第二个目标意味着我们着眼于可读性和清晰度,
精心选用算法、数据结构和渲染技术。
因为比起其他渲染系统,我们的实现要接受更多读者的检验,
所以我们尽力选择已知的最优算法并将其实现。
这个目标也要求系统要小到一个人能完全理解的程度。
我们用可扩展的架构实现了pbrt,
即系统核心采用精心设计的抽象基类,
且这些基类尽量实现足够多特定功能。
这样读者不用理解所有特定细节就能明白系统的基本结构。
这更易于钻研感兴趣的部分并跳过其他内容,
且不影响对系统整体配合的把握。

完整性和解说性目标之间是存在矛盾的。
涵盖所有可能有用的技术不仅让本书过于冗长,
而且对于大多数读者而言也太复杂。
针对万一pbrt缺少某项有用功能的情况,
我们尽量使架构便于增添功能而不用改变系统整体设计。

基于物理的渲染的基础是物理定律及其数学表达式。
pbrt的设计对所计算的量和实现的算法使用正确的物理单位。
这样配置后,pbrt能计算出{\itshape 物理正确}的图像;
它们像在真实世界场景中那样准确反映光照。
这样的好处是它为程序正确性提供了具体标准:
对于预期结果可用解析解计算的简单场景,
如果pbrt没有算出相同结果,我们就能知道实现一定有bug。
类似地,如果pbrt中基于物理光照的不同算法对同一场景给出了不同结果,
或者pbrt所得结果和另一个基于物理的渲染器不一致,
则它们中必有一个出错了。
最后,我们认为基于物理的渲染方法是有价值的,因为它是严格的。
当不清楚特定计算该如何执行时,物理学会给出确保一致的答案。

效率的优先度低于以上三个目标。
既然渲染系统生成一张图像通常要花费数分钟或小时,
效率显然是很重要的。
然而我们主要关注{\itshape 算法}层面的效率而非底层代码优化。
尽管系统中计算量集中的部分已尽力做了优化,
但有时明显而微小的优化会让位于清晰的代码组织。

在介绍pbrt和讨论其实现时,
我们希望传授多年来渲染研究和开发的经验教训。
编写好一个渲染器比串接一堆快速算法更需要付出;
让系统既灵活又稳定是项困难的任务。
随着增添越来越多的几何体或光源,
或者其他复杂维度上升,
系统的性能将逐渐下降。
严谨处理数值稳定性、
算法不浪费浮点精度也至关重要。

开发出解决所有这些问题的系统大有益处——
编写新的渲染器或向已有渲染器添加新功能并用它创作出以往无法生成的图片是多么地快乐。
我们编写本书最基本的目标就是给广大读者这样的机会。
我们鼓励读者在阅读本书时使用该系统渲染pbrt发行的示例场景。
每章末的习题会要求修改系统以加深对内部工作原理的理解,
或者完成添加新功能等更复杂的工程。

本书官网为\href{www.pbrt.org}{\ttfamily pbrt.org},
可从该站获取pbrt最新版源码。
我们也会发布勘误、修复bug、新增渲染场景和补充材料。
遇到网站尚未列出的pbrt中的任何bug或行文错误
均发送到邮箱\href{mailto:bugs@pbrt.org}{\url{bugs@pbrt.org}}。
我们非常重视您的反馈\sidenote{译者注:我也欢迎您的反馈!详见扉页更新网址。}!

\section*{第一版和第二版的区别}\label{sec:第一版和第二版的区别}

{\itshape 详见英文原版}

\section*{第二版和第三版的区别}\label{sec:第二版和第三版的区别}

{\itshape 详见英文原版}

\section*{致谢}\label{sec:致谢}

{\itshape 详见英文原版}

\section*{出版}\label{sec:出版}

{\itshape 详见英文原版}

\section*{场景和模型}\label{sec:场景和模型}

{\itshape 详见英文原版}

\section*{关于封面}\label{sec:关于封面}

{\itshape 详见英文原版}

\section*{扩展阅读}\label{sec:扩展阅读0}

\citet{10.1093/comjnl/27.2.97}的论文《\emph{Literate Programming}》
描述了文学编程背后的主要思想以及他的{\ttfamily web}编程环境\sidenote{译者注:一种计算机编程语言系统。}。
开创性的\TeX 排版系统是用网络写成的并出版了一系列书籍\citep{10.5555/536126,10.5555/536123}。
最近,\citet{10.1145/164984}在《\citetitle{10.1145/164984}》中
以文学格式出版了图表算法集。
这些程序读起来很有趣,各个算法也展示得很好。
网站\url{www.literateprogramming.com}指向了许多关于文学编程的论文、程序以及大量系统;
自Knuth最初提出该思想以来,文学编程已经进行了许多改进。


我们所知的其他出版成书的文学程序只有对lcc编译器的实现——
由\citet{10.5555/555424}编写并出版的《\citetitle{10.5555/555424}》,
以及\citet{10.5555/1036653}关于MP3音频格式的书《\citetitle{10.5555/1036653}》。



\chapter{在线版序言}\label{chap:在线版序言}

2004年发行的第一版《基于物理的渲染》只有纸质书。
2010年第二版新增了Kindle版,但不幸的是
所有交叉引用和索引都在转换中丢失了。
终于,2016年发布的第三版转换出了良好的Kindle版和PDF版。
尽管电子版有所改进,但我们觉得它还远称不上完美。

文学编程是《基于物理的渲染》的核心。
它是Donald Knuth提出的一种软件编写方法,
比起在计算机上把源码转换为可执行指令,
它更重视人类阅读源码时的直观性。
文学编程将复杂程序分解为便于理解的片段,
并提供多种方式对其交叉引用,以帮助理解每个片段的内容。

对于纸上的文学编程,每页都有丰富的辅助信息并编有定向页码。
边栏有索引指向当前页所用标识符对应的代码定义所在的页码,
并且每个代码段都有表示其它部分定义所在页码和被引用的页码。

这种格式很有效,但页码绕得烦人。
而且翻书找页也很麻烦。
我们想到,如果把电子设备——台式机、笔记本、平板甚至手机——
作为本书内容的主要交付工具,我们会获得怎样的阅读体验?
没有页码了,取而代之的是超链接,直接把读者引向目标且容易返回原处。

改善的不仅只有导航:
现代显示器比打印纸有好得多的色彩保真度和动态范围,
结合计算机使用还能与书中图示交互。
对于通篇在讲图像与三维世界的书籍而言这是极大的优点。

2018年夏季,我们获取到出版商的授权;
我们非常感谢他们慷慨归还版权。
这样我们就能自由决定是否尝试以这种新形式呈现本书内容。
我们的答案是肯定的。
废寝忘食一个月后,
我们实现了一个系统,
把本书从以前编写时所用的标记语言转换为HTML。
你现在读的就是它\sidenote{译者注:原作者可能没料到有人又把它翻译回PDF了。}。

本书在线版与第三版《基于物理的渲染》很接近。
我们只作了以下修改:
\begin{itemize}
    \item 更新一些插图所用的渲染图像,
    \item 为图像查看增加交互,
    \item 重画所有插图,
    \item 把比较同一场景的不同渲染结果的多幅图像合并为一幅图像,
    \item 合并读者反馈的勘误。
\end{itemize}

前两点还需要说明一下。关于更新图像:
纸质书的一大挑战是确保诸如蒙特卡罗噪声等图像痕迹在页面上可见。
我们担心印刷会引入多余的模糊,
也担心成书过程中有好心人帮倒忙给图像降噪使其看起来更清晰。
因此我们用最近邻滤波器放大了本应展现差错的图像,
使这些差错能在印刷过程中保留下来。

现在这个担心是多余的了。
很高兴能重新渲染这些图像,
连一个像素大小的差错都能保留了。

第二点修改标志着我们朝交互式内容探究迈出了第一步:
在网页浏览器中,可以细究渲染图像的细节和区别,
这是纸质书做不到的。
在线版大多数渲染图像都能放大、全屏查看以及和其他图像比较。
它们都有此图标
\sidenote{译者注:在线版是一片雪花图案。
    不过PDF没法实现网页端那么强大的交互功能,
    所以阅读本书时就无视它吧。}:*。
将鼠标悬停在该图标上可获取相关操作的详细信息。

\section*{路线图}\label{sec:路线图}

本书计划大致每年发布一次更新
\sidenote{译者注:本书在翻译时已经发布第四版书籍与代码了。}。
尽管在线版比纸质版更容易快速更新,
我们还是认为适当的更新速度有利于对下次发布做严谨的检查和编辑。

除了扩展pbrt功能跟进最新研究,
我们还计划为新版增加更多交互元素。
\citet{4b212a02-105c-42a2-ad5c-91c16a06e815}
编写的《\citetitle{4b212a02-105c-42a2-ad5c-91c16a06e815}》\footnote{\citeurl{4b212a02-105c-42a2-ad5c-91c16a06e815}}
一书展现了这种媒体的无限可能。

我们会在线保留本书的早前版本,URL均和首发时保持一致;
新版会放在单独的目录中。
因此链接到此处的内容是安全的,不必担心未来断链
\sidenote{译者注:本中译版不作此承诺。}。

\section*{报告错误}\label{sec:报告错误}

{\itshape 详见英文原版}

\section*{致谢}\label{sec:致谢a}

{\itshape 详见英文原版}

\section*{许可证}\label{sec:许可证}

{\itshape 详见英文原版}

{\Huge\bfseries 译者序}\vspace{30pt}\\

“为什么要翻译这本书?”
这是自我翻译本书以来被问过的最多的问题。
同学问过、父母问过、连面试官也问过。
我回答过各种各样的理由,
甚至连自己也不知道哪个理由才是最重要最关键的。

我曾经有过比较低谷的学业经历,
这主要是心中的完美主义作祟。
当主业不能使自己获得一丝满足时,
我不由得把目光投向了其他领域——渲染便是其中之一。
之后很快我便在GitHub上偶遇了pbrt-v4项目。
虽然这并不是我第一次与渲染相遇了——

若干年前读初中的时候,家里终于有了第一台电脑。
那个时候老家电脑并不算普及。爸爸为了学会怎么用
还得去书店买本《电脑入门基础教程》之类的书回来查阅。
我在学校微机课倒是学会了开关机和使用开始菜单,所以对这些书已没多大兴趣。
不过我在这些教程书区中发现了其他宝贝——与电脑相关的视觉设计类书籍,
就是那些封面图片非常有“质感”的与渲染相关的书,厚度通常也很离谱。
同龄人如果家里新买了电脑,几乎都会兴奋地琢磨怎么安装上时下最火的游戏好好玩一把,
而我却被这些书“带偏”了——虽然完全看不懂其中的内容,甚至不知道
里面偶尔出现的单词“run”是“运行”的意思而不是“奔跑”,
但书中花花绿绿的模型和最终展现的成品渲染图深深地吸引了我。
在电脑里生成一幅现实生活中不存在但分毫毕现的图像,
这样的技术对于一个刚刚学会怎么计算有理数的学生而言是非常震撼的。
我让爸爸买回其中一本,回去对照着折腾起怎么安装3ds Max\textsuperscript{\textregistered}
和V-Ray\textsuperscript{\textregistered},
然后磕磕绊绊地按书中步骤设置材质并花三个多小时渲染出一幅室内装修图。
现在想来,当年那台连独显都没有的电脑承受了太多不该它承受的计算开销。
我也只渲染过那么一次——模型、材质都是附带光盘里的,
照着做了一遍后我也学不会什么,只是觉得好玩。
也许那时候对渲染的兴趣种子就这么埋进土里了。

所以当我与它再会时,曾经的回忆便苏醒了。
那时神经网络的黑箱特性让我厌倦,
而渲染技术明晰的数学原理如同命中十环那般精准地满足了我的口味,
让我有一种从迷茫中解脱的释然感。
事实上,我一直有完成一部“作品”的愿望。
在我心里“作品”这个词是有很高门槛的。
我发表了主业相关的论文,但我并不喜欢它——那远远算不上“作品”。
作为骨灰级动漫爱好者,我甚至觉得自己倾注情感剪辑
的MAD(对动画原片进行重新剪辑配乐做成的视频)更算得上“作品”。
何况动漫粉丝的身份让我对渲染技术的滤镜又加深了一层。
在确认pbrt是一套可以自学搞定的完整教程后,
我认定这本内容详实的著作就是我心中追求的“作品”的模样。
我下定决心以翻译的方式学习它,也许要很久,但不怕学不懂。
毕竟检验是否学会的最好办法就是看能否教会别人。
毫不避讳地说,我就是希望从这个过程中获取自我认同,满足完美主义心理。

这本书翻译起来确实不轻松,曾经我只能抽课余时间写,现在只能抽业余时间写。
越写就越觉得原作者能把如此丰富的内容无偿公开是何其慷慨,
所以我效仿着不对获取译本设任何门槛——而且许可证也不允许。
知识本应自由而充分地交流。我崇尚开源精神,并通过这种方式为其贡献力量。
比起从中获取什么经济利益,我更看中是否有人因此而顺利入门渲染技术,成为又一个充满潜力的领域新人。
当然坦诚地说,我没有从翻译本书中获得个人利益是不可能的——
我收获了一些零散的朋友圈点赞满足了自己的虚荣心,
还把翻译经历写进简历里帮助自己熬过了求职关卡。
这也算是对自己一点小小的犒劳吧。

最后是一点关于匿名发表的解释。
“为什么不用真名而是用笔名Kanition署名译作?”
“对你找工作有什么用吗?”
“不怕别人冒充你去攫取个人名利吗?”
“花了这么多心血难道不希望自己的名字在业内传开吗?”
我当然想!可是我起初并不知道自己能翻译到什么水平。
要知道当初我读第一章面对那么多陌生概念是非常痛苦的。
与其搞砸后被人拿去嘲笑黑历史,不如留下一丝神秘感,
像江湖上不见其人的侠者一般只留下一个名号。
此外,这个笔名对我而言有特殊的意义。
它改编自某家我十分钟爱的动画公司名。
在我心中,这家公司就是耕耘“作品”的代表,
正是它制作的动画让我撑过翻译本书之前那段煎熬的岁月。
然而一场人祸夺取了许多鲜活的生命,那些作品成为了永恒……
无论是渲染还是动画,它们都在构建一个更美丽纯粹的世界,
Kanition这个名字正是代表着为此努力的人。
所以即便这本译作将来获得好评,我也不会改回真名署名,
这是我为数不多的纪念方式了。

\vspace{15pt}
{\hfill {\itshape 译者 Kanition}\qquad}

\vspace{15pt}
\noindent{\LARGE\bfseries 致谢}

感谢\href{https://zixuan-zhang.com}{Zixuan Zhang}参与翻译\refsec{相机模型}部分段落;
感谢\href{https://github.com/xwc2021}{xwc2021}对\refsub{辐射度量}一节的讨论;
感谢\href{https://github.com/OverflowCat}{OverflowCat}、
\href{https://www.zhihu.com/people/zhong-ling-xiao}{钟凌霄}、
\href{https://github.com/star-hengxing}{star-hengxing}、
\href{https://theigrams.github.io/}{张晋}、
\href{https://github.com/leemeans}{leemeans}、
\href{https://www.zhihu.com/people/itact/posts}{DeJhon-Huang}改进部分词句。

\begin{center}
      {\LARGE\bfseries 关于本仓库拒绝刊载于GitCode\\及其关联网站的声明}
\end{center}\vspace{30pt}

鉴于GitCode及其关联网站的过往劣迹,
尤其是近来未经授权大量镜像GitHub高流量账号和仓库,
故意伪造出原作者在其网站上开设账号和发布内容的假象,
并使用AI手段批量生成质量堪忧的引流推广软文,以此为网站牟取商业利益,
本仓库作者(署名Kanition)在此严正声明:

\begin{enumerate}
      \item 强烈谴责这些网站违背开源精神、污染开源社区环境、冒充开源作者身份、
            窃取开源作品成果、误导广大用户的无耻行径。
            强调任何举措都应遵守相应的许可协议、通行的开源社区准则与基本的社会公共道德。
            敦促有关实体和个人悬崖勒马,立即停止侵权行为。
      \item {\bfseries 未经本人许可,本仓库拒绝以任何形式被网站官方刊载在GitCode及其关联网站上。
            对于已经刊载的内容,本人要求立即无条件进行下架与彻底删除处理。}
      \item 本声明中的“刊载”形式包括但不限于由网站官方
            (包括但不限于其操控的公共账号、下属员工个人账号和伪装成普通用户的机器账号)
            镜像发布本仓库的源码、编译产物、提交记录、问题与计划列表、讨论区记录等的全部或部分内容,
            以及发布任何提及本仓库的内容。
      \item 本声明中的“GitCode及其关联网站”包括:
            \begin{itemize}
                  \item GitCode:gitcode.com
                  \item 与GitCode实际上存在或疑似存在用户数据共享的网站,
                        或由GitCode的实控组织机构或个人开办的其他网站,包括但不限于:
                        \begin{itemize}
                              \item CSDN:csdn.net
                              \item OSCHINA:oschina.net
                              \item Gitee:gitee.com
                              \item 华为云:huaweicloud.com
                              \item InsCode:inscode.csdn.net
                        \end{itemize}
                  \item 以上述网站名义在其他网络平台上开办的公共账号空间,
                        包括但不限于微博、微信公众号、知乎、抖音、bilibili、
                        百度百家号、今日头条、简书、YouTube、X(原Twitter)、GitLab、极狐等。
            \end{itemize}
      \item 普通个人用户(上述网站的实控组织机构成员除外)可在遵守许可协议的前提下
            自由地于上述网站发布本仓库相关内容而不受前述条款限制。
      \item {\bfseries 任何用户发布与本仓库相关的内容都不得冒用本人署名“Kanition”,也不得使用极其相似的署名误导用户。}
      \item 呼吁上述实体和个人发扬开源共享精神,维护开源社区良好秩序,共同推动开源事业高质量发展。
      \item 本人保留依据事态后续发展作出进一步反应并调整本声明内容的权利。
\end{enumerate}


%----------------------------------------------------------------------------------------
%	TABLE OF CONTENTS
%----------------------------------------------------------------------------------------
\pagestyle{fancy}
\renewcommand{\contentsname}{目录}
\renewcommand{\figurename}{图}
\renewcommand{\tablename}{表}

%\usechapterimagefalse % If you don't want to include a chapter image, use this to toggle images off - it can be enabled later with \usechapterimagetrue
\usechapterimagetrue
\chapterimage{Pictures/measure-one180-cut1260.png} % Table of contents heading image

\pagestyle{empty} % Disable headers and footers for the following pages

\tableofcontents % Print the table of contents itself

\cleardoublepage % Forces the first chapter to start on an odd page so it's on the right side of the book

\pagestyle{fancy} % Enable headers and footers again

%----------------------------------------------------------------------------------------
%	PART
%----------------------------------------------------------------------------------------
\setcounter{page}{1}
\mainmatter % 控制正文部分
\part{绪论}
\input{content/chap01.tex}

\part{主要几何功能}
\input{content/chap02.tex}

\input{content/chap03.tex}

\input{content/chap04.tex}

\part{成像过程}
\chapterimage{Pictures/chap05/measure-one180-1260x630.png}
\chapter{颜色和辐射度学}\label{chap:颜色和辐射度学}
\setcounter{sidenote}{1}

为了精确描述光是怎样被表示和采样以计算图像的,
我们必须首先建立一些\keyindex{辐射度学}{radiometry}{}的背景——
\keyindex{电磁辐射}{electromagnetic radiation}{}在环境中的传播研究。
渲染中尤其感兴趣的是\keyindex{波长}{wavelength}{}($\lambda$)约在
380nm到780nm的电磁辐射\sidenote{译者注:{\normalfont nm}即长度单位纳米。$1$纳米=$10^{-9}$米。},
主要是人眼可见光\footnote{可感知波长的完整范围稍微超出了该区间,
但眼睛在这些波长上的敏感性低了许多量级。当绘制光谱曲线时常把范围360-830nm用作保守边界。}。
较短波长($\lambda\approx400\text{nm}$)是偏蓝色的,
中间波长($\lambda\approx550\text{nm}$)是绿色的,
而较长波长($\lambda\approx650\text{nm}$)是红色的。

本章中,我们将介绍描述电磁辐射的四个关键量:
\keyindex{通量}{flux}{}、\keyindex{强度}{intensity}{}、
\keyindex{辐射照度}{irradiance}{}和\keyindex{辐射亮度}{radiance}{}。
这些辐射量每一个都以它们的\keyindex{光谱功率分布}{spectral power distribution}{}(SPD)来描述,
即在各波长上的光量关于波长的分布函数。
pbrt中用定义于\refsec{光谱表示}的类\refvar{Spectrum}{}来表示SPD。

\input{content/chap0501.tex}

\input{content/chap0502.tex}

\input{content/chap0503.tex}

\input{content/chap0504.tex}

\input{content/chap0505.tex}

\section{表面反射}\label{sec:表面反射}

当光入射到表面时,表面会散射该光,将其一部分反射回环境中。
有两个需要描述的效应以对反射建模:反射光的光谱分布和其方向分布。
例如,柠檬皮大都吸收了蓝波长的光而反射了大部分红和绿波长的光
(回想\reffig{5.1}中柠檬皮的反射SPD)。
因此,当用白光照射它时,其颜色是黄色。
无论从哪个方向观察,皮的颜色都相当一致,
但有的方向会有\keyindex{高光}{highlight}{}——会看见与其说黄色不如说白色的更亮区域
\sidenote{译者注:有高光区的柠檬照片。}。
\begin{marginfigure}
    \includegraphics[width=\linewidth]{chap05/lemon.jpg}
\end{marginfigure}
相反,镜子一点反射的光几乎完全取决于观察方向。
在镜子的固定点上,当观察角度变化时,镜子反射的物体也随之变化。

来自\keyindex{半透明}{translucent}{}表面的反射更复杂;
从皮和叶子到蜡和液体的各种材料都
表现出\keyindex{次表面光传输}{subsurface light transport}{light transport光传输},
即进入表面一点的光在有一定距离的地方退出。
(例如考虑在一个人的嘴巴里开手电筒会让他的脸颊被照亮,
因为进入脸颊内侧的光穿过了皮肤并从脸上退出。)

有两种抽象来为光的反射描述这些机制:
\refsub{BRDF}和\refsub{BSSRDF}分别介绍的BRDF和BSSRDF。
BRDF描述一点的表面反射而忽略次表面光传输效应;
对于不受该传输机制明显影响的材料,
这一简化会减少报错并让渲染算法的实现高效得多。
BSSRDF推广了BRDF并描述来自半透明材料光反射的更一般设置。

\subsection{BRDF}\label{sub:BRDF}
\keyindex{双向反射分布函数}{bidirectional reflectance distribution function}{}(BRDF)
为描述来自表面的反射给出了形式。考虑\reffig{5.18}中的设置:
我们想知道,作为沿方向${\bm\omega}_{\mathrm{i}}$
入射辐亮度$L_{\mathrm{i}}({\bm p},{\bm\omega}_{\mathrm{i}})$的结果,
在朝向观察者的方向${\bm\omega}_{\mathrm{o}}$中
有多少辐射亮度$L_{\mathrm{o}}({\bm p},{\bm\omega}_{\mathrm{o}})$离开表面。
\begin{figure}[htbp]
    \centering\includegraphics[width=0.5\linewidth]{chap05/BRDF.eps}
    \caption{BRDF。双向反射分布函数是在一对方向${\bm\omega}_{\mathrm{i}}$
        和${\bm\omega}_{\mathrm{o}}$上描述有多少沿${\bm\omega}_{\mathrm{i}}$
        入射的光从表面朝方向${\bm\omega}_{\mathrm{o}}$散射的4D函数。}
    \label{fig:5.18}
\end{figure}

如果方向${\bm\omega}_{\mathrm{i}}$视作方向的微分锥,则$\bm p$处的微分辐照度是
\begin{align}\label{eq:5.7}
    \mathrm{d}E({\bm p},{\bm\omega}_{\mathrm{i}})=L_{\mathrm{i}}({\bm p},{\bm\omega}_{\mathrm{i}})\cos\theta_{\mathrm{i}}\mathrm{d}{\bm\omega}_{\mathrm{i}}\, .
\end{align}

要被反射到方向${\bm\omega}_{\mathrm{o}}$的辐射亮度微分量取决于该辐射照度。
因为几何光学的线性假设,反射的微分辐射亮度正比于辐射照度
\begin{align*}
    \mathrm{d}L_{\mathrm{o}}({\bm p},{\bm\omega}_{\mathrm{o}})\propto\mathrm{d}E({\bm p},{\bm\omega}_{\mathrm{i}})\, .
\end{align*}

比例常数为这对特定方向${\bm\omega}_{\mathrm{i}}$和${\bm\omega}_{\mathrm{o}}$定义了曲面的BRDF:
\begin{align}\label{eq:5.8}
    f_{\mathrm{r}}({\bm p},{\bm \omega}_\mathrm{o},{\bm \omega}_\mathrm{i})=\frac{\mathrm{d}L_{\mathrm{o}}({\bm p},{\bm\omega}_{\mathrm{o}})}{\mathrm{d}E({\bm p},{\bm\omega}_{\mathrm{i}})}=\frac{\mathrm{d}L_{\mathrm{o}}({\bm p},{\bm\omega}_{\mathrm{o}})}{L_{\mathrm{i}}({\bm p},{\bm\omega}_{\mathrm{i}})\cos\theta_{\mathrm{i}}\mathrm{d}{\bm\omega}_{\mathrm{i}}}\, .
\end{align}

基于物理的BRDF有两个重要性质
\sidenote{译者注:事实上它还有非负性。}:
\begin{enumerate}
    \item \keyindex{互易性}{reciprocity}{}:对所有方向对${\bm\omega}_{\mathrm{i}}$和${\bm\omega}_{\mathrm{o}}$,
          $f_{\mathrm{r}}({\bm p},{\bm \omega}_\mathrm{i},{\bm \omega}_\mathrm{o})=f_{\mathrm{r}}({\bm p},{\bm \omega}_\mathrm{o},{\bm \omega}_\mathrm{i})$.
    \item {\sffamily 能量守恒}:光反射的总能量少于或等于入射光的能量。
          对于所有方向${\bm\omega}_{\mathrm{o}}$,
          \begin{align*}
              \int\limits_{H^2({\bm n})}f_{\mathrm{r}}({\bm p},{\bm \omega}_\mathrm{o},{\bm \omega}')\cos\theta'\mathrm{d}{\bm\omega}'\le1\, .
          \end{align*}
\end{enumerate}

曲面的\keyindex{双向透射分布函数}{bidirectional transmittance distribution function}{}
(BTDF)描述透射光的分布,可以用和BRDF一样的方法定义。
BTDF一般表示为$f_{\mathrm{t}}({\bm p},{\bm \omega}_\mathrm{o},{\bm \omega}_\mathrm{i})$,
其中${\bm\omega}_{\mathrm{i}}$和${\bm\omega}_{\mathrm{o}}$在绕$\bm p$的相对半球内。
要注意的是,BTDF不遵循上面定义的互异性;
我们将在\refsec{镜面反射与透射}和\refsub{非对称散射}详细讨论该问题。

为了等式的方便,我们把统一考虑时的BRDF和BTDF表示为$f({\bm p},{\bm \omega}_\mathrm{o},{\bm \omega}_\mathrm{i})$;
我们称之为\keyindex{双向散射分布函数}{bidirectional scattering distribution function}{}(BSDF)。
第\refchap{反射模型}将完全专注于描述对渲染有用的各种BSDF。

利用BSDF的定义,我们有
\begin{align*}
    \mathrm{d}L_{\mathrm{o}}({\bm p},{\bm\omega}_{\mathrm{o}})=f({\bm p},{\bm \omega}_\mathrm{o},{\bm \omega}_\mathrm{i})L_{\mathrm{i}}({\bm p},{\bm\omega}_{\mathrm{i}})|\cos\theta_{\mathrm{i}}|\mathrm{d}{\bm\omega}_{\mathrm{i}}\, .
\end{align*}
这里给项$\cos\theta_{\mathrm{i}}$加上了绝对值。
这样做是因为pbrt中曲面法线并没有调整为和${\bm\omega}_{\mathrm{i}}$位于曲面同侧
(许多其他渲染系统都这样做,但我们发现让它们留在原来\refvar{Shape}{}给出的自然朝向会更有用)。
这样做更容易一致地在系统别处运用如“假设曲面法线指向曲面外侧”那样的约定。
因此,像这样给项$\cos\theta_{\mathrm{i}}$加上绝对值保证了实际计算出所需的数量。
我们可以在绕$\bm p$的入射方向球内对该等式积分以计算由于各个方向对$\bm p$的照射
而得到的沿方向${\bm\omega}_{\mathrm{o}}$的出射辐亮度:
\begin{align}\label{eq:5.9}
    L_{\mathrm{o}}({\bm p},{\bm\omega}_{\mathrm{o}})=\int\limits_{S^2}f({\bm p},{\bm \omega}_\mathrm{o},{\bm \omega}_\mathrm{i})L_{\mathrm{i}}({\bm p},{\bm\omega}_{\mathrm{i}})|\cos\theta_{\mathrm{i}}|\mathrm{d}{\bm\omega}_{\mathrm{i}}\, .
\end{align}
这是渲染中的基本方程;它描述了一点的入射光分布
是怎样基于表面的散射性质转化为出射分布的。
当(像这里)球$S^2$作为积分域时它常常称为\keyindex{散射方程}{scattering equation}{},
当只在上半球$H^2({\bm n})$积分时则称\keyindex{反射方程}{reflection equation}{}。
\refchap{光传输I:表面反射}和\refchap{光传输III:双向方法}中积分例程的
关键任务之一就是计算场景中曲面上的点的该积分值。

\subsection{BSSRDF}\label{sub:BSSRDF}
\keyindex{双向散射表面反射分布函数}{bidirectional scattering surface reflectance distribution function}{}(BSSRDF)
是描述展现出明显程度次表面光传输的材料之散射的形式。
它是一个描述点${\bm p}_{\mathrm{o}}$在方向${\bm\omega}_{\mathrm{o}}$的出射微分辐亮度
与点${\bm p}_{\mathrm{i}}$来自方向${\bm\omega}_{\mathrm{i}}$的入射微分通量之比
的分布函数$S({\bm p}_{\mathrm{o}},{\bm\omega}_{\mathrm{o}},{\bm p}_{\mathrm{i}},{\bm\omega}_{\mathrm{i}})$
(\reffig{5.19}):
\begin{align}\label{eq:5.10}
    S({\bm p}_{\mathrm{o}},{\bm\omega}_{\mathrm{o}},{\bm p}_{\mathrm{i}},{\bm\omega}_{\mathrm{i}})=\frac{\mathrm{d}L_{\mathrm{o}}({\bm p}_{\mathrm{o}},{\bm\omega}_{\mathrm{o}})}{\mathrm{d}\varPhi({\bm p}_{\mathrm{i}},{\bm\omega}_{\mathrm{i}})}\, .
\end{align}
\begin{figure}[htbp]
    \centering\includegraphics[width=0.5\linewidth]{chap05/BSSRDF.eps}
    \caption{双向散射表面反射分布函数推广了BSDF以考虑光在不同于其入射处的一点离开表面。
        尽管次表面光传输对许多真实世界物体的外观做出了巨大贡献,但它常常比BSDF更难算。}
    \label{fig:5.19}
\end{figure}

BSSRDF散射方程的推广需要在曲面面积\emph{以及}入射方向上积分,
将2D散射方程\refeq{5.9}转化为4D积分。
伴随着新增的两个积分维度,在渲染算法中使用它要复杂得多。
\begin{align}\label{eq:5.11}
    L_{\mathrm{o}}({\bm p}_{\mathrm{o}},{\bm\omega}_{\mathrm{o}})=\int\limits_A\int\limits_{H^2({\bm n})}S({\bm p}_{\mathrm{o}},{\bm\omega}_{\mathrm{o}},{\bm p}_{\mathrm{i}},{\bm\omega}_{\mathrm{i}})L_{\mathrm{i}}({\bm p}_{\mathrm{i}},{\bm\omega}_{\mathrm{i}})|\cos\theta_{\mathrm{i}}|\mathrm{d}{\bm\omega}_{\mathrm{i}}\mathrm{d}A\, .
\end{align}

随着点${\bm p}_{\mathrm{i}}$和${\bm p}_{\mathrm{o}}$的距离增加,$S$的值一般会减小。
这一事实对次表面散射算法的实现可以有很大帮助。

描述表面之下光传输的原则和介质中体积光传输方程一样,
由\refsec{转移方程}介绍的转移方程描述。
因此次表面散射基于和烟雾中光散射一样的效应,只是尺度更小。

\input{content/chap0507.tex}

\input{content/chap0508.tex}

\section{译者补充:辐射度学、光度学与色度学}\label{sec:译者补充:辐射度学、光度学与色度学}

\begin{remark}
      本节内容不是原书内容,而是译者根据有关资料\citep{978-7-5640-0658-7,
            wiki:solidangle,GB3102.6-93,enwiki:1052681830,enwiki:SRGB,
            wiki:candela,Hoffmann2015,wiki:eye,BERTALMIO2020131}补充的,请酌情参考和斧正。
\end{remark}

\subsection{辐射度量}\label{sub:辐射度量}
立体角表示一个物体对特定点的三维空间角度,是平面角在三维空间中的类比。
它描述在某一点观测到的物体大小尺度。
例如对于一特定观察点,一个在该点附近的小物体
可能和一个远处的大物体有着相同的立体角。
\begin{definition}\label{definition:SolidAngle}
      锥体的\keyindex{立体角}{solid angle}{}大小定义为:
      以锥体的顶点为球心作球面,该锥体在球表面截取的面积与球半径平方之比。
\end{definition}
立体角的单位为\keyindex{球面度}{steradian}{}(sr),是无量纲的导出单位。

\begin{figure}[htbp]
      \centering\includegraphics[width=0.4\linewidth]{chap05/ex-solidangle.eps}
      \caption{立体角的定义。}
      \label{fig:5.ex01}
\end{figure}

立体角常用字母$\varOmega$表示。\reffig{5.ex01}展示了一种简单的情形。
以$\bm p$为顶点的微分锥体${\bm p}-BCDF$在同样以$\bm p$为球心且半径为$r$的球面上截取下面元$BCDF$.
建立以$\bm p$为原点的球坐标系,则弧线$\wideparen{BC}$的\keyindex{方位角}{azimuth angle}{}
(在$xy$平面上的投影与原点连线和$x$正半轴所成角)为$\varphi$,
$\wideparen{DF}$方位角为$\varphi+\mathrm{d}\varphi$;
弧线$\wideparen{BF}$的\keyindex{天顶角}{zenith angle}{}(与原点连线和$z$正半轴所成角)为$\theta$,
$\wideparen{CD}$的天顶角为$\theta+\mathrm{d}\theta$.
因此点$B$到其在$z$轴投影点$G$的距离为$r\sin\theta$,
$\wideparen{BF}$的弧长为$r\sin\theta\mathrm{d}\varphi$;
同时$\wideparen{BC}$的弧长为$r\mathrm{d}\theta$.
将面元$BCDF$视作矩形,可求得其微分面积为
\begin{align}
      \mathrm{d}A=r\sin\theta\mathrm{d}\varphi\cdot r\mathrm{d}\theta\, .
\end{align}
依据定义,则相应的立体角元为
\begin{align}
      \label{eq:05ex-solidangle}
      \mathrm{d}\varOmega=\frac{\mathrm{d}A}{r^2}=\sin\theta\mathrm{d}\theta\mathrm{d}\varphi\, .
\end{align}
对立体角元做曲面积分则可得立体角
\begin{align}
      \varOmega=\iint\limits_S \mathrm{d}\varOmega=\iint\limits_S \sin\theta\mathrm{d}\theta\mathrm{d}\varphi\, .
\end{align}

\begin{corollary}
      封闭曲面对于其内任意一点的立体角均为$4\pi$sr.
\end{corollary}

在连续意义下,我们定义以下辐射度量。

\begin{definition}
      \keyindex{辐射能}{radiant energy}{}指以电磁波形式发射、传播或接收的能量。
\end{definition}
辐射能常用$Q$表示,单位为\keyindex{焦耳}{joule}{}(焦,J)。

\begin{definition}
      \keyindex{辐射能通量}{radiant energy flux}{},
      也称\keyindex{辐射功率}{radiant power}{},
      指电磁辐射通过某一面积发射、传播或接收的功率。
\end{definition}
辐通量常用$\varPhi$表示,单位为\keyindex{瓦特}{watt}{}(瓦,W)。
它描述单位时间内的辐射能:
\begin{align}
      \varPhi=\frac{\mathrm{d}Q}{\mathrm{d}t}\, ,
\end{align}
其中$t$表示时间。

\begin{definition}
      \keyindex{辐射照度}{irradiance}{}指辐射接收面单位面积内收到的辐射能通量。
      \keyindex{辐射出射度}{radiant exitance}{}指辐射源面单位面积内向半空间发射的辐射能通量。
\end{definition}
辐射照度常用$E$表示,辐射出射度常用$M$表示,单位均为$\text{W}/\text{m}^2$.
它们定义中面元所对应的立体角是辐射的整个半球空间,与辐通量的关系为
\begin{align}
      E(\text{或}M)=\frac{\mathrm{d}\varPhi}{\mathrm{d}A}\, ,
\end{align}
其中$A$为表面面积。

\begin{definition}
      \keyindex{辐射强度}{radiant intensity}{}指
      辐射源在给定方向上发射在单位立体角内的辐射通量。
\end{definition}
辐射强度常用$I$表示,单位为$\text{W}/\text{sr}$.
它一般适合于描述(近似)点光源的辐射方向特性,
与辐射通量和立体角的关系为
\begin{align}
      I=\frac{\mathrm{d}\varPhi}{\mathrm{d}\varOmega}\, .
\end{align}

\begin{definition}
      面辐射源上一点沿给定方向的\keyindex{辐射亮度}{radiance}{}指
      包含该点的面元朝该方向上的辐射强度与面元在垂直于该方向的平面上的正投影面积之比。
\end{definition}
辐射亮度常用$L$表示,单位为W$/$(sr$\cdot$m$^2$)。
它与其他辐射度量的关系为
\begin{align}\label{eq:5.ex-radiance}
      L=\frac{\mathrm{d}I}{\cos\theta\mathrm{d}A}=\frac{\mathrm{d}^2\varPhi}{\cos\theta\mathrm{d}A\mathrm{d}\varOmega}=\frac{\mathrm{d}E}{\cos\theta\mathrm{d}\varOmega}\, ,
\end{align}
其中$\theta$是面元法线$\bm n$与给定方向的夹角(\reffig{5.ex01-add01})。
\begin{figure}[htbp]
      \centering\includegraphics[width=0.4\linewidth]{chap05/ex-radiance.eps}
      \caption{辐射亮度示意图。}
      \label{fig:5.ex01-add01}
\end{figure}

\reffig{5.ex02}展示了辐射度量之间的微分关系。
\begin{figure}[htbp]
      \centering
      \begin{picture}(370,50)
            \put(0,0){辐射能$Q$}
            \put(45,3){\vector(1,0){40}}
            \put(50,6){时间$t$}
            \put(90,0){辐射通量$\varPhi$}
            \put(150,3){\vector(1,0){50}}
            \put(152,6){立体角$\varOmega$}
            \put(205,0){辐射强度$I$}
            \put(260,3){\vector(1,0){50}}
            \put(262,6){投影面积}
            \put(315,0){辐射亮度$L$}
            \put(120,12){\line(0,1){25}}
            \put(120,37){\vector(1,0){80}}
            \put(152,40){面积$A$}
            \put(205,35){辐射照度$E$}
            \put(260,37){\line(1,0){80}}
            \put(340,37){\vector(0,-1){25}}
            \put(262,40){余弦加权立体角}
      \end{picture}
      \caption{辐射度量的微分关系。}
      \label{fig:5.ex02}
\end{figure}

满足辐射亮度在任意方向大小都一样的辐射体
称为\keyindex{理想漫辐射体}{ideal diffuse radiator}{radiator辐射体}。
它遵循朗伯余弦定律。
\begin{proposition}[\keyindex{朗伯余弦定律}{Lambert's cosine law}{}]
      理想漫辐射体面元在给定方向的辐射强度与该方向和面元法线夹角的余弦成正比。
\end{proposition}
\begin{prove}
      由\refeq{5.ex-radiance},理想漫辐射体在微小面积$\Delta A$上有恒定的辐射亮度
      \begin{align*}
            L=\frac{I}{\cos\theta\Delta A}\, .
      \end{align*}
      因为在法线方向上有$\theta=0$,此时对应的辐射强度为$I_0=L\Delta A$;
      而其他方向上的辐射强度为$I_{\theta}=L\Delta A\cos\theta$,于是
      \begin{align*}\label{eq:5.ex-LambertCosine}
            I_{\theta}=I_0\cos\theta\, .
      \end{align*}
      因此理想漫辐射体也称\keyindex{朗伯辐射体}{Lambertian radiator}{}、余弦辐射体。
\end{prove}
\begin{example}
      若一个球是朗伯辐射体,则视觉上它看起来像个均匀的发光盘。
\end{example}

\begin{proposition}
      假设辐射在传播介质中不损失能量(例如被吸收、散射),
      则点光源对物体表面的照度还满足距离\keyindex{平方反比定律}{inverse-square law}{}。
\end{proposition}
\begin{figure}[htbp]
      \centering
      \includegraphics[width=0.5\linewidth]{chap05/ex-InverseSquareLaw.eps}
      \caption{点光源光照的距离平方反比定律。}
      \label{fig:5.ex-InverseSquareLaw}
\end{figure}
\begin{prove}
      如\reffig{5.ex-InverseSquareLaw},一强度为$I$的点光源
      以倾角$\theta$照亮物体表面上与其距离为$r$的点$P$。
      记该点处的曲面法线为$\bm n$,考虑点$P$周围被照射的面元$\mathrm{d}A$,
      设其关于光源的立体角为$\mathrm{d}\varOmega$,辐射能通量为$\mathrm{d}\varPhi$。
      由立体角的定义\refeq{05ex-solidangle},有
      \sidenote{注意理解为什么这里多了一项$\cos\theta$——
            因为立体角原始定义中面元与到球心的方向垂直,但此处有倾角,需折算。}
      \begin{align}
            \mathrm{d}\varOmega=\frac{\cos\theta\mathrm{d}A}{r^2}\, .
      \end{align}
      结合辐射照度与辐射强度的定义,点$P$处的辐射照度$E$为
      \begin{align}
            E=\frac{\mathrm{d}\varPhi}{\mathrm{d}A}=\frac{I\mathrm{d}\varOmega}{\mathrm{d}A}=\frac{I\cos\theta\mathrm{d}A}{r^2\mathrm{d}A}=\frac{I}{r^2}\cos\theta\, .
      \end{align}
      也即当倾角$\theta$固定时,辐射照度$E$与点光源到被照射处距离的平方$r^2$成反比。
      此外还能看出,当距离固定时,辐射照度也按倾角的余弦变化。
\end{prove}

\begin{proposition}[\keyindex{亮度守恒}{conservation of radiance}{}]
      在没有耗散的理想光学系统中,辐射亮度是守恒的。
\end{proposition}
\begin{prove}
      \begin{figure}[htbp]
            \centering
            \includegraphics[width=0.75\linewidth]{chap05/ex-ConservationRadiance.eps}
            \caption{辐射亮度的守恒特性。}
            \label{fig:5.ex-ConservationRadiance}
      \end{figure}
      如\reffig{5.ex-ConservationRadiance}所示:
      忽略介质对辐射的耗散、折射等影响,
      设点$S$向点$R$直接发出的辐射亮度为$L_S$,
      而点$R$从点$S$直接接收到的辐射亮度为$L_R$.
      我们记两点间的距离为$l$,并分别考虑这两点周围的
      面元$\mathrm{d}A_S$和$\mathrm{d}A_R$,
      记相应的法线分别为${\bm n}_S$和${\bm n}_R$,
      并设两点连线$SR$与两条法线形成的夹角分别为$\theta_S$和$\theta_R$.

      现在考虑两面元相对于对方的立体角。面元$\mathrm{d}A_R$相对于点$S$的立体角为
      \begin{align}\label{eq:5.ex-source-solid-angle}
            \mathrm{d}\varOmega_S=\frac{\cos\theta_R\mathrm{d}A_R}{l^2}\, .
      \end{align}
      同理,面元$\mathrm{d}A_S$相对于点$R$的立体角为
      \begin{align}\label{eq:5.ex-receiver-solid-angle}
            \mathrm{d}\varOmega_R=\frac{\cos\theta_S\mathrm{d}A_S}{l^2}\, .
      \end{align}
      又因为辐射能没有耗散,所以点$S$的面元向点$R$的面元发射的辐射能通量
      等于点$R$的面元从点$S$的面元接收的辐射能通量,因此根据辐射亮度定义有
      \begin{align}
            L_S & =\frac{\mathrm{d}^2\varPhi}{\cos\theta_S\mathrm{d}A_S\mathrm{d}\varOmega_S}\, , \\
            L_R & =\frac{\mathrm{d}^2\varPhi}{\cos\theta_R\mathrm{d}A_R\mathrm{d}\varOmega_R}\, .
      \end{align}
      将\refeq{5.ex-source-solid-angle}和\refeq{5.ex-receiver-solid-angle}代入上面两式可得
      \begin{align}
            L_S=L_R=l^2\frac{\mathrm{d}^2\varPhi}{\cos\theta_S\cos\theta_R\mathrm{d}A_S\mathrm{d}A_R}\, .
      \end{align}
      也即从点$S$到点$R$的亮度是守恒的。
\end{prove}



\subsection{光度学}\label{sub:光度学}
光度量是光辐射能为平均人眼接收所引起的视觉刺激大小的度量。
光度量和辐射度量的定义是一一对应的,都可以用来定量描述辐射的大小。
辐射度量是辐射能本身的客观度量,是纯粹的物理量;
而光度量则还考虑了生理学、心理学等因素。

\keyindex{发光强度}{luminous intensity}{}与辐射强度对应,
单位为坎德拉。
\begin{definition}
      频率为$540\times10^{12}\text{Hz}$(对应空气中555nm的波长)的单色辐射光源
      在给定方向上辐射强度为$\displaystyle\frac{1}{683}$W$/$sr时,
      其发光强度为1\keyindex{坎德拉}{candela}{}(cd)。
\end{definition}
坎德拉是国际单位制七个基本单位之一。
一只普通蜡烛的发光强度约为1cd。

\keyindex{光通量}{luminous flux}{}与辐射通量对应,
单位为\keyindex{流明}{lumen}{}(lm),1lm$=$1cd$\cdot$sr。

\begin{notation}
      我们约定后文光度量和对应的辐射度量所用字母相同,作区分时两者分别添加下标$\mathrm{v}$和$\mathrm{e}$.
      例如辐射强度记为$I_{\mathrm{e}}$,发光强度记为$I_{\mathrm{v}}$.
\end{notation}

\begin{notation}
      某一量的光谱密集度通常表示为波长(或频率)的函数,它具有该量除以波长的量纲,
      有时也称分布函数,为了简便也可用形容词“光谱(的)”代替“光谱密集度”。
      后文中我们约定这类量用下标$\lambda$标记。
      例如“辐射能通量的光谱密集度”可简称为“光谱辐射能通量”,记为$\varPhi_{\lambda}$,
      与辐射能通量$\varPhi$的关系是$\displaystyle\varPhi=\int \varPhi_{\lambda}\mathrm{d}\lambda$.
      但应注意到形容词“光谱(的)”也可用来表示某个量是波长(或频率)的函数,
      这类量的记号把$\lambda$写在圆括号内。
\end{notation}

人感知到的光的强弱与光的频率(波长)有关,这是人眼视觉系统特性决定的。
例如在辐射功率一定时,人眼会感觉黄绿光比红光和蓝光看起来更明亮。
为了描述光源发出可见光的能力,我们引入新的指标。

\begin{definition}
      \keyindex{光视效能}{luminous efficacy}{}是
      目视引起刺激的光通量与光源发出的辐射通量之比,
      记作$K$,单位为lm$/$W:
      \begin{align}
            K=\frac{\varPhi_{\mathrm{v}}}{\varPhi_{\mathrm{e}}}\, .
      \end{align}
\end{definition}

\reftab{5.ex01}列出了常见光源的光视效能。
\begin{table}[htbp]
      \centering
      \begin{tabular}{lc|lc}
            \toprule
            \textbf{光源类型} & \textbf{光视效能(lm$/$W)} & \textbf{光源类型} & \textbf{光视效能(lm$/$W)} \\
            \midrule
            钨丝灯(真空)    & 8.0 - 9.2                 & 日光灯            & 27 - 41                   \\
            钨丝灯(充气)    & 9.2 - 21.0                & 高压水银灯        & 34 - 45                   \\
            石英卤钨灯        & 30                        & 超高压水银灯      & 40.0 - 47.5               \\
            气体放电管        & 16 - 30                   & 钠光灯            & 60                        \\
            \bottomrule
      \end{tabular}
      \caption{常见光源的光视效能。}
      \label{tab:5.ex01}
\end{table}

\begin{definition}
      \keyindex{光谱光视效能}{spectral luminous efficacy}{luminous efficacy光视效能}记作$K(\lambda)$,是光视效能关于波长的函数,即
      \begin{align}
            K(\lambda)=\frac{\varPhi_{\mathrm{v}\lambda}}{\varPhi_{\mathrm{e}\lambda}}\, .
      \end{align}
\end{definition}

\begin{definition}
      $K(\lambda)$的最大值称为\keyindex{最大光谱光视效能}{maximum spectral luminous efficacy}{luminous efficacy光视效能},
      记作$K_{\mathrm{m}}$.它在频率为$540\times10^{12}\text{Hz}$时取得,值为683lm$/$W。
\end{definition}
这也是坎德拉的定义以该频率为标准的原因。

\begin{definition}
      \keyindex{光视效率}{luminous efficiency}{}记作$V$,定义为光视效能与最大光谱光视效能的比值,量纲为1:
      \begin{align}
            V=\frac{K}{K_{\mathrm{m}}}\, .
      \end{align}
\end{definition}

\begin{definition}
      \keyindex{光谱光视效率}{spectral luminous efficiency}{}函数,
      即光视效率关于波长的函数,也称\keyindex{光度函数}{luminosity function}{}、
      相对\keyindex{视见函数}{visual sensitivity function}{}等,记作$V(\lambda)$:
      \begin{align}
            V(\lambda)=\frac{K(\lambda)}{K_{\mathrm{m}}}\, .
      \end{align}
      它表征了人眼对各波长单色光的视觉灵敏度。
\end{definition}

因为人眼在不同亮度环境下的视觉灵敏度不同,所以$V(\lambda)$
分为\keyindex{明视觉}{photopic}{}和\keyindex{暗视觉}{scotopic}{}两种常用版本(\reffig{5.ex03})。
1971年国际照明委员会公布的明视觉的$V(\lambda)$标准值已于1972年由国际计量委员会批准。
\begin{figure}[htbp]
      \centering\includegraphics[width=0.75\linewidth]{chap05/spectralluminousefficiency.eps}
      \put(0,0){$\lambda/$nm}
      \put(-315,110){$V(\lambda)$}
      \caption{光谱光视效率曲线。其中橙色实线对应明视觉($2^{\circ}$视场角),蓝色虚线对应暗视觉。
            数据来源于\protect\url{http://www.cvrl.org}。}
      \label{fig:5.ex03}
\end{figure}

依据以上定义,可以推导出如下关系:
\begin{align}
      K                    & =\frac{\displaystyle\int \varPhi_{\mathrm{v}\lambda}\mathrm{d}\lambda}{\displaystyle\int \varPhi_{\mathrm{e}\lambda}\mathrm{d}\lambda}=\frac{\displaystyle\int K(\lambda)\varPhi_{\mathrm{e}\lambda}\mathrm{d}\lambda}{\displaystyle\int \varPhi_{\mathrm{e}\lambda}\mathrm{d}\lambda}\, , \\
      \varPhi_{\mathrm{v}} & =\displaystyle\int K(\lambda)\varPhi_{\mathrm{e}\lambda}\mathrm{d}\lambda=K_{\mathrm{m}}\int V(\lambda)\varPhi_{\mathrm{e}\lambda}\mathrm{d}\lambda\, ,                                                                                                                                    \\
      V                    & =\frac{\displaystyle\int V(\lambda)\varPhi_{\mathrm{e}\lambda}\mathrm{d}\lambda}{\displaystyle\int \varPhi_{\mathrm{e}\lambda}\mathrm{d}\lambda}\, .
\end{align}

\keyindex{光照度}{illuminance}{}与辐射照度对应,
\keyindex{光出射度}{luminous exitance}{}与辐射出射度对应,
单位均为\keyindex{勒克斯}{lux}{}(lx),1lx=1lm$/$m$^2$.

\keyindex{光亮度}{luminance}{}与辐射亮度对应,单位为cd$/$m$^2$.

\subsection{色度学}\label{sub:色度学}
\keyindex{色度学}{colorimetry}{}是在物理上量化描述人类颜色知觉的科学技术。
\subsubsection*{人眼视觉特性与颜色视觉理论}
\begin{figure}[htbp]
      \centering\includegraphics[width=0.5\linewidth]{chap05/Schematic_diagram_of_the_human_eye_zh-hans.eps}
      \caption{人眼结构。}
      \label{fig:5.ex04}
\end{figure}
人眼(\reffig{5.ex04})中负责感光的部分是\keyindex{视网膜}{retina}{},
其中具有两种\keyindex{感光细胞}{photoreceptor cell}{},
即\keyindex{视杆细胞}{rod cell}{}和\keyindex{视锥细胞}{cone cell}{}。
视杆细胞主要分布在视网膜中心周围,几乎全部用于夜视力,数量达到一亿量级。
1个光子就足以激发视杆细胞的活动,其对单个光子的敏感程度是视锥细胞的一百多倍,
因此视杆细胞建立人类在夜晚最基本的视觉,即暗视觉。
暗视觉只有视杆细胞起作用,由其仅含的视紫红色素吸收光子,所以不能分辨颜色,只有明暗感觉。
视锥细胞大多分布在视网膜黄斑处,因树突呈锥形得名,约有六百万个。
它主要负责颜色识别,在相对较亮的光照条件下发挥作用(一般需要数十到上百个光子激发),
按所含有的视色素分为三种,分别对黄绿色、绿色和蓝紫色的光最为敏感(\reffig{5.ex05}),
其形成的视觉信号复合后呈现出色彩缤纷的世界。

\begin{figure}[htbp]
      \centering\includegraphics[width=0.75\linewidth]{chap05/NormalizedResponsivitySpectra.eps}
      \put(0,0){$\lambda/$nm}
      \put(-300,35){\rotatebox{90}{规范化视锥响应度(线性能量)}}
      \caption{人眼三种视锥细胞的规范化响应光谱($2^{\circ}$)。
            蓝、绿、红曲线分别对应S、M、L型视锥细胞。
            注意它们的响应绝对峰值并不相同,图中是规范化到0-1后的结果。
            数据来源于\protect\url{http://www.cvrl.org}。}
      \label{fig:5.ex05}
\end{figure}

现代颜色视觉理论主要有两大类,分别从是Yang-Helmholtz的三色学说和Hering的“对立”颜色学说发展起来的。
两者学说都能解释大量事实,但也都有不足之处。
三色学说能很好地说明各种颜色的混合现象,但不能很好地解释色盲现象;“对立”颜色学说则相反。
也有学者试图调和两者的观点。

\subsubsection*{颜色匹配}
\keyindex{颜色混合}{color mixing}{}可以是颜色光的混合,也可以是染料的混合,两种混合方法的结果是不同的,
前者称为\keyindex{加色混合}{additive mixing}{color mixing颜色混合},
后者称为\keyindex{减色混合}{subtractive mixing}{color mixing颜色混合}(\reffig{5.ex06})。
\begin{figure}[htb]
      \centering
      \includegraphics[width=0.4\linewidth]{chap05/additivemixing.eps}
      \includegraphics[width=0.4\linewidth]{chap05/subtractivemixing.eps}
      \caption{颜色混合:加色混合(左)与减色混合(右)。}
      \label{fig:5.ex06}
\end{figure}

1853年,德国学者格拉斯曼(Hermann Günther Gra{\ss}mann)总结出加色混合的性质,
即格拉斯曼定律,为现代色度学奠定了基础。

\begin{proposition}
      \keyindex{格拉斯曼定律}{Grassmann's laws}{}的现代解释有四点内容:
      \begin{enumerate}
            \item 人的视觉只能分辨颜色的三种变化(例如明度、色调、饱和度)。
            \item 在由两种成分组成的混合色光中,若一种成分连续变化,则混合色光外观也连续变化。
            \item 存在光谱功率分布不同但外观相同的色光。
            \item 混合色光的总亮度是各成分亮度之和。
      \end{enumerate}
\end{proposition}
\begin{corollary}
      每种色光都存在\keyindex{互补色}{complementary color}{},
      使得两者混合后要么产生无色(白或灰)光,要么产生近似比重大的颜色成分的非饱和色。
\end{corollary}
\begin{corollary}
      任意两种非互补色混合会产生中间色,其色调和饱和度决定于两种成分的色调与相对数量。
\end{corollary}
\begin{corollary}
      若两种色光的外观相同,则它们在加色或减色混合中作成分时的效应是等价的。
\end{corollary}

例如若有两对颜色外观相同,即$A\equiv B$,$C\equiv D$,则有
\begin{align}
      A+C      & \equiv B+D\, ,      \\
      A-C      & \equiv B-D\, ,      \\
      \alpha A & \equiv \alpha B\, ,
\end{align}
其中符号“$\equiv$”表示颜色互相匹配。
即加色、减色混合的外观相同,颜色同时扩大或缩小相同倍数$\alpha>0$的外观相同。
这揭示了颜色混合的线性性质。

\begin{figure}[htbp]
      \centering\includegraphics[width=0.75\linewidth]{chap05/colormatching.eps}
      \put(-120,195){\color[RGB]{6,139,190}蓝}
      \put(-105,178){\color[RGB]{31,129,49}绿}
      \put(-105,153){\color[RGB]{187,42,33}红}
      \put(-180,30){\color[RGB]{222,206,186}待匹配颜色光}
      \put(-155,95){\color{white}黑挡片}
      \put(-40,95){\color{white}眼}
      \put(-50,170){\color{white}背景光}
      \caption{颜色匹配实验。}
      \label{fig:5.ex07}
\end{figure}

结合来自1928年莱特(W. David Wright)、1931年吉尔德(John Guild)
以及1931年国际照明委员会(CIE)的颜色匹配实验数据,
CIE提出了“CIE 1931标准色度系统”。它们的实验方法都如\reffig{5.ex07}所示。
左边是一块白色屏幕,上方为红、绿、蓝三原色光,下方为待测色光。
三原色光照射白色屏幕的上半部分,待测色光照射下半部分,中间用黑挡片隔开。
从白色屏幕反射出来的光通过小孔抵达观察者人眼,视场为$2^{\circ}$并被分为两部分。
此外右上方还有一束颜色和强度可调的光照射在小孔周围的背景白板上。
在实验中,调节红、绿、蓝三原色光,直到观察者认为与待测色光的外观相同,
即视场中分界线感觉消失,两部分合为整体,
此时即三原混合色光与待测色光达到\keyindex{颜色匹配}{color matching}{}。
达到匹配后,改变背景光,此时视场中的颜色会变化,但仍能匹配。
颜色匹配时所需三原色的数量称为\keyindex{三刺激值}{tristimulus values}{}。
若分别以$\compcolor{C},\compcolor{R},\compcolor{G},\compcolor{B}$表示
被匹配的颜色以及红、绿、蓝三原色光的单位,以实数$C,R,G,B$表示相应颜色的数量,
则颜色匹配方程可写作
\begin{align}\label{eq:colormatchrgb}
      C\compcolor{C}\equiv R\compcolor{R}+G\compcolor{G}+B\compcolor{B}\, ,
\end{align}
其中符号“$\equiv$”表示颜色外观相同,$R,G,B$可以为负,$C=R+G+B$.

实验还证明了颜色匹配恒常律,即互相匹配的颜色在观察环境变化后依然保持匹配。
实验中在相应规定条件下视场为$2^{\circ}$的观察者
称为\keyindex{CIE 1931标准色度观察者}{CIE 1931 standard colorimetric observer}{}。

颜色匹配实验中有个重要问题是:以什么样的颜色作为三原色光?
三刺激值的单位$\compcolor{R},\compcolor{G},\compcolor{B}$如何确定?
原则上,三原色可以任意选定,但其中任何一种颜色不能由其他两种加色混合得到,最常用的是红、绿、蓝。
CIE在实验中使用波长分别为700nm、546.1nm、435.8nm的\keyindex{单色光}{monochromatic light}{}作为三原色,
其中700nm是可见光谱的红色末端,546.1nm和435.8nm是明显的汞谱线,三者在实验中都能比较精确地产生。
为了确定单位,CIE规定能匹配\keyindex{等能白光}{equal-energy white}{}(
也称\keyindex{E光源}{illuminant E}{},即整个光谱功率分布为常数的混合光,
因颜色接近白色得名,是一种理论光源,现实中暂无法模拟出来)且使得三刺激值全等
(即$R=G=B$)的三原色比例作为相应的色度学单位。
结果是,红绿蓝按光亮度之比为$1:4.5907:0.0601$
(对应于辐亮度之比$72.0962:1.3791:1$)作为三刺激值的单位。
\begin{example}
      按$0.1\compcolor{R}+0.2\compcolor{G}+0.3\compcolor{B}$所得的色光
      指按$(0.1\times1):(0.2\times4.5907):(0.3\times0.0601)=0.1:0.91814:0.01803$的
      光亮度比例混合波长分别700nm、546.1nm、435.8nm的单色光所得的结果。
\end{example}

实验中还发现,有一些颜色无论如何也无法被三原色匹配出来。
此时将某些三原色挪到与待测色光的同一侧进行实验则能实现匹配。
例如将红光与待测色光放在一侧,绿和蓝色光放在另一侧,当达到匹配时有
\begin{align}
      C\compcolor{C}+R'\compcolor{R}\equiv G\compcolor{G}+B\compcolor{B}\, .
\end{align}
于是可视作
\begin{align}
      C\compcolor{C}\equiv -R'\compcolor{R}+G\compcolor{G}+B\compcolor{B}\, ,
\end{align}
即三刺激值的红色分量为负值$-R'$.

单个波长的可见光称为\keyindex{光谱色}{spectral color}{}。
CIE依据历史与实验测定了匹配等能光谱色的RGB三刺激值,规范化后%规范化的具体做法未查到资料
得到RGB的\keyindex{颜色匹配函数}{color matching functions}{color matching颜色匹配},
记作$\bar{r}(\lambda),\bar{g}(\lambda),\bar{b}(\lambda)$,
并对应构建了\keyindex{CIE 1931 RGB 颜色空间}{CIE 1931 RGB color space}{}。
\reffig{5.ex08}显示$\bar{r}(\lambda)$在一定波长范围内取负值。
\begin{figure}[htbp]
      \centering\includegraphics[width=0.75\linewidth]{chap05/CIERGBcolormatchingfunctions.eps}
      \put(10,0){$\lambda/$nm}
      \put(-110,125){$\bar{r}(\lambda)$}
      \put(-170,125){$\bar{g}(\lambda)$}
      \put(-235,125){$\bar{b}(\lambda)$}
      \caption{CIE RGB颜色匹配函数($2^{\circ}$视场),图示来自\citet{SETCHELL2012219}。}
      \label{fig:5.ex08}
\end{figure}

对于光谱分布为$S(\lambda)$的光刺激,相应的RGB值通过在颜色匹配函数上积分算得:
\begin{align}
      R & =k\int \bar{r}(\lambda)S(\lambda)\mathrm{d}\lambda\, , \\
      G & =k\int \bar{g}(\lambda)S(\lambda)\mathrm{d}\lambda\, , \\
      B & =k\int \bar{b}(\lambda)S(\lambda)\mathrm{d}\lambda\, ,
\end{align}
其中$k$为适当的规范化系数。

\subsubsection*{CIE 1931颜色空间}
\begin{definition}
      对于RGB三刺激值为$R,G,B$的颜色,定义其RGB\keyindex{色品}{chromaticity}{}坐标为:
      \begin{align}
            r & =\frac{R}{R+G+B}\, , \\
            g & =\frac{G}{R+G+B}\, , \\
            b & =\frac{B}{R+G+B}\, .
      \end{align}
\end{definition}
\begin{corollary}\label{corollary:chromaticity}
      色品坐标之和恒为1,即$r+g+b=1$.
\end{corollary}
这意味着色品坐标中只有两个独立分量。

依据RGB颜色匹配函数$\bar{r}(\lambda),\bar{g}(\lambda),\bar{b}(\lambda)$计算
出光谱色的色品坐标$r,g,b$,并以其中的$r$为横坐标,$g$为纵坐标,
按波长顺序将光谱色的色品点连接起来,可以得到舌形的轨迹(\reffig{5.ex09}),
称为\keyindex{光谱轨迹}{spectral locus}{}。
该轨迹在CIE RGB三原色波长处与坐标轴相交\sidenote{请读者思考一下为什么。},
即在435.8nm处有$r=0,g=0$,在546.1nm处有$r=0,g=1$,在700nm处有$r=1,g=0$.
光谱轨迹及其首尾连线围成封闭区域,
即构成\keyindex{CIE 1931 RGB 颜色空间}{CIE 1931 RGB color space}{}
\sidenote{它包含了人眼可见的全部颜色。}。
注意与\reffig{5.ex08}结合起来思考波长较短的轨迹为何进入$r<0$的半平面。
\begin{figure}[htbp]
      \centering\includegraphics[width=0.75\linewidth]{chap05/CIE1931rgxy.png}
      \put(-205,9){\small -}
      \put(-270,9){\small -}
      \caption{CIE $rg$色品空间。其中E为等能白光;
      $\text{C}_{\mathrm{r}},\text{C}_{\mathrm{g}},\text{C}_{\mathrm{b}}$分别为
      $\compcolor{X},\compcolor{Y},\compcolor{Z}$三原色的色品点。
      $\text{C}_{\mathrm{r}}$与$\text{C}_{\mathrm{b}}$的连线为零亮度线。}
      \label{fig:5.ex09}
\end{figure}

依据颜色混合的线性性质,我们有:
\begin{corollary}
      对于由任意两种颜色$A,B$加色混合所得的颜色$C$,
      其色品点一定在$A$与$B$色品点所连的线段上。
\end{corollary}

虽然RGB颜色空间的$\bar{r}(\lambda),\bar{g}(\lambda),\bar{b}(\lambda)$
可用于色度计算,但其出现的负值不易理解且使用不便。
因此CIE定义了三种假想的三原色$\compcolor{X},\compcolor{Y},\compcolor{Z}$,
构建新的颜色空间以消除负值。其在构建过程中还考虑了以下问题:
\begin{enumerate}
      \item 规定$\compcolor{X}$和$\compcolor{Z}$只代表色度,与亮度无关。
            光亮度只与$\compcolor{Y}$的刺激值$Y$成正比。
            称$\compcolor{X}$和$\compcolor{Z}$在色品图中的连线为\keyindex{零亮度线}{alychne}{}。
            因为$\compcolor{R},\compcolor{G},\compcolor{B}$的光亮度之比为$1:4.5907:0.0601$,
            故零亮度线上的色品坐标应满足
            \begin{align}
                  r+4.5907g+0.0601b=0\, ,
            \end{align}
            代入$b=1-r-g$得
            \begin{align}
                  0.9399r+4.5306g+0.0601=0\, .
            \end{align}
      \item 所有光谱三刺激值均为非负数。因此$\compcolor{X},\compcolor{Y},\compcolor{Z}$所
            围成的三角形必须能包围整个光谱轨迹。
      \item $rg$色品图中光谱轨迹从540nm附近到700nm几乎是一条直线,
            这段线段上的任意两种颜色加色混合可匹配两色之间的各种光谱色。
            因此希望$\compcolor{X}$和$\compcolor{Y}$的连线与该线段重合,
            使其只与$\compcolor{X}$和$\compcolor{Y}$的变化有关,与$\compcolor{Z}$无关。
\end{enumerate}

此外$\compcolor{YZ}$边取光谱轨迹在波长503nm处的切线。
于是直线$\compcolor{XY}$、$\compcolor{YZ}$和$\compcolor{XZ}$均得以确定。
它们的交点即$\compcolor{X},\compcolor{Y},\compcolor{Z}$在$rg$色品图中的位置。
以$\compcolor{X},\compcolor{Y},\compcolor{Z}$为三原色的颜色空间
即\keyindex{CIE 1931 XYZ 颜色空间}{CIE 1931 XYZ color space}{}。
注意$\compcolor{X},\compcolor{Y},\compcolor{Z}$是现实中观察不到的理论假想色。
在该颜色空间中,颜色匹配方程为
\begin{align}\label{eq:colormatchxyz}
      C\compcolor{C}\equiv X\compcolor{X}+Y\compcolor{Y}+Z\compcolor{Z}\, .
\end{align}
其中$X,Y,Z$为相应的三刺激值。
依据构建过程中考虑的第1点问题,$Y$的光谱分布就是光谱光视效率函数,即
\begin{align}
      Y(\lambda)=V(\lambda)\, .
\end{align}
与$X,Y,Z$对应的XYZ色品坐标为
\begin{align}
      x & =\frac{X}{X+Y+Z}\, , \\
      y & =\frac{Y}{X+Y+Z}\, , \\
      z & =\frac{Z}{X+Y+Z}\, .
\end{align}
它同样满足推论\ref{corollary:chromaticity},即$x+y+z=1$.

与RGB类似,我们可以得到对应于光谱色的
CIE XYZ 颜色匹配函数$\bar{x}(\lambda),\bar{y}(\lambda),\bar{z}(\lambda)$(\reffig{5.ex10})
以及马蹄形的光谱轨迹与$xy$色品图(\reffig{5.ex11})。
可以看到$\bar{x}(\lambda),\bar{y}(\lambda),\bar{z}(\lambda)$不再含有负数,
光谱轨迹完全在$xy$第一象限。其中光谱轨迹的首尾连线称为\keyindex{紫线}{line of purples}{}。
\begin{figure}[htbp]
      \centering\includegraphics[width=0.75\linewidth]{chap05/cie31xyz.eps}
      \put(0,0){$\lambda/$nm}
      \put(-110,100){$\bar{x}(\lambda)$}
      \put(-170,125){$\bar{y}(\lambda)$}
      \put(-250,125){$\bar{z}(\lambda)$}
      \caption{CIE XYZ颜色匹配函数($2^{\circ}$视场),数据来源于\protect\url{http://www.cvrl.org}。}
      \label{fig:5.ex10}
\end{figure}

类似于RGB,对于光谱分布为$S(\lambda)$的光刺激,相应的XYZ值通过在颜色匹配函数上积分算得:
\begin{align}
      X & =k\int \bar{x}(\lambda)S(\lambda)\mathrm{d}\lambda\, , \\
      Y & =k\int \bar{y}(\lambda)S(\lambda)\mathrm{d}\lambda\, , \\
      Z & =k\int \bar{z}(\lambda)S(\lambda)\mathrm{d}\lambda\, ,
\end{align}
其中$k$为适当的规范化系数。

\begin{figure}[htbp]
      \centering\includegraphics[width=0.6\linewidth]{chap05/CIE1931xyCIERGB.pdf}
      \caption{CIE 1931 XYZ颜色空间色品图。图中的三角形标出了CIE RGB三原色的位置。
            注意因为当下显示和打印设备的色域均不能覆盖整个颜色空间,加之图片格式的限制,
            所以该图中的一部分颜色一定会显示得不准确。}
      \label{fig:5.ex11}
\end{figure}

注意到当$X+Y+Z=1$时,有$x=X,y=Y,z=Z$.
这说明$xy$色品图是在$XYZ$坐标系中
截取平面$X+Y+Z=1$再向$XY$平面投影的结果(\reffig{5.ex12})。
\begin{figure}[htbp]
      \centering\includegraphics[width=0.5\linewidth]{chap05/Projectionandchromaticityplane.pdf}
      \put(-80,38){\color{white}$X$}
      \put(-165,130){\color{white}$Y$}
      \put(-160,4){\color{white}$Z$}
      \caption{XYZ体与$xy$色品图。左图是XYZ体,右上是截平面$X+Y+Z=1$,
            右下是将截平面投影到$XY$平面得到的$xy$色品图。图示来自\protect\citet{BERTALMIO2020131}。}
      \label{fig:5.ex12}
\end{figure}

\subsubsection*{颜色空间的转化}
由于选择的三原色以及三刺激值单位不同,出现了多种颜色空间。
这里我们介绍RGB与XYZ颜色空间相互转化的推导过程。

以XYZ转化到RGB空间为例。XYZ为旧空间,RGB为新空间。
设新空间的三原色$\compcolor{R},\compcolor{G},\compcolor{B}$在
旧空间中的三刺激值为$X_i,Y_i,Z_i,(i=\mathrm{r},\mathrm{g},\mathrm{b})$,即
\begin{align}\label{eq:colorbasis}
      \compcolor{R} & =X_\mathrm{r}\compcolor{X}+Y_\mathrm{r}\compcolor{Y}+Z_\mathrm{r}\compcolor{Z}\, , \\
      \compcolor{G} & =X_\mathrm{g}\compcolor{X}+Y_\mathrm{g}\compcolor{Y}+Z_\mathrm{g}\compcolor{Z}\, , \\
      \compcolor{B} & =X_\mathrm{b}\compcolor{X}+Y_\mathrm{b}\compcolor{Y}+Z_\mathrm{b}\compcolor{Z}\, .
\end{align}
对照颜色匹配方程\refeq{colormatchrgb}与\refeq{colormatchxyz},可得
\begin{align}
      X\compcolor{X}+Y\compcolor{Y}+Z\compcolor{Z}=R\compcolor{R}+G\compcolor{G}+B\compcolor{B}\, .
\end{align}
联立以上式子,由对应系数相等整理可得
\begin{align}
      \left[\begin{array}{c}
                  X \\Y\\Z
            \end{array}\right]=
      \left[\begin{array}{ccc}
                  X_\mathrm{r} & X_\mathrm{g} & X_\mathrm{b} \\
                  Y_\mathrm{r} & Y_\mathrm{g} & Y_\mathrm{b} \\
                  Z_\mathrm{r} & Z_\mathrm{g} & Z_\mathrm{b}
            \end{array}\right]
      \left[\begin{array}{c}
                  R \\G\\B
            \end{array}\right]\, .
\end{align}
上式表明两种颜色空间是线性转换关系,只要知道九个系数$X_i,Y_i,Z_i,(i=\mathrm{r},\mathrm{g},\mathrm{b})$即可。
实际中更常见的是这些系数待定,但知道对应的色品坐标$x_i,y_i,z_i,(i=\mathrm{r},\mathrm{g},\mathrm{b})$.设
\begin{align}
      C_i=X_i+Y_i+Z_i\, ,\quad (i=\mathrm{r},\mathrm{g},\mathrm{b})\, ,
\end{align}
则有
\begin{align}\quad
      X_i=C_ix_i\, ,\quad Y_i=C_iy_i\, ,\quad Z_i=C_iz_i\, ,\quad (i=\mathrm{r},\mathrm{g},\mathrm{b})\, .
\end{align}
记
\begin{align}
      P & =\left[\begin{array}{ccc}
                  x_\mathrm{r} & x_\mathrm{g} & x_\mathrm{b} \\
                  y_\mathrm{r} & y_\mathrm{g} & y_\mathrm{b} \\
                  z_\mathrm{r} & z_\mathrm{g} & z_\mathrm{b}
            \end{array}\right]\, , \\
      D & =\left[\begin{array}{ccc}
                  C_\mathrm{r} &              &              \\
                               & C_\mathrm{g} &              \\
                               &              & C_\mathrm{b}
            \end{array}\right]\, ,
\end{align}
于是
\begin{align}
      [X\quad Y\quad Z]^{\mathrm{T}}=PD[R\quad G\quad B]^{\mathrm{T}}\, .
\end{align}
其中矩阵$P$是已知的,还需确定$D$中的三个系数$C_i$,
一般我们通过选择一种颜色(例如参照白)
在旧空间的三刺激值$X_0,Y_0,Z_0$以及规定它在新空间的三刺激值$R_0,G_0,B_0$来求解,即
\begin{align}
      [X_0\quad Y_0\quad Z_0]^{\mathrm{T}}=PD[R_0\quad G_0\quad B_0]^{\mathrm{T}}\, .
\end{align}
可解得
\begin{align}
      \left[\begin{array}{c}
                  C_\mathrm{r} \\C_\mathrm{g}\\C_\mathrm{b}
            \end{array}\right]=\left(P
      \left[\begin{array}{ccc}
                  R_0 &     &     \\
                      & G_0 &     \\
                      &     & B_0
            \end{array}\right]\right)^{-1}
      \left[\begin{array}{c}
                  X_0 \\Y_0\\Z_0
            \end{array}\right]\, .
\end{align}
此时转化关系得以完全确定。还可得逆转化关系相应为
\begin{align}
      [R\quad G\quad B]^{\mathrm{T}}=(PD)^{-1}[X\quad Y\quad Z]^{\mathrm{T}}\, .
\end{align}

对于CIE 1931 RGB颜色空间与XYZ空间的转化,以等能白光为参照取
\begin{align}
      \begin{array}{lll}
            x_\mathrm{r}=0.73467\, , & x_\mathrm{g}=0.27376\, , & x_\mathrm{b}=0.16658\, , \\
            y_\mathrm{r}=0.26533\, , & y_\mathrm{g}=0.71741\, , & y_\mathrm{b}=0.00886\, , \\
            X_0=Y_0=Z_0=1\, ,        & R_0=G_0=B_0=1\, .        &
      \end{array}
\end{align}
算得
\begin{align}
      PD        & =\left[\begin{array}{rrr}
                  0.4900  & 0.3100 & 0.2000 \\
                  0.1770  & 0.8124 & 0.0106 \\
                  -0.0000 & 0.0100 & 0.9900
            \end{array}\right]\, , \\
      (PD)^{-1} & =\left[\begin{array}{rrr}
                  2.3647  & -0.8966 & -0.4681 \\
                  -0.5152 & 1.4264  & 0.0887  \\
                  0.0052  & -0.0144 & 1.0092
            \end{array}\right]\, .
\end{align}

\keyindex{sRGB颜色空间}{standard RGB color space}{}是惠普与微软等企业
于1996年共同开发的用于显示器、打印机以及互联网的一种色域标准。
对于sRGB的线性值与XYZ空间的转化,以D65为参照白\sidenote{D65是CIE规定的标准日光光源,色温约6500{\normalfont K}。}取
\begin{align}
      \begin{array}{lll}
            x_\mathrm{r}=0.6400\, , & x_\mathrm{g}=0.3000\, , & x_\mathrm{b}=0.1500\, , \\
            y_\mathrm{r}=0.3300\, , & y_\mathrm{g}=0.6000\, , & y_\mathrm{b}=0.0600\, , \\
            x_0=0.3127\, ,          & y_0=0.3290\, ,          & Y_0=1.0000\, ,          \\
            R_0=G_0=B_0=1\, .       &                         &
      \end{array}
\end{align}
算得
\begin{align}
      PD        & =\left[\begin{array}{rrr}
                  0.4124 & 0.3576 & 0.1805 \\
                  0.2126 & 0.7152 & 0.0722 \\
                  0.0193 & 0.1192 & 0.9505
            \end{array}\right]\, , \\
      (PD)^{-1} & =\left[\begin{array}{rrr}
                  3.2410  & -1.5374 & -0.4986 \\
                  -0.9692 & 1.8760  & 0.0416  \\
                  0.0556  & -0.2040 & 1.0570
            \end{array}\right]\, .
\end{align}

sRGB在完成RGB线性值(在0到1之间)的计算后,还要进行非线性的伽马校正才得到最终结果。
对于线性值$C(=R,G,B)$,相应最终值为
\begin{align}
      C_{\text{sRGB}}=\left\{
      \begin{array}{ll}
            \displaystyle12.92C\, ,                       & \text{若}C\le0.0031308\, , \\
            \displaystyle1.055C^{\frac{1}{2.4}}-0.055\, , & \text{其他}\, .
      \end{array}
      \right.
\end{align}
由最终值到线性值的逆变换为
\begin{align}
      C=\left\{
      \begin{array}{ll}
            \displaystyle\frac{C_{\text{sRGB}}}{12.92}\, ,                          & \text{若}C_{\text{sRGB}}\le0.04045\, , \\
            \displaystyle\left(\frac{C_{\text{sRGB}}+0.055}{1.055}\right)^{2.4}\, , & \text{其他}\, .
      \end{array}
      \right.
\end{align}

\reffig{5.ex13}展示了多种\keyindex{色域}{color gamut}{}标准。
\begin{figure}[htbp]
      \centering\includegraphics[width=0.6\linewidth]{chap05/CIE1931xygamutcomparison.pdf}
      \caption{一些RGB与CMYK色域在CIE 1931 $xy$色品图中的范围。}
      \label{fig:5.ex13}
\end{figure}

\input{content/chap06.tex}

\chapterimage{Pictures/chap07/checkerboard-ref-465x930.png}
\chapter{采样与重构}\label{chap:采样与重构}
\setcounter{sidenote}{1}
尽管像pbrt那样的渲染器最终输出的是彩色像素的2D网格,
但实际上入射辐射是定义在胶片平面上的连续函数。
从该连续函数计算出离散像素值的方法会显著影响渲染器生成的最终图像的质量;
如果没有仔细执行该过程,则会出现伪影\sidenote{译者注:原文artifact。}。
反之,如果执行得很好,则为此进行相对少量的额外计算就能极大提升渲染图像的质量。

本章从介绍\emph{采样理论}开始——即从定义在连续域上的函数
取出离散样本值并用它们重建与原本类似的新函数的理论。
在采样理论和低偏差点集(一种均匀分布的样本点类型)思想的基础上,
本章定义的\refvar{Sampler}{}以不同方式生成$n$维样本向量
\footnote{回想上一章中\refvar{Camera}{}用\refvar{CameraSample}{}
    在胶片平面、透镜上以及时间域中取点——通过取用这些样本向量前几维来设定\refvar{CameraSample}{}值。}。
本章将介绍五种\refvar{Sampler}{}实现,涵盖了采样问题的各种方法。

本章以类\refvar{Filter}{}和\refvar{Film}{}作结。
\refvar{Filter}{}用于确定每个像素周围要融合多少倍样本量来计算最终像素值,
类\refvar{Film}{}则积累图像样本对图中像素的贡献量。

\section{采样理论}\label{sec:采样理论}
数字图像表示为一组像素值,通常对齐到矩形网格。
当在物理设备上展示数字图像时,这些值用于确定显示器上像素发射的光谱功率。
当考虑数字图像时,区分图像像素与显示器像素很重要,
前者表示一个函数在特定样本位置的值,后者是具有某个发光分布的物理对象
(例如对于LCD显示器,当以倾斜角度观察它时,颜色和亮度可能会极大变化)。
显示器用图像像素值在显示器表面构造新的图像函数。
该函数定义在显示器所有点位上,而不只是数字图像像素的无穷小点上。
这样取一组样本值并将其转换回连续函数的过程称为\keyindex{重建}{reconstruction}{}。

为了计算数字图像中的离散像素值,必须采样原始连续定义的图像函数。
在pbrt中,像大多数其他光线追踪渲染器那样,
获取图像函数有关信息的唯一方法就是通过追踪光线来对其采样。
例如,能计算胶片平面上两点间的图像函数变化边界的通用方法是不存在的。
尽管可以通过在像素位置上精确采样该函数来生成图像,
但通过在不同位置上取用更多样本并将这些关于图像函数的
额外信息融合到最终的像素值中能得到更好的结果。
实际上,为了有最佳质量的结果,计算像素值时应使得
在显示设备上重建的图像尽可能与虚拟相机胶片平面上的场景原始图像逼近。
注意这和希望显示器像素在其位置上取用图像函数实际值的目标有些微妙区别。
处理这一区别是本章实现的算法的主要目标
\footnote{本书中我们将忽略物理显示器像素特性相关问题并
    在显示器执行本节后面所述理想重建过程的假设下处理。
    该假设显然与真实显示器的工作方式不符,但这里它避免了不必要的复杂分析。
    \citet{GLASSNER1995}的第3章很好地处理了非理想显示设备
    及其对图像采样和重建过程的影响。}。

因为采样和重建过程涉及估值,所以它引入了称作\keyindex{混叠}{aliasing}{alias混叠}的误差,
并会以许多方式表现出来,包括锯齿状边缘或动画中的闪烁。
产生这些误差的原因是采样过程不能捕获来自连续定义的图像函数的全部信息。

作为这些思想的一个例子,考虑一个1D函数(我们也会称之为信号)即$f(x)$,
我们可以求函数定义域中任意期望位置$x'$处的值$f(x')$.
每个这样的$x'$称为\keyindex{样本位置}{sample position}{},
$f(x')$的值称为\keyindex{样本值}{sample value}{}。
\reffig{7.1}展示了光滑1D函数的样本集,以及逼近原始函数$f$的重建信号$\tilde{f}$.
本例中,$\tilde{f}$是分段线性函数,通过线性插值相邻样本值来逼近$f$
(已经熟悉采样理论的读者会认出这是用帽函数\sidenote{译者注:原文hat function。}重建的)。
因为关于$f$唯一可用的信息是来自在位置$x'$处的采样值,
且没有关于$f$在样本间特性的信息,所以$\tilde{f}$不能完全匹配$f$.
\begin{figure}[htbp]
    \centering
    \subfloat[]{\includegraphics[width=0.4\linewidth]{chap07/point-sampling.eps}}\,\,
    \subfloat[]{\includegraphics[width=0.4\linewidth]{chap07/linear-reconstruction.eps}}
    \caption{(a)通过取$f(x)$的\emph{样本点}集(标实心记),我们确定了函数在这些位置处的值。
        (b)样本值可用于\emph{重建}逼近$f(x)$的函数$\tilde{f}(x)$.
        \refsub{混叠}介绍的采样原理准确描述了关于$f(x)$的条件、
        所需样本的数目,以及使得$\tilde{f}(x)$和$f(x)$一模一样的重建技术。
        原始函数有时能只从样本点中完全重建的事实令人瞩目。}
    \label{fig:7.1}
\end{figure}

\keyindex{傅里叶分析}{Fourier analysis}{}可用于评估重建函数与原始函数间的匹配质量。
本节将用丰富细节来介绍一部分采样和重建过程中涉及的傅里叶分析主要思想,
但略去了许多性质的证明并跳过了与pbrt所用的采样算法没有直接关系的细节。
本章“扩展阅读”一节有关于这些话题详细信息的指引。

\subsection{频域与傅里叶变换}\label{sub:频域与傅里叶变换}
傅里叶分析的基础之一是\keyindex{傅里叶变换}{Fourier transform}{},
它在\keyindex{频域}{frequency domain}{}中来表示函数(我们称
通常的函数是在\keyindex{空域}{spatial domain}{}中表示的)。
考虑\reffig{7.2}中的两个函数。\reffig{7.2.1}中$x$的函数变化得相对较慢,
而\reffig{7.2.2}中的函数变化得迅速得多。称变化越慢的函数有越低频的分量。
\begin{figure}[htbp]
    \centering
    \subfloat[]{\includegraphics[width=0.32\linewidth]{chap07/func-lowfreq.eps}\label{fig:7.2.1}}\,\,\,\,
    \subfloat[]{\includegraphics[width=0.32\linewidth]{chap07/func-highfreq.eps}\label{fig:7.2.2}}
    \caption{(a)低频函数和(b)高频函数。粗略地说,函数频率越高,在给定区域内变化得越快。}
    \label{fig:7.2}
\end{figure}

\reffig{7.3}展示了这两个函数在频率空间的表示;低频函数的表示比高频函数更快变为0.
\begin{figure}[htbp]
    \centering
    \subfloat[]{\includegraphics[width=0.32\linewidth]{chap07/fourier-lowfreq.eps}}\,\,\,\,
    \subfloat[]{\includegraphics[width=0.32\linewidth]{chap07/fourier-highfreq.eps}}
    \caption{\reffig{7.2}中的函数的频率空间表示。本图展示了每个频率$\omega$对空域中每个函数的贡献。}
    \label{fig:7.3}
\end{figure}

许多函数可以分解为平移过的正弦曲线的加权和。
约瑟夫·傅里叶\sidenote{译者注:Jean-Baptiste Joseph Fourier,
    18至19世纪法国著名数学家和物理学家。其中文译名还常作“傅立叶”。}
首先描述了这一奇特事实,傅里叶变换即将函数转换为该表示。
函数的频率空间表示便于深入了解其一些特点——正弦函数的频率分布对应于原函数的频率分布。
使用该形式后,就能用傅里叶分析深入了解采样和重建过程引入的误差以及如何降低该误差带来的感知影响。

1D函数$f(x)$的傅里叶变换为\footnote{要告知读者的是,
    在不同领域中该积分前的常数并不总是一样的。例如一些作者(包括许多物理界的)
    更喜欢在两个积分前乘上$\frac{1}{\sqrt{2\pi}}$.}
\begin{align}\label{eq:7.1}
    F(\omega)=\int_{-\infty}^{\infty}f(x)\mathrm{e}^{-\mathrm{i}2\pi\omega x}\mathrm{d}x\, .
\end{align}
(回想$\mathrm{e}^{-\mathrm{i}x}=\cos x+\mathrm{i}\sin x$,其中$\mathrm{i}=\sqrt{-1}$.)
为了简便,这里我们将只考虑\keyindex{偶函数}{even function}{},
即$f(-x)=f(x)$,这种情况下$f$的傅里叶变换没有虚数项。
新函数$F$是\keyindex{频率}{frequency}{}$\omega$的函数
\footnote{本章中,我们将用符号$\omega$表示频率。在本书剩下部分中,$\omega$表示规范化的方向向量。
    这种记号的重复在使用它们的给定上下文中不应混淆。简单来说,当我们说函数的“频谱”(spectrum)时,
    我们是在说它在其频率空间表示中的频率分布,而不是和颜色相关的东西。}。
我们将用$\mathcal{F}$表示傅里叶变换运算
\sidenote{译者注:原文使用的符号是$\mathrm{F}$,这里译者换用更常用的花体$\mathcal{F}$,
    也更利于阅读时与其他符号区分。},即$\mathcal{F}\{f(x)\}=F(\omega)$.
$\mathcal{F}$显然是线性运算——即对任意标量$a$都有$\mathcal{F}\{af(x)\}=a\mathcal{F}\{f(x)\}$,
且$\mathcal{F}\{f(x)+g(x)\}=\mathcal{F}\{f(x)\}+\mathcal{F}\{g(x)\}$.

\refeq{7.1}称为傅里叶分析方程,
有时简称\keyindex{傅里叶变换}{Fourier transform}{}
\sidenote{译者注:一般要求函数$f(x)$是实数域上
    的\keyindex{可积函数}{integrable function}{}。}。
我们也可用\keyindex{傅里叶合成}{Fourier synthesis}{}方程从频域变换回空域,
也称作\keyindex{傅里叶逆变换}{inverse Fourier transform}{}
\sidenote{译者注:本书定义中$\omega$是频率,此外用角频率的定义也很常见。}:
\begin{align}\label{eq:7.2}
    f(x)=\int_{-\infty}^{\infty}F(\omega)\mathrm{e}^{\mathrm{i}2\pi\omega x}\mathrm{d}\omega\, .
\end{align}

\reftab{7.1}展示了许多重要函数及其频率空间表示
\sidenote{译者注:表中原文将频域函数写作$f(\omega)$,译者改为了$F(\omega)$.
    译者还把原文余弦函数的频域表示改为等价但更常见的写法;
    此外,原文表中shah函数一行系数与后文矛盾,已对其作了修正。
    具体推导过程可参考译者补充的\refsec{译者补充:傅里叶变换}。}。
这些函数中许多都基于\keyindex{狄拉克$\delta$分布}{Dirac delta distribution}{},
该空间函数的定义使得$\displaystyle\int_{-\infty}^{\infty}\delta(x)\mathrm{d}x=1$,且对任意$x\neq0$,都有$\delta(x)=0$.
这些性质的一个重要结论是\sidenote{译者注:我为该式加上了积分上下限以表明它是定积分。}
\begin{align*}
    \int_{-\infty}^{\infty} f(x)\delta(x)\mathrm{d}x=f(0)\, .
\end{align*}

$\delta$分布不能表示为标准数学函数,但通常可以视作以原点为中心且宽度逼近0的
单位面积矩形函数\sidenote{译者注:原文box function。}的极限。
\begin{table}[htbp]
    \centering\begin{tabular}{l p{170pt}}
        \toprule
        {\bfseries 空域}                                         & {\bfseries 频率空间表示}                                                                                       \\
        \midrule
        矩形函数:$f(x)=\left\{\begin{array}{ll}
                1, & \text{若}|x|<\frac{1}{2}, \\
                0, & \text{其他}.
            \end{array}\right.$ & sinc函数:$\displaystyle F(\omega)=\mathrm{sinc}(\omega)=\frac{\sin(\pi\omega)}{\pi\omega}$                    \\
        \hline
        高斯函数:$f(x)=\mathrm{e}^{-\pi x^2}$                   & 高斯函数:$F(\omega)=\mathrm{e}^{-\pi \omega^2}$                                                               \\
        \hline
        常函数:$f(x)=1$                                         & $\delta$函数:$F(\omega)=\delta(\omega)$                                                                       \\
        \hline
        余弦函数:$f(x)=\cos x$                                  & 平移的$\delta$函数:
        $F(\omega)=\frac{1}{2}\left(\delta\left(\omega-\frac{1}{2\pi}\right)+\delta\left(\omega+\frac{1}{2\pi}\right)\right)$                                                     \\
        \hline
        shah函数:$f(x)=III_T(x)=\sum\limits_k\delta(x-kT)$      & $F(\omega)=\frac{1}{T}III_{\frac{1}{T}}(\omega)=\frac{1}{T}\sum\limits_k\delta\left(\omega-\frac{k}{T}\right)$ \\
        \bottomrule
    \end{tabular}
    \caption{傅里叶变换对。空域中的函数及其频率空间表示。
        因为傅里叶变换的对称性,如果左边一列被当作频率空间,
        则右边一列是这些函数的空间等价。}
    \label{tab:7.1}
\end{table}

\subsection{理想采样与重建}\label{sub:理想采样与重建}
利用频率空间分析,我们现在能正式研究采样的性质了。
回想采样过程要求我们选择一组等间隔样本位置并计算这些位置的函数值。
形式上,这对应于让该函数乘以“shah”——或称“冲激串”函数
\sidenote{译者注:也称\keyindex{狄拉克梳状函数}{Dirac comb function}{}、
    \keyindex{shah函数}{shah function}{}或\keyindex{采样函数}{sampling function}{}。},
即无数等间隔的$\delta$函数之和。
shah函数$III_T(x)$定义为
\sidenote{译者注:为了和虚数单位$\mathrm{i}$更好区分,
    译者将式子中的下标$i$改为$k$,下同;
    此外,译者去掉了该式系数$T$.}
\begin{align*}
    III_T(x)=\sum\limits_{k=-\infty}^{\infty}\delta(x-kT)\, ,
\end{align*}
其中$T$定义了\keyindex{周期}{period}{},也称\keyindex{采样率}{sampling rate}{}。
\reffig{7.4}展示了采样的正式定义。相乘后得到函数在等间隔点处取值的无限序列
\sidenote{译者注:我去掉了该式系数$T$.}:
\begin{align*}
    III_T(x)f(x)=\sum\limits_k\delta(x-kT)f(kT)\, .
\end{align*}
\begin{figure}[htbp]
    \centering
    \subfloat[]{\includegraphics[width=0.32\linewidth]{chap07/func-to-sample.eps}}\,
    \subfloat[]{\includegraphics[width=0.32\linewidth]{chap07/shah-samples.eps}}\,
    \subfloat[]{\includegraphics[width=0.32\linewidth]{chap07/shah-sampled-function.eps}}
    \caption{形式化的采样过程。(a)函数$f(x)$乘以(b)shah函数$III_T(x)$,
        得到(c)表示其在每个样本点处取值的缩放后的$\delta$函数的无限序列。}
    \label{fig:7.4}
\end{figure}

通过选择重建\keyindex{滤波}{filter}{}函数$r(x)$并计算\keyindex{卷积}{convolution}{},
这些样本值可用于定义重建的函数$\tilde{f}$,即
\begin{align*}
    (III_T(x)f(x))\otimes r(x)\, ,
\end{align*}
其中卷积运算$\otimes$定义为
\begin{align*}
    f(x)\otimes g(x)=\int_{-\infty}^{\infty}f(x')g(x-x')\mathrm{d}x'\, .
\end{align*}

对于重建,卷积给出以样本点为中心缩放后的重建滤波器实例加权和
\sidenote{译者注:我去掉了该式系数$T$.}:
\begin{align*}
    \tilde{f}(x)=\sum\limits_{k=-\infty}^{\infty}f(kT)r(x-kT)\, .
\end{align*}

例如\reffig{7.1}中使用了三角形重建\keyindex{滤波器}{filter}{}$r(x)=\max(0,1-|x|)$.
\reffig{7.5}展示了为该例所用的缩放后的三角形函数。
\begin{figure}[htbp]
    \centering\includegraphics[width=0.6\linewidth]{chap07/func-tri-reconstruction.eps}
    \caption{虚线表示的三角形重建滤波器实例的和给出了实线表示的对原始函数的重建逼近。}
    \label{fig:7.5}
\end{figure}

为了得到直观的结果,我们经历了看似不用这么复杂的过程:
用一些方法在样本间插值也能得到重建函数$\tilde{f}(x)$.
然而通过仔细构建这些背景,傅里叶分析现在能更简单地用于该过程。

通过在频域分析采样函数,我们能更深入理解采样过程。
特别地,我们将能确定原始函数能从其在样本位置的取值中完全恢复的条件——一个很强的结论。
对于此处的讨论,我们现在假设函数$f(x)$是\keyindex{带限}{band limited}{}的——
存在某个频率$\omega_0$使得$f(x)$在大于$\omega_0$处不再包含任何频率。
根据定义,带限函数具有紧支撑\sidenote{译者注:compact support。}的频率空间表示,
即对于所有$|\omega|>\omega_0$都有$F(\omega)=0$.\reffig{7.3}中的两个频谱都是带限的。

傅里叶分析所用的一个重要思想是两个函数之积的傅里叶变换$\mathcal{F}\{f(x)g(x)\}$可
表示为它们各自傅里叶变换$F(\omega)$和$G(\omega)$的卷积:
\begin{align*}
    \mathcal{F}\{f(x)g(x)\}=F(\omega)\otimes G(\omega)\, .
\end{align*}

类似地,空域卷积等价于频域乘积:
\begin{align}\label{eq:7.3}
    \mathcal{F}\{f(x)\otimes g(x)\}=F(\omega)G(\omega)\, .
\end{align}

这些性质是傅里叶分析的标准参考文献中得来的。
利用这些思想可以发现,空域中原始的采样步骤,即shah函数与原始函数$f(x)$相乘,
可等价描述为频域中$F(\omega)$与另一shah函数的卷积。

从\reftab{7.1}中我们还知道shah函数$III_T(x)$的频谱;
周期为$T$的shah函数的傅里叶变换是另一个周期为$\displaystyle\frac{1}{T}$的shah函数。
牢记周期间的倒数关系很重要:它意味着如果样本在空域中隔得较远,
则它们在频域中离得更近。

因此采样信号的频域表示通过$F(\omega)$和新的shah函数的卷积给出。
让$\delta$函数与一个函数卷积得到该函数副本,故用shah函数卷积
得到原始函数副本的无限序列,间隔等于该shah的周期(\reffig{7.6})。
它是样本序列的频率空间表示。
\begin{figure}[htbp]
    \centering\includegraphics[width=0.45\linewidth]{chap07/func-convolve-shah.eps}
    \caption{$F(\omega)$与shah函数的卷积。结果是$F$的无数个副本。}
    \label{fig:7.6}
\end{figure}

现在我们有了该函数频谱副本的无限集,我们该怎样重建原始函数呢?
观察\reffig{7.6},答案很明显:只需要抹除除了以原点为中心外的所有频谱副本,就能得到原始的$F(\omega)$.
\begin{figure}[htbp]
    \centering
    \subfloat[]{\includegraphics[width=0.32\linewidth]{chap07/func-convolve-shah-a.eps}}\,
    \subfloat[]{\includegraphics[width=0.32\linewidth]{chap07/unit-box-filter.eps}}\,
    \subfloat[]{\includegraphics[width=0.32\linewidth]{chap07/single-func-after-box.eps}}
    \caption{(a)$F(\omega)$的副本序列和(b)合适的矩形函数相乘得到(c)原始频谱。}
    \label{fig:7.7}
\end{figure}

为了丢弃除了中间外的所有频谱副本,我们乘以具有合适宽度的矩形函数(\reffig{7.7})。
宽度为$T$的矩形函数$\textstyle\prod_T(x)$定义为
\begin{align*}
    {\textstyle\prod_T}(x)=\left\{\begin{array}{ll}
        \displaystyle\frac{1}{T}, & \text{若}\displaystyle|x|<\frac{T}{2}, \\
        0,                        & \text{其他}.
    \end{array}\right.
\end{align*}

该相乘步骤对应了用重建滤波器在空域做卷积。这是理想采样与重建过程。总结为
\sidenote{译者注:我增加了系数$\frac{1}{T}$.}:
\begin{align*}
    \tilde{F}=(F(\omega)\otimes \frac{1}{T}III_{\frac{1}{T}}(\omega))\textstyle\prod_{\frac{1}{T}}(\omega)\, .
\end{align*}

这是个重要结论:仅仅通过采样一组均匀间隔的点,我们就能确定$f(x)$的精准频率空间表示。
除了知道该函数是带限的外,没有使用关于函数成分的额外信息。

在空域运用等价过程同样能完全恢复$f(x)$.因为矩形函数的傅里叶逆变换是sinc函数,
所以空域中的理想重建是
\begin{align*}
    \tilde{f}=(f(x)III_T(x))\otimes \mathrm{sinc}_T(x)\, ,
\end{align*}
其中$\mathrm{sinc}_T(x)=\mathrm{sinc}(Tx)$,因此\sidenote{译者注:原文该式
    将$\mathrm{sinc}_T(x-kT)$误写为$\mathrm{sinc}(x-kT)$,已修正。}
\begin{align*}
    \tilde{f}(x)=\sum\limits_{k=-\infty}^{\infty}\mathrm{sinc}_T(x-kT)f(kT)\, .
\end{align*}

不幸的是,因为sinc函数有无限定义域,所以必须用所有采样值$f(kT)$来计算空域中$\tilde{f}(x)$的任一特定值。
实际实现中更爱用空间范围有限的滤波器,即使它们不能完美重建原始函数。

图形学常用的可选方法是用矩形函数做重建,即高效地对$x$附近区域内的全部样本值做平均。
通过考虑矩形滤波器的频域表现可以看到这是非常糟糕的选择:
该技术试图通过\emph{乘以sinc}来分离出函数频谱中间的副本,
这不仅在选出函数频谱中央副本方面做得很差,
还包含了无限序列中其他副本的高频贡献。

\subsection{混叠}\label{sub:混叠}
除了sinc函数无限定义域的问题外,理想采样与重建方法一个最严重的实际问题是它假设信号是带限的。
对于非带限信号,或者没能以足够高的采样率采样其频率成分的信号,
之前描述的过程会重建出与原始信号不同的函数。

成功重建的关键是用宽度合适的矩形函数相乘以完全恢复原始频谱$F(\omega)$的能力。
注意在\reffig{7.6}中,信号频谱的副本被空白空间分隔,所以能够被完美重建。
然而如果以更低的采样率采样原始函数,考虑一下会发生什么。
回想周期为$T$的shah函数$III_T$的傅里叶变换是周期为$\displaystyle\frac{1}{T}$的新shah函数。
这意味着如果空域中样本间的距离增大,频域的样本间隔会变小,
将频谱$F(\omega)$的副本挤在一起。如果副本挨得太近,它们就开始重叠。

因为副本被加在一起,所以得到的频谱看起来不再像许多原始的副本(\reffig{7.8})。
当该新频谱乘以矩形函数后,结果是相似但不等于原始$F(\omega)$的频谱:
原始信号的高频细节渗入到重建信号频谱的低频区域。
这些新的低频伪影称作\keyindex{混叠}{alias}{}(因为高频“伪装”成低频),
得到的信号被称是\keyindex{混叠的}{aliased}{alias混叠}。
\begin{figure}
    \centering
    \subfloat[]{\includegraphics[width=0.4\linewidth]{chap07/freq-space-overlap.eps}}\,
    \subfloat[]{\includegraphics[width=0.4\linewidth]{chap07/freq-space-aliasing.eps}}
    \caption{(a)采样率过低时,函数频谱副本会重叠,当执行重建时会导致(b)混叠。}
    \label{fig:7.8}
\end{figure}

\reffig{7.9}\sidenote{译者注:原文该图题注函数表达式有笔误,已修正。}
展示了欠采样并重建1D函数$f(x)=1+\cos(4\pi x^2)$时的混叠效应。
\begin{figure}[htbp]
    \centering
    \subfloat[]{\includegraphics[width=0.4\linewidth]{chap07/freq-increasing-func.eps}}\,
    \subfloat[]{\includegraphics[width=0.4\linewidth]{chap07/freq-increasing-aliased.eps}}
    \caption{函数$f(x)=1+\cos(4\pi x^2)$采样点的混叠。(a)该函数。
        (b)以0.125单位为样本间隔采样并用sinc滤波器执行完美重建后所重建出的函数。
        混叠造成原始函数的高频信息被丢失了并作为低频误差重新出现。}
    \label{fig:7.9}
\end{figure}

一种可能解决重叠频谱问题的办法是简单地增加采样率
直到频谱的副本隔得足够远而不再重叠,进而完全消除混叠。
事实上,\keyindex{采样定理}{sampling theorem}{}准确告诉我们所需的采样率。
该定理说只要均匀样本点的频率$\omega_s$大于信号中出现的最大频率$\omega_0$的两倍,
就能从样本中完美重建原始信号。该最小采样频率称为\keyindex{奈奎斯特频率}{Nyquist frequency}{frequency频率}
\sidenote{译者注:哈里·奈奎斯特(Harry Nyquist),19至20世纪瑞典裔美国著名物理学家,通讯理论的奠基者之一。}。

对于非带限信号($\omega_0=\infty$),不可能以足够高的采样率执行完美重建。
非带限信号有无限支撑的频谱,所以无论其频谱副本隔得多远(即无论我们用多高的采样率),
都总会有重叠。不幸的是,计算机图形学中要处理的函数很少是带限的。
特别地,任何不连续的函数都不是带限的,因此我们不能完美采样和重建它。
这是有道理的,因为两个样本间的函数连续性是不清楚的,样本没有提供不连续处的信息。
因此除了提高采样率外还必须用不同方法来消除混叠可能引入到渲染器结果中的误差。

\subsection{抗锯齿技术}\label{sub:抗锯齿技术}
如果不仔细对待采样和重建,则最终图像中可能出现大量伪影。
有时区分采样伪影与重建伪影很有用;确切地说,我们会称采样伪影为\keyindex{预混叠}{prealiasing}{alias混叠},
称重建伪影为\keyindex{后混叠}{postaliasing}{alias混叠}。任意想要修正这些误差的尝试都
大体划分为\keyindex{抗锯齿}{antialiasing}{}\sidenote{译者注:也称反混叠。}。
本节将回顾除了只增加采样率外的大量抗锯齿技术。

\subsubsection*{非均匀采样}
尽管知道我们要采样的图像函数有无穷的频率成分因而不能从样本点中完美重建,
但以非均匀的方式改变样本间隔有可能降低混叠的视觉影响。
如果$\xi$表示在0到1间的随机数,则基于冲激串的非均匀样本集为
\begin{align*}
    \sum\limits_{k=-\infty}^{\infty}\delta\left(x-(k+\frac{1}{2}-\xi)T\right)\, .
\end{align*}

对于不足以刻画该函数的固定采样率,均匀和非均匀采样都会得到不正确的重建信号。
然而非均匀采样倾向于将规则的混叠伪影转化为不容易引起人类视觉系统注意的噪声。
在频率空间,采样信号的副本最终也被随机平移,
所以当执行重建时结果是随机误差而不是有条理的混叠。

\subsubsection*{自适应采样}
另一个对抗混叠的建议方法是\keyindex{自适应超采样}{adaptive supersampling}{}:
如果我们能辨别出频率高于奈奎斯特上限的信号区域,
则我们可以在这些区域再取额外样本而不用承担在每处都增加采样频率所致的计算开销。
在实际中让该方法奏效是很困难的,因为寻找所有需要超采样的地方会很难。
大多数这样做的技术都基于测试相邻样本值并找到两值间有明显变化的地方;
然后假设该区域信号有较高频率。

通常相邻样本值不能确定地告诉我们它们之间到底发生了什么:
即使它们值相同,函数也可能在它们间有巨大变化。
或者相邻样本可能有相差很大的值但实际上并没有出现任何混叠。
例如,第\refchap{纹理}的纹理滤波算法全力消除场景中图像贴图和表面过程纹理造成的混叠;
我们不想让自适应采样例程在纹理值迅速变化但实际上没有出现过高频率的区域不必要地采额外样本。

\subsubsection*{预滤波}
采样理论提供的另一个消除混叠的方法是对原始函数滤波(即模糊)
使得所用采样率不能精确捕获的高频率不再保留下来。
该方法应用于第\refchap{纹理}的纹理函数。
该技术通过从被采样函数中移除信息来改变其特性,模糊一般不如混叠令人讨厌。

回想我们要用选定宽度的矩形滤波器与原始函数频谱相乘使得
奈奎斯特上限之上的频率被移除。在空域,这对应于原始函数与sinc滤波器做卷积,
\begin{align*}
    f(x)\otimes \mathrm{sinc}(2\omega_sx)\, .
\end{align*}

在实际中,我们也可以用范围有限的滤波器。该滤波器的频率空间表示能
帮助弄清它能有多逼近理想sinc滤波器的表现。

\reffig{7.10}展示了函数$1+\cos(4\pi x^2)$与\refsec{图像重建}
介绍的范围有限的sinc变种的卷积\sidenote{译者注:原文正文与插图
    题注的函数表达式均有笔误,与图示不符,译者已修改。}。
注意高频细节被消除了;该函数可用\reffig{7.9}的采样率采样和重建而无混叠。
\begin{figure}[htbp]
    \centering\includegraphics[width=0.4\linewidth]{chap07/highfreq-prefiltered.eps}
    \caption{函数$1+\cos(4\pi x^2)$与移除采样率$T=0.125$对应的
        奈奎斯特上限之外频率的滤波器的卷积。高频细节已从该函数移除掉,
        使得新函数至少能无混叠地被采样与重建。}
    \label{fig:7.10}
\end{figure}

\subsection{应用到图像合成}\label{sub:应用到图像合成}
这些思想应用到2D情况的渲染场景的图像采样和重建很简单:
我们有一张图像可视作2D图像位置$(x,y)$到辐亮度值$L$的函数:
\begin{align*}
    f(x,y)\rightarrow L\, .
\end{align*}

好消息是,有了我们的光线追踪器,我们能在我们所选的任意点$(x,y)$处求该函数的值。
坏消息是,一般不太可能在采样前对$f$预滤波来从中移除高频。
因此本章采样器使用两种策略,既将采样率提升至超过最终图像基础像素间隔,
也有非均匀分布样本以将混叠转化为噪声。

将场景函数的定义推广为也依赖于时间$t$以及采样处的透镜位置$(u,v)$的更高维函数很有用。
因为来自相机的光线基于这五个量,它们中任意一个变化都会得到不同的光线,
进而可能是不同的$f$值。对于特定的图像位置,该点的辐亮度一般
随时间(如果场景中有运动物体)和透镜上的位置(如果相机有光圈有限的透镜)变化。

更一般地,因为第\refchap{光传输I:表面反射}到第\refchap{光传输III:双向方法}
定义的许多积分器都用统计技术来估计沿给定光线的辐亮度,
所以当重复给定相同光线时它们可能返回不同的辐亮度值。
如果我们进一步将场景辐亮度函数扩展至包含积分器所用的样本值
(例如,为了照明计算而用于在面光源上选点的值),
我们就有甚至更高维的图像函数
\begin{align*}
    f(x,y,t,u,v,i_1,i_2,\ldots)\rightarrow L\, .
\end{align*}

采样好所有这些维度是高效生成高质量图像的重要一部分。
例如如果我们保证图像上位置$(x,y)$附近倾向于在透镜上有不同的$(u,v)$,
则得到的渲染图像会有更小的误差,因为每个样本更有可能考虑了其相邻样本没有考虑的场景信息。
后面几节的类\refvar{Sampler}{}会解决高效采样所有这些维度的问题。

\subsection{渲染中的混叠来源}\label{sub:渲染中的混叠来源}
几何体是在渲染图像中造成混叠的最常见因素之一。
当投影到图像平面时,物体的边界引入了\keyindex{阶跃函数}{step function}{}——
图像函数的值突然从一个值跳到另一个值。
不仅阶跃函数像前面所述那样有无穷的频率成分,
而且更糟糕的是,完美重建滤波器在运用于混叠样本时也会造成伪影:
重建函数中出现\keyindex{振铃}{ringing}{}伪影,
即称作\keyindex{吉布斯现象}{Gibbs phenomenon}{}的效应。
\reffig{7.11}为1D函数展示了该效应的例子。
正如我们将在本章后续所看到的,选择有效的重建滤波器来面对混叠需要科学、艺术以及个人品味。
\begin{figure}[htbp]
    \centering\includegraphics[width=0.5\linewidth]{chap07/gibbs-phenomenon.eps}
    \caption{吉布斯现象的图示。当没有以奈奎斯特采样率采样函数却又用
        sinc滤波器重建一组混叠的样本时,重建的函数会有“振铃”伪影,它在真实函数附近振荡。
        这里用0.125的样本间隔采样1D阶跃函数(虚线)。当用sinc重建时,振铃出现了(实线)。}
    \label{fig:7.11}
\end{figure}

场景中非常小的物体也会造成几何混叠。如果几何体小到落入图像平面样本之间,
则它会在一个动画的若干帧中不可预测地消失和重现。

混叠的另一个来源可能来自物体上的纹理和材质。没有被正确滤波的纹理贴图
(解决该问题是第\refchap{纹理}的主要内容)或光泽表面的小高光
可能造成\keyindex{着色混叠}{shading aliasing}{alias混叠}。
如果采样率没有高到足以充分采样这些特征,则会导致混叠。
此外,一个物体投射的尖锐阴影会在最终图像中引入另一个阶跃函数。
尽管有可能从图像平面上的几何边来辨别阶跃函数的位置,
但从阴影边界中检测阶跃函数则更加困难。

对于渲染图像中混叠的关键认识是,我们永远不可能移除所有这些来源,
所有我们必须开发技术来减轻其对最终图像质量的影响。

\subsection{理解像素}\label{sub:理解像素}
在本章剩余内容中牢记两个关于像素的观点很重要。
第一,一定记住构成图像的像素是图像函数在图像平面上离散位置的样本点;这样的像素没有相应的“面积”。
正如\citet{Smith95apixel}着重指出的,将像素视作具有有限面积的小方形是错误的认知模型,
会导致一系列问题。通过用信号处理的方法介绍本章话题,我们尝试为更准确的认知模型奠定基础。

第二个问题是最终图像中的像素是在像素网格上的离散整数坐标$(x,y)$处自然定义的,
但本章的\refvar{Sampler}{}是在连续的浮点位置$(x,y)$处生成图像样本的。
映射这两个域的自然方法是将连续坐标舍入到最近的离散坐标;
既然它把刚好和离散坐标有相同值的连续坐标就映射为那个离散坐标,该方法看起来不错。
然而结果是,给定覆盖范围$[x_0,x_1]$的离散坐标集,则连续坐标集覆盖范围为$\displaystyle\left[x_0-\frac{1}{2},x_1+\frac{1}{2}\right)$.
因此任何为给定离散像素范围生成连续样本位置的代码都被$\displaystyle\frac{1}{2}$的偏移量扰乱。
它们很容易被忘记并导致隐晦的错误。

如果我们改用
\begin{align*}
    d=\lfloor c\rfloor\, ,
\end{align*}
将连续坐标$c$截断为离散坐标$d$,并通过
\begin{align*}
    c=d+\frac{1}{2}\, ,
\end{align*}
将离散转换为连续,则离散范围$[x_0,x_1]$对应的连续坐标范围
自然是$[x_0,x_1+1)$且所得代码会简单得多\citep{HECKBERT1990246}。
\reffig{7.12}展示了我们将在pbrt中采用的这一转化。
\begin{figure}[htbp]
    \centering\includegraphics[width=0.5\linewidth]{chap07/Pixelsdiscretecontinuous.eps}
    \caption{图像中的像素可以解释为\emph{离散}或\emph{连续}坐标。
    离散图像五个像素的宽度覆盖了连续像素范围$[0,5)$.
    特定离散像素$d$的坐标在连续表示中为$\displaystyle d+\frac{1}{2}$.}
    \label{fig:7.12}
\end{figure}

\section{采样接口}\label{sec:采样接口}
正如先在\refsub{应用到图像合成}介绍的,
pbrt中实现的渲染方法包含了在图像平面的2D点之外的额外维度上选择样本点。
各种算法将用于生成这些点,但它们的所有实现都继承自定义其接口的抽象类\refvar{Sampler}{}。
核心采样声明和函数在文件\href{https://github.com/mmp/pbrt-v3/blob/master/src/core/sampler.h}{\ttfamily core/sampler.h}
和\href{https://github.com/mmp/pbrt-v3/blob/master/src/core/sampler.cpp}{\ttfamily core/sampler.cpp}中。
样本生成的每种实现都在目录{\ttfamily samplers/}下其自己的源文件内。

\refvar{Sampler}{}的任务是生成$[0,1)^n$中$n$维样本的序列,
其中每个图像样本都要为其生成这样的样本向量,且每个样本中的维数$n$可能会变,
这取决于光传输算法执行的计算(见\reffig{7.13})。
\begin{figure}[htbp]
    \centering\includegraphics[width=0.9\linewidth]{chap07/Samplerndimensional.eps}
    \caption{采样器为每个图像样本生成用来合成最终图像的$n$维样本向量。
        这里,像素$(3,8)$正被采样,且在该像素区域内有两个图像样本。
        样本向量的前两维给出样本在该像素内的偏移量$(x,y)$,
        接下来三维决定相应相机光线的时间和透镜位置。后续维度用于
        第\refchap{光传输I:表面反射}、\refchap{光传输II:体积渲染}和\refchap{光传输III:双向方法}中
        的蒙特卡罗光传输算法。这里,光传输算法已经请求了样本向量中的四个2D数组样本;
        例如,这些值可能用于选择面光源上的四个点来为图像样本计算辐亮度。}
    \label{fig:7.13}
\end{figure}

因为样本值必须严格小于1,所以定义一个常数\refvar{OneMinusEpsilon}{}很有用,
它表示小于1的最大可表示浮点常数。然后,我们会截断样本向量值使之不大于该值。
\begin{lstlisting}
`\initcode{Random Number Declarations}{=}`
#ifdef PBRT_FLOAT_IS_DOUBLE
static const `\refvar{Float}{}` `\initvar{OneMinusEpsilon}{}` = 0x1.fffffffffffffp-1;
#else
static const `\refvar{Float}{}` OneMinusEpsilon = 0x1.fffffep-1;
#endif
\end{lstlisting}

可能最简单的\refvar{Sampler}{}实现是当每次需要样本向量的额外分量时直接返回$[0,1)$内的均匀随机值。
这样的采样器可产生正确的图像但会需要非常多的样本(以及更多要追踪的光线与更多的时间)来
创建用更先进采样器所能取得的相同质量的图像。
使用更佳采样模式的运行时间开销大致和用诸如均匀随机数的低质量模式相同;
因为为每个图像样本计算辐亮度比计算样本的分量值会有大得多的开销,
所以这样做是有回报的(\reffig{7.14})。
\begin{figure}[htbp]
    \centering
    \subfloat[差的采样]{\includegraphics[width=\linewidth]{chap07/spheres-bad-sampler.png}\label{fig:7.14.1}}\\
    \subfloat[更好的采样]{\includegraphics[width=\linewidth]{chap07/spheres-better-sampler.png}\label{fig:7.14.2}}
    \caption{用(a)相对低效的采样器和(b)精心设计的采样器渲染的场景,
        它们用了同样多的样本。从高光边缘到光泽反射,图像质量的提升是明显的。}
    \label{fig:7.14}
\end{figure}

下面假设这些样本向量的一些特性:
\begin{itemize}
    \item \refvar{Sampler}{}生成的前五维通常由\refvar{Camera}{}使用。
          这种情况下,前两个专门用于选择图像上当前像素区域内的点;
          第三个用于计算应该取用该样本的时间;第四和五维为景深给出透镜位置$(u,v)$.
    \item 一些采样算法在样本向量的某些维度中生成的样本比其他维度更好。
          在系统其他地方,我们假设一般前面的维度具有放置得最好的样本值。
\end{itemize}

还要注意\refvar{Sampler}{}生成的$n$维样本通常不会整个显式表示或存储,
而是常常按照光传输算法的需要逐步生成。(然而,存储整个样本向量并对其分量逐渐作出调整
是\refsub{基本样本空间采样器}中\refvar{MLTSampler}{}的基础,
它用于\refsub{MLT积分器}的\refvar{MLTIntegrator}{}。)

\subsection{评估样本模式:偏差}\label{sub:评价样本模式:偏差}
\begin{remark}
    本节含有高级内容,第一次阅读时可以跳过。
\end{remark}

傅里叶分析给了我们一种方法来评估2D采样模式的质量,
但它只是让我们能够在可表示的带限频率方面量化增加更均匀间隔的样本所带来的提升。
考虑到图像中出现了来自边缘的无穷频率成分以及蒙特卡罗光传输算法对$(n>2)$维样本向量的需求,
傅里叶分析对于我们的需求而言是不够的。

给定一个渲染器和放置样本的候选算法,一种评估该算法效果的方法是用其
采样模式来渲染图像并计算它和用大量样本渲染的参考图像相比的误差。
本章后面我们将用该方法比较采样算法,不过它只告诉了我们该算法对于特定场景的表现如何,
且若没有经过渲染过程它将不能让我们感觉出样本点的质量。

除了傅里叶分析,数学家还发明了一个称作\keyindex{偏差}{discrepancy}{}的概念
用于评估$n$维样本位置模式的质量。分布良好(稍后形式化说明)的模式有低的偏差值,
且因此该样本模式生成问题可以考虑成寻找点的合适的\emph{低偏差}模式
\footnote{当然,这样使用偏差隐含假设了用于计算偏差的度量
    对于图像采样而言是与模式的质量有良好关联性的,这可能会有所区别,
    尤其是当人类视觉系统参与该过程时。}。
大量确定性技术已经被开发出来,甚至能在高维空间中生成低偏差点集
(本章后面所用的大多数采样算法都使用这些技术)。

偏差的基本思想是$n$维空间$[0,1)^n$中点集的质量可通过查看域$[0,1)^n$中的各区域、
数出每个区域内的点数并拿每个区域的体积和其内的样本点数作比较来评估。
通常,给定的某一占比体积内应该大致含有样本点总数的相同比例。
尽管不可能总是这种情况,但我们仍可尽量使用让实际体积与点估计的体积间的最大差异(即偏差)最小化的模式。
\reffig{7.15}展示了该思想在二维下的例子。
\begin{figure}[htbp]
    \centering\includegraphics[width=0.5\linewidth]{chap07/Boxdiscrepancy.eps}
    \caption{给定$[0,1)^2$中2D样本点集后矩形(阴影)的偏差。
    四个样本点中的一个在矩形内,所以该点集将把矩形的面积估为$\frac{1}{4}$.
    该矩形是真实面积是$0.3\times0.3=0.09$,所以该特定矩形的偏差
    为$0.25-0.09=0.16$.通常,我们关心的是找出所有可能的矩形(或某些其他形状)中的最大偏差。}
    \label{fig:7.15}
\end{figure}

为了计算点集的偏差,我们首先取作为$[0,1)^n$子集的一簇形状$B$.
例如常用一个角位于原点的方盒。其对应于
\begin{align*}
    B=\{[0,v_1]\times[0,v_2]\times\cdots\times[0,v_n]\}\, ,
\end{align*}
其中$0\le v_i<1$.给定样本点序列$P=x_1,\ldots,x_N$,
$P$关于$B$的偏差为\footnote{算符$\sup$,也称作\emph{上确界},给出了定义域内函数值的最紧上界。}
\begin{align}\label{eq:7.4}
    D_N(B,P)=\sup\limits_{b\in B}\left|\frac{\#\{x_i\in b\}}{N}-V(b)\right|\, ,
\end{align}
其中$\#\{x_i\in b\}$是$b$中的点数,$V(b)$是$b$的体积。

对于为什么\refeq{7.4}是合理的质量度量的直观解释是,
值$\displaystyle\frac{\#\{x_i\in b\}}{N}$是
由特定点集$P$给出的方盒$b$体积的近似。
因此,偏差是所有可能的方盒用这种办法逼近其体积时的最差误差。
当形状集$B$是一个角在原点的方盒集时,
该值称为\keyindex{星偏差}{star discrepancy}{discrepancy偏差}
\sidenote{译者注:也称“均匀性偏差”。},$D^*_N(P)$.
对于$B$的另一个流行选择是全体轴对齐框的集合,即去掉了一个角在原点的限制。

对于一些特定点集,可以解析计算偏差。例如考虑一维中的点集
\begin{align*}
    x_i=\frac{i}{N}\, .
\end{align*}
我们可以看到$x_i$的星偏差为\sidenote{译者注:原文将$x_N$误写为$x_n$,已修正。}
\begin{align*}
    D^*_N(x_1,\ldots,x_N)=\frac{1}{N}\, .
\end{align*}
例如,取区间$b=\displaystyle\left[0,\frac{1}{N}\right)$.则$V(b)=\displaystyle\frac{1}{N}$,
但$\#\{x_i\in b\}=0$.该区间(以及区间$\displaystyle\left[0,\frac{2}{N}\right)$等)
的体积与体积内所见点的比例有最大的差异。

该序列的星偏差可通过稍微对其改动来改进:
\begin{align}\label{eq:7.5}
    x_i=\frac{i-\frac{1}{2}}{N}\, .
\end{align}
则
\begin{align*}
    D^*_N(x_i)=\frac{1}{2N}\, .
\end{align*}
说明一维点序列的星偏差边界为
\sidenote{译者注:这里我简单推导了该式:假设$P=\{x_i\}_{i=1}^{N}$已经按照
升序排列。构造一个新序列$Q=\left\{\frac{j}{N}\right\}_{j=0}^{N}$,然后让$P$和$Q$的元素交错排列构造
新序列$U=\{u_k\}_{k=1}^{2N+1}=\left\{0,x_1,\frac{1}{N},\ldots,x_i,\frac{i}{N},\ldots,x_N,1\right\}$.
则依据偏差的定义可以证明,$D^*_N(x_i)=\max\limits_{1\le k\le 2N}{|u_{k+1}-u_k|}=\max\limits_{1\le i\le N}{d_i}$,
其中$d_i=\max{\left(\left|x_i-\frac{i-1}{N}\right|,\left|x_i-\frac{i}{N}\right|\right)}$.
又注意到$d_i=\frac{1}{2N}+\left|x_i-\frac{2i-1}{2N}\right|$,于是得到文中该式。}
\begin{align*}
    D^*_N(x_i)=\frac{1}{2N}+\max\limits_{1\le i\le N}\left|x_i-\frac{2i-1}{2N}\right|\, .
\end{align*}
因此,之前\refeq{7.5}中的序列具有1D序列中所能取到的最低偏差。
通常,分析和计算1D序列的偏差边界比高维简单得多。
对于构造更复杂的点序列、高维序列以及比方盒更不规则的形状,
通常必须通过构造大量形状$b$、计算其偏差并报告找到的最大值来数值地估计偏差。

聪明的读者会注意到根据低偏差度量,1D中该均匀序列是最优的,
但本章前面我们说过,对于2D中的图像采样,不规则的抖动模式优于均匀模式,
因为它们将混叠替换为噪声。在这一框架下,均匀样本显然不是最好的。
幸运的是,更高维的低偏差模式比在一维中更不均匀得多,
因此实际中通常能作为样本模式工作得很好。
然而,其根本的均匀性意味着低偏差模式比伪随机变化的模式更可能倾向于视觉上令人讨厌的混叠。

只靠偏差并不一定是好的度量:一些低偏差点集表现出样本的聚集性,
其中两个或以上样本可能靠得很近。\refsec{Sobol采样器}的Sobol采样器
\sidenote{译者注:得名于立陶宛裔俄罗斯著名数学家Ilya Meyerovich Sobol。
原文对“Sobol”这一名称均加上了撇号,译文将其略去了,下同。}
尤其困扰于该问题——见\reffig{7.36},它展示了其前两维的图示。
直觉上,靠得太近的样本不能很好地利用采样资源:
一个样本离另一个越近,它就越不可能给出关于被采样函数的新信息。
因此,计算点集中任意两个样本间的最小距离也已被证明是一种有用的样本模式质量度量;最小距离越大越好。

有各种算法用来生成在该度量下得分不错的\keyindex{泊松圆盘}{Poisson disk}{}采样模式。
通过构造,泊松圆盘模式内没有两个点比某一距离$d$更近。
研究已表明眼睛的视杆细胞和视锥细胞也按该方式分布,
这进一步验证了该分布适合用来成像的观点。
实际中,我们发现泊松圆盘模式对于采样2D图像工作得很好,
但对于更复杂的渲染情形中的更高维采样会比更好的低偏差模式更低效;
见“扩展阅读”一节了解更多信息。

\subsection{基本采样器接口}\label{sub:基本采样器接口}
基类\refvar{Sampler}{}不仅定义了采样器的接口,
还提供了一些通用功能供\refvar{Sampler}{}
的实现使用。
\begin{lstlisting}
`\initcode{Sampler Declarations}{=}\initnext{SamplerDeclarations}`
class `\initvar{Sampler}{}` {
public:
    `\refcode{Sampler Interface}{}`
    `\refcode{Sampler Public Data}{}`
protected:
    `\refcode{Sampler Protected Data}{}`
private:
    `\refcode{Sampler Private Data}{}`
};
\end{lstlisting}

所有\refvar{Sampler}{}实现必须提供指定了要为最终图像中每个像素生成的样本数量的构造函数。
在罕见情况下,将胶片建模为只有单个覆盖整个可视区域的“像素”可能对系统有用
(这种过载的像素定义有点夸张,但我们允许它简化某些实现方面)。
既然该“像素”可能有数十亿个样本,我们就用64位精度的变量来存储样本数量。
\begin{lstlisting}
`\initcode{Sampler Method Definitions}{=}\initnext{SamplerMethodDefinitions}`
`\refvar{Sampler}{}`::`\refvar{Sampler}{}`(int64_t samplesPerPixel)
: `\refvar{samplesPerPixel}{}`(samplesPerPixel) { }
\end{lstlisting}
\begin{lstlisting}
`\initcode{Sampler Public Data}{=}`
const int64_t `\initvar{samplesPerPixel}{}`;
\end{lstlisting}

当渲染算法准备好在给定像素上工作时,它通过提供
该像素在图像中的坐标并调用\refvar{StartPixel}{()}来开始。
一些\refvar{Sampler}{}实现利用哪个像素正被采样的知识来提升
其为该像素生成的样本整体分布,而其他的则忽略该信息。
\begin{lstlisting}
`\initcode{Sampler Interface}{=}\initnext{SamplerInterface}`
virtual void `\initvar{StartPixel}{}`(const `\refvar{Point2i}{}` &p);
\end{lstlisting}

方法\refvar{Get1D}{()}为当前样本向量的下一维返回样本值,
\refvar{Get2D}{()}则为下两维返回样本值。
尽管能通过使用调取一对\refvar{Get1D}{()}返回的值来构造2D样本值,
但一些采样器如果知道两维会一起用时能够生成更好的点分布。
\begin{lstlisting}
`\refcode{Sampler Interface}{+=}\lastnext{SamplerInterface}`
virtual `\refvar{Float}{}` `\initvar{Get1D}{}`() = 0;
virtual `\refvar{Point2f}{}` `\initvar{Get2D}{}`() = 0;
\end{lstlisting}

在pbrt中,我们不支持从采样器中获取3D或更高维度的样本值,
因为它们一般对于这里实现的渲染算法类型而言是非必需的。
如果需要,可用来自低维分量的多个值来构造高维样本点。

这些接口的一个显著特点是必须仔细编写使用样本值的代码使其
总是以同样的顺序获取样本维度。考虑下列代码:\\
{\ttfamily
sampler->StartPixel(p);\\
do \{\\
\indent Float v = a(sampler->Get1D());\\
\indent if (v > 0)\\
\indent \indent v += b(sampler->Get1D());\\
\indent v += c(sampler->Get1D());\\
\} while (sampler->StartNextSample());
}

情况下,样本向量的第一维总会传给函数{\ttfamily a()};
当执行调用{\ttfamily b()}的代码路径时,{\ttfamily b()}会收到第二维。
然而,若{\ttfamily if}测试并不总是为真或假,则{\ttfamily c()}有时
会从样本向量的第二维收到样本,否则从第三维收到样本。
因此,采样器为了提供在每个维度评估都分布良好的样本点所作的努力就白费了。
故需要仔细编写使用\refvar{Sampler}{}的代码使得它始终如一地用掉样本向量维度以避免该问题。

为了方便,基类\refvar{Sampler}{}提供了为给定像素初始化\refvar{CameraSample}{}的方法。
\begin{lstlisting}
`\refcode{Sampler Method Definitions}{+=}\lastnext{SamplerMethodDefinitions}`
`\refvar{CameraSample}{}` `\refvar{Sampler}{}`::`\initvar{GetCameraSample}{}`(const `\refvar{Point2i}{}` &pRaster) {
    `\refvar{CameraSample}{}` cs;
    cs.`\refvar{pFilm}{}` = (`\refvar{Point2f}{}`)pRaster + `\refvar{Get2D}{}`();
    cs.`\refvar[CameraSample::time]{time}{}` = `\refvar{Get1D}{}`();
    cs.`\refvar{pLens}{}` = `\refvar{Get2D}{}`();
    return cs;
}
\end{lstlisting}

一些渲染算法为其采样的某些维度利用了样本值数组;
比起生成一系列单独的样本,大多数样本生成算法通过考虑
数组所有元素上以及一个像素内所有样本上的样本值分布可生成更高质量的样本数组。

如果需要样本数组,则必须在渲染开始前请求之。
在渲染开始前——例如在重写了方法\refvar{SamplerIntegrator::Preprocess}{()}的方法中,
应该为每个这样维度的数组调用方法\refvar[Request1DArray]{Request[12]DArray}{()}。
例如,在具有两个面光源的场景中,当积分器追踪了四条阴影射线到第一个光源,
八条到第二个光源时,积分器会为每个图像样本请求两个2D样本数组,
每个分别有四个和八个样本(需要2D数组是因为需要两个维度来参数化光源表面)。
在\refsec{俄罗斯轮盘赌与划分}中,我们将看到使用样本数组会怎样对应于
用“划分”\sidenote{译者注:原文splitting。}的蒙特卡罗技术更密集地采样光传输积分的某些维度。
\begin{lstlisting}
`\refcode{Sampler Interface}{+=}\lastnext{SamplerInterface}`
void `\initvar{Request1DArray}{}`(int n);
void `\initvar{Request2DArray}{}`(int n);
\end{lstlisting}

大多数\refvar{Sampler}{}能更好地生成某些特定大小的数组。
例如,在数量为2的幂时,来自\refvar{ZeroTwoSequenceSampler}{}的样本
则分布得好得多。方法\linebreak
\refvar[RoundCount]{Sampler::RoundCount}{()}
帮助传达该信息。需要样本数组的代码应该用想要取用的样本数目调用该方法,
以给\refvar{Sampler}{}机会把样本数目调整到更好的值。
然后应该用返回的值作为从\refvar{Sampler}{}实际请求样本的数目。
默认实现返回不变的给定数目。
\begin{lstlisting}
`\refcode{Sampler Interface}{+=}\lastnext{SamplerInterface}`
virtual int `\initvar{RoundCount}{}`(int n) const {
    return n;
}
\end{lstlisting}

在渲染时,可调用方法\refvar[Get1DArray]{Get[12]DArray}{()}获取
指向之前请求的当前维度下样本数组起点的指针。
在\refvar{Get1DArray}{()}和\refvar{Get2DArray}{()}的代码行中
\sidenote{译者注:原文疑似笔误写成了\refvar{Get1D}{()}和\refvar{Get2D}{()},已修正。},
它们返回指向样本数组的指针,数组大小由参数{\ttfamily n}传给
初始化时对\refvar[Request1DArray]{Request[12]DArray}{()}
的相应调用。调用者也必须提供数组大小以“获取”方法用于验证返回的缓冲区具有期望大小。
\begin{lstlisting}
`\refcode{Sampler Interface}{+=}\lastnext{SamplerInterface}`  
const `\refvar{Float}{}` *`\initvar{Get1DArray}{}`(int n);
const `\refvar{Point2f}{}` *`\initvar{Get2DArray}{}`(int n);
\end{lstlisting}

为一个样本完成这些工作后,积分器调用\refvar{StartNextSample}{()}。
该调用为当前像素通知\refvar{Sampler}{}针对
样本分量的后续请求应该返回下一个样本从第一维起的值。
该方法返回{\ttfamily true},直到已经为每个像素生成了请求的原始数目样本
(此时调用者应该要么在另一个像素上开始工作要么停止试图使用更多样本)。
\begin{lstlisting}
`\refcode{Sampler Interface}{+=}\lastnext{SamplerInterface}`
virtual bool `\initvar{StartNextSample}{}`();
\end{lstlisting}

\refvar{Sampler}{}的实现存储了关于当前样本的各种状态:
正在采样哪个像素,用了该样本的多少维度等等。
因此对于多个线程同时使用的单个\refvar{Sampler}{}而言自然是不安全的。
方法\refvar{Clone}{()}生成一个初始\refvar{Sampler}{}的新实例给渲染线程使用;
它为采样器的随机数生成器(如果有)接收一个种子值,这样不同线程会有不同的随机数序列。
在多个图块间复用相同伪随机数序列可能导致微妙的图像伪影,例如重复的噪声模式。

方法\refvar{Clone}{()}的各种实现一般并不有趣,所以这里文中没有包含它们。
\begin{lstlisting}
`\refcode{Sampler Interface}{+=}\lastnext{SamplerInterface}`
virtual std::unique_ptr<`\refvar{Sampler}{}`> `\initvar{Clone}{}`(int seed) = 0;
\end{lstlisting}

一些光传输算法(特别是\refsec{随机渐进光子映射}的随机渐进光子映射)
在进行到下一像素前并不使用当前像素内的所有样本,
而是跳跃到周围的像素,每个里面每次取一个样本。
方法\refvar{SetSampleNumber}{()}允许积分器在当前像素内设置样本的索引以生成下一个。
一旦{\ttfamily sampleNum}大于或等于每个像素请求的原始样本数目该方法就返回{\ttfamily false}。
\begin{lstlisting}
`\refcode{Sampler Interface}{+=}\lastcode{SamplerInterface}`
virtual bool `\initvar{SetSampleNumber}{}`(int64_t sampleNum);
\end{lstlisting}

\subsection{采样器实现}\label{sub:采样器实现}
基类\refvar{Sampler}{}在其接口内提供了一些方法的实现。
首先,方法\refvar[Sampler::StartPixel]{StartPixel}{()}的实现记录当前正被采样的像素坐标
并置零\refvar{currentPixelSampleIndex}{}即像素中当前正被生成的样本数量。
注意这是有一个实现的虚方法;重载该方法的子类需要显式调用\refvar{Sampler::StartPixel}{()}。
\begin{lstlisting}
`\refcode{Sampler Method Definitions}{+=}\lastnext{SamplerMethodDefinitions}`
void `\initvar[Sampler::StartPixel]{\refvar{Sampler}{}::\refvar{StartPixel}{}}{}`(const `\refvar{Point2i}{}` &p) {
    `\refvar{currentPixel}{}` = p;
    `\refvar{currentPixelSampleIndex}{}` = 0;
    `\refcode{Reset array offsets for next pixel sample}{}`
}
\end{lstlisting}

\refvar{Sampler}{}子类可获取当前像素坐标和像素内的样本数量,
但它们应当将其作为只读值对待。
\begin{lstlisting}
`\initcode{Sampler Protected Data}{=}\initnext{SamplerProtectedData}`
`\refvar{Point2i}{}` `\initvar{currentPixel}{}`;
int64_t `\initvar{currentPixelSampleIndex}{}`;
\end{lstlisting}

当像素样本被更新或显式设置时,\refvar{currentPixelSampleIndex}{}也随之更新。
像\refvar[Sampler::StartPixel]{StartPixel}{()}那样,
方法\refvar[Sampler::StartNextSample]{StartNextSample}{()}和\refvar[Sampler::SetSampleNumber]{SetSampleNumber}{()}也
都是虚实现;这些实现也必须由\refvar{Sampler}{}子类中重载它们的实现来显式调用。
\begin{lstlisting}
`\refcode{Sampler Method Definitions}{+=}\lastnext{SamplerMethodDefinitions}`
bool `\initvar[Sampler::StartNextSample]{\refvar{Sampler}{}::\refvar{StartNextSample}{}}{}`() {
    `\refcode{Reset array offsets for next pixel sample}{}`
    return ++`\refvar{currentPixelSampleIndex}{}` < `\refvar{samplesPerPixel}{}`;
}
\end{lstlisting}
\begin{lstlisting}
`\refcode{Sampler Method Definitions}{+=}\lastnext{SamplerMethodDefinitions}`
bool `\initvar[Sampler::SetSampleNumber]{\refvar{Sampler}{}::\refvar{SetSampleNumber}{}}{}`(int64_t sampleNum) {
    `\refcode{Reset array offsets for next pixel sample}{}`
    `\refvar{currentPixelSampleIndex}{}` = sampleNum;
    return `\refvar{currentPixelSampleIndex}{}` < `\refvar{samplesPerPixel}{}`;
}
\end{lstlisting}

基类\refvar{Sampler}{}的实现也仔细记录
对样本分量数组的请求并为这些值分配存储空间。
所需的样本数组大小存于\refvar{samples1DArraySizes}{}和\refvar{samples2DArraySizes}{},
整个像素的样本数组值的内存分配于\refvar{sampleArray1D}{}和\refvar{sampleArray2D}{}。
每份分配中前{\ttfamily n}个值用于像素中首个样本的相应数组,以此类推。
\begin{lstlisting}
`\refcode{Sampler Method Definitions}{+=}\lastnext{SamplerMethodDefinitions}`
void `\initvar[Sampler::Request1DArray]{\refvar{Sampler}{}::\refvar{Request1DArray}{}}{}`(int n) {
    `\refvar{samples1DArraySizes}{}`.push_back(n);
    `\refvar{sampleArray1D}{}`.push_back(std::vector<`\refvar{Float}{}`>(n * `\refvar{samplesPerPixel}{}`));
}
\end{lstlisting}
\begin{lstlisting}
`\refcode{Sampler Method Definitions}{+=}\lastnext{SamplerMethodDefinitions}`
void `\initvar[Sampler::Request2DArray]{\refvar{Sampler}{}::\refvar{Request2DArray}{}}{}`(int n) {
    `\refvar{samples2DArraySizes}{}`.push_back(n);
    `\refvar{sampleArray2D}{}`.push_back(std::vector<`\refvar{Point2f}{}`>(n * `\refvar{samplesPerPixel}{}`));
}
\end{lstlisting}
\begin{lstlisting}
`\refcode{Sampler Protected Data}{+=}\lastcode{SamplerProtectedData}`
std::vector<int> `\initvar{samples1DArraySizes}{}`, `\initvar{samples2DArraySizes}{}`;
std::vector<std::vector<`\refvar{Float}{}`>> `\initvar{sampleArray1D}{}`;
std::vector<std::vector<`\refvar{Point2f}{}`>> `\initvar{sampleArray2D}{}`;
\end{lstlisting}

像方法\refvar[Get1DArray]{Get[12]DArray}{()}获取当前样本内的数组那样,\refvar{array1DOffset}{}和
\refvar{array2DOffset}{}被更新成将为样本向量返回的下一数组的索引。
\begin{lstlisting}
`\initcode{Sampler Private Data}{=}`
size_t `\initvar{array1DOffset}{}`, `\initvar{array2DOffset}{}`;
\end{lstlisting}
当处理新像素或当前像素中样本数量改变时,这些数组偏移量必须重置为0.
\begin{lstlisting}
`\initcode{Reset array offsets for next pixel sample}{=}`
`\refvar{array1DOffset}{}` = `\refvar{array2DOffset}{}` = 0;
\end{lstlisting}

要返回合适的数组指针,首先要基于当前样本向量内已经消耗了多少来选择合适的数组,
然后基于当前像素样本索引返回其合适的实例。
\begin{lstlisting}
`\refcode{Sampler Method Definitions}{+=}\lastnext{SamplerMethodDefinitions}`
const `\refvar{Float}{}` *`\initvar[Sampler::Get1DArray]{\refvar{Sampler}{}::\refvar{Get1DArray}{}}{}`(int n) {
    if (`\refvar{array1DOffset}{}` == `\refvar{sampleArray1D}{}`.size())
        return nullptr;
    return &`\refvar{sampleArray1D}{}`[`\refvar{array1DOffset}{}`++][`\refvar{currentPixelSampleIndex}{}` * n];
}
\end{lstlisting}
\begin{lstlisting}
`\refcode{Sampler Method Definitions}{+=}\lastnext{SamplerMethodDefinitions}`
const `\refvar{Point2f}{}` *`\initvar[Sampler::Get2DArray]{\refvar{Sampler}{}::\refvar{Get2DArray}{}}{}`(int n) {
    if (`\refvar{array2DOffset}{}` == `\refvar{sampleArray2D}{}`.size())
        return nullptr;
    return &`\refvar{sampleArray2D}{}`[`\refvar{array2DOffset}{}`++][`\refvar{currentPixelSampleIndex}{}` * n];
}
\end{lstlisting}

\subsection{像素采样器}\label{sub:像素采样器}
尽管一些采样算法很容易递进生成每个样本向量的元素,但其他算法会更自然地为一个像素同时生成
所有样本向量所有维度上的样本值。类\refvar{PixelSampler}{}
实现了一些对该类采样器的实现有用的功能。
\begin{lstlisting}
`\refcode{Sampler Declarations}{+=}\lastnext{SamplerDeclarations}`
class `\initvar{PixelSampler}{}` : public `\refvar{Sampler}{}` {
public:
    `\refcode{PixelSampler Public Methods}{}`
protected:
    `\refcode{PixelSampler Protected Data}{}`
};
\end{lstlisting}
\begin{lstlisting}
`\initcode{PixelSampler Public Methods}{=}`
`\refvar{PixelSampler}{}`(int64_t samplesPerPixel, int nSampledDimensions);
bool `\refvar[PixelSampler::StartNextSample]{StartNextSample}{}`();
bool `\refvar[PixelSampler::SetSampleNumber]{SetSampleNumber}{}`(int64_t);
`\refvar{Float}{}` `\refvar[PixelSampler::Get1D]{Get1D}{}`();
`\refvar{Point2f}{}` `\refvar[PixelSampler::Get2D]{Get2D}{}`();
\end{lstlisting}

渲染算法要用的样本向量维数是不能提前知道的
(确实,它只隐式取决于调用\refvar{Get1D}{()}和\refvar{Get2D}{()}的次数
以及请求的数组)。因此,\refvar{PixelSampler}{}构造函数
接收\refvar{Sampler}{}要计算的非数组样本值的最大维数。
如果所有这些分量维度都用掉了,则\refvar{PixelSampler}{}直接为额外维度返回均匀随机值。

对于每个预先计算的维度,构造函数都分配一个{\ttfamily vector}来存储样本值,
像素内的每个样本对应一个值。这些向量按{\ttfamily\refvar{samples1D}{}[dim][pixelSample]}来索引
\sidenote{译者注:原文将\refvar{samples1D}{}误写为{\ttfamily sample1D},已修正。};
尽管交换这些索引的顺序可能看起来更合理,但现在这样的内存排布——
对于给定维度,所有样本的所有样本分量值在内存中是连续的
\sidenote{译者注:指这些值的内存地址是连续的。}——
对于生成这些值的代码而言变得更方便了。
\begin{lstlisting}
`\refcode{Sampler Method Definitions}{+=}\lastnext{SamplerMethodDefinitions}`
`\refvar{PixelSampler}{}`::`\refvar{PixelSampler}{}`(int64_t samplesPerPixel,
        int nSampledDimensions)
    : `\refvar{Sampler}{}`(samplesPerPixel) {
    for (int i = 0; i < nSampledDimensions; ++i) {
        `\refvar{samples1D}{}`.push_back(std::vector<`\refvar{Float}{}`>(samplesPerPixel));
        `\refvar{samples2D}{}`.push_back(std::vector<`\refvar{Point2f}{}`>(samplesPerPixel));
    }
}
\end{lstlisting}

继承自\refvar{PixelSampler}{}的\refvar{Sampler}{}实现的
关键责任接着是在其方法\refvar{StartPixel}{()}
中填充数组\refvar{samples1D}{}和\refvar{samples2D}{}
(以及\refvar{sampleArray1D}{}和\refvar{sampleArray2D}{})。

\refvar{current1DDimension}{}和\refvar{current2DDimension}{}保存了
当前像素样本针对对应数组的偏移量。在开始处理每个新样本前必须将它们重置为0.
\begin{lstlisting}
`\initcode{PixelSampler Protected Data}{=}\initnext{PixelSamplerProtectedData}`
std::vector<std::vector<`\refvar{Float}{}`>> `\initvar{samples1D}{}`;
std::vector<std::vector<`\refvar{Point2f}{}`>> `\initvar{samples2D}{}`;
int `\initvar{current1DDimension}{}` = 0, `\initvar{current2DDimension}{}` = 0;
\end{lstlisting}
\begin{lstlisting}
`\refcode{Sampler Method Definitions}{+=}\lastnext{SamplerMethodDefinitions}`
bool `\initvar[PixelSampler::StartNextSample]{\refvar{PixelSampler}{}::\refvar{StartNextSample}{}}`() {
    `\refvar{current1DDimension}{}` = `\refvar{current2DDimension}{}` = 0;
    return `\refvar{Sampler}{}::\refvar[Sampler::StartNextSample]{StartNextSample}{}`();
}
\end{lstlisting}
\begin{lstlisting}
`\refcode{Sampler Method Definitions}{+=}\lastnext{SamplerMethodDefinitions}`
bool `\initvar[PixelSampler::SetSampleNumber]{\refvar{PixelSampler}{}::\refvar{SetSampleNumber}{}}{}`(int64_t sampleNum) {
    `\refvar{current1DDimension}{}` = `\refvar{current2DDimension}{}` = 0;
    return `\refvar{Sampler}{}::\refvar[Sampler::SetSampleNumber]{SetSampleNumber}{}`(sampleNum);
}
\end{lstlisting}

有了子类\refvar{PixelSampler}{}计算的数组中的样本值,
实现\refvar{Get1D}{()}只需依维度返回值直到算出的
所有维度都已被用掉,此时返回均匀随机值。
\begin{lstlisting}
`\refcode{Sampler Method Definitions}{+=}\lastnext{SamplerMethodDefinitions}`
`\refvar{Float}{}` `\initvar[PixelSampler::Get1D]{\refvar{PixelSampler}{}::\refvar{Get1D}{}}{}`() {
    if (`\refvar{current1DDimension}{}` < `\refvar{samples1D}{}`.size())
        return `\refvar{samples1D}{}`[`\refvar{current1DDimension}{}`++][`\refvar{currentPixelSampleIndex}{}`];
    else
        return `\refvar[PixelSampler::rng]{rng}{}`.`\refvar{UniformFloat}{}`();
}
\end{lstlisting}

{\initvar{PixelSampler::Get2D}{()}}同理,所以这里不再介绍。

\refvar{PixelSampler}{}用的随机数生成器是{\ttfamily protected}的
而不是{\ttfamily private}的。这对于其一些也需要随机数
来初始化\refvar{samples1D}{}和\refvar{samples2D}{}的子类会很方便。
\begin{lstlisting}
`\refcode{PixelSampler Protected Data}{+=}\lastcode{PixelSamplerProtectedData}`
`\refvar{RNG}{}` `\initvar[PixelSampler::rng]{rng}{}`;
\end{lstlisting}

\subsection{全局采样器}\label{sub:全局采样器}
其他生成样本的算法很少基于像素而是自然地生成分布于整幅图像的连续样本,
连续访问完全不同的像素(许多这样的采样器会高效地放置每个追加的样本
使其填充$n$维样本空间中的最大空洞,这自然导致后续样本在不同像素内)。
这些采样算法对于目前描述的\refvar{Sampler}{}接口有点问题:
例如考虑一个为前两维生成如\reftab{7.2}中间一列所示的一系列样本值的采样器。
这些样本值乘以图像每维分辨率得到图像平面中的样本位置
(这里我们为了简化考虑一幅$2\times3$的图像)。
注意对于这里的采样器(其实是\refvar{HaltonSampler}{}),
每六个样本就访问每个像素。若我们正渲染的图像每个像素用三个样本,
则为了给像素$(0,0)$生成所有的样本,我们需要生成索引为0、6和12的样本,以此类推。
\begin{table}[htb]
    \centering
    \begin{tabular}{lll}
        \toprule
        样本索引 & $[0,1)^2$的样本坐标   & 像素样本坐标          \\
        \midrule
        0        & $(0.000000,0.000000)$ & $(0.000000,0.000000)$ \\
        1        & $(0.500000,0.333333)$ & $(1.000000,1.000000)$ \\
        2        & $(0.250000,0.666667)$ & $(0.500000,2.000000)$ \\
        3        & $(0.750000,0.111111)$ & $(1.500000,0.333333)$ \\
        4        & $(0.125000,0.444444)$ & $(0.250000,1.333333)$ \\
        5        & $(0.625000,0.777778)$ & $(1.250000,2.333333)$ \\
        6        & $(0.375000,0.222222)$ & $(0.750000,0.666667)$ \\
        7        & $(0.875000,0.555556)$ & $(1.750000,1.666667)$ \\
        8        & $(0.062500,0.888889)$ & $(0.125000,2.666667)$ \\
        9        & $(0.562500,0.037037)$ & $(1.125000,0.111111)$ \\
        10       & $(0.312500,0.370370)$ & $(0.625000,1.111111)$ \\
        11       & $(0.812500,0.703704)$ & $(1.625000,2.111111)$ \\
        12       & $(0.187500,0.148148)$ & $(0.375000,0.444444)$ \\
        $\vdots$ &                       &                       \\
        \bottomrule
    \end{tabular}
    \caption{\refvar{HaltonSampler}{}生成中间一列坐标的前两维。
        因为它是个\refvar{GlobalSampler}{},所以它必须定义从像素坐标到样本索引的逆映射;
        这里,它通过将第一维坐标放大2倍、第二维坐标放大3倍
        以在$2\times3$像素的图像上放置样本,得到右边一列的像素样本坐标。}
    \label{tab:7.2}
\end{table}

若有了这样的采样器,我们就能定义\refvar{Sampler}{}接口使得
它为每个样本指定正在渲染的像素而不是相反(即告诉\refvar{Sampler}{}要渲染哪个像素)。

然而,采用目前的设计也有很好的理由:该方法更易把胶片分解为
小的图块以供多线程渲染,每个线程计算一个可高效并入最终图像的局部区域内的像素。
因此,我们必须要求这样的采样器能无序生成样本,使得每个像素的全部样本是连续生成的。

\refvar{GlobalSampler}{}帮助沟通\refvar{Sampler}{}接口的要求
与这类采样器的合理操作。它提供了\refvar{Sampler}{}所有纯虚方法的实现,
即代之以其子类必须实现的两个新的纯虚方法。
\begin{lstlisting}
`\refcode{Sampler Declarations}{+=}\lastcode{SamplerDeclarations}`
class `\initvar{GlobalSampler}{}` : public `\refvar{Sampler}{}` {
public:
    `\refcode{GlobalSampler Public Methods}{}`
private:
    `\refcode{GlobalSampler Private Data}{}`
};
\end{lstlisting}
\begin{lstlisting}
`\initcode{GlobalSampler Public Methods}{=}\initnext{GlobalSamplerPublicMethods}`
`\refvar{GlobalSampler}{}`(int64_t samplesPerPixel) : `\refvar{Sampler}{}`(samplesPerPixel) { }
\end{lstlisting}

有两个方法是实现必须提供的。第一个是\refvar{GetIndexForSample}{()},
它执行从当前像素和给定样本索引到样本向量全集中全局索引的逆映射。
例如,对于生成\reftab{7.2}中值的\refvar{Sampler}{},
如果\refvar{currentPixel}{}是$(0,2)$,则\refvar{GetIndexForSample}{(0)}会返回2,
因为样本索引2相应的像素样本坐标$(0.5,2)$对应着该像素区域中的首个样本
\sidenote{译者注:原文写的坐标值是$(0.25,0.666667)$,疑是笔误,已修改。}。
\begin{lstlisting}
`\refcode{GlobalSampler Public Methods}{+=}\lastnext{GlobalSamplerPublicMethods}`
virtual int64_t `\initvar{GetIndexForSample}{}`(int64_t sampleNum) const = 0;
\end{lstlisting}

紧密相关的\refvar{SampleDimension}{()}为
序列中第{\ttfamily index}个样本向量的给定维度返回样本值。
因为前两维用于偏移到当前像素,所以它们要做特殊处理:
该方法的实现返回的值应该是当前像素内的样本偏移量,
而不是原始的$[0,1)^2$样本值。例如\reftab{7.2}中,
\refvar{SampleDimension}{(4,1)}中会返回0.333333,
因为索引为4的样本的第二维相对于像素$(0,1)$偏移了这么多。
\begin{lstlisting}
`\refcode{GlobalSampler Public Methods}{+=}\lastcode{GlobalSamplerPublicMethods}`
virtual `\refvar{Float}{}` `\initvar{SampleDimension}{}`(int64_t index, int dimension) const = 0;
\end{lstlisting}

当开始为一个像素生成样本时,必须重置样本的维度并找到像素内首个样本的索引。
像所有采样器那样,接下来生成样本数组的所有值。
\begin{lstlisting}
`\refcode{Sampler Method Definitions}{+=}\lastnext{SamplerMethodDefinitions}`
void `\initvar[GlobalSampler::StartPixel]{\refvar{GlobalSampler}{}::\refvar{StartPixel}{}}{}`(const `\refvar{Point2i}{}` &p) {
    `\refvar{Sampler}{}`::`\refvar[Sampler::StartPixel]{StartPixel}{}`(p);
    `\refvar[GlobalSampler::dimension]{dimension}{}` = 0;
    `\refvar{intervalSampleIndex}{}` = `\refvar{GetIndexForSample}{}`(0);
    `\refcode{Compute arrayEndDim for dimensions used for array samples}{}`
    `\refcode{Compute 1D array samples for GlobalSampler}{}`
    `\refcode{Compute 2D array samples for GlobalSampler}{}`
}
\end{lstlisting}

成员变量\refvar[GlobalSampler::dimension]{dimension}{}跟踪
采样器实现将被要求生成的样本值的下一维;
当调用\refvar[GlobalSampler::Get1D]{Get1D}{()}和
\refvar[GlobalSampler::Get2D]{Get2D}{()}时它是递增的。
\refvar{intervalSampleIndex}{}记录当前像素内当前样本$s_i$对应的样本索引。
\begin{lstlisting}
`\initcode{GlobalSampler Private Data}{=}\initnext{GlobalSamplerPrivateData}`
int `\initvar[GlobalSampler::dimension]{dimension}{}`;
int64_t `\initvar{intervalSampleIndex}{}`;
\end{lstlisting}

必须决定为数组样本使用样本向量的哪些维度。
在靠前的维度比后面的维度质量更好的假设下,
为\refvar{CameraSample}{}留出前几个维度很重要,
因为这些样本值的质量经常对最终图像质量有很大影响。

因此,\refvar{arrayStartDim}{}前的维度用于常规的1D和2D样本,
而后续维度用于先1D再2D的数组样本。最后,起始于\refvar{arrayEndDim}{}的更高维
进一步用于非数组的1D和2D样本。当\refvar{GlobalSampler}{}构造函数运行时
不可能计算\refvar{arrayEndDim}{},因为目前还没有积分器请求数组样本。
因此,该值在方法\refvar[GlobalSampler::StartPixel]{StartPixel}{()}中
(重复且冗余地)计算。
\begin{lstlisting}
`\refcode{GlobalSampler Private Data}{+=}\lastcode{GlobalSamplerPrivateData}`
static const int `\initvar{arrayStartDim}{}` = 5;
int `\initvar{arrayEndDim}{}`;
\end{lstlisting}

所有像素样本的数组样本总数由像素样本数量与请求的样本数组尺寸的乘积给出。
\begin{lstlisting}
`\initcode{Compute arrayEndDim for dimensions used for array samples}{=}`
`\refvar{arrayEndDim}{}` = `\refvar{arrayStartDim}{}` +
              `\refvar{sampleArray1D}{}`.size() + 2 * `\refvar{sampleArray2D}{}`.size();
\end{lstlisting}

实际生成数组样本只需计算当前样本维度内所需值的数量。
\begin{lstlisting}
`\initcode{Compute 1D array samples for GlobalSampler}{=}`
for (size_t i = 0; i < `\refvar{samples1DArraySizes}{}`.size(); ++i) {
    int nSamples = `\refvar{samples1DArraySizes}{}`[i] * `\refvar{samplesPerPixel}{}`;
    for (int j = 0; j < nSamples; ++j) {
        int64_t index = `\refvar{GetIndexForSample}{}`(j);
        `\refvar{sampleArray1D}{}`[i][j] =
            `\refvar{SampleDimension}{}`(index, `\refvar{arrayStartDim}{}` + i);
    }
}
\end{lstlisting}

2D样本数组的生成类似;这里不再介绍
代码片\refcode{Compute 2D array samples for GlobalSampler}{}
\sidenote{译者注:我补充回来了。}。
\begin{lstlisting}
`\initcode{Compute 2D array samples for GlobalSampler}{=}`
int dim = `\refvar{arrayStartDim}{}` + `\refvar{samples1DArraySizes}{}`.size();
for (size_t i = 0; i < `\refvar{samples2DArraySizes}{}`.size(); ++i) {
    int nSamples = `\refvar{samples2DArraySizes}{}`[i] * `\refvar{samplesPerPixel}{}`;
    for (int j = 0; j < nSamples; ++j) {
        int64_t idx = `\refvar{GetIndexForSample}{}`(j);
        `\refvar{sampleArray2D}{}`[i][j].x = `\refvar{SampleDimension}{}`(idx, dim);
        `\refvar{sampleArray2D}{}`[i][j].y = `\refvar{SampleDimension}{}`(idx, dim+1);
    }
    dim += 2;
}
`\refvar{Assert}{}`(dim == `\refvar{arrayEndDim}{}`);
\end{lstlisting}

当像素样本变化时,必须重置当前样本维度计数器并计算像素内下一样本的样本索引。
\begin{lstlisting}
`\refcode{Sampler Method Definitions}{+=}\lastnext{SamplerMethodDefinitions}`
bool `\initvar[GlobalSampler::StartNextSample]{\refvar{GlobalSampler}{}::\refvar{StartNextSample}{}}{}`() {
    `\refvar[GlobalSampler::dimension]{dimension}{}` = 0;
    `\refvar{intervalSampleIndex}{}` = `\refvar{GetIndexForSample}{}`(`\refvar{currentPixelSampleIndex}{}` + 1);
    return `\refvar{Sampler}{}`::`\refvar[Sampler::StartNextSample]{StartNextSample}{}`();
}
\end{lstlisting}
\begin{lstlisting}
`\refcode{Sampler Method Definitions}{+=}\lastnext{SamplerMethodDefinitions}`
bool `\initvar[GlobalSampler::SetSampleNumber]{\refvar{GlobalSampler}{}::\refvar{SetSampleNumber}{}}{}`(int64_t sampleNum) {
    `\refvar[GlobalSampler::dimension]{dimension}{}` = 0;
    `\refvar{intervalSampleIndex}{}` = `\refvar{GetIndexForSample}{}`(sampleNum);
    return `\refvar{Sampler}{}`::`\refvar[Sampler::SetSampleNumber]{SetSampleNumber}{}`(sampleNum);
}
\end{lstlisting}

有了该机制,获取常规1D样本值只需跳过分配给数组样本的维度
并把当前样本索引和维度传给实现的方法\refvar{SampleDimension}{()}。
\begin{lstlisting}
`\refcode{Sampler Method Definitions}{+=}\lastnext{SamplerMethodDefinitions}`
`\refvar{Float}{}` `\initvar[GlobalSampler::Get1D]{\refvar{GlobalSampler}{}::\refvar{Get1D}{}}{}`() {
    if (`\refvar[GlobalSampler::dimension]{dimension}{}` >= `\refvar{arrayStartDim}{}` && `\refvar[GlobalSampler::dimension]{dimension}{}` < `\refvar{arrayEndDim}{}`)
        `\refvar[GlobalSampler::dimension]{dimension}{}` = `\refvar{arrayEndDim}{}`;
    return `\refvar{SampleDimension}{}`(`\refvar{intervalSampleIndex}{}`, `\refvar[GlobalSampler::dimension]{dimension}{}`++);
}
\end{lstlisting}

2D样本样本同理。
\begin{lstlisting}
`\refcode{Sampler Method Definitions}{+=}\lastcode{SamplerMethodDefinitions}`
`\refvar{Point2f}{}` `\initvar[GlobalSampler::Get2D]{\refvar{GlobalSampler}{}::\refvar{Get2D}{}}{}`() {
    if (`\refvar[GlobalSampler::dimension]{dimension}{}` + 1 >= `\refvar{arrayStartDim}{}` && `\refvar[GlobalSampler::dimension]{dimension}{}` < `\refvar{arrayEndDim}{}`)
        `\refvar[GlobalSampler::dimension]{dimension}{}` = `\refvar{arrayEndDim}{}`;
    `\refvar{Point2f}{}` p(`\refvar{SampleDimension}{}`(`\refvar{intervalSampleIndex}{}`, `\refvar[GlobalSampler::dimension]{dimension}{}`),
              `\refvar{SampleDimension}{}`(`\refvar{intervalSampleIndex}{}`, `\refvar[GlobalSampler::dimension]{dimension}{}` + 1));
    `\refvar[GlobalSampler::dimension]{dimension}{}` += 2;
    return p;
}
\end{lstlisting}

\input{content/chap0703.tex}

\input{content/chap0704.tex}

\input{content/chap0705.tex}

\input{content/chap0706.tex}

\input{content/chap0707.tex}

\input{content/chap0708.tex}

\input{content/chap0709.tex}

\input{content/chap0710.tex}

\section{习题}\label{sec:习题07}
\begin{enumerate}
      \item \circletwo 可以根据\refvar{RadicalInverse}{()}中的实现代码,
            为基2实现一个专门版本的\refvar{ScrambledRadicalInverse}{()}。
            确定怎样将随机数字排列映射为单个数位运算并实现该方法。
            比较算出的值和当前实现生成的值以确保你的方法是对的
            并通过编写一个小巧的基准程序来度量你的方法有多快。
      \item \circletwo 当前,每个样本的第三到五维是被时间和镜头样本用掉的,
            即便并非所有场景都需要这些样本值。因为样本向量中的更低维常比
            后面的分布得更好,所以这会造成不必要的图像质量下降。
            修改pbrt使得相机能表明其样本需求然后在需要样本来
            初始化\refvar{CameraSample}{}时利用该信息。
            别忘了更新\refvar[arrayStartDim]{GlobalSampler::arrayStartDim}{}的值。
            用\refvar{DirectLightingIntegrator}{}
            渲染图像并和当前实现的结果比较。你看到有改进吗?
            用不同采样器时结果有何区别?你怎样解释你所见到的各采样器间的任何区别?
      \item \label{sub:7.11.3}\circletwo 把\citet{Kensler2013Pixar}介绍
            改进的多重扰动采样方法实现为pbrt中的新
            \refvar{Sampler}{}。比较它和用\refvar{StratifiedSampler}{}、
            \refvar{HaltonSampler}{}以及\refvar{SobolSampler}{}渲染时
            的图像质量和渲染时间。
      \item \circletwo\citet{10.1007/3-540-31186-6_14}\sidenote{译者注:
                  在Springer获取的引用条目标注为2006年,
                  属于2004年的会议;原文则均标为2004年。}
            和\citet{10.1007/978-3-540-74496-2_12}\sidenote{译者注:
                  在Springer获取的引用条目标注为2008年,属于2006年的会议;
                  原文则均标为2006年。}描述了图像合成中\keyindex{一阶点阵}{rank-1 lattices}{}的应用。
            一阶点阵是另一种高效生成高质量低偏差样本点序列的方式。
            阅读他们的论文并基于该方法实现一个\refvar{Sampler}{}。
            比较它和pbrt中其他采样器的结果。
      \item \circletwo 用pbrt当前的\refvar{FilmTile}{}实现时,
            若重新渲染一幅图像,由于线程在后续运行中以不同顺序完成图块,
            图像中的像素值可能有轻微变化。例如一个像素最终值取自三个不同
            图像采样块中的样本,$v_1+v_2+v_3$,其值可能有时算作$(v_1+v_2)+v_3$
            而有时为$v_1+(v_2+v_3)$.由于浮点舍入,这两个值可能不同。
            尽管这些区别通常不成问题,但当想用自动化测试脚本验证对系统
            作出的无伤大雅的更改不会在渲染图像中实际引发任何区别时,
            它们就会造成灾难。修改\refvar[MergeFilmTile]{Film::MergeFilmTile}{()}使其
            以一致的顺序合并图块,从而让最终像素值不再被该不一致性干扰
            (例如你的实现可能缓存\refvar{FilmTile}{}并只在一个图块的
            上方和左侧相邻图块都已被合并时才合并它)。确保你的实现不引入
            任何意义上的性能倒退。度量因\refvar{FilmTile}{}生命期更长
            而新增的内存使用量;它和总内存使用量有什么关系?
      \item \circletwo 如\refsec{胶片与成像管道}中所述,
            方法\refvar[AddSplat]{Film::AddSplat}{()}没用滤波函数
            而是代之高效地用矩形滤波器把样本溅射到其最靠近的单个像素上。
            为了应用任意滤波器,必须规范化滤波器使得它在定义域上的积分为一;
            pbrt目前并不要求\refvar{Filter}{}满足该约束。
            修改\refvar{Film}{}构造函数中\refvar[Film::filterTable]{filterTable}{}的计算,
            使得制表函数规范化(别忘了在计算规范化因子时,
            表格只保存函数四分之一的范围)。然后修改方法\refvar{AddSplat}{()}的
            实现以使用该滤波器。研究其导致的执行时间和图像质量的区别。
      \item \circleone 修改pbrt以创建为每条相机光线存于\refvar{Film}{}的值
            都正比于计算该光线辐亮度所花时长的图像(一个1像素宽的矩形滤波器
            可能是对该习题最有用的滤波器)。用该技术渲染各种场景。
            得到的图像对于系统性能带来了怎样的启发?当你查看它们时
            你可能需要缩放像素值或取其对数来看到有意义的变化。
      \item \circletwo 辐射度量学中线性假设的一个优点是场景的最终图像和
            分别考虑每个光源分布的图像之和是一样的(假设使用不会
            截断像素辐亮度值的浮点图像块格式)。该性质意味着如果渲染器为
            每个光源创建单独的图像,可以写个交互式灯光设计工具让
            快速查看缩放场景中单个光源作用的影响而无需重新渲染成为可能。
            可代之以缩放一个光源的单独图像然后再对所有光源图像求和
            重新生成最终图像(该技术首先应用于\citet{10.1145/122718.122723}的
            歌剧灯光设计)。修改pbrt来为场景中的每个光源输出单独的图像,
            并写一个按该方式利用它们的交互式灯光设计工具。
      \item \circlethree \citet{10.1145/54852.378514}注意到
            有一簇重建滤波器同时用了函数值和它在该点的导数来进行
            比只知道函数值好得多的重建。此外,他们报告他们已为朗伯和
            冯氏反射模型\sidenote{译者注:原文Lambertian and Phong reflection models。}的
            屏幕空间导数推导出解析解,然而他们没有在其论文中包含这些表达式。
            研究基于导数的重建,扩展pbrt以支持该技术。
            因为给一般形状和BSDF模型的屏幕空间导数推导表达式可能很难,
            研究基于有限差分的近似即可。\refsec{采样与抗锯齿}射线差分背后
            基于该思想的技术可能对该尝试有成效。
      \item \circlethree \keyindex{基于图像的渲染}{image-based rendering}{render渲染}是
            使用一个场景一幅或多幅图像合成不同于原始视角的新视角图像的一组技术的总称。
            其中一种方法是\keyindex{光场渲染}{light field rendering}{render渲染},
            即用一组来自密集间隔位置的图像\citep{10.1145/237170.237199,10.1145/237170.237200}。
            阅读这两篇关于光场的论文,并修改pbrt以直接生成场景的光场,
            而不需要渲染器运行多次,每次只针对一个相机位置。
            为此可能有必要编写专门的\refvar{Camera}{}、\refvar{Sampler}{}和\refvar{Film}{}。
            此外,编写一个交互式光场查看器来加载你的实现生成的光场并生成场景的新视角。
      \item \circlethree 比起只保存图像中的光谱值,常常更有用的是
            保存场景中在每个像素处可见的物体的额外信息。
            例如见\citet{10.1145/325334.325247}和\citet{10.1145/97879.97901}的SIGGRAPH论文。
            例如,如果保存每个像素处物体的3D位置、曲面法线以及BRDF,
            则移动光源后可高效地重新渲染场景\citep{10.1145/344779.344938}。
            或者,如果每个样本保存沿其相机光线可见的所有物体信息而不是只存第一个,
            则可以重新渲染移动视点后的新图像\citep{10.1145/280814.280882}。
            研究深度帧缓冲区\sidenote{译者注:原文deep frame buffer,不确定该词翻译。}的表示
            和利用它的算法;扩展pbrt以支持创建像这样的图像,并开发对它们进行操作的工具。
      \item \circletwo 为图像重建实现中值滤波器:对于每个像素,保存
            滤波器范围内其周围所有样本的中值。该任务很复杂,因为事实上
            当前\refvar{Film}{}实现中的滤波器必须是\keyindex{线性的}{linear}{}——
            滤波函数值只取决于样本相对于像素位置的位置,样本值对滤波函数值没有影响。
            因为实现假设滤波器是线性的,且因为它在把样本值的贡献加到图像中后就不再保存了,
            所以实现中值滤波器要求一般化\refvar{Film}{}或开发新的\refvar{Film}{}实现。
            用像\refvar{PathIntegrator}{}那样搭配常规图像滤波器会有讨厌的图像噪声的积分器渲染图像。
            中值滤波器在减少噪声上有多成功?用中值滤波器有视觉缺陷吗?
            你能实现该方法而无需在计算最终像素值前保存所有图像样本值吗?
      \item \circletwo 中值滤波器的一种替代是丢弃像素滤波器区域中
            具有最小贡献的样本和具有最大贡献的样本。该方法更多使用采样期间收集的信息。
            实现该方法并比较它和中值滤波器的结果。
      \item \circlethree 实现\citeauthor{keller1998quasi}及其合作者
            介绍的非连续缓冲区\citep{keller1998quasi,10.2312:EGWR:EGWR02:015-024}。
            你可能需要修改\refvar{Integrator}{}的接口使得它们可以独自返回
            直接和间接照明贡献然后独立将其传给\refvar{Film}{}。
            渲染图像以展示其在用间接照明渲染图像时的高效性。
      \item \circlethree 实现近年一种自适应采样和重建技术,
            例如\citet{10.1145/1360612.1360632}、\citet{10.1145/1531326.1531399}、
            \citet{10.1145/1618452.1618486}或者\citet{10.1145/2641762}介绍的。
            比起只用高采样率的均匀采样它们生成同等质量图像要高效多少?
            对于无需自适应采样的简单场景它们如何影响运行时间?
      \item \circlethree 调研色调重建算法的当前研究
            (例如见\citet{reinhard2010high}、\citet{10.1145/2366145.2366220}),
            并实现其中一个或多个算法。对pbrt渲染的大量场景使用你的实现,
            并讨论比起查看无色调重建的图像你所见到的改进。
\end{enumerate}

\section{译者补充:傅里叶变换}\label{sec:译者补充:傅里叶变换}
\begin{remark}
    本节内容不是原书内容,而是译者根据\citet{enwiki:1115652231,enwiki:1115414995,
        enwiki:1098200554,enwiki:1114206769}、\citet{DigitalSignalProcessing}补充的,
    请酌情参考和斧正。
\end{remark}
\begin{notation}
    本节所指的时域和原书前文中的空域是等价的概念,不影响本质。
\end{notation}
\subsection{单位冲激函数}\label{sub:单位冲激函数}
\begin{definition}
    数学中,\keyindex{狄拉克$\delta$分布}{Dirac delta distribution}{}是定义在实数域上的广义分布。
    它在除零以外的点上都取零,且在整个实数域上的积分等于一。通常记作$\delta(\cdot)$.
\end{definition}

狄拉克$\delta$分布也称\keyindex{狄拉克$\delta$函数}{Dirac delta function}{},
简称\keyindex{$\delta$分布}{delta distribution}{}或
\keyindex{$\delta$函数}{delta function}{},
它最早由英国理论物理学家保罗·狄拉克(Paul Adrien Maurice Dirac)提出,
在物理和工程界有广泛应用,也称作\keyindex{单位冲激函数}{unit impulse function}{}。

单位冲激函数不是严格意义上的函数,但形式上遵守微积分运算法则。
可以将其视作在非零处取零,在零处取无穷大,即
\begin{align}
    \delta(t)\approx\left\{
    \begin{array}{ll}
        +\infty, & \text{当}t=0,     \\
        0,       & \text{当}t\neq 0,
    \end{array}
    \right.
\end{align}
且满足如下积分约束的函数:
\begin{align}
    \int_{-\infty}^{\infty}\delta(t)\mathrm{d}t=1\, .
\end{align}

依据单位冲激函数的定义,可推导出以下性质:
\begin{theorem}\label{theorem:7.ex01.symmetry}
    单位冲激函数具有缩放性质:对任意实数$\alpha\neq0$,有
    \begin{align}
        \delta(\alpha t)=\frac{\delta(t)}{|\alpha|}\, .
    \end{align}
    特别地,单位冲激函数具有对称性,即
    \begin{align}
        \delta(t)=\delta(-t)\, .
    \end{align}
\end{theorem}
\begin{definition}
    称实数域上满足$\displaystyle\int_{-\infty}^{\infty}|f(x)|\mathrm{d}x<\infty$的
    函数$f$为\keyindex{可积函数}{integrable function}{}。
\end{definition}
\begin{theorem}\label{theorem:7.ex01.1}
    单位冲激函数具有时延性质,也称平移性质或筛选性质,
    即对于可积函数$f$,它可以采样出$t=\tau$处的值:
    \begin{align}
        \int_{-\infty}^{\infty}\delta(t-\tau)f(t)\mathrm{d}t=f(\tau)\, .
    \end{align}
\end{theorem}
\subsection{傅里叶变换的定义}\label{sub:傅里叶变换的定义}
\begin{definition}
    对于可积函数$f(t)$,其(一元)\keyindex{傅里叶变换}{Fourier transform}{}为
    \begin{align}
        F(\omega)=\mathcal{F}\{f(t)\}=\int_{-\infty}^{\infty}f(t)\mathrm{e}^{-\mathrm{i}2\pi\omega t}\mathrm{d}t\, ,
    \end{align}
    其中$\mathrm{i}$为虚数单位,$\mathrm{e}$为自然对数的底;
    称$F(\omega)$为$f(t)$的频域表示,也有文献记作$\mathcal{F}\{f\}(\omega)$,
    其中$\omega$表示\keyindex{频率}{frequency}{};
    也称$f(t)$和$F(\omega)$构成一个傅里叶变换对,记作$f(t)\leftrightarrow F(\omega)$;
    同时,相应的(一元)\keyindex{傅里叶逆变换}{inverse Fourier transform}{}为
    \begin{align}\label{eq:7.ex01.inverseFourier}
        f(t)=\mathcal{F}^{-1}\{F(\omega)\}=\int_{-\infty}^{\infty}F(\omega)\mathrm{e}^{\mathrm{i}2\pi\omega t}\mathrm{d}\omega\, .
    \end{align}
\end{definition}

\begin{theorem}\label{theorem:7.ex01.2}
    对于单位冲激函数$\delta(t)$,其频率表示为$F(\omega)=1$.
\end{theorem}
\begin{prove}
    由傅里叶变换定义,
    \begin{align}
        F(\omega)=\int_{-\infty}^{\infty}\delta(t)\mathrm{e}^{-\mathrm{i}2\pi\omega t}\mathrm{d}t
        =\mathrm{e}^{-\mathrm{i}2\pi\omega\cdot0}
        =1\, .
    \end{align}
\end{prove}

\begin{theorem}\label{theorem:7.ex01.3}
    对于单位常函数$f(t)=1$,其频率表示为$\delta(\omega)$.
\end{theorem}
\begin{prove}
    定义\keyindex{双边指数衰减函数}{two-sided decaying exponential function}{}为
    \sidenote{属于\keyindex{拉普拉斯分布}{Laplace distribution}{distribution分布}。}
    \begin{align}
        f_a(t)=\mathrm{e}^{-a|t|},\quad (a>0)\, ,
    \end{align}
    则单位常函数可视作该函数的极限,即
    \begin{align}
        f(t)=\lim\limits_{a\rightarrow0^+}f_a(t)=1\, .
    \end{align}
    于是常函数的频率表示满足
    \begin{align}
        F(\omega) & =\int_{-\infty}^{\infty}f(t)\mathrm{e}^{-\mathrm{i}2\pi\omega t}\mathrm{d}t
        =\int_{-\infty}^{\infty}\lim\limits_{a\rightarrow0^+}\mathrm{e}^{-a|t|}\mathrm{e}^{-\mathrm{i}2\pi\omega t}\mathrm{d}t
        =\lim\limits_{a\rightarrow0^+}\int_{-\infty}^{\infty}\mathrm{e}^{-a|t|-\mathrm{i}2\pi\omega t}\mathrm{d}t\nonumber                                                                                  \\
                  & =\lim\limits_{a\rightarrow0^+}\left(\int_{-\infty}^0\mathrm{e}^{(a-\mathrm{i}2\pi\omega)t}\mathrm{d}t+\int_0^{\infty}\mathrm{e}^{-(a+\mathrm{i}2\pi\omega)t}\mathrm{d}t\right)\nonumber \\
                  & =\lim\limits_{a\rightarrow0^+}\left(\frac{\mathrm{e}^{(a-\mathrm{i}2\pi\omega)t}}{a-\mathrm{i}2\pi\omega}\bigg|_{t=-\infty}^0
        +\frac{\mathrm{e}^{-(a+\mathrm{i}2\pi\omega)t}}{-(a+\mathrm{i}2\pi\omega)}\bigg|_{t=0}^{\infty}\right)\nonumber                                                                                     \\
                  & =\lim\limits_{a\rightarrow0^+}\left(\frac{1}{a-\mathrm{i}2\pi\omega}+\frac{1}{a+\mathrm{i}2\pi\omega}\right)=\lim\limits_{a\rightarrow0^+}\frac{2a}{a^2+4\pi^2\omega^2}\nonumber        \\
                  & =\left\{\begin{array}{ll}
            0,      & \text{若}\omega\neq0, \\
            \infty, & \text{若}\omega=0.
        \end{array}\right.
    \end{align}
    注意到上式取极限的部分
    \sidenote{属于\keyindex{柯西分布}{Cauchy distribution}{distribution分布}。}
    在实数域上积分与$a$无关且为
    \begin{align}
        \int_{-\infty}^{\infty}\frac{2a}{a^2+4\pi^2\omega^2}\mathrm{d}\omega
        =\frac{1}{\pi}\int_{-\infty}^{\infty}\frac{1}{1+\left(\frac{2\pi\omega}{a}\right)^2}\mathrm{d}\frac{2\pi\omega}{a}
        =\frac{1}{\pi}\arctan\frac{2\pi\omega}{a}\bigg|_{\omega=-\infty}^{\infty}=1\, .
    \end{align}
    因此它实际上就是单位冲激函数,即
    \begin{align}
        F(\omega)=\delta(\omega)\, .
    \end{align}
\end{prove}
\subsection{傅里叶变换的性质}\label{sub:傅里叶变换的性质}
\begin{theorem}
    傅里叶变换具有线性性质:对于傅里叶变换对$g(t)\leftrightarrow G(\omega)$
    与$h(t)\leftrightarrow H(\omega)$,给定任意实数$\alpha,\beta$,则
    \begin{align}
        \alpha g(t)+\beta h(t)\leftrightarrow \alpha G(\omega)+\beta H(\omega)\, .
    \end{align}
\end{theorem}

\begin{theorem}
    傅里叶变换具有缩放性质:对于傅里叶变换对$f(t)\leftrightarrow F(\omega)$,
    给定任意实数$\alpha\neq0$,则
    \begin{align}
        f(\alpha t)\leftrightarrow\frac{1}{|\alpha|} F\left(\frac{\omega}{\alpha}\right)\, .
    \end{align}
\end{theorem}
\begin{prove}
    依照傅里叶变换定义,
    \begin{align}\label{eq:7.ex01.scale}
        \mathcal{F}\{f(\alpha t)\}=\int_{-\infty}^{\infty}f(\alpha t)\mathrm{e}^{-\mathrm{i}2\pi\omega t}\mathrm{d}t
        =\frac{1}{\alpha}\int_{-\infty}^{\infty}f(\alpha t)\mathrm{e}^{-\mathrm{i}2\pi\frac{\omega}{\alpha}\alpha t}\mathrm{d}(\alpha t)\, .
    \end{align}
    当$\alpha>0$时,\refeq{7.ex01.scale}化为
    \begin{align}
        \mathcal{F}\{f(\alpha t)\}=\frac{1}{\alpha}\int_{-\infty}^{\infty}f(t)\mathrm{e}^{-\mathrm{i}2\pi\frac{\omega}{\alpha}t}\mathrm{d}t
        =\frac{1}{\alpha}F\left(\frac{\omega}{\alpha}\right)\, .
    \end{align}
    当$\alpha<0$时,\refeq{7.ex01.scale}化为
    \begin{align}
        \mathcal{F}\{f(\alpha t)\}=\frac{1}{\alpha}\int_{\infty}^{-\infty}f(t)\mathrm{e}^{-\mathrm{i}2\pi\frac{\omega}{\alpha}t}\mathrm{d}t
        =-\frac{1}{\alpha}F\left(\frac{\omega}{\alpha}\right)\, .
    \end{align}
    于是综合起来表示有
    \begin{align}
        \mathcal{F}\{f(\alpha t)\}=\frac{1}{|\alpha|}F\left(\frac{\omega}{\alpha}\right)\, .
    \end{align}
\end{prove}

\begin{theorem}\label{theorem:7.ex01.4}
    傅里叶变换具有频移与时移性质,即对于傅里叶变换对$f(t)\leftrightarrow F(\omega)$,
    给定任意常数$\omega_0$和$\tau$,则有相应的变换对
    \begin{align}
        f(t)\mathrm{e}^{\mathrm{i}2\pi\omega_0 t} & \leftrightarrow F(\omega-\omega_0)\, ,                              \\
        f(t-\tau)                                 & \leftrightarrow F(\omega)\mathrm{e}^{-\mathrm{i}2\pi\omega\tau}\, .
    \end{align}
\end{theorem}
\begin{prove}
    对于时域表示$f(t)\mathrm{e}^{\mathrm{i}2\pi\omega_0 t}$,其傅里叶变换为
    \begin{align}
        \int_{-\infty}^{\infty}f(t)\mathrm{e}^{\mathrm{i}2\pi\omega_0 t}\mathrm{e}^{-\mathrm{i}2\pi\omega t}\mathrm{d}t
        =\int_{-\infty}^{\infty}f(t)\mathrm{e}^{-\mathrm{i}2\pi(\omega-\omega_0) t}\mathrm{d}t=F(\omega-\omega_0)\, .
    \end{align}
    对于频率表示$F(\omega)\mathrm{e}^{-\mathrm{i}2\pi\omega\tau}$,其傅里叶逆变换为
    \begin{align}
        \int_{-\infty}^{\infty}F(\omega)\mathrm{e}^{-\mathrm{i}2\pi\omega\tau}\mathrm{e}^{\mathrm{i}2\pi\omega t}\mathrm{d}\omega
        =\int_{-\infty}^{\infty}F(\omega)\mathrm{e}^{\mathrm{i}2\pi\omega(t-\tau)}\mathrm{d}\omega
        =f(t-\tau)\, .
    \end{align}
\end{prove}

\begin{theorem}
    傅里叶变换和逆变换互为逆运算,即
    \begin{align}
        \mathcal{F}^{-1}\{\mathcal{F}\{f(t)\}\}      & =f(t)\, ,      \\
        \mathcal{F}\{\mathcal{F}^{-1}\{F(\omega)\}\} & =F(\omega)\, .
    \end{align}
\end{theorem}
\begin{prove}
    利用定理\ref{theorem:7.ex01.symmetry}、\ref{theorem:7.ex01.1}、\ref{theorem:7.ex01.2}以及\ref{theorem:7.ex01.4}可得
    \begin{align}
        \mathcal{F}^{-1}\{\mathcal{F}\{f(t)\}\}= & \int_{-\infty}^{\infty}\left(\int_{-\infty}^{\infty}f(\tau)\mathrm{e}^{-\mathrm{i}2\pi\omega\tau}\mathrm{d}\tau\right)\mathrm{e}^{\mathrm{i}2\pi\omega t}\mathrm{d}\omega\nonumber \\
        =                                        & \int_{-\infty}^{\infty}f(\tau)\left(\int_{-\infty}^{\infty}\mathrm{e}^{\mathrm{i}2\pi\omega(t-\tau)}\mathrm{d}\omega\right)\mathrm{d}\tau\nonumber                                 \\
        =                                        & \int_{-\infty}^{\infty}f(\tau)\delta(t-\tau)\mathrm{d}\tau\nonumber                                                                                                                \\
        =                                        & f(t)\, .
    \end{align}
    第二个式子同理可证。
\end{prove}

\begin{theorem}
    傅里叶变换具有微分性质:对于绝对连续可微函数$f$及其傅里叶变换$F(\omega)$,有
    \begin{align}
        \frac{\mathrm{d}f(t)}{\mathrm{d}t}\leftrightarrow\mathrm{i}2\pi\omega F(\omega)\, .
    \end{align}
\end{theorem}
\begin{prove}
    对\refeq{7.ex01.inverseFourier}两边求导即可得证明:
    \begin{align}
        \frac{\mathrm{d}f(t)}{\mathrm{d}t} & =\frac{\mathrm{d}}{\mathrm{d}t}\int_{-\infty}^{\infty}F(\omega)\mathrm{e}^{\mathrm{i}2\pi\omega t}\mathrm{d}\omega\nonumber              \\
                                           & =\int_{-\infty}^{\infty}\frac{\mathrm{d}}{\mathrm{d}t}\left(F(\omega)\mathrm{e}^{\mathrm{i}2\pi\omega t}\right)\mathrm{d}\omega\nonumber \\
                                           & =\int_{-\infty}^{\infty}(\mathrm{i}2\pi\omega F(\omega))\mathrm{e}^{\mathrm{i}2\pi\omega t}\mathrm{d}\omega\, .
    \end{align}
\end{prove}

\begin{theorem}
    当有傅里叶变换对$f(t)\leftrightarrow F(\omega)$,
    则$f(t)$的\keyindex{直流分量}{DC component}{}为
    \begin{align}
        \int_{-\infty}^{\infty}f(t)\mathrm{d}t=F(0)\, .
    \end{align}
\end{theorem}

\begin{definition}
    称满足$\displaystyle\int_{-\infty}^{\infty}|f(x)|^2\mathrm{d}x<\infty$的
    函数$f$为\keyindex{平方可积函数}{square-integrable function}{}。
\end{definition}
\begin{theorem}[\keyindex{普朗歇尔定理}{Plancherel theorem}{}]
    对于平方可积函数$f(t)$及其傅里叶变换$F(\omega)$,有等式
    \begin{align}
        \int_{-\infty}^{\infty}|f(t)|^2\mathrm{d}t=\int_{-\infty}^{\infty}|F(\omega)|^2\mathrm{d}\omega\, .
    \end{align}
\end{theorem}
\begin{prove}
    依照傅里叶变换定义,有\sidenote{$\overline{f(t)}$表示$f(t)$的共轭。}
    \begin{align}
        \int_{-\infty}^{\infty}|f(t)|^2\mathrm{d}t & =\int_{-\infty}^{\infty}f(t)\overline{f(t)}\mathrm{d}t\nonumber                                                                                                                                                                                \\
                                                   & =\int_{-\infty}^{\infty}\left(\int_{-\infty}^{\infty}F(\xi)\mathrm{e}^{\mathrm{i}2\pi\xi t}\mathrm{d}\xi\right)\left(\overline{\int_{-\infty}^{\infty}F(\omega)\mathrm{e}^{\mathrm{i}2\pi\omega t}\mathrm{d}\omega}\right)\mathrm{d}t\nonumber \\
                                                   & =\int_{-\infty}^{\infty}\int_{-\infty}^{\infty}\int_{-\infty}^{\infty}F(\xi)\overline{F(\omega)}\mathrm{e}^{\mathrm{i}2\pi(\xi-\omega) t}\mathrm{d}\xi\mathrm{d}\omega\mathrm{d}t\nonumber                                                     \\
                                                   & =\int_{-\infty}^{\infty}\int_{-\infty}^{\infty}F(\xi)\overline{F(\omega)}\left(\int_{-\infty}^{\infty}\mathrm{e}^{\mathrm{i}2\pi(\xi-\omega) t}\mathrm{d}t\right)\mathrm{d}\xi\mathrm{d}\omega\nonumber                                        \\
                                                   & =\int_{-\infty}^{\infty}\int_{-\infty}^{\infty}F(\xi)\overline{F(\omega)}\delta(\xi-\omega)\mathrm{d}\xi\mathrm{d}\omega\nonumber                                                                                                              \\
                                                   & =\int_{-\infty}^{\infty}\left(\int_{-\infty}^{\infty}F(\xi)\delta(\xi-\omega)\mathrm{d}\xi\right)\overline{F(\omega)}\mathrm{d}\omega\nonumber                                                                                                 \\
                                                   & =\int_{-\infty}^{\infty}F(\omega)\overline{F(\omega)}\mathrm{d}\omega=\int_{-\infty}^{\infty}|F(\omega)|^2\mathrm{d}\omega\, .
    \end{align}
\end{prove}

\begin{definition}
    对于可积函数$g(t)$与$h(t)$,称
    \begin{align}
        g(t)\otimes h(t)=\int_{-\infty}^{\infty}g(\tau)h(t-\tau)\mathrm{d}\tau
    \end{align}
    为$g(t)$与$h(t)$的\keyindex{卷积}{convolution}{},更多文献记作$g\ast h$.
\end{definition}
\begin{theorem}[\keyindex{卷积定理}{convolution theorem}{}]
    函数在时域上的卷积和在频域上的乘积等价;在时域上的乘积和在频域上的卷积等价。
    具体地,对于傅里叶变换对$g(t)\leftrightarrow G(\omega)$与$h(t)\leftrightarrow H(\omega)$,有
    \begin{align}
        \mathcal{F}\{g(t)\otimes h(t)\} & =G(\omega)H(\omega)\, ,         \\
        \mathcal{F}\{g(t)h(t)\}         & =G(\omega)\otimes H(\omega)\, .
    \end{align}
\end{theorem}
\begin{prove}
    由傅里叶变换定义,
    \begin{align}
        \mathcal{F}\{g(t)\otimes h(t)\} & =\int_{-\infty}^{\infty}(g(t)\otimes h(t))\mathrm{e}^{-\mathrm{i}2\pi\omega t}\mathrm{d}t\nonumber                                                                                              \\
                                        & =\int_{-\infty}^{\infty}\left(\int_{-\infty}^{\infty}g(\tau)h(t-\tau)\mathrm{d}\tau\right)\mathrm{e}^{-\mathrm{i}2\pi\omega t}\mathrm{d}t\nonumber                                              \\
                                        & =\int_{-\infty}^{\infty}g(\tau)\left(\int_{-\infty}^{\infty}h(t-\tau)\mathrm{e}^{-\mathrm{i}2\pi\omega t}\mathrm{d}t\right)\mathrm{d}\tau\nonumber                                              \\
                                        & =\int_{-\infty}^{\infty}g(\tau)\mathrm{e}^{-\mathrm{i}2\pi\omega\tau}\left(\int_{-\infty}^{\infty}h(t-\tau)\mathrm{e}^{-\mathrm{i}2\pi\omega (t-\tau)}\mathrm{d}t\right)\mathrm{d}\tau\nonumber \\
                                        & =\int_{-\infty}^{\infty}g(\tau)\mathrm{e}^{-\mathrm{i}2\pi\omega\tau}H(\omega)\mathrm{d}\tau\nonumber                                                                                           \\
                                        & =H(\omega)\int_{-\infty}^{\infty}g(\tau)\mathrm{e}^{-\mathrm{i}2\pi\omega\tau}\mathrm{d}\tau\nonumber                                                                                           \\
                                        & =G(\omega)H(\omega)\, .
    \end{align}
    \begin{align}
        \mathcal{F}\{g(t)h(t)\} & =\int_{-\infty}^{\infty}g(t)h(t)\mathrm{e}^{-\mathrm{i}2\pi\omega t}\mathrm{d}t\nonumber                                                                                    \\
                                & =\int_{-\infty}^{\infty}\left(\int_{-\infty}^{\infty}G(\xi)\mathrm{e}^{\mathrm{i}2\pi\xi t}\mathrm{d}\xi\right)h(t)\mathrm{e}^{-\mathrm{i}2\pi\omega t}\mathrm{d}t\nonumber \\
                                & =\int_{-\infty}^{\infty}G(\xi)\left(\int_{-\infty}^{\infty}h(t)\mathrm{e}^{-\mathrm{i}2\pi(\omega-\xi)t}\mathrm{d}t\right)\mathrm{d}\xi\nonumber                            \\
                                & =\int_{-\infty}^{\infty}G(\xi)H(\omega-\xi)\mathrm{d}\xi\nonumber                                                                                                           \\
                                & =G(\omega)\otimes H(\omega)\, .
    \end{align}
\end{prove}

\subsection{常见傅里叶变换对}\label{sub:常见傅里叶变换对}
\begin{theorem}
    对于\keyindex{矩形函数}{rectangular function}{}
    \begin{align}
        f(t)=\left\{\begin{array}{ll}
            1,                        & \displaystyle\text{若}|t|<\frac{1}{2}, \\
            \displaystyle\frac{1}{2}, & \displaystyle\text{若}|t|=\frac{1}{2}, \\
            0,                        & \displaystyle\text{若}|t|>\frac{1}{2},
        \end{array}\right.
    \end{align}
    其频率表示为
    \begin{align}
        F(\omega)=\frac{\sin(\pi\omega)}{\pi\omega}\, .
    \end{align}
\end{theorem}
\begin{prove}
    由傅里叶变换定义,
    \begin{align}
        F(\omega) & =\int_{-\infty}^{\infty}f(t)\mathrm{e}^{-\mathrm{i}2\pi\omega t}\mathrm{d}t
        =\int_{-\frac{1}{2}}^{\frac{1}{2}}\mathrm{e}^{-\mathrm{i}2\pi\omega t}\mathrm{d}t
        =-\frac{\mathrm{e}^{-\mathrm{i}2\pi\omega t}}{\mathrm{i}2\pi\omega}\bigg|_{t=-\frac{1}{2}}^{\frac{1}{2}}\nonumber \\
                  & =-\frac{\mathrm{e}^{-\mathrm{i}\pi\omega}-\mathrm{e}^{\mathrm{i}\pi\omega}}{\mathrm{i}2\pi\omega}
        =\frac{\mathrm{i}2\sin(\pi\omega)}{\mathrm{i}2\pi\omega}
        =\frac{\sin(\pi\omega)}{\pi\omega}\, .
    \end{align}
\end{prove}

\begin{theorem}
    对于\keyindex{高斯函数}{Gaussian function}{}$f(t)=\mathrm{e}^{-\pi t^2}$,
    其频率表示为$F(\omega)=\mathrm{e}^{-\pi\omega^2}$.
\end{theorem}
\begin{prove}
    由傅里叶变换定义,
    \begin{align}
        F(\omega) & =\int_{-\infty}^{\infty}f(t)\mathrm{e}^{-\mathrm{i}2\pi\omega t}\mathrm{d}t
        =\int_{-\infty}^{\infty}\mathrm{e}^{-\pi t^2}\mathrm{e}^{-\mathrm{i}2\pi\omega t}\mathrm{d}t
        =\int_{-\infty}^{\infty}\mathrm{e}^{-\pi((t+\mathrm{i}\omega)^2+\omega^2)}\mathrm{d}t\nonumber                  \\
                  & =\mathrm{e}^{-\pi\omega^2}\int_{-\infty}^{\infty}\mathrm{e}^{-\pi(t+\mathrm{i}\omega)^2}\mathrm{d}t
        =\mathrm{e}^{-\pi\omega^2}\int_{-\infty}^{\infty}\mathrm{e}^{-\pi t^2}\mathrm{d}t
        =\mathrm{e}^{-\pi\omega^2}\, .
    \end{align}
\end{prove}

\subsubsection*{余弦函数}
对于余弦函数
\begin{align}
    f(t)=\cos t\, ,
\end{align}
其频率表示为
\begin{align}
    F(\omega) & =\int_{-\infty}^{\infty}f(t)\mathrm{e}^{-\mathrm{i}2\pi\omega t}\mathrm{d}t\nonumber                                                                                                \\
              & =\int_{-\infty}^{\infty}(\cos t)\mathrm{e}^{-\mathrm{i}2\pi\omega t}\mathrm{d}t\nonumber                                                                                            \\
              & =\int_{-\infty}^{\infty}\frac{1}{2}(\mathrm{e}^{\mathrm{i}t}+\mathrm{e}^{-\mathrm{i}t})\mathrm{e}^{-\mathrm{i}2\pi\omega t}\mathrm{d}t\nonumber                                     \\
              & =\frac{1}{2}\int_{-\infty}^{\infty}(\mathrm{e}^{\mathrm{i}2\pi\frac{1}{2\pi}t}+\mathrm{e}^{\mathrm{i}2\pi\frac{-1}{2\pi}t})\mathrm{e}^{-\mathrm{i}2\pi\omega t}\mathrm{d}t\nonumber \\
              & =\frac{1}{2}\left(\delta\left(\omega-\frac{1}{2\pi}\right)+\delta\left(\omega+\frac{1}{2\pi}\right)\right)\, .
\end{align}

\subsubsection*{shah函数}
\begin{theorem}
    周期为$T$的函数$f(t)$可被展开为唯一的\keyindex{傅里叶级数}{Fourier series}{},其指数形式为
    \begin{align}
        f(t)=\sum\limits_{n=-\infty}^{\infty}c_n\mathrm{e}^{\mathrm{i}2\pi\frac{n}{T}t}\, ,
    \end{align}
    其中系数
    \begin{align}
        c_n=\frac{1}{T}\int\limits_T f(t)\mathrm{e}^{-\mathrm{i}2\pi\frac{n}{T}t}\mathrm{d}t\, .
    \end{align}
\end{theorem}

对于周期为$T$的shah函数
\begin{align}
    f(t)=\sum\limits_{k=-\infty}^{\infty}\delta(t-kT)\, ,
\end{align}
其傅里叶展开中的系数为
\begin{align}
    c_n=\frac{1}{T}\int_{-\frac{T}{2}}^{\frac{T}{2}}f(t)\mathrm{e}^{-\mathrm{i}2\pi\frac{n}{T}t}\mathrm{d}t
    =\frac{1}{T}\int_{-\frac{T}{2}}^{\frac{T}{2}}\delta(t)\mathrm{e}^{-\mathrm{i}2\pi\frac{n}{T}t}\mathrm{d}t
    =\frac{1}{T}\, .
\end{align}
于是shah函数可展开为
\begin{align}
    f(t)=\frac{1}{T}\sum\limits_{n=-\infty}^{\infty}\mathrm{e}^{\mathrm{i}2\pi\frac{n}{T}t}\, .
\end{align}
因此其频域表示为
\begin{align}
    F(\omega) & =\int_{-\infty}^{\infty}f(t)\mathrm{e}^{-\mathrm{i}2\pi\omega t}\mathrm{d}t\nonumber                                                                                            \\
              & =\int_{-\infty}^{\infty}\left(\frac{1}{T}\sum\limits_{n=-\infty}^{\infty}\mathrm{e}^{\mathrm{i}2\pi\frac{n}{T}t}\right)\mathrm{e}^{-\mathrm{i}2\pi\omega t}\mathrm{d}t\nonumber \\
              & =\frac{1}{T}\sum\limits_{n=-\infty}^{\infty}\int_{-\infty}^{\infty}\mathrm{e}^{-\mathrm{i}2\pi(\omega-\frac{n}{T})t}\mathrm{d}t\nonumber                                        \\
              & =\frac{1}{T}\sum\limits_{n=-\infty}^{\infty}\delta\left(\omega-\frac{n}{T}\right)\, .
\end{align}

\section{译者补充:初等数论基础}\label{sec:译者补充:初等数论基础}

\begin{remark}
    本节内容不是原书内容,而是译者根据\citet{ElementaryNumberTheory}
    以及\citet{wiki:ExtendedEuclideanAlgorithm}补充的,请酌情参考和斧正。
\end{remark}

\begin{notation}
    本节我们重申以下记号:
    \begin{itemize}
        \item 用$\mathbb{N}$表示全体正整数构成的集合;$\mathbb{Z}$表示全体整数构成的集合。
        \item 若命题$p$能推出命题$q$,则记为$p\Rightarrow q$;若$p$与$q$等价,则记为$p\Leftrightarrow q$.
    \end{itemize}
\end{notation}


\begin{theorem}[\protect\keyindex{最小自然数原理}{least number principle}{}]\label{theorem:7.ex02.1}
    设$T$是$\mathbb{N}$的一非空子集,则必有$t_0\in T$,
    使对任意的$t\in T$有$t_0\le t$,即$t_0$是$T$中最小的自然数。
\end{theorem}

% \begin{theorem}[最大自然数原理]
%     设$M$是$\mathbb{N}$的一非空子集,若$M$有上界(即存在$a\in \mathbb{N}$使
%     对任意的$m\in M$有$m\le a$),则必有$m_0\in M$,使对任意的$m\in M$有$m\le m_0$,
%     即$m_0$是$M$中最大的自然数。
% \end{theorem}

% \begin{theorem}[\protect\keyindex{归纳原理}{principle of induction}{}]
%     设$S\subseteq \mathbb{N}$,且满足
%     \begin{enumerate}
%         \item 有$1\in S$;
%         \item 对任意$n\in S$都有$n+1\in S$;
%     \end{enumerate}
%     则$S=\mathbb{N}$.
% \end{theorem}

% \begin{theorem}[\protect\keyindex{数学归纳法}{mathematical induction}{}]
%     设$P(n)$是关于自然数$n$的命题,若
%     \begin{enumerate}
%         \item 当$n=1$时,$P(1)$成立;
%         \item $P(n)$成立时必能推出$P(n+1)$成立;
%     \end{enumerate}
%     则$P(n)$对所有自然数$n$均成立。
% \end{theorem}

% \begin{theorem}[\protect 第二种数学归纳法]
%     设$P(n)$是关于自然数$n$的命题,若
%     \begin{enumerate}
%         \item 当$n=1$时,$P(1)$成立;
%         \item 设$n>1$,对所有自然数$m<n$都有$P(m)$成立时必能推出$P(n)$成立;
%     \end{enumerate}
%     则$P(n)$对所有自然数$n$均成立。
% \end{theorem}

\begin{theorem}[\protect\keyindex{鸽巢原理}{pigeonhole principle}{}]\label{theorem:7.ex02.2}
    对于某$n\in\mathbb{N}$,现有$n$个笼子和$n+1$只鸽子,
    所有的鸽子都被关在鸽笼里,那么至少有一个笼子有至少2只鸽子。
    也称\keyindex{狄利克雷抽屉原理}{Dirichlet's drawer principle}{}。
\end{theorem}

\subsection{整除与带余除法}\label{sub:整除与带余除法}
\begin{definition}
    设$a,b\in\mathbb{Z}$且$a\neq0$,若存在$q\in\mathbb{Z}$使得$b=aq$,
    则称$a$\keyindex{整除}{divide evenly}{}$b$,或说$b$能被$a$整除,记作$a|b$,
    并称$a$是$b$的\keyindex{因数}{divisor}{},也称{\sffamily 约数}、{\sffamily 除数},
    $b$是$a$的\keyindex{倍数}{multiple}{}。$a$不能整除$b$时记作$a\nmid b$.
\end{definition}

\begin{example}
    6能整除18,记作$6|18$,6是18的因数,18是6的倍数。
\end{example}

\begin{theorem}\label{theorem:7.ex02.3}
    整除具有以下性质:
    \begin{enumerate}
        \item $a|b\Leftrightarrow -a|b \Leftrightarrow a|-b \Leftrightarrow |a|||b|$;
        \item $a|b$且$b|c \Rightarrow a|c$;
        \item $a|b$且$a|c \Leftrightarrow$对任意的$x,y\in\mathbb{Z}$有$a|bx+cy$;
        \item 设$m\neq0$,则$a|b\Leftrightarrow ma|mb$;
        \item $a|b$且$b|a\Rightarrow b=\pm a$;
        \item 设$b\neq0$,则$a|b\Rightarrow |a|\le|b|$.
    \end{enumerate}
\end{theorem}
% \begin{corollary}
%     非零整数的因数只有有限个。
% \end{corollary}
% \begin{theorem}
%     设整数$b\neq0$,而$d_1,d_2,\ldots,d_k$是$b$的全体因数,
%     则$\displaystyle\frac{b}{d_1},\frac{b}{d_2},\ldots,\frac{b}{d_k}$也是
%     $b$的全体因数。此外,若$b>0$,则当$d$遍历$b$的全体正因数时,
%     $\displaystyle\frac{b}{d}$也遍历$b$的全体正因数。
% \end{theorem}
\begin{definition}
    设整数$p\neq0,\pm1$,若$p$除了$\pm1,\pm p$外没有其他因数,
    则称$p$为\keyindex{质数}{prime number}{},也称{\sffamily 素数}、{\sffamily 不可约数}。
    若$a\neq0,\pm1$且$a$不是质数,则称$a$是\keyindex{合数}{composite number}{}。
\end{definition}
\begin{example}
    3、5、11是质数,4、6、12是合数。0和1既不是质数也不是合数。
\end{example}
\begin{notation}
    下文若无特别说明,所指的质数总是正的。
\end{notation}
% \begin{theorem}
%     \begin{enumerate}
%         \item $a>1$是合数$\Leftrightarrow$$a=de,1<d<a,1<e<a$;
%         \item 若$d>1$,$p$是质数且$d|p$,则$d=p$.
%     \end{enumerate}
% \end{theorem}
% \begin{theorem}
%     若$a$是合数,则必存在质数$p|a$.
% \end{theorem}
% \begin{definition}
%     若一个整数的因数是质数时,称该因数为\keyindex{质因数}{prime factor}{}。
% \end{definition}
% \begin{theorem}
%     设整数$a\ge2$,则$a$一定可表示为质数的乘积(包括$a$本身是质数),即
%     \begin{align}\label{eq:7.ex02.primefactor}
%         a=p_1p_2\cdots p_s\, ,
%     \end{align}
%     其中$p_j(1\le j\le s)$是质数。
% \end{theorem}
% \begin{example}
%     1260共有6个质因数(包括相同的),其中不相同的有4个,即
%     $1260=2\times2\times3\times3\times5\times7=2^2\times3^2\times5\times7$.
% \end{example}
% \begin{corollary}
%     设整数$a\ge2$,
%     \begin{enumerate}
%         \item 若$a$是合数,则必有质数$p|a$且$p\le\sqrt{a}$;
%         \item 若$a$有表示\refeq{7.ex02.primefactor},则必有质数$p|a$且$p\le a^{\frac{1}{s}}$.
%     \end{enumerate}
% \end{corollary}
% \begin{theorem}
%     质数有无穷多个。
% \end{theorem}
% \begin{theorem}
%     设全体质数按大小排序成
%     \begin{align}
%         p_1=2,\quad p2=3,\quad p_3=5,\ldots\, .
%     \end{align}
%     则有
%     \begin{align}
%         p_n\le2^{2^{n-1}},\quad n=1,2,\ldots\, ,
%     \end{align}
%     及
%     \begin{align}
%         \pi(x)>\log_2{\log_2{x}},\quad x\ge2\, ,
%     \end{align}
%     其中$\pi(x)$表示不超过$x$的质数个数。
% \end{theorem}

初等数论的证明中最重要、最基本、最直接的工具是下面的
\keyindex{带余除法}{division with remainder}{},
也称\keyindex{欧几里德除法}{Euclidean division}{}。
\begin{theorem}\label{theorem:7.ex02.4}
    % \label{theorem:7.ex02.EuclideanDivision}
    对于给定的$a,b\in\mathbb{Z}$且$a\neq0$,必存在唯一一对$q,r\in\mathbb{Z}$,满足
    \begin{align}\label{eq:7.ex02.EuclideanDivision}
        b=qa+r,\quad 0\le r<|a|\, .
    \end{align}
    此外,$a|b \Leftrightarrow r=0$.
\end{theorem}
\begin{prove}
    {\sffamily 唯一性}\quad 若还有整数$q'$与$r'$满足
    \begin{align}\label{eq:7.ex02.prove-theorem4-01}
        b=q'a+r',\quad 0\le r'<|a|\, ,
    \end{align}
    不妨设$r'\ge r$.则由\refeq{7.ex02.EuclideanDivision}和\refeq{7.ex02.prove-theorem4-01}得$0\le r'-r<|a|$,及
    \begin{align}
        r'-r=(q-q')a\, .
    \end{align}
    若$r'-r>0$,则由上式及定理\ref{theorem:7.ex02.3}(6)
    推出$|a|\le r'-r$.这和$r'-r<|a|$矛盾。所以必有$r'=r$,进而得$q'=q$.

        {\sffamily 存在性}\quad 当$a|b$时,可取$q=\displaystyle\frac{b}{a}$,$r=0$.
    当$a\nmid b$时。考虑集合
    \begin{align}
        T=\{b-ka:k=0,\pm1,\pm2,\ldots\}\, .
    \end{align}
    容易看出,集合$T$中必有正整数,所以由定理\ref{theorem:7.ex02.1}知,
    $T$中必有一个最小正整数,设为
    \begin{align}
        t_0=b-k_0a>0\, .
    \end{align}
    现在来证明必有$t_0<|a|$.因$a\nmid b$,所以$t_0\neq |a|$.
    若$t_0>|a|$,则$t_1=t_0-|a|>0$,显然$t_1\in T$,$t_1<t_0$.
    这和$t_0$的最小性矛盾。取$q=k_0$,$r=t_0$就满足要求。

    最后,显然当$b=qa+r$时,$a|b \Leftrightarrow a|r$.
    当满足$0\le r<|a|$时,由定理\ref{theorem:7.ex02.3}(6)就
    推出$a|r \Leftrightarrow r=0$.这就证明了定理的最后一部分。
\end{prove}

上述定理还有更灵活的形式。
\begin{theorem}\label{theorem:7.ex02.5}
    对于给定的$a,b,d\in\mathbb{Z}$且$a\neq0$,必存在唯一一对$q_1,r_1\in\mathbb{Z}$,满足
    \begin{align}\label{eq:7.7.ex02.remainder}
        b=q_1a+r_1,\quad d\le r_1<|a|+d\, .
    \end{align}
    此外,$a|b \Leftrightarrow a|r_1$.
\end{theorem}

只要对$a$和$b-d$用定理\ref{theorem:7.ex02.4}即可推出定理\ref{theorem:7.ex02.5}。
适当选取$d$可令\refeq{7.7.ex02.remainder}变形为下面的形式:
\begin{align}
    b & =q_1a+r_1, & -\frac{|a|}{2}< r_1\le\frac{|a|}{2}\, ,\label{eq:7.ex02.remainder02} \\
    b & =q_1a+r_1, & -\frac{|a|}{2}\le r_1<\frac{|a|}{2}\, ,\label{eq:7.ex02.remainder03} \\
    b & =q_1a+r_1, & 1\le r_1\le |a|\, .\label{eq:7.ex02.remainder04}
\end{align}
通常称\refeq{7.ex02.EuclideanDivision}中的$r$为$b$被$a$除后的\keyindex{最小非负余数}{least non-negative remainder}{remainder余数},
\refeq{7.ex02.remainder02}和\refeq{7.ex02.remainder03}中的$r_1$都称为\keyindex{绝对最小余数}{least absolute remainder}{remainder余数},
\refeq{7.ex02.remainder04}中的$r_1$称为\keyindex{最小正余数}{least positive remainder}{remainder余数},
\refeq{7.7.ex02.remainder}中的$r_1$统称为\keyindex{余数}{remainder}{}。

% \begin{corollary}
%     设$a>0$,任意整数被$a$除后所得的最小非负余数是且仅是$0,1,\ldots,a-1$这$a$个数中的一个。
% \end{corollary}
\begin{corollary}
    给定正整数$a\ge2$,则任一正整数$n$必可唯一表示为
    \begin{align}
        n=r_ka^k+r_{k-1}a^{k-1}+\cdots+r_1a+r_0\, ,
    \end{align}
    其中整数$k\ge0,0\le r_j\le a-1(0\le j\le k),r_k\neq0$.
    这即正整数的$a$进制表示。
\end{corollary}
\begin{prove}
    对正整数$n$必有唯一的$k\ge 0$,使得$a^k\le n<a^{k+1}$.
    由带余除法知,必有唯一的$q_0,r_0$满足
    \begin{align}
        n=q_0a+r_0,\quad 0\le r_0<a\, .
    \end{align}
    若$k=0$,则必有$q_0=0$,$1\le r_0<a$,所以结论成立。
    设结论对$k=m\ge0$成立,则当$k=m+1$时,上式中的$q_0$必满足
    \begin{align}
        a^m\le q_0<a^{m+1}\, .
    \end{align}
    由假设知
    \begin{align}
        q_0=s_ma^m+\cdots+s_0\, ,
    \end{align}
    其中$0\le s_j\le a-1(0\le j\le m-1)$,$1\le s_m\le a-1$.因而有
    \begin{align}
        n=s_ma^{m+1}+\cdots+s_0a+r_0\, ,
    \end{align}
    即结论对$m+1$也成立。由数学归纳法,推论得证。
\end{prove}

\subsection{最大公因数与最小公倍数}\label{sub:最大公因数与最小公倍数}
\begin{definition}
    设$a_1,a_2\in\mathbb{Z}$,若$d|a_1$且$d|a_2$,则称$d$是
    $a_1$与$a_2$的\keyindex{公因数}{common divisor}{divisor因数}。
    一般地,设$a_1,\ldots,a_k$是$k$个整数,若$d|a_1,\cdots,d|a_k$,
    则称$d$是$a_1,\ldots,a_k$的公因数。
\end{definition}
\begin{example}
    12和18的公因数是$\pm1,\pm2,\pm3,\pm6$.$n$和$n+1$的公因数是$\pm1$.
    当$a_1,\ldots,a_k$中有一个不为零时,它们的公因数个数有限。
\end{example}
\begin{definition}
    设$a_1,a_2\in\mathbb{Z}$不全为零,称$a_1$和$a_2$的公因数中
    最大的为$a_1$和$a_2$的\keyindex{最大公因数}{greatest common divisor}{divisor因数}(GCD),
    记作$(a_1,a_2)$.一般地,设$a_1,\ldots,a_k$是$k$个不全为零的整数,
    称$a_1,\ldots,a_k$的公因数中最大的为$a_1,\ldots,a_k$的最大公因数,
    记作$(a_1,\ldots,a_k)$.用$\mathcal{D}(a_1,\ldots,a_k)$表示$a_1,\ldots,a_k$的
    所有公因数组成的集合。于是
    \begin{align}
        (a_1,a_2)        & =\max\limits_{d\in\mathcal{D}(a_1,a_2)}{d}\, ,        \\
        (a_1,\ldots,a_k) & =\max\limits_{d\in\mathcal{D}(a_1,\ldots,a_k)}{d}\, .
    \end{align}
\end{definition}
\begin{example}
    $\mathcal{D}(12,16)=\{\pm1,\pm2,\pm3,\pm6\}$,$(12,18)=6$;
    $\mathcal{D}(6,10,-15)=\{\pm1\}$,$(6,10,-15)=1$;
    $(n,n+1)=1$.
\end{example}
\begin{theorem}\label{theorem:7.ex02.6}
    最大公因数满足以下性质:
    \begin{enumerate}
        \item $(a_1,a_2)=(a_2,a_1)=(-a_1,a_2)$;一般地,\\
              $(a_1,a_2,\ldots,a_i,\ldots,a_k)=(a_i,a_2,\ldots,a_1,\ldots,a_k)=(-a_1,a_2,\ldots,a_i,\ldots,a_k)$;
        \item $a_1|a_j(j=2,\ldots,k)\Rightarrow (a_1,a_2)=(a_1,a_2,\ldots,a_k)=|a_1|$;
        \item 对任意整数$x$,$(a_1,a_2)=(a_1,a_2,a_1x)$;$(a_1,\ldots,a_k)=(a_1,\ldots,a_k,a_1x)$;
        \item 对任意整数$x$,$(a_1,a_2)=(a_1,a_2+a_1x)$;\\
              $(a_1,a_2,a_3,\ldots,a_k)=(a_1,a_2+a_1x,a_3,\ldots,a_k)$;
        \item 若$p$是质数,则
              \begin{align}
                  (p,a_1)=\left\{
                  \begin{array}{ll}
                      p, & \text{若}p|a_1\, ,      \\
                      1, & \text{若}p\nmid a_1\, ;
                  \end{array}
                  \right.
              \end{align}
              一般地
              \begin{align}
                  (p,a_1,\ldots,a_k)=\left\{
                  \begin{array}{ll}
                      p, & \text{若}p|a_j,\quad j=1,2,\ldots,k, \\
                      1, & \text{其他。}
                  \end{array}
                  \right.
              \end{align}
    \end{enumerate}
\end{theorem}
\begin{definition}
    若$(a_1,a_2)=1$,则称$a_1$和$a_2$是\keyindex{互质}{coprime}{}
    (或relatively prime、mutually prime)的,也称{\sffamily 互素}、{\sffamily 既约}。
    一般地,若$(a_1,\ldots,a_k)=1$,则称$a_1,\ldots,a_k$是互质的。
\end{definition}
\begin{theorem}\label{theorem:7.ex02.7}
    若存在整数$x_1,\ldots,x_k$使得$a_1x_1+\cdots+a_kx_k=1$,则$a_1,\ldots,a_k$是互质的。
\end{theorem}
\begin{prove}
    因为$a_1,\ldots,a_k$的任意公因数$d$一定要整除1,所以必有$d=\pm1$.定理得证。
\end{prove}
\begin{theorem}\label{theorem:7.ex02.8}
    设正整数$m|(a_1,\ldots,a_k)$,则
    \begin{align}\label{eq:7.ex02.prove-theorem8-01}
        m\left(\frac{a_1}{m},\cdots,\frac{a_k}{m}\right)=(a_1,\ldots,a_k)\, .
    \end{align}
    特别地有
    \begin{align}\label{eq:7.ex02.prove-theorem8-02}
        \left(\frac{a_1}{(a_1,\cdots,a_k)},\ldots,\frac{a_k}{(a_1,\cdots,a_k)}\right)=1\, .
    \end{align}
\end{theorem}
\begin{prove}
    记$D=(a_1,\ldots,a_k)$.由$m|D$,$D|a_j(1\le j \le k)$知
    $m|a_j(1\le j \le k)$,故
    \begin{align}
        \frac{D}{m}\bigg|\frac{a_j}{m},\quad j=1,\ldots,k\, ,
    \end{align}
    即$\displaystyle\frac{D}{m}$是$\displaystyle\frac{a_1}{m},\ldots,\frac{a_k}{m}$的公因数
    且为正,所以由定义知
    \begin{align}\label{eq:7.ex02.prove-theorem8-03}
        \frac{D}{m}\le\left(\frac{a_1}{m},\ldots,\frac{a_k}{m}\right)\, .
    \end{align}
    另一方面,若$\displaystyle d\bigg|\frac{a_j}{m}(1\le j\le k)$,
    则$md|a_j(j=1,\ldots,k)$,由定义知
    \begin{align}
        md\le D,\quad \text{即}d\le\frac{D}{m}\, .
    \end{align}
    取$d=\displaystyle\left(\frac{a_1}{m},\ldots,\frac{a_k}{m}\right)$,
    由此及\refeq{7.ex02.prove-theorem8-03}即得\refeq{7.ex02.prove-theorem8-01}。
    在\refeq{7.ex02.prove-theorem8-01}中取$m=(a_1,\ldots,a_k)$即得\refeq{7.ex02.prove-theorem8-02}。
\end{prove}
\begin{definition}
    设$a_1,a_2\in\mathbb{Z}$均不为零,若$a_1|l$且$a_2|l$,
    则称$l$是$a_1$和$a_2$的\keyindex{公倍数}{common multiple}{multiple倍数}。
    一般地,设$a_1,\ldots,a_k$是$k$个均不为零的整数,
    若$a_1|l,\ldots,a_k|l$,则称$l$是$a_1,\ldots,a_k$的公倍数。
    此外,以$\mathcal{L}(a_1,\ldots,a_k)$表示$a_1,\ldots,a_k$的所有公倍数构成的集合。
\end{definition}
\begin{example}
    $\mathcal{L}(2,3)=\{0,\pm6,\pm12,\ldots,\pm6k,\ldots\}$.
\end{example}
\begin{definition}
    设$a_1,a_2\in\mathbb{Z}$均不为零,我们把$a_1$和$a_2$公倍数中的最小正数
    称为$a_1$和$a_2$的\keyindex{最小公倍数}{least common multiple}{multiple倍数},记作$[a_1,a_2]$,即
    \begin{align}
        [a_1,a_2]=\min\limits_{l\in\mathcal{L}(a_1,a_2),l>0}{l}\, .
    \end{align}
    一般地,设$a_1,\ldots,a_k\in\mathbb{Z}$均不为零,我们把
    $a_1,\ldots,a_k$公倍数中的最小正数称为$a_1,\ldots,a_k$的最小公倍数,
    记作$[a_1,\ldots,a_k]$,即
    \begin{align}
        [a_1,\ldots,a_k]=\min\limits_{l\in\mathcal{L}(a_1,\ldots,a_k),l>0}{l}\, .
    \end{align}
\end{definition}
\begin{example}
    $[2,3]=6$;$[2,3,4]=12$.
\end{example}
\begin{theorem}\label{theorem:7.ex02.9}
    最小公倍数满足以下性质:
    \begin{enumerate}
        \item $[a_1,a_2]=[a_2,a_1]=[-a_1,a_2]$;一般有\\
              $[a_1,a_2,\ldots,a_i,\ldots,a_k]=[a_i,a_2,\ldots,a_1,\ldots,a_k]=[-a_1,a_2,\ldots,a_i,\ldots,a_k]$;
        \item $a_2|a_1\Rightarrow [a_1,a_2]=|a_1|$;\\
              $a_j|a_1(2\le j\le k)\Rightarrow [a_1,\ldots,a_k]=|a_1|$;
        \item 对任意的$d|a_1$,有$[a_1,a_2]=[a_1,a_2,d]$;$[a_1,\ldots,a_k]=[a_1,\ldots,a_k,d]$.
    \end{enumerate}
\end{theorem}
\begin{theorem}\label{theorem:7.ex02.10}
    设$m>0$,则$[ma_1,\ldots,ma_k]=m[a_1,\ldots,a_k]$.
\end{theorem}
\begin{prove}
    设$L=[ma_1,\ldots,ma_k], L'=[a_1,\ldots,a_k]$.
    由$ma_j|L(1\le j\le k)$推出$\displaystyle a_j\bigg|\frac{L}{m}(1\le j\le k)$,
    进而由最小公倍数定义知$L'\le\displaystyle\frac{L}{m}$.
    另一方面,由$a_j|L'(1\le j\le k)$推出$ma_j|mL'(1\le j\le k)$,
    进而由最小公倍数定义得$L\le mL'$.由此定理得证。
\end{prove}
\begin{theorem}\label{theorem:7.ex02.11}
    $a_j|c(1\le j\le k)\Leftrightarrow [a_1,\ldots,a_k]|c$.
\end{theorem}
\begin{prove}
    $[a_1,\ldots,a_k]|c\Rightarrow a_j|c(1\le j\le k)$是显然的。
    下面证$a_j|c(1\le j\le k)\Rightarrow [a_1,\ldots,a_k]|c$.
    设$L=[a_1,\ldots,a_k]$.由定理\ref{theorem:7.ex02.4}知,有$q,r$使得
    \begin{align}
        c=qL+r,\quad 0\le r<L\, .
    \end{align}
    由此及$a_j|c$推出$a_j|r(1\le j\le k)$,所以$r$是公倍数。
    进而由最小公倍数的定义及$0\le r<L$可得$r=0$,即$L|c$.
    结论表明:公倍数一定是最小公倍数的倍数。
\end{prove}
\begin{theorem}\label{theorem:7.ex02.12}
    设$D$为正整数,则$D=(a_1,\ldots,a_k)$的充要条件是
    \begin{enumerate}
        \item $D|a_j(1\le j\le k)$;
        \item 若$d|a_j(1\le j\le k)$,则$d|D$.
    \end{enumerate}
\end{theorem}
\begin{prove}
    {\sffamily 充分性}\quad 由第一个条件知$D$是$a_j(1\le j\le k)$的公因数,
    由第二个条件、定理\ref{theorem:7.ex02.3}(6)及$D\ge1$知,
    $a_j(1\le j\le k)$的任一公因数$d$有$|d|\le D$.
    因而由定义知$D=(a_1,\ldots,a_k)$.

        {\sffamily 必要性}\quad 设$d_1,\ldots,d_s$是$a_1,\ldots,a_k$的
    全体公因数,$L=[d_1,\ldots,d_s]$.由定理\ref{theorem:7.ex02.11}
    知$L|a_j(1\le j\le k)$,因此$L$满足了两个条件。
    由上面充分性的证明知$L=(a_1,\ldots,a_k)=D$.必要性得证。
    结论表明:公因数一定是最大公因数的因数。
\end{prove}
\begin{theorem}\label{theorem:7.ex02.13}
    设$m>0$,则$m(b_1,\ldots,b_k)=(mb_1,\ldots,mb_k)$.
\end{theorem}
\begin{prove}
    在定理\ref{theorem:7.ex02.8}中取$a_j=mb_j(1\le j\le k)$,
    由定理\ref{theorem:7.ex02.12}可得$m|(a_1,\ldots,a_k)$.
    因此\refeq{7.ex02.prove-theorem8-01}成立,即本定理结论成立。
\end{prove}
\begin{theorem}\label{theorem:7.ex02.14}
    \begin{enumerate}
        \item $(a_1,a_2,a_3,\ldots,a_k)=((a_1,a_2),a_3,\ldots,a_k)$;
        \item $(a_1,\ldots,a_{k+r})=((a_1,\ldots,a_k),(a_{k+1},\ldots,a_{k+r}))$.
    \end{enumerate}
\end{theorem}
\begin{prove}
    对于第一个结论:若$d|a_j(1\le j\le k)$,则由定理\ref{theorem:7.ex02.12}知,
    $d|(a_1,a_2)$,$d|a_j(3\le j\le k)$;反过来,若$d|(a_1,a_2)$,$d|a_j(3\le j\le k)$,
    则由定义知,$d|a_j(1\le j\le k)$.这就证明了
    \begin{align}
        \mathcal{D}(a_1,a_2,a_3,\ldots,a_k)=\mathcal{D}((a_1,a_2),a_3,\ldots,a_k)\, .
    \end{align}
    故第一个结论成立。由它可立即推出第二个结论。
\end{prove}
\begin{theorem}\label{theorem:7.ex02.15}
    设$(m,a)=1$,则$(m,ab)=(m,b)$.
\end{theorem}
\begin{prove}
    $m=0$时$a=\pm1$,结论显然成立。当$m\neq0$时,
    由定理\ref{theorem:7.ex02.6}、定理\ref{theorem:7.ex02.13}和定理\ref{theorem:7.ex02.14}可得
    \begin{align}
        (m,b)=(m,b(m,a))=(m,(mb,ab))=(m,mb,ab)=(m,ab)\, .
    \end{align}
    得证。
\end{prove}
\begin{theorem}\label{theorem:7.ex02.16}
    设$(m,a)=1$,那么,若$m|ab$,则$m|b$.
\end{theorem}
\begin{prove}
    由定理\ref{theorem:7.ex02.6}和定理\ref{theorem:7.ex02.15}得
    $|m|=(m,ab)=(m,b)$,于是$m|b$.
\end{prove}
% \begin{theorem}
%     $[a_1,a_2](a_1,a_2)=|a_1a_2|$.
% \end{theorem}
\begin{theorem}\label{theorem:7.ex02.17}
    设$a_1,\ldots,a_k\in\mathbb{Z}$不全为零,则有
    \begin{enumerate}
        \item $(a_1,\ldots,a_k)=\min\{s=a_1x_1+\cdots+a_kx_k:x_j\in\mathbb{Z}(1\le j\le k),s>0\}$,即
              $a_1,\ldots,a_k$的最大公因数等于$a_1,\ldots,a_k$的所有整系数线性组合
              构成的集合$S$中的最小正整数。
        \item 一定存在一组整数$x'_1,\ldots,x'_k$使得
              \begin{align}\label{eq:7.ex02.theorem17-02}
                  (a_1,\ldots,a_k)=a_1x'_1+\cdots+a_kx'_k\, .
              \end{align}
    \end{enumerate}
\end{theorem}
\begin{prove}
    由于$0<a_1^2+\cdots+a_k^2\in S$,所以集合$S$中有正整数,
    由定理\ref{theorem:7.ex02.1}知$S$中必有最小正整数,记为$s_0$.
    显然对任一公因数$d|a_j(1\le j \le k)$必有$d|s_0$,所以$|d|\le s_0$.
    另一方面,对任一$a_j$由定理\ref{theorem:7.ex02.4}知存在$q_j,r_j$满足
    \begin{align}
        a_j=q_js_0+r_j,\quad 0\le r_j<s_0\, .
    \end{align}
    显然$r_j\in S$.若$r_j>0$,则和$s_0$的最小性矛盾,所以$r_j=0$,
    即$s_0|a_j(1\le j \le k)$.所以$s_0$是最大公因数。$s_0$当然是
    \refeq{7.ex02.theorem17-02}右边的形式。
\end{prove}

% \subsection*{算术基本定理}
% \begin{theorem}
%     设$p$是质数,$p|a_1a_2$,则$p|a_1$或$p|a_2$至少有一个成立。
%     一般地,若$p|a_1\cdots a_k$,则$p|a_1,\ldots,p|a_k$至少有一个成立。
% \end{theorem}
% \begin{theorem}[\protect\keyindex{算术基本定理}{fundamental theorem of arithmetic}{}]
%     设$a>1$,则必有
%     \begin{align}\label{eq:7.ex02.arithmeticfundamental}
%         a=p_1p_2\cdots p_s\, ,
%     \end{align}
%     其中$p_j(1\le j\le s)$是质数,且在不计次序的意义下,
%     表示\refeq{7.ex02.arithmeticfundamental}是唯一的。
% \end{theorem}

\subsection{辗转相除法}\label{sub:辗转相除法}
\keyindex{辗转相除法}{Euclidean algorithm}{},
也称{\sffamily 欧几里得算法},是指下面求取最大公因数的算法。
它最早出现于欧几里得的《几何原本》中,我国则可追溯至约东汉出现的《九章算术》。
\begin{theorem}\label{theorem:7.ex02.18}
    给定$u_0,u_1\in\mathbb{Z}$,且$u_1\neq0,u_1\nmid u_0$.
    我们一定可以反复应用定理\ref{theorem:7.ex02.4}得到下面$k+1$个等式:
    \begin{align}\label{eq:7.ex02.EuclideanAlgorithm}
        u_0     & =q_0u_1+u_2,             &  & 0<u_2<|u_1|,\nonumber    \\
        u_1     & =q_1u_2+u_3,             &  & 0<u_3<u_2,\nonumber      \\
        u_2     & =q_2u_3+u_4,             &  & 0<u_4<u_3,\nonumber      \\
        \cdots  & \cdots\cdots\cdots\cdots &  & \cdots\cdots\cdots\cdots \\
        u_{k-2} & =q_{k-2}u_{k-1}+u_k,     &  & 0<u_k<u_{k-1},\nonumber  \\
        u_{k-1} & =q_{k-1}u_k+u_{k+1},     &  & 0<u_{k+1}<u_k,\nonumber  \\
        u_k     & =q_ku_{k+1}.             &  & \nonumber
    \end{align}
\end{theorem}
\begin{prove}
    对$u_0,u_1$应用定理\ref{theorem:7.ex02.4},由$u_1\nmid u_0$知
    必有第一式成立。同样地,如果$u_2\nmid u_1$就得到第二式。
    如果$u_2\nmid u_1$就证明定理对$k=1$成立。以此类推,就得到
    \begin{align}
        |u_1|>u_2>u_3\cdots>u_{j+1}>0
    \end{align}
    以及前面$j$个等式成立。若$u_{j+1}|u_j$,则定理对$k=j$成立;
    若$u_{j+1}\nmid u_j$,则继续对$u_j,u_{j+1}$用定理\ref{theorem:7.ex02.4}。
    由于小于$|u_1|$的正整数只有有限个,而1整除任一整数,
    所以该过程不能无限进行下去,一定会出现某个$k$,要么$1<u_{k+1}|u_k$,
    要么$1=u_{k+1}|u_k$,证毕。
\end{prove}
\begin{theorem}\label{theorem:7.ex02.19}
    在定理\ref{theorem:7.ex02.18}的条件和符号下,我们有
    \begin{enumerate}
        \item $u_{k+1}=(u_0,u_1)$;
        \item $d|u_0$且$d|u_1$的充要条件是$d|u_{k+1}$;
        \item 存在整数$x_0,x_1$,使$u_{k+1}=x_0u_0+x_1u_1$.
    \end{enumerate}
\end{theorem}
\begin{prove}
    利用定理\ref{theorem:7.ex02.6}(1)、(4),从\refeq{7.ex02.EuclideanAlgorithm}的
    最后一式开始依次往上推,可得
    \begin{align}
        u_{k+1} & =(u_{k+1},u_k)=(u_k,u_{k-1})=(u_{k-1},u_{k-2})=\cdots\nonumber \\
                & =(u_4,u_3)=(u_3,u_2)=(u_2,u_1)=(u_1,u_0)\, ,
    \end{align}
    这就得到了第一个结论。利用定理\ref{theorem:7.ex02.3}(2)、(3),
    从\refeq{7.ex02.EuclideanAlgorithm}立即推出第二个结论。
    由\refeq{7.ex02.EuclideanAlgorithm}的第$k$式知$u_{k+1}$可
    表示为$u_{k-1}$和$u_k$的整系数线性组合,
    利用\refeq{7.ex02.EuclideanAlgorithm}的第$k-1$式
    可消去该表示中的$u_k$,将$u_{k+1}$表示为$u_{k-2}$和$u_{k-1}$的
    整系数线性组合。以此类推利用\refeq{7.ex02.EuclideanAlgorithm}的
    第$k-2,k-3,\ldots,2,1$式,就能相应地消去$u_{k-1},u_{k-2},\ldots,u_3,u_2$,
    最后将$u_{k+1}$表示为$u_0$和$u_1$的整系数线性组合,即证明了第三个结论。
\end{prove}
% \begin{example}
%     利用辗转相除法求198和252的最大公因数,并将其表示为198和252的整系数线性组合。因为
%     \begin{align*}
%         252 & =1\times198+54\, , \\
%         198 & =3\times54+36\, ,  \\
%         54  & =1\times36+18\, ,  \\
%         36  & =2\times18\, ,
%     \end{align*}
%     于是$(252,198)=(198,54)=(54,36)=(36,18)=18$,且得
%     \begin{align*}
%         18 & =54-1\times36                 \\
%            & =54-(198-3\times54)           \\
%            & =-198+4\times54               \\
%            & =-198+4\times(252-1\times198) \\
%            & =4\times252-5\times198\, .
%     \end{align*}
% \end{example}

\keyindex{扩展欧几里得算法}{extended Euclidean algorithm}{}是
辗转相除法的扩展。对于给定的不全为零整数$a,b$(不妨设$b\neq0$),
它求解关于整数变量$x,y$的方程
\begin{align}\label{eq:7.ex02.ExtendedEuclideanAlgorithm}
    ax+by=(a,b)\, .
\end{align}
由定理\ref{theorem:7.ex02.17}知该方程一定有解。
如果$a$为负数,则可以转化为
\begin{align}
    |a|(-x)+by=(|a|,b)\, ,
\end{align}
$b$为负数时同理。因此我们只考虑$a,b$不小于零的情况。
在定理\ref{theorem:7.ex02.18}的记号下,原始的
辗转相除法求解$(a,b)$的递推过程是
\begin{align}
    u_0     & =a\, ,\nonumber                  \\
    u_1     & =b\, ,\nonumber                  \\
    u_2     & =u_0-q_0u_1\, ,\nonumber         \\
    u_3     & =u_1-q_1u_2\, ,\nonumber         \\
    \ldots\nonumber                            \\
    u_{k+1} & =u_{k-1}-q_{k-1}u_k\, ,\nonumber \\
    u_{k+2} & =u_k-q_ku_{k+1}=0\, .
\end{align}
最后一步得到$u_{k+2}=0$时算法终止,此时$u_{k+1}|u_k$,即$(a,b)=u_{k+1}$.

扩展欧几里得算法则还利用了以上步骤中的商$q_i$以求解\refeq{7.ex02.ExtendedEuclideanAlgorithm}:
它新引入了两组序列$s_i,t_i$,并初始化$s_0=1,s_1=0,t_0=0,t_1=1$,
在辗转相除法每步计算$u_{i+1}=u_{i-1}-q_{i-1}u_i$后额外计算
\begin{align}
    s_{i+1} & =s_{i-1}-q_{i-1}s_i\, , \\
    t_{i+1} & =t_{i-1}-q_{i-1}t_i\, ,
\end{align}
则当$u_{k+2}=0$算法终止时,求得\refeq{7.ex02.ExtendedEuclideanAlgorithm}的解
即为$x=s_{k+1},y=t_{k+1}$.
\begin{prove}
    当$i=0,1$时,显然可以验证
    \begin{align}\label{eq:7.ex02.ExtendedEuclidean-prove-01}
        as_i+bt_i=u_i
    \end{align}
    成立。若\refeq{7.ex02.ExtendedEuclidean-prove-01}对某个$i$成立,则有
    \begin{align}
        u_{i+1} & =u_{i-1}-q_{i-1}u_i\nonumber                          \\
                & =(as_{i-1}+bt_{i-1})-q_{i-1}(as_i+bt_i)\nonumber      \\
                & =a(s_{i-1}-q_{i-1}s_i)+b(t_{i-1}-q_{i-1}t_i)\nonumber \\
                & =as_{i+1}+bt_{i+1}\, ,
    \end{align}
    即\refeq{7.ex02.ExtendedEuclidean-prove-01}对$i+1$也成立。
    由数学归纳法,\refeq{7.ex02.ExtendedEuclidean-prove-01}对
    算法步骤中的所有$i$都成立,故
    \begin{align}
        as_{k+1}+bt_{k+1}=u_{k+1}=(a,b)\, ,
    \end{align}
    即求得\refeq{7.ex02.ExtendedEuclideanAlgorithm}的解为$x=s_{k+1},y=t_{k+1}$.
\end{prove}
\begin{example}
    以$a=240,b=46$为例演示扩展欧几里得法,具体步骤是
    \begin{align}
        \begin{array}{lrrrr}
            i & q_{i-2}     & u_i              & s_i             & t_i                \\
            0 & -           & 240              & 1               & 0                  \\
            1 & -           & 46               & 0               & 1                  \\
            2 & 240\div46=5 & 240-5\times46=10 & 1-5\times0=1    & 0-5\times1=-5      \\
            3 & 46\div10=4  & 46-4\times10=6   & 0-4\times1=-4   & 1-4\times(-5)=21   \\
            4 & 10\div6=1   & 10-1\times6=4    & 1-1\times(-4)=5 & -5-1\times21=-26   \\
            5 & 6\div4=1    & 6-1\times4=2     & -4-1\times5=-9  & 21-1\times(-26)=47 \\
            6 & 4\div2=2    & 4-2\times2=0     & -               & -
        \end{array}\nonumber
    \end{align}
    算法在$i=6$时终止,由$i=5$时的$u_i,s_i,t_i$可得$-9\times240+47\times46=(240,46)=2$.
\end{example}

\subsection{同余}\label{sub:同余}
\begin{definition}
    设$a,b,m\in\mathbb{Z}$且$m\neq0$,若$m|a-b$,则称$a$与$b$\keyindex{模$m$同余}{congruent modulo $m$}{},
    也称$a$同余于$b$模$m$、$b$是$a$对模$m$的剩余,记作
    \begin{align}\label{eq:7.ex02.congruent}
        a\equiv b\pmod{m}\, ,
    \end{align}
    其中$m$称为\keyindex{模}{modulus}{},称\refeq{7.ex02.congruent}为模$m$的同余式;
    否则称$a$不同余于$b$模$m$、$b$不是$a$对模$m$的剩余,记作
    \begin{align}
        a\not\equiv b\pmod{m}\, .
    \end{align}
\end{definition}

因为$m|a-b\Leftrightarrow -m|a-b$,所以\refeq{7.ex02.congruent}等价于$a\equiv b\pmod{-m}$.
由此,下文均假定模$m\ge1$.\refeq{7.ex02.congruent}中,
若$0\le b<m$,则称$b$是$a$对模$m$的最小非负剩余;
若$1\le b\le m$,则称$b$是$a$对模$m$的最小正剩余;
若$\displaystyle -\frac{m}{2}<b\le\frac{m}{2}$(或$\displaystyle -\frac{m}{2}\le b<\frac{m}{2}$),
则称$b$是$a$对模$m$的绝对最小剩余。
\begin{example}
    $m|a$可记为$a\equiv 0\pmod{m}$;偶数可记为$a\equiv 0\pmod{2}$;
    奇数可记为$a\equiv 1\pmod{2}$.
\end{example}

% \begin{theorem}
%     $a$与$b$模$m$同余的充要条件是$a$和$b$被$m$除后的最小非负余数相等,即若
%     \begin{align}
%         a & =q_1m+r_1, & 0\le r_1<m\, , \\
%         b & =q_2m+r_2, & 0\le r_2<m\, ,
%     \end{align}
%     则$r_1=r_2$.
% \end{theorem}
% \begin{prove}
%     因为$a-b=(q_1-q_2)m+(r_1-r_2)$,所以$m|a-b$的充要条件是$m|r_1-r_2$,
%     由此及$0\le |r_1-r_2|<m$即得$r_1=r_2$.
% \end{prove}

% 容易证明,$a$对模$m$的最小非负剩余、最小正剩余、绝对最小剩余
% 正好分别是$a$被$m$除后的最小非负余数、最小正余数、绝对最小余数。

\begin{theorem}\label{theorem:7.ex02.20}
    同余是一种等价关系,即有
    \begin{enumerate}
        \item $a\equiv a\pmod{m}$;
        \item $a\equiv b\pmod{m} \Leftrightarrow b\equiv a\pmod{m}$;
        \item $a\equiv b\pmod{m}, b\equiv c\pmod{m} \Rightarrow a\equiv c\pmod{m}$.
    \end{enumerate}
\end{theorem}
\begin{prove}
    由$m|a-a=0$,$m|a-b\Leftrightarrow m|b-a$,以及
    $m|a-b,m|b-c\Rightarrow m|(a-b)+(b-c)=a-c$,就推出这三个性质。
\end{prove}
\begin{theorem}\label{theorem:7.ex02.21}
    同余式可以相加,即若
    \begin{align}\label{eq:7.ex02.addcongruent}
        a\equiv b\pmod{m},\qquad c\equiv d\pmod{m}\, ,
    \end{align}
    则
    \begin{align}
        a+c\equiv b+d\pmod{m}\, .
    \end{align}
\end{theorem}
\begin{prove}
    由$m|a-b,m|c-d\Rightarrow m|(a-b)+(c-d)=(a+c)-(b+d)$,就证明该结论。
\end{prove}
\begin{theorem}\label{theorem:7.ex02.22}
    同余式可以相乘,即若\refeq{7.ex02.addcongruent}成立,则有
    \begin{align}
        ac\equiv bd\pmod{m}\, .
    \end{align}
\end{theorem}
\begin{prove}
    由$a=b+k_1m$,$c=d+k_2m$推出$ac=bd+(bk_2+dk_1+k_1k_2m)m$,就证明该结论。
\end{prove}
% \begin{theorem}
%     设$f(x)=a_nx^n+\cdots+a_0$,$g(x)=b_nx^n+\cdots+b_0$是
%     两个整系数多项式,满足
%     \begin{align}\label{eq:7.ex02.polynomialcongruent}
%         a_j\equiv b_j\pmod{m},\quad 0\le j\le n\, .
%     \end{align}
%     那么若$a\equiv b\pmod{m}$,则
%     \begin{align}
%         f(a)\equiv g(b)\pmod{m}\, .
%     \end{align}
% \end{theorem}
% \begin{definition}
%     把满足\refeq{7.ex02.polynomialcongruent}的这两个多项式
%     称作多项式$f(x)$与$g(x)$模$m$同余,记作
%     \begin{align}
%         f(x)\Equiv g(x)\pmod{m}\, .
%     \end{align}
% \end{definition}

% \begin{theorem}
%     设$d\ge1$, $d|m$,则$a\equiv b\pmod{m} \Rightarrow a\equiv b\pmod{d}$.
% \end{theorem}
% \begin{theorem}
%     设$d\neq0$,则$a\equiv b\pmod{m} \Leftrightarrow da\equiv db\pmod{|d|m}$.
% \end{theorem}

注意在模不变的条件下,同余式两边不能相约。
\begin{example}
    $6\times3\equiv6\times8\pmod{10}$,但是$3\not\equiv8\pmod{10}$.
\end{example}

\begin{theorem}\label{theorem:7.ex02.23}
    同余式$\displaystyle ca\equiv cb\pmod{m}\Leftrightarrow a\equiv b\pmod{\frac{m}{(c,m)}}$.
    特别地,当$(c,m)=1$时可得$a\equiv b\pmod{m}$,即此时可两边约去$c$.
\end{theorem}
\begin{prove}
    $ca\equiv cb\pmod{m}$即$m|c(a-b)$,这等价于
    \begin{align}
        \frac{m}{(c,m)}\bigg|\frac{c}{(c,m)}(a-b)\, .
    \end{align}
    由定理\ref{theorem:7.ex02.16}以及$\displaystyle\left(\frac{m}{(c,m)},\frac{c}{(c,m)}\right)=1$知
    这等价于
    \begin{align}
        \frac{m}{(c,m)}\bigg|a-b\, .
    \end{align}
    就证明了该结论。
\end{prove}
\begin{theorem}\label{theorem:7.ex02.24}
    若$m\ge1$,$(a,m)=1$,则存在$c$使得
    \begin{align}\label{eq:7.ex02.modularinverse}
        ca\equiv1\pmod{m}\, .
    \end{align}
    我们把$c$称作$a$对模$m$的逆,或\keyindex{模逆元}{modular multiplicative inverse}{},
    记作$a^{-1}\pmod{m}$或$a^{-1}$.
\end{theorem}
\begin{prove}
    由定理\ref{theorem:7.ex02.17}知,存在$x_0,y_0$,
    使得$ax_0+my_0=1$,取$c=x_0$即满足要求。
\end{prove}
$a$对模$m$的逆不是唯一的。若$c$是$a$对模$m$的逆,
则任一$\bar{c}\equiv c\pmod{m}$也必是$a$对模$m$的逆;
$a$对模$m$的任意两个逆$c_1,c_2$必有$c_1\equiv c_2\pmod{m}$;
若$(a,m)=1$,则$(a^{-1},m)=1$,及$(a^{-1})^{-1}\equiv a\pmod{m}$.
\begin{notation}
    下文中约定$a^{-1}\pmod{m}$或$a^{-1}$指
    任一取定的满足\refeq{7.ex02.modularinverse}的$c$.
\end{notation}

\begin{example}
    $a$对模7的逆(只列出了一个值):
    \begin{table}[htbp]
        \centering
        \begin{tabular}{c|cccccc}
            \toprule
            $a$              & 1 & 2 & 3 & 4 & 5 & 6 \\
            \midrule
            $a^{-1}\pmod{7}$ & 1 & 4 & 5 & 2 & 3 & 6 \\
            \bottomrule
        \end{tabular}
        \caption{$a$对模7的逆(只列出了一个值)。}
        \label{tab:7.ex02.modularinverse}
    \end{table}
\end{example}

\begin{theorem}\label{theorem:7.ex02.25}
    同余式组
    \begin{align}
        a\equiv b\pmod{m_j}\, \quad j=1,2,\ldots,k
    \end{align}
    同时成立的充要条件是
    \begin{align}
        a\equiv b\pmod{[m_1,\ldots,m_k]}\, .
    \end{align}
\end{theorem}
\begin{prove}
    由定理\ref{theorem:7.ex02.11}知,$m_j|a-b(j=1,2,\ldots,k)$同时成立的
    充要条件是$[m_1,\ldots,m_k]|a-b$,得证。
\end{prove}
\begin{definition}[同余类(剩余类)]
    由定理\ref{theorem:7.ex02.20}知,对于给定的模$m$,
    整数的同余关系是一个等价关系,因此全体整数可按对模$m$是否同余
    分为若干个两两不相交的集合,使得在同一个集合中的任意两个数
    对模$m$一定同余,而属于不同集合中的两个数对模$m$一定不同余。
    每一个这样的集合称为是模$m$的\keyindex{同余类}{congruence class}{},
    或模$m$的\keyindex{剩余类}{residue class}{}。
    我们把$r$所属的模$m$的同余类表示为$r\mod{m}$.
\end{definition}

\begin{theorem}\label{theorem:7.ex02.26}
    同余类具有以下性质:
    \begin{enumerate}
        \item $r\mod{m}=\{r+km:k\in\mathbb{Z}\}$;
        \item $r\mod{m}=s\mod{m}\Leftrightarrow r\equiv s\pmod{m}$;
        \item 对任意的$r,s$,要么$r\mod{m}=s\mod{m}$,要么$r\mod{m}$与$s\mod{m}$的交集为空集。
    \end{enumerate}
\end{theorem}
\begin{theorem}\label{theorem:7.ex02.27}
    对于给定的模$m$,有且恰有$m$个不同的模$m$的同余类,即
    \begin{align}\label{eq:7.ex02.theorem27-02}
        0\mod{m},\quad 1\mod{m},\quad \ldots,\quad (m-1)\mod{m}\, .
    \end{align}
\end{theorem}
\begin{prove}
    由定理\ref{theorem:7.ex02.26}(2)知这是$m$个两两不同的同余类。
    对每个整数$a$,由定理\ref{theorem:7.ex02.4}知
    \begin{align}
        a=qm+r,\quad 0\le r<m\, .
    \end{align}
    故由定理\ref{theorem:7.ex02.26}(1)知,$a\in r\mod{m}$,
    即必属于\refeq{7.ex02.theorem27-02}中的某个同余类。
\end{prove}
\begin{theorem}\label{theorem:7.ex02.28}
    同余具有以下性质:
    \begin{enumerate}
        \item 在任意取定的$m+1$个整数中,必有两个数对模$m$同余;
        \item 存在$m$个数两两对模$m$不同余。
    \end{enumerate}
\end{theorem}
\begin{prove}
    由定理\ref{theorem:7.ex02.27},对模$m$共有$m$个由\refeq{7.ex02.theorem27-02}给出的同余类,
    所以根据定理\ref{theorem:7.ex02.2},$m+1$个数中必有两个数属于同一个模$m$的同余类,
    这两个数就对模$m$同余,第一个结论得证。在每个同余类$r\mod{m}(0\le r<m)$中
    取定一个数$x_r$作代表,就得到$m$个两两对模$m$不同余的数$x_0,x_1,\ldots,x_{m-1}$,第二个结论得证。
\end{prove}

由定理\ref{theorem:7.ex02.28}可引进以下概念:
\begin{definition}
    一组数$y_1,\ldots,y_s$称为是模$m$的\keyindex{完全剩余系}{complete residue system}{},
    如果对任意的$a$有且仅有一个$y_j$满足$a\equiv y_j\pmod{m}$.
\end{definition}

\subsection{同余方程}\label{sub:同余方程}
\begin{definition}
    设整系数多项式
    \begin{align}
        f(x)=a_nx^n+\cdots+a_1x+a_0\, ,
    \end{align}
    我们称含有变量$x$的同余式
    \begin{align}\label{eq:7.ex02.congruenceequation}
        f(x)\equiv0\pmod{m}
    \end{align}
    为模$m$的\keyindex{同余方程}{congruence equation}{equation方程}。
    若整数$c$满足
    \begin{align}
        f(c)\equiv0\pmod{m}\, ,
    \end{align}
    则称$c$是同余方程\refeq{7.ex02.congruenceequation}的\keyindex{解}{solution}{}。
\end{definition}

在上述定义中,显然同余类$c\mod{m}$中的任一整数也是
同余方程\refeq{7.ex02.congruenceequation}的解。
我们把这些解都看作是相同的,也常说同余类$c\mod{m}$是该方程的解,
写为$x\equiv c\pmod{m}$.当$c_1,c_2$均为该同余方程的解且对模$m$不同余时
才把它们看作是不同的解。我们把所有对模$m$两两不同余的解的个数
称为是同余方程\refeq{7.ex02.congruenceequation}
的\keyindex{解数}{number of solutions}{}。
因此我们只需要在模$m$的一组完全剩余系中来解模$m$的同余方程。
显然模$m$的同余方程的解数至多为$m$.
\begin{example}
    对于同余方程$4x^2+27x-12\equiv0\pmod{15}$,
    取模15的一个完全剩余系$-7,-6,\ldots,-1,0,1,\ldots,6,7$,
    直接代入验算知$x=-6,3$是解,所以该同余方程的解
    是$x\equiv -6,3\pmod{15}$,解数为2.
\end{example}
\begin{definition}
    设$m\nmid a$,称
    \begin{align}\label{eq:7.ex02.linearcongruence}
        ax\equiv b\pmod{m}
    \end{align}
    为模$m$的{\sffamily 一次同余方程}。
\end{definition}
\begin{example}
    同余方程$6x\equiv2\pmod{8}$的解是$x\equiv-1,3\pmod{8}$,解数为2.
\end{example}
\begin{theorem}\label{theorem:7.ex02.29}
    当$(a,m)=1$时,同余方程\refeq{7.ex02.linearcongruence}必有解,且其解数为1.
\end{theorem}
\begin{prove}
    当$(a,m)=1$时,由定理\ref{theorem:7.ex02.24}知,
    $a$对模$m$有逆$a^{-1}$(任取一个)满足
    \begin{align}
        aa^{-1}\equiv1\pmod{m}\, .
    \end{align}
    容易看出
    \begin{align}
        x_1=a^{-1}b
    \end{align}
    就满足同余方程\refeq{7.ex02.linearcongruence}。若还有解$x_2$,则有
    \begin{align}
        ax_2\equiv ax_1\pmod{m}\, ,
    \end{align}
    由此根据定理\ref{theorem:7.ex02.23}得
    \begin{align}
        x_2\equiv x_1\pmod{m}\, .
    \end{align}
    这就证明了解数为1.
\end{prove}
% \begin{theorem}
%     同余方程\refeq{7.ex02.linearcongruence}有解的充要条件是
%     \begin{align}\label{eq:7.ex02.conditionsolution}
%         (a,m)|b\, .
%     \end{align}
%     在有解时,其解数等于$(a,m)$;若$x_0$是它的解,则它的$(a,m)$个解是
%     \begin{align}
%         x\equiv x_0+\frac{m}{(a,m)}t\pmod{m},\quad t=0,\ldots,(a,m)-1\, .
%     \end{align}
% \end{theorem}
% \begin{theorem}\label{theorem:7.ex02.hassolutionlinear}
%     当$(a,m)=1$时,同余方程\refeq{7.ex02.linearcongruence}必有解,且其解数为1.
% \end{theorem}
% \begin{prove}
%     当$(a,m)=1$时,由定理\ref{theorem:7.ex02.modularinverse}知,
%     $a$对模$m$有逆$a^{-1}$(任取一个)满足
%     \begin{align}
%         aa^{-1}\equiv1\pmod{m}\, .
%     \end{align}
%     容易看出
%     \begin{align}
%         x_1=a^{-1}b
%     \end{align}
%     就满足同余方程\refeq{7.ex02.linearcongruence}。若还有解$x_2$,则有
%     \begin{align}
%         ax_2\equiv ax_1\pmod{m}\, ,
%     \end{align}
%     则从定理\ref{theorem:7.ex02.congruentreduce}推出
%     \begin{align}
%         x_2\equiv x_1\pmod{m}\, .
%     \end{align}
%     这就证明了解数为1.
% \end{prove}
% \begin{theorem}
%     同余方程\refeq{7.ex02.linearcongruence}有解的充要条件
%     是\refeq{7.ex02.conditionsolution}成立。在有解时,
%     它的解数等于$(a,m)$,以及若$x_0$是\refeq{7.ex02.linearcongruence}的解,
%     则它的$(a,m)$个解是
%     \begin{align}
%         x\equiv x_0+\frac{m}{(a,m)}t\pmod{m},\quad t=0,1,\ldots,(a,m)-1\, .
%     \end{align}
% \end{theorem}

\begin{definition}
    设$f_j(x), j=1,2,\ldots,k$是整系数多项式,我们把含有变量$x$的一组同余式
    \begin{align}\label{eq:7.ex02.congruencegroup}
        f_j(x)\equiv0\pmod{m_j},\quad 1\le j\le k\, ,
    \end{align}
    称为{\sffamily 同余方程组}。若整数$c$同时满足
    \begin{align}
        f_j(c)\equiv0\pmod{m_j},\quad 1\le j\le k\, ,
    \end{align}
    则称$c$是同余方程组\refeq{7.ex02.congruencegroup}的\keyindex{解}{solution}{}。
\end{definition}

显然在上述定义中,同余类
\begin{align}\label{eq:7.ex02.groupsolution}
    c\mod{m},\quad m=[m_1,\ldots,m_k]
\end{align}
中任一整数也是同余方程组\refeq{7.ex02.congruencegroup}的解,
我们把它们看作是相同的,也常说同余类\refeq{7.ex02.groupsolution}是
该同余方程组的一个解,写作$x\equiv c\pmod{m}$.
当$c_1,c_2$均为该同余方程组的解且对模$m$不同余时
才把它们看作是不同的解。我们把所有对模$m$两两不同余的解的个数
称为是同余方程组\refeq{7.ex02.congruencegroup}的\keyindex{解数}{number of solutions}{}。
因此我们只需要在模$m$的一组完全剩余系中来解该同余方程组,
它的解数至多为$m$.此外,只要同余方程组中任一一个方程无解,
则\refeq{7.ex02.congruencegroup}一定无解。
\begin{theorem}[\protect\keyindex{中国剩余定理}{Chinese remainder theorem}{}(CRT)]\label{theorem:7.ex02.30}
    也称{\sffamily 孙子定理}:设$m_1,\ldots,m_k$是两两互质的正整数,
    则对任意整数$a_1,\ldots,a_k$,一次同余方程组
    \begin{align}\label{eq:7.ex02.CRT}
        x\equiv a_j\pmod{m_j},\quad 1\le j\le k\, ,
    \end{align}
    必有解,且解数为1.事实上,该同余方程组的解是
    \begin{align}
        x\equiv M_1M_1^{-1}a_1+\ldots+M_kM_k^{-1}a_k\pmod{m}\, ,
    \end{align}
    这里$m=m_1m_2\cdots m_k$,$m=m_jM_j(1\le j\le k)$,以及$M_j^{-1}$是满足
    \begin{align}
        M_jM_j^{-1}\equiv1\pmod{m_j},\quad 1\le j\le k
    \end{align}
    的一个整数(即是$M_j$对模$m_j$的逆)。
\end{theorem}
\begin{prove}
    首先指出一个事实:若$x_0$满足同余方程组\refeq{7.ex02.CRT},
    且$x_0'$满足下面的另一同余方程组
    \begin{align}
        x\equiv a_j'\pmod{m_j},\quad 1\le j\le k\, ,
    \end{align}
    则$x_0+x_0'$一定是同余方程组
    \begin{align}
        x\equiv a_j+a_j'\pmod{m_j},\quad 1\le j\le k
    \end{align}
    的解。因此,我们可用下面的叠加方法来求同余方程组\refeq{7.ex02.CRT}的解。设
    \begin{align}\label{eq:7.ex02.proveCRT03}
        a_j^{(i)}=\left\{\begin{array}{ll}
            a_j, & \text{若}i=j\, ,     \\
            0,   & \text{若}i\neq j\, .
        \end{array}\right.
    \end{align}
    对每个固定的$i(1\le j\le k)$考虑同余方程组
    \begin{align}\label{eq:7.ex02.proveCRT01}
        x\equiv a_j^{(i)}\pmod{m_j},\quad 1\le j\le k\, .
    \end{align}
    注意到$j\neq i$时$a_j^{(i)}=0$,结合$m_j$两两互质,
    由这个方程组的第$1,\ldots,i-1,i+1,\ldots,k$个方程知
    \begin{align}
        x\equiv0\pmod{M_i}\, ,
    \end{align}
    即存在整数$y$使得
    \begin{align}\label{eq:7.ex02.proveCRT02}
        x=M_iy\, .
    \end{align}
    代入第$i$个方程得
    \begin{align}
        M_iy\equiv a_i\pmod{m_i}\, .
    \end{align}
    由定理\ref{theorem:7.ex02.29}的证明知
    \begin{align}
        y\equiv M_i^{-1}a_i\pmod{m_i}\, ,
    \end{align}
    即
    \begin{align}
        M_iy\equiv M_iM_i^{-1}a_i\pmod{m}\, .
    \end{align}
    由此及\refeq{7.ex02.proveCRT02}得
    \begin{align}
        x\equiv M_iM_i^{-1}a_i\pmod{m}\, .
    \end{align}
    容易验证,$M_iM_i^{-1}a_i$确是同余方程组\refeq{7.ex02.proveCRT01}的解,
    且由定理\ref{theorem:7.ex02.29}知解数为1.
    注意到由\refeq{7.ex02.proveCRT03}可得
    \begin{align}
        a_j^{(1)}+a_j^{(2)}+\cdots+a_j^{(k)}=a_j\, ,
    \end{align}
    所以$M_1M_1^{-1}a_1+\cdots+M_kM_k^{-1}a_k$一定是同余方程组\refeq{7.ex02.CRT}的解。
    若$c_1,c_2$均是同余方程组\refeq{7.ex02.CRT}的解,
    则必有
    \begin{align}
        c_1\equiv c_2\pmod{m_j},\quad 1\le j\le k\, .
    \end{align}
    又因为$m_1,\ldots,m_k$两两互质,所以
    \begin{align}
        m=m_1m_2\cdots m_k=[m_1,\ldots,m_k]\, .
    \end{align}
    利用定理\ref{theorem:7.ex02.25}结合上两式可得
    \begin{align}
        c_1\equiv c_2\pmod{m}\, ,
    \end{align}
    即同余方程组\refeq{7.ex02.CRT}的解数必为1.
\end{prove}
\begin{example}
    解同余方程组
    \begin{align}
        \left\{
        \begin{array}{l}
            x\equiv1\pmod{3}\, ,  \\
            x\equiv-1\pmod{5}\, , \\
            x\equiv2\pmod{7}\, ,  \\
            x\equiv-2\pmod{11}\, .
        \end{array}
        \right.
    \end{align}
    {\sffamily 解}\quad 取$m_1=3,m_2=5,m_3=7,m_4=11$,
    满足定理\ref{theorem:7.ex02.30}的条件。这时,
    $M_1=5\times7\times11,M_2=3\times7\times11,M_3=3\times5\times11,M_4=3\times5\times7$.
    由于$M_1\equiv(-1)\times1\times(-1)\equiv1\pmod{3}$,所以令
    \begin{align}
        1\equiv M_1M_1^{-1}\equiv M_1^{-1}\pmod{3}\, ,
    \end{align}
    因此可取$M_1^{-1}=1$.由于$M_2\equiv(-2)\times2\times1\equiv1\pmod{5}$,令
    \begin{align}
        1\equiv M_2M_2^{-1}\equiv M_2^{-1}\pmod{5}\, ,
    \end{align}
    因此可取$M_2^{-1}=1$.由于$M_3\equiv3\times5\times4\equiv4\pmod{7}$,令
    \begin{align}
        1\equiv M_3M_3^{-1}\equiv4M_3^{-1}\pmod{7}\, ,
    \end{align}
    因此可取$M_3^{-1}=2$.由$M_4\equiv3\times5\times7\equiv4\times7\equiv6\pmod{11}$,令
    \begin{align}
        1\equiv M_4M_4^{-1}\equiv6M_4^{-1}\pmod{11}\, ,
    \end{align}
    因此可取$M_4^{-1}=2$.进而由定理\ref{theorem:7.ex02.30}知
    同余方程组的解为
    \begin{align}
        x & \equiv(5\times7\times11)\times1\times1+(3\times7\times11)\times1\times(-1)\nonumber                               \\
          & \qquad+(3\times5\times11)\times2\times2+(3\times5\times7)\times2\times(-2)\pmod{3\times5\times7\times11}\nonumber \\
          & \equiv385-231+660-420\equiv394\pmod{1155}\, .
    \end{align}
\end{example}

至此我们介绍了Halton样本生成中求解一次同余方程组所需的全部背景知识。
代码片{\refcode{Compute Halton sample offset for currentPixel}{}}中
即采用了定理\ref{theorem:7.ex02.30}:其中{\ttfamily dimOffset}对应了$a_j$,
{\ttfamily\refvar{sampleStride}{} / \refvar{baseScales}{}}对应了$M_j$,
{\refvar{multInverse}{}}对应了$M_j^{-1}$,它由扩展欧几里得法求得,代码为
\begin{lstlisting}
static uint64_t `\initvar{multiplicativeInverse}{}`(int64_t a, int64_t n) {
    int64_t x, y;
    `\refvar{extendedGCD}{}`(a, n, &x, &y);
    return `\refvar{Mod}{}`(x, n);
}
\end{lstlisting}
\begin{lstlisting}
static void `\initvar{extendedGCD}{}`(uint64_t a, uint64_t b, int64_t *x, int64_t *y) {
    if (b == 0) {
        *x = 1;
        *y = 0;
        return;
    }
    int64_t d = a / b, xp, yp;
    `\refvar{extendedGCD}{}`(b, a % b, &xp, &yp);
    *x = yp;
    *y = xp - (d * yp);
}
\end{lstlisting}



\part{光的散射}
\chapterimage{Pictures/chap08/dragons-fourier-600x1200.png}
\chapter{反射模型}\label{chap:反射模型}
\setcounter{sidenote}{1}
本章定义一组类来描述光在表面上散射的方式。回想\refsub{BRDF}中
我们介绍了双向反射分布函数(BRDF)抽象来描述表面的光反射,
双向透射分布函数(BTDF)来描述表面的透射,以及
双向散射分布函数(BSDF)来统合这两种效应。
本章中,我们将从为这些表面反射和透射函数定义通用接口开始。

许多来自表面的散射通常最好描述为多个BRDF和BTDF随空间变化的混合体;
在第\refchap{材质},我们将介绍结合了多个BRDF和BTDF的BSDF对象
以表示来自表面的整体散射。本章回避了反射和折射性质随表面变化的问题;
第\refchap{纹理}的纹理类将解决该问题。
BRDF和BTDF只显式建模了在表面上同一点入射和出射的光的散射。
对于展现出有意义的次表面光传输的曲面,我们将引入类\refvar{BSSRDF}{},
在第\refchap{体积散射}介绍一些相关理论后,它将在\refsec{BSSRDF}对次表面散射建模。

表面反射模型有以下几个来源:
\begin{itemize}
      \item \emph{测量的数据}:许多真实世界表面的反射分布性质已在实验室中测定。
            这样的数据可直接以表格形式使用或用来为一组基函数计算系数。
      \item \emph{现象模型}\sidenote{译者注:原文phenomenological models。}:
            试图描述真实世界表面定性性质的方程在仿真时可能很有效。
            这类BSDF可能很容易使用,因为它们常常有直观的参数来修改其表现(例如“粗糙度”)。
      \item \emph{模拟}:有时关于表面组成的底层信息是已知的。
            例如,我们可能知道涂料是悬浮在介质中的平均大小彩色颗粒组成的,
            或者某种布料是两种织线组成的,且知道每种的反射性质。
            在这些情况下,可以模拟来自微观几何体的光散射来生成反射数据。
            该模拟可在渲染时进行,或作为预处理完成后去适配一组基函数供渲染时使用。
      \item \emph{物理(波动)光学}:一些反射模型是用详细的光模型推导出的,
            将其视作波并计算麦克斯韦方程组的解以求解光是怎么从已知性质的表面散射的。
            这些模型常常计算量很大,然而对于渲染应用而言它们通常并不比基于几何光学的模型精确多少。
      \item \emph{几何光学}:像模拟方法那样,如果表面的底层散射和几何性质已知,
            则有时能直接从这些描述中推出解析式的反射模型。几何光学让建模光与表面的交互
            更加容易处理,因为可以忽略像偏振那样的复杂波动效应。
\end{itemize}
本章末的“扩展阅读”一节给出了许多这样的反射模型索引。

在我们定义相关接口前,简要回顾下它们是怎么嵌入整个系统的。
如果用了\refvar{SamplerIntegrator}{},则会为每条光线
调用方法\refvar[Li]{SamplerIntegrator::Li}{()}的实现。
在找到与几何图元最近的相交处后,它调用与该图元关联的表面着色器。
表面着色器实现为\refvar{Material}{}子类的方法并负责决定表面上特定点的BSDF是什么;
它返回的BSDF对象持有BRDF和BTDF且已分配内存和初始化来表示该点的散射。
然后积分器基于该点的入射光照用BSDF计算该点的散射光
(使用\refvar{BDPTIntegrator}{}、\refvar{MLTIntegrator}{}或\refvar{SPPMIntegrator}{}而
不是\refvar{SamplerIntegrator}{}的过程大致相同)。

\subsection{基本术语}\label{sub:基本术语}
为了能比较不同反射模型的视觉表现,我们将介绍一些基本术语以描述来自表面的反射。

来自表面的反射可分为四大类:\keyindex{漫反射}{diffuse}{}、\keyindex{光泽镜面}{glossy specular}{}、
\keyindex{完美镜面}{perfect specular}{}和\keyindex{逆反射}{retro-reflective}{}(\reffig{8.1})。
大多数真实表面展现的反射都是这四种的混合。漫反射表面在所有方向均等地散射光。
尽管完美的漫反射表面是不可物理实现的,但几乎是漫反射表面的例子包括暗沉的黑板和哑光的油漆。
光泽镜面表面例如塑料或高光泽涂料优先在一组反射方向上散射光——
它们展示了其他物体的模糊反射。完美镜像表面将入射光朝单个出射方向散射。
镜子和玻璃就是完美镜面表面的例子。
最后,像天鹅绒或月壤那样的逆反射表面主要沿着入射方向把光散射回去。
本章的图像将展示渲染场景中使用这几种反射的差别。
\begin{figure}[htbp]
      \centering
      \subfloat[漫反射BSDF]{\includegraphics[width=0.5\linewidth]{chap08/brdf-diffuse-plot.jpg}\label{fig:8.1.1}}
      \subfloat[光泽BRDF]{\includegraphics[width=0.5\linewidth]{chap08/brdf-glossy-plot.jpg}\label{fig:8.1.2}}\\
      \subfloat[几乎完美的镜面BRDF]{\includegraphics[width=0.5\linewidth]{chap08/brdf-specular-plot.jpg}\label{fig:8.1.3}}
      \subfloat[逆反射BRDF]{\includegraphics[width=0.5\linewidth]{chap08/brdf-retro-plot.jpg}\label{fig:8.1.4}}
      \caption{来自表面的反射通常可按反射光相对于入射方向(粗线)的分布来划分:
            (1)漫反射、(2)光泽镜面、(3)几乎完美的镜面、(4)逆反射分布。}
      \label{fig:8.1}
\end{figure}

对于特定的某种反射,反射分布函数可能是\keyindex{各向同性}{isotropic}{}
或\keyindex{各向异性}{anisotropic}{}的。大部分物体是各向同性的:
如果你在表面上选一点并绕该处的法线轴旋转它,反射光的分布不变。
相反,当你像这样旋转各向异性材料时,它们反射的光量会不同。
各向异性表面的例子包括拉丝金属、多种布料和压缩光盘。

\subsection{几何设置}\label{sub:几何设置}
pbrt中的反射计算是在反射坐标系中进行的,
被着色点的两个切向量和法向量分别对齐到$x$、$y$和$z$轴(\reffig{8.2})。
所有传入BRDF和BTDF例程及其返回的各方向向量都在该坐标系下定义。
为了理解本章的BRDF和BTDF实现,理解该坐标系很重要。
\begin{figure}[htbp]
      \centering
      \includegraphics[width=0.7\linewidth]{chap08/BSDFcoordinatesystem.eps}
      \caption{基本BSDF接口设置。着色坐标系由正交基向量$({\bm s},{\bm t},{\bm n})$定义。
            我们将调整这些向量朝向使得它们在该坐标系中沿着$x$、$y$和$z$轴。
            在调用任何BRDF或BTDF方法前,世界空间中的方向向量$\bm \omega$会被
            变换到着色坐标系。}
      \label{fig:8.2}
\end{figure}

着色坐标系也给出了表示球面坐标$(\theta,\varphi)$中方向的坐标系;
角度$\theta$从给定方向测量到$z$轴,$\varphi$是方向投影到$xy$平面后
与$x$轴所成角度。给定该坐标系下的方向向量$\bm\omega$,
很容易计算与法向夹角的余弦等量:
\begin{align*}
      \cos\theta=({\bm n}\cdot{\bm\omega})=((0,0,1)\cdot{\bm\omega})=\omega_z\, .
\end{align*}

我们将提供实用函数来计算这些值和一些有用的变量;
它们的使用有助于阐明BRDF和BTDF的实现。
\begin{lstlisting}
`\initcode{BSDF Inline Functions}{=}\initnext{BSDFInlineFunctions}`
inline `\refvar{Float}{}` `\initvar{CosTheta}{}`(const `\refvar{Vector3f}{}` &w) { return w.z; }
inline `\refvar{Float}{}` `\initvar{Cos2Theta}{}`(const `\refvar{Vector3f}{}` &w) { return w.z * w.z; }
inline `\refvar{Float}{}` `\initvar{AbsCosTheta}{}`(const `\refvar{Vector3f}{}` &w) { return std::abs(w.z); }
\end{lstlisting}

$\sin^2\theta$的值可用三角恒等式$\sin^2\theta+\cos^2\theta=1$算得,
但我们要注意避免取负数的平方根,罕见情况下浮点舍入误差
会让{\ttfamily 1 - \refvar{Cos2Theta}{}(w)}小于零。
\begin{lstlisting}
`\refcode{BSDF Inline Functions}{+=}\lastnext{BSDFInlineFunctions}`
inline `\refvar{Float}{}` `\initvar{Sin2Theta}{}`(const `\refvar{Vector3f}{}` &w) {
    return std::max((`\refvar{Float}{}`)0, (`\refvar{Float}{}`)1 - `\refvar{Cos2Theta}{}`(w));
}
inline `\refvar{Float}{}` `\initvar{SinTheta}{}`(const `\refvar{Vector3f}{}` &w) {
    return std::sqrt(`\refvar{Sin2Theta}{}`(w));
}
\end{lstlisting}

角$\theta$的正切可以由恒等式$\displaystyle\tan\theta=\frac{\sin\theta}{\cos\theta}$计算。
\begin{lstlisting}
`\refcode{BSDF Inline Functions}{+=}\lastnext{BSDFInlineFunctions}`
inline `\refvar{Float}{}` `\initvar{TanTheta}{}`(const `\refvar{Vector3f}{}` &w) {
    return `\refvar{SinTheta}{}`(w) / `\refvar{CosTheta}{}`(w);
}
inline `\refvar{Float}{}` `\initvar{Tan2Theta}{}`(const `\refvar{Vector3f}{}` &w) {
    return `\refvar{Sin2Theta}{}`(w) / `\refvar{Cos2Theta}{}`(w);
}
\end{lstlisting}

我们可以类似地用着色坐标系来简化角$\varphi$的正余弦计算(\reffig{8.3})。
在被着色点的平面内,向量$\bm\omega$的坐标$(x,y)$分别由$r\cos\varphi$和$r\sin\varphi$给出。
半径$r$是$\sin\theta$,所以
\begin{align*}
      \cos\varphi & =\frac{x}{r}=\frac{x}{\sin\theta}\,   \\
      \sin\varphi & =\frac{y}{r}=\frac{y}{\sin\theta}\, .
\end{align*}

\begin{figure}[htbp]
      \centering
      \includegraphics[width=0.7\linewidth]{chap08/BSDFthetaphiangles.eps}
      \caption{$\sin\varphi$和$\cos\varphi$的值可用球面坐标
            方程$x=r\cos\varphi$和$y=r\sin\varphi$算得,其中$r$是虚线的长度,等于$\sin\theta$.}
      \label{fig:8.3}
\end{figure}

\begin{lstlisting}
`\refcode{BSDF Inline Functions}{+=}\lastnext{BSDFInlineFunctions}`
inline `\refvar{Float}{}` `\initvar{CosPhi}{}`(const `\refvar{Vector3f}{}` &w) {
    `\refvar{Float}{}` sinTheta = `\refvar{SinTheta}{}`(w);
    return (sinTheta == 0) ? 1 : `\refvar{Clamp}{}`(w.x / sinTheta, -1, 1);
}
inline `\refvar{Float}{}` `\initvar{SinPhi}{}`(const `\refvar{Vector3f}{}` &w) {
    `\refvar{Float}{}` sinTheta = `\refvar{SinTheta}{}`(w);
    return (sinTheta == 0) ? 0 : `\refvar{Clamp}{}`(w.y / sinTheta, -1, 1);
}
\end{lstlisting}
\begin{lstlisting}
`\refcode{BSDF Inline Functions}{+=}\lastnext{BSDFInlineFunctions}`
inline `\refvar{Float}{}` `\initvar{Cos2Phi}{}`(const `\refvar{Vector3f}{}` &w) {
    return `\refvar{CosPhi}{}`(w) * `\refvar{CosPhi}{}`(w);
}
inline `\refvar{Float}{}` `\initvar{Sin2Phi}{}`(const `\refvar{Vector3f}{}` &w) {
    return `\refvar{SinPhi}{}`(w) * `\refvar{SinPhi}{}`(w);
}
\end{lstlisting}

两个向量在着色坐标系下$\varphi$值间的角$\Delta\varphi$的余弦
可以通过置零两个向量的$z$坐标获得2D向量后再规范化求得。
这两个向量的点积给出了它们夹角的余弦。
为了高效,下面的实现重新排列了项使得只需执行一次平方根运算。
\begin{lstlisting}
`\refcode{BSDF Inline Functions}{+=}\lastnext{BSDFInlineFunctions}`
inline `\refvar{Float}{}` `\initvar{CosDPhi}{}`(const `\refvar{Vector3f}{}` &wa, const `\refvar{Vector3f}{}` &wb) {
    return `\refvar{Clamp}{}`((wa.x * wb.x + wa.y * wb.y) /
                 std::sqrt((wa.x * wa.x + wa.y * wa.y) *
                           (wb.x * wb.x + wb.y * wb.y)), -1, 1);
}
\end{lstlisting}

当阅读本章代码和向pbrt增添BRDF和BTDF时,需要记住一些重要约定和实现细节。
\begin{itemize}
      \item 入射光方向${\bm\omega}_{\mathrm{i}}$和出射查看方向${\bm\omega}_{\mathrm{o}}$都是
            规范化的,且变换到表面的局部坐标系后都是朝外指的。
      \item 按pbrt的约定,曲面法线$\bm n$总是指向物体的“外侧”,这让确定光是进入还是射出透明物体更简单:
            如果入射光方向${\bm\omega}_{\mathrm{i}}$和$\bm n$在同一半球,
            则光在射入;否则光在射出。因此,要记住的一个细节是法线可能相对于
            一个或两个方向向量${\bm\omega}_{\mathrm{i}}$和${\bm\omega}_{\mathrm{o}}$
            在表面的对侧。不像许多其他渲染器那样,pbrt不会翻转法线使其和${\bm\omega}_{\mathrm{o}}$在同侧。
      \item 用于着色的局部坐标系可能并不和来自第\refchap{形状}的例程\refvar{Shape::Intersect}{()}
            返回的坐标系一样;它们在相交和着色间为了达到凹凸贴图等效果可能会被修改。见第\refchap{材质}这类修改的例子。
      \item 最后,BRDF和BTDF的实现不应关心${\bm\omega}_{\mathrm{i}}$和${\bm\omega}_{\mathrm{o}}$是否在同一半球。
            例如,尽管反射BRDF原则上应该检测是否入射方向在表面之上而出射方向在下面并在这种情况下总是不返回反射,
            但这里我们希望反射函数代之以利用其反射模型的合适公式计算和返回反射的光量,
            忽略它们不在同一半球的细节。pbrt中的高层级代码会保证只有反射或透射散射例程会适当求值。
            该约定的价值将在\refsec{BSDF}解释。
\end{itemize}

\section{基本接口}\label{sec:基本接口}
我们将首先定义单个BRDF和BTDF函数的接口。
BRDF和BTDF共享共同的基类\refvar{BxDF}{}。
因为两者都有一样的接口,共享相同的基类减少了重复代码并
允许系统的一些部分和一般的\refvar{BxDF}{}配合而不用区分BRDF和BTDF。
\begin{lstlisting}
`\initcode{BxDF Declarations}{=}\initnext{BxDFDeclarations}`
class `\initvar{BxDF}{}` {
public:
    `\refcode{BxDF Interface}{}`
    `\refcode{BxDF Public Data}{}`
};
\end{lstlisting}

\refsec{BSDF}将要介绍的类\refvar{BSDF}{}持有一系列\refvar{BxDF}{}对象
来一起描述表面上一点的散射。尽管我们把\refvar{BxDF}{}的实现细节隐藏到
反射和透射材质的公共接口后,第\refchap{光传输I:表面反射}到\refchap{光传输III:双向方法}的
一些光传输算法还是需要区分这两个类型。因此,所有\refvar{BxDF}{}都
有成员\refvar{BxDF::type}{}持有来自\refvar{BxDFType}{}的标志。
对于每个\refvar{BxDF}{},该标志应至少有一个置为\refvar[BSDFREFLECTION]{BSDF\_REFLECTION}{}
或\refvar[BSDFTRANSMISSION]{BSDF\_TRANSMISSION}{},且恰有一个漫反射、光泽或镜面标志。
注意没有逆反射标志;这里的分类中逆反射被当作光泽反射。

\begin{lstlisting}
`\initcode{BSDF Declarations}{=}\initnext{BSDFDeclarations}`
enum `\initvar{BxDFType}{}` {
    `\initvar[BSDFREFLECTION]{BSDF\_REFLECTION}{}` = 1 << 0,
    `\initvar[BSDFTRANSMISSION]{BSDF\_TRANSMISSION}{}` = 1 << 1,
    `\initvar[BSDFDIFFUSE]{BSDF\_DIFFUSE}{}` = 1 << 2,
    `\initvar[BSDFGLOSSY]{BSDF\_GLOSSY}{}` = 1 << 3,
    `\initvar[BSDFSPECULAR]{BSDF\_SPECULAR}{}` = 1 << 4,
    `\initvar[BSDFALL]{BSDF\_ALL}{}` = BSDF_DIFFUSE | BSDF_GLOSSY | BSDF_SPECULAR |
                        BSDF_REFLECTION | BSDF_TRANSMISSION,
};
\end{lstlisting}

\begin{lstlisting}
`\initcode{BxDF Interface}{=}\initnext{BxDFInterface}`
`\refvar{BxDF}{}`(`\refvar{BxDFType}{}` type) : `\refvar[BxDF::type]{type}{}`(type) { }
\end{lstlisting}

\begin{lstlisting}
`\initcode{BxDF Public Data}{=}`
const `\refvar{BxDFType}{}` `\initvar[BxDF::type]{type}{}`;
\end{lstlisting}

实用方法\refvar{MatchesFlags}{()}确定\refvar{BxDF}{}是否匹配用户提供的类型标志:
\begin{lstlisting}
`\refcode{BxDF Interface}{+=}\lastnext{BxDFInterface}`
bool `\initvar{MatchesFlags}{}`(`\refvar{BxDFType}{}` t) const {
    return (`\refvar[BxDF::type]{type}{}` & t) == `\refvar[BxDF::type]{type}{}`;
}
\end{lstlisting}

\refvar{BxDF}{}提供的关键方法是\refvar{BxDF::f}{()}。
它为给定的方向对返回分布函数的值。该接口隐式假设了不同波长的光是解耦的——
某一波长的能量不会反射成不同波长。通过作出该假设,反射函数的效应可以直接用\refvar{Spectrum}{}表示。
支持该假设不成立的荧光材料则要求该方法返回一个$n\times n$矩阵以编码光谱样本间的能量转化
(其中$n$是\refvar{Spectrum}{}表示中的样本数量)。
\begin{lstlisting}
`\refcode{BxDF Interface}{+=}\lastnext{BxDFInterface}`
virtual `\refvar{Spectrum}{}` `\initvar[BxDF::f]{f}{}`(const `\refvar{Vector3f}{}` &wo, const `\refvar{Vector3f}{}` &wi) const = 0;
\end{lstlisting}

不是所有\refvar{BxDF}{}都能用方法\refvar[BxDF::f]{f}{()}求值。
例如,像镜子、玻璃或水那样的完美镜面物体只把来自单个入射方向的光朝单个出射方向散射。
这样的\refvar{BxDF}{}最好用$\delta$分布描述,即除了光散射的单个方向外都取零。
pbrt中这些\refvar{BxDF}{}需要特殊处理,所以我们也会提供方法\refvar[BxDF::Samplef]{BxDF::Sample\_f}{()}。
该方法既能用于处理由$\delta$分布描述的散射,
也能从散射光有多个方向的\refvar{BxDF}{}中随机采样方向;
第二种应用将在\refsec{采样反射函数}中讨论蒙特卡罗BSDF采样时解释。

\refvar[BxDF::Samplef]{BxDF::Sample\_f}{()}计算给定出射方向${\bm\omega}_{\mathrm{o}}$的
入射光方向${\bm\omega}_{\mathrm{i}}$并为这对方向返回\refvar{BxDF}{}的值。
对于$\delta$分布,\refvar{BxDF}{}有必要这样选择入射光方向,因为调用者
无法生成合适的方向${\bm\omega}_{\mathrm{i}}$
\footnote{反射函数中的$\delta$分布对于光传输算法有一些额外微妙的影响。
    \refsub{镜面反射与透射}和\refsub{被积函数中的delta分布}详细描述了该问题。}。
$\delta$分布的\refvar{BxDF}{}不需要参数{\ttfamily sample}和{\ttfamily pdf},
所以它们会在后面的\refsec{采样反射函数}解释,到时我们将为非镜面反射函数提供该方法的实现。
\begin{lstlisting}
`\refcode{BxDF Interface}{+=}\lastnext{BxDFInterface}`
virtual `\refvar{Spectrum}{}` `\initvar[BxDF::Samplef]{Sample\_f}{}`(const `\refvar{Vector3f}{}` &wo, `\refvar{Vector3f}{}` *wi,
    const `\refvar{Point2f}{}` &sample, `\refvar{Float}{}` *pdf,
    `\refvar{BxDFType}{}` *sampledType = nullptr) const;
\end{lstlisting}

\subsection{反射}\label{sub:反射}
将4D的BRDF或BTDF的表现聚合起来定义为一对方向上的函数,
并将其简化为单个方向上的2D函数甚至是描述其整体散射表现的常数值很有用。

\keyindex{半球定向反射率}{hemispherical-directional reflectance}{}是
一个2D函数,它给出了半球上常量照明于给定方向上的反射率,
或者等价地,因来自给定方向的光而在半球上的总反射率
\footnote{这两个量相等的事实源自反射函数的互易性。BTDF通常不互易;见\refsub{非对称散射}。}。
它定义为
\begin{align}
    \label{eq:8.1}
    \rho_{\mathrm{hd}}({\bm\omega}_{\mathrm{o}})=\int\limits_{H^2({\bm n})}{f_{\mathrm{r}}({\bm p},{\bm \omega}_\mathrm{o},{\bm \omega}_\mathrm{i})|\cos{\theta_{\mathrm{i}}}|\mathrm{d}{\bm \omega}_\mathrm{i}}\, .
\end{align}

方法\refvar{BxDF::rho}{()}计算反射函数$\rho_{\mathrm{hd}}$.
一些\refvar{BxDF}{}能解析地计算该值,然而大部分用蒙特卡罗积分来计算其近似值。
对于那些\refvar{BxDF}{},参数{\ttfamily nSamples}和{\ttfamily samples}供
蒙特卡罗算法的实现使用;它们将在\refsub{应用:估计反射率}解释。
\begin{lstlisting}
`\refcode{BxDF Interface}{+=}\lastnext{BxDFInterface}`
virtual `\refvar{Spectrum}{}` `\initvar[BxDF::rho]{rho}{}`(const `\refvar{Vector3f}{}` &wo, int nSamples,
                     const `\refvar{Point2f}{}` *samples) const;
\end{lstlisting}

表面的\keyindex{半球半球反射率}{hemispherical-hemispherical reflectance}{}记为$\rho_{\mathrm{hh}}$,
该光谱值给出了当各方向入射光相同时表面反射的入射光比例。它是
\begin{align*}
    \rho_{\mathrm{hh}}=\frac{1}{\pi}\int\limits_{H^2({\bm n})}\int\limits_{H^2({\bm n})}f_{\mathrm{r}}({\bm p},{\bm \omega}_\mathrm{o},{\bm \omega}_\mathrm{i})|\cos{\theta_{\mathrm{o}}}\cos{\theta_{\mathrm{i}}}|\mathrm{d}{\bm \omega}_\mathrm{o}\mathrm{d}{\bm \omega}_\mathrm{i}\, .
\end{align*}

如果不提供方向${\bm\omega}_\mathrm{o}$,则方法\refvar[BxDF::rho2]{BxDF::rho}{()}计算$\rho_{\mathrm{hh}}$.
剩下的参数又是在需要时用于计算$\rho_{\mathrm{hh}}$值的蒙特卡罗估计。
\begin{lstlisting}
`\refcode{BxDF Interface}{+=}\lastcode{BxDFInterface}`
virtual `\refvar{Spectrum}{}` `\initvar[BxDF::rho2]{rho}{}`(int nSamples, const `\refvar{Point2f}{}` *samples1,
                     const `\refvar{Point2f}{}` *samples2) const;
\end{lstlisting}

\subsection{BxDF缩放适配器}\label{sub:BxDF缩放适配器}
取一个给定的\refvar{BxDF}{}并用一个\refvar{Spectrum}{}值
缩放它的作用也很有用。\refvar{ScaledBxDF}{}
封装器持有一个\refvar{BxDF}{*}和\refvar{Spectrum}{}并实现其功能。
该类由\refvar{MixMaterial}{}(定义于\refsub{混合材料})使用,
它基于另两种材料的加权和创建\refvar{BSDF}{}。
\begin{lstlisting}
`\refcode{BxDF Declarations}{+=}\lastnext{BxDFDeclarations}`
class `\initvar{ScaledBxDF}{}` : public `\refvar{BxDF}{}` {
public:
    `\refcode{ScaledBxDF Public Methods}{}`
private:
    `\refvar{BxDF}{}` *`\initvar[ScaledBxDF::bxdf]{bxdf}{}`;
    `\refvar{Spectrum}{}` `\initvar[ScaledBxDF::scale]{scale}{}`;
};
\end{lstlisting}
\begin{lstlisting}
`\initcode{ScaledBxDF Public Methods}{=}`
`\refvar{ScaledBxDF}{}`(`\refvar{BxDF}{}` *bxdf, const `\refvar{Spectrum}{}` &scale)
    : `\refvar{BxDF}{}`(`\refvar{BxDFType}{}`(bxdf->`\refvar[BxDF::type]{type}{}`)), `\refvar[ScaledBxDF::bxdf]{bxdf}{}`(bxdf), `\refvar[ScaledBxDF::scale]{scale}{}`(scale) {
}
\end{lstlisting}

\refvar{ScaledBxDF}{}的方法实现很简单;我们这里只介绍\refvar[ScaledBxDF::f]{f}{()}。
\begin{lstlisting}
`\initcode{BxDF Method Definitions}{=}\initnext{BxDFMethodDefinitions}`
`\refvar{Spectrum}{}` `\refvar{ScaledBxDF}{}`::`\initvar[ScaledBxDF::f]{f}{}`(const `\refvar{Vector3f}{}` &wo, const `\refvar{Vector3f}{}` &wi) const {
    return `\refvar[ScaledBxDF::scale]{scale}{}` * `\refvar[ScaledBxDF::bxdf]{bxdf}{}`->`\refvar[BxDF::f]{f}{}`(wo, wi);
}
\end{lstlisting}

\section{镜面反射与透射}\label{sec:镜面反射与透射}

绝对光滑表面上光的特性较容易用物理和几何光学模型分析刻画。
这些表面展现出入射光的完美镜像反射和透射;
对于给定的方向${\bm\omega}_{\mathrm{i}}$,
所有光都散射到单个出射方向${\bm\omega}_{\mathrm{o}}$.
对于镜面反射,该出射方向和法线所成角与入射方向相同:
\begin{align*}
    \theta_{\mathrm{i}}=\theta_{\mathrm{o}}\, ,
\end{align*}
且其中$\varphi_{\mathrm{o}}=\varphi_{\mathrm{i}}+\pi$.
对于透射,我们也有$\varphi_{\mathrm{o}}=\varphi_{\mathrm{i}}+\pi$,
且出射方向$\theta_{\mathrm{t}}$由斯涅尔定律给出,
它将折射方向与曲面法线$\bm n$的夹角$\theta_{\mathrm{t}}$与
入射光线与曲面法线$\bm n$的夹角$\theta_{\mathrm{i}}$联系起来
(本章末的习题之一是用光学的费马原理推导斯涅尔定律)。
斯涅尔定律基于入射光线所在介质的\keyindex{折射率}{index of refraction}{}和
要进入的介质的折射率。折射率描述了光在特定介质中相比在真空中传播要慢多少。
我们用希腊字母$\eta$表示折射率,读作“eta”。斯涅尔定律是
\begin{align}
    \label{eq:8.2}
    \eta_{\mathrm{i}}\sin\theta_{\mathrm{i}}=\eta_{\mathrm{t}}\sin\theta_{\mathrm{t}}\, .
\end{align}

通常,折射率随波长变化。因此,在两种不同介质界面上入射光通常散射到多个方向,
该效应称为\keyindex{色散}{dispersion}{}。
当入射白光被棱镜分出光谱成分时可以观察到该效应。
图形学中的通行做法是忽略该波长依赖性,
因为该效应通常对视觉准确性并不关键且忽略它能极大简化光传输计算。
可选地,可以在有色散物体的环境中追踪多束光路(例如一系列离散波长)。
第\refchap{光传输I:表面反射}末的“扩展阅读”一节有关于该话题的更多信息指引。
\begin{figure}[htbp]
    \centering
    \subfloat[镜面反射]{\includegraphics[width=0.75\linewidth]{chap08/dragon-specular-reflect.png}\label{fig:8.4.1}}\\
    \subfloat[镜面透射]{\includegraphics[width=0.75\linewidth]{chap08/dragon-specular-transmit.png}\label{fig:8.4.2}}
    \caption{用(1)完美镜面反射和(2)完美镜面折射渲染的龙模型。图像(2)排除了
        内外反射的影响;导致的能量损失产生了显眼的暗区(感谢Christian Schüller提供模型)。}
    \label{fig:8.4}
\end{figure}

\reffig{8.4}展示了完美镜面反射和透射的效果。

\subsection{菲涅尔反射率}\label{sub:菲涅尔反射率}
除了反射和透射方向,还有必要计算反射或透射的入射光占比。
为了物理上准确反射或折射,该项依赖于方向,而不能用每个表面的缩放常数表征。
\keyindex{菲涅耳方程}{Fresnel equations}{}
\sidenote{译者注:得名于法国物理学家奥古斯丁·菲涅耳(Augustin-Jean Fresnel)。}描述了
表面上反射光的量;它们是麦克斯韦方程在光滑表面上的解。

给定折射率和入射光与曲面法线所成角度,菲涅耳方程
指定了材料对两种不同偏振状态的入射照明相应的反射率。
因为偏振的视觉效果在大多数环境下是受限的,
所以在pbrt中我们将作出光是无偏振的常用假设;
即光波是随机朝向的。有了该简化假设,菲涅尔反射率就是
平行和垂直偏振项的均方。

此刻有必要指出几个重要材料类别的差异:
\begin{enumerate}
    \item 第一类是\keyindex{介电质}{dielectric}{},
          是不会导电的材料。它们有实数值的折射率(通常在范围1-3内)且
          透射\footnote{注意介电质可能充满能吸收大部分或所有透射光的粒子(例如石油)。
              像水那样的介电质也能通过添加离子使之导电而变成电解质溶液。
              这两方面都和材料本身划分为介电质或导体无关。}一部分入射照明。
          介电质的例子有玻璃、矿油、水和空气。
    \item 第二类组成是\keyindex{导体}{conductor}{}例如金属。
          价电子可以自由地在原子晶格中移动,允许电流从一个地方流到另一处。
          当导体受到电磁辐射例如可见光时,这一基本的原子属性就会转化为完全不同的特性:
          该材料是不透明的并反射回大部分照明。一部分光也透射进导体内部并被迅速吸收:
          总吸收通常发生在材料表面0.1$\mu\mathrm{m}$内,因此只有极薄的金属膜才能透射足够的光量。
          我们在pbrt中忽略该效应而只建模导体的反射部分。与介电质相反,
          导体有复数值的折射率$\bar{\eta}=\eta+\mathrm{i}k$.
    \item \keyindex{半导体}{semiconductor}{}例如硅或锗是本书中我们不予考虑的第三类。
\end{enumerate}

导体和介电质都由同一组菲涅尔方程表征。
尽管如此,我们更喜欢为介电质创建特殊的求值函数,
这样当折射率保证为实数值时,这些方程会取特别简单的形式。

\begin{table}[htbp]
    \centering
    \begin{tabular}{ll}
        \toprule
        \textbf{介质}  & \textbf{折射率}$\eta$ \\
        \midrule
        真空           & 1.0                   \\
        海平面上的空气 & 1.00029               \\
        冰             & 1.31                  \\
        水(20℃)      & 1.333                 \\
        熔融石英       & 1.46                  \\
        玻璃           & 1.5-1.6               \\
        蓝宝石         & 1.77                  \\
        钻石           & 2.42                  \\
        \bottomrule
    \end{tabular}
    \caption{各种物体的折射率,给出了光在真空中的速度与
        光在介质中的速度的比值。它们通常是与波长相关的量;
        这些值是在可见波长上的均值。}
    \label{tab:8.1}
\end{table}

为了计算两种介电质界面处的菲涅尔反射率,我们
需要知道这两种介质的折射率。\reftab{8.1}有许多介电质的折射率。
介电质的菲涅尔反射率公式是
\begin{align*}
    r_{\parallel} & =\frac{\eta_{\mathrm{t}}\cos\theta_{\mathrm{i}}-\eta_{\mathrm{i}}\cos\theta_{\mathrm{t}}}{\eta_{\mathrm{t}}\cos\theta_{\mathrm{i}}+\eta_{\mathrm{i}}\cos\theta_{\mathrm{t}}}\, , \\
    r_{\perp}     & =\frac{\eta_{\mathrm{i}}\cos\theta_{\mathrm{i}}-\eta_{\mathrm{t}}\cos\theta_{\mathrm{t}}}{\eta_{\mathrm{i}}\cos\theta_{\mathrm{i}}+\eta_{\mathrm{t}}\cos\theta_{\mathrm{t}}}\, ,
\end{align*}
其中$r_{\parallel}$是平行偏振光的菲涅尔反射率,
$r_{\perp}$是垂直偏振光的反射率,$eta_{\mathrm{i}}$和$\eta_{\mathrm{t}}$是
入射和透射介质的折射率,$\bm\omega_{\mathrm{i}}$和$\bm\omega_{\mathrm{t}}$是
入射和透射方向。$\bm\omega_{\mathrm{t}}$由斯涅尔定律算出(见\refsub{镜面透射})。

余弦项应该大于或等于零;出于计算这些值的目的,
当计算$\cos\theta_{\mathrm{i}}$和$\cos\theta_{\mathrm{t}}$时,
几何法线应该分别翻转到和$\bm\omega_{\mathrm{i}}$或$\bm\omega_{\mathrm{t}}$同侧。

对于非偏振光,菲涅尔反射率是
\begin{align*}
    F_{\mathrm{r}}=\frac{1}{2}(r_{\parallel}^2+r_{\perp}^2)\, .
\end{align*}

因为能量守恒,介电质透射的能量为$1-F_{\mathrm{r}}$.

函数\refvar{FrDielectric}{()}为介电质材料和非偏振光计算菲涅尔反射率公式。
量$\cos\theta_{\mathrm{i}}$作为参数{\ttfamily cosThetaI}传入。
\begin{lstlisting}
`\initcode{BxDF Utility Functions}{=}`
`\refvar{Float}{}` `\initvar{FrDielectric}{}`(`\refvar{Float}{}` cosThetaI, `\refvar{Float}{}` etaI, `\refvar{Float}{}` etaT) {
    cosThetaI = `\refvar{Clamp}{}`(cosThetaI, -1, 1);
    `\refcode{Potentially swap indices of refraction}{}`
    `\refcode{Compute cosThetaT using Snell's law}{}`
    `\refvar{Float}{}` Rparl = ((etaT * cosThetaI) - (etaI * cosThetaT)) /
                  ((etaT * cosThetaI) + (etaI * cosThetaT));
    `\refvar{Float}{}` Rperp = ((etaI * cosThetaI) - (etaT * cosThetaT)) /
                  ((etaI * cosThetaI) + (etaT * cosThetaT));
    return (Rparl * Rparl + Rperp * Rperp) / 2;
}
\end{lstlisting}

为了求得折射角的余弦{\ttfamily cosThetaT},
首先需要确定入射方向是在介质的外面还是里面,这样才能恰当解释两个折射率。

入射角余弦的符号表明了入射光在介质的哪一侧(\reffig{8.5})。
如果余弦在0到1间,则光线在外侧,如果在-1到0间,则光线在内侧。
调整参数{\ttfamily etaI}和{\ttfamily etaT}使得{\ttfamily etaI}
是入射介质的折射率,这样保证了{\ttfamily cosThetaI}非负。

\begin{figure}[htbp]
    \centering
    \includegraphics[width=0.7\linewidth]{chap08/BSDFanglegivesinout.eps}
    \caption{方向$\bm\omega$和几何曲面法线间夹角$\theta$的余弦
        表明方向是指向表面外侧(和法线在同一半球)还是表面内侧。
        在标准反射坐标系中,该测试只要求检查方向向量的$z$分量。
        这里,$\bm\omega$在上半球,取正值余弦,而${\bm\omega}'$在下半球。}
    \label{fig:8.5}
\end{figure}

\begin{lstlisting}
`\initcode{Potentially swap indices of refraction}{=}`
bool entering = cosThetaI > 0.f;
if (!entering) {
    std::swap(etaI, etaT);
    cosThetaI = std::abs(cosThetaI);
}
\end{lstlisting}

一旦确定折射率,我们就能用斯涅尔定律(\refeq{8.2})计算
透射方向和曲面法线夹角的正弦$\sin\theta_{\mathrm{t}}$.
最后,用恒等式$\sin^2\theta+\cos^2\theta=1$求得该角的余弦。
\begin{lstlisting}
`\initcode{Compute cosThetaT using Snell's law}{=}`
`\refvar{Float}{}` sinThetaI = std::sqrt(std::max((`\refvar{Float}{}`)0,
                                     1 - cosThetaI * cosThetaI));
`\refvar{Float}{}` sinThetaT = etaI / etaT * sinThetaI;
`\refcode{Handle total internal reflection}{}`
`\refvar{Float}{}` cosThetaT = std::sqrt(std::max((`\refvar{Float}{}`)0,
                                     1 - sinThetaT * sinThetaT));
\end{lstlisting}

当从一种介质传播到另一种折射率更低的介质时,入射角接近掠角的光不能进入另一介质。
发生该现象的最大角称为\keyindex{临界角}{critical angle}{};
当$\theta_{\mathrm{i}}$大于临界角时,发生\keyindex{全内反射}{total internal reflection}{reflection反射},
所有光都被反射。这里通过$\sin\theta_{\mathrm{t}}$大于1检测到该情况;
此时不需要菲涅尔方程。
\begin{lstlisting}
`\initcode{Handle total internal reflection}{=}`
if (sinThetaT >= 1)
    return 1;
\end{lstlisting}

我们现在聚焦一般情况下的复数折射率$\bar{\eta}=\eta+\mathrm{i}k$,
其中一些入射光被材料部分吸收并变为热量。
除了实部,一般菲涅尔公式现在也依赖于虚部$k$,
称为\keyindex{吸收系数}{absorption coefficient}{}。

\reffig{8.6}展示了金的折射率和吸收系数图示。两者都是与波长相关的量。
pbrt发行版中目录{\ttfamily scenes/spds/metals}下有各种金属的$\eta$与$k$与波长相关的数据。
下章的\reffig{9.4}展示了用金属材料渲染的模型。
\begin{figure}[htbp]
    \centering
    \includegraphics[width=0.7\linewidth]{chap08/au-k-eta.eps}
    \caption{金的吸收系数和折射率。该图展示了金的吸收系数$k$(实线)
        和折射率$\eta$(虚线)随光谱变化的值,横轴是波长,单位纳米。}
    \label{fig:8.6}
\end{figure}

导体和介电质界面处的菲涅尔反射率是
\begin{align}
    \label{eq:8.3}
    r_{\perp}     & =\frac{a^2+b^2-2a\cos\theta+\cos^2\theta}{a^2+b^2+2a\cos\theta+\cos^2\theta}\, ,\nonumber                                                     \\
    r_{\parallel} & =r_{\perp}\frac{(a^2+b^2)\cos^2\theta-2a\cos\theta\sin^2\theta+\sin^4\theta}{(a^2+b^2)\cos^2\theta+2a\cos\theta\sin^2\theta+\sin^4\theta}\, ,
\end{align}
其中
\begin{align*}
    a^2+b^2=\sqrt{(\eta^2-k^2-\sin^2\theta)^2+4\eta^2k^2}\, ,
\end{align*}
且$\displaystyle\eta+\mathrm{i}k=\frac{\bar{\eta_\mathrm{t}}}{\bar{\eta_\mathrm{i}}}$是
用复数除法算出的相对折射率。然而,通常$\bar{\eta_\mathrm{i}}$是介电质的
所以可以替代使用普通的实数除法。

该计算由函数\refvar{FrConductor}{()}实现
\footnote{注意这稍微用词不当,因为函数在技术上包含了介电质$k=0$的情况。
    也就是说,我们选这个名称是为了表明该函数应只用于处理导体,
    因为它比求\refvar{FrDielectric}{()}的开销更大。};
该实现直接对应\refeq{8.3}所以这里不介绍了。
\begin{lstlisting}
`\initcode{Reflection Declarations}{=}`
`\refvar{Spectrum}{}` `\initvar{FrConductor}{}`(`\refvar{Float}{}` cosThetaI, const `\refvar{Spectrum}{}` &etaI,
    const `\refvar{Spectrum}{}` &etaT, const `\refvar{Spectrum}{}` &k);
\end{lstlisting}

为了方便,我们定义抽象类\refvar{Fresnel}{}以提供接口计算菲涅尔反射系数。
用该接口的实现帮助简化后续可能需要支持两种形式的BRDF实现。
\begin{lstlisting}
`\refcode{BxDF Declarations}{+=}\lastnext{BxDFDeclarations}`
class `\initvar{Fresnel}{}` {
public:
    `\refcode{Fresnel Interface}{}`
};
\end{lstlisting}

\refvar{Fresnel}{}接口提供的唯一函数是\refvar{Fresnel::Evaluate}{()}。
给定入射方向和曲面法线夹角的余弦,它返回表面反射的光量。
\begin{lstlisting}
`\initcode{Fresnel Interface}{=}`
virtual `\refvar{Spectrum}{}` `\initvar[Fresnel::Evaluate]{Evaluate}{}`(`\refvar{Float}{}` cosI) const = 0;
\end{lstlisting}

\subsubsection*{菲涅尔导体}
\refvar{FresnelConductor}{}为导体实现该接口。
\begin{lstlisting}
`\refcode{BxDF Declarations}{+=}\lastnext{BxDFDeclarations}`
class `\initvar{FresnelConductor}{}` : public `\refvar{Fresnel}{}` {
public:
    `\refcode{FresnelConductor Public Methods}{}`
private:
    `\refvar{Spectrum}{}` `\initvar[FresnelConductor::etaI]{etaI}{}`, `\initvar[FresnelConductor::etaT]{etaT}{}`, `\initvar[FresnelConductor::k]{k}{}`;
};
\end{lstlisting}

其构造函数存有给定的折射率$\eta$和吸收系数$k$.
\begin{lstlisting}
`\initcode{FresnelConductor Public Methods}{=}`
`\refvar{FresnelConductor}{}`(const `\refvar{Spectrum}{}` &etaI, const `\refvar{Spectrum}{}` &etaT,
    const `\refvar{Spectrum}{}` &k) : `\refvar[FresnelConductor::etaI]{etaI}{}`(etaI), `\refvar[FresnelConductor::etaT]{etaT}{}`(), `\refvar[FresnelConductor::k]{k}{}`(k) { }
\end{lstlisting}

\refvar{FresnelConductor}{}的求值例程也很简单;它只需调用之前定义的函数
\refvar{FrConductor}{()}。注意在调用\refvar{FrConductor}{()}
前{\ttfamily cosThetaI}要取绝对值,因为\refvar{FrConductor}{()}
要求该余弦是在法线和${\bm\omega}_{\mathrm{i}}$在表面的同一侧时测出的,
或者等价地,应该用$\cos\theta_{\mathrm{i}}$的绝对值。
\begin{lstlisting}
`\refcode{BxDF Method Definitions}{+=}\lastnext{BxDFMethodDefinitions}`
`\refvar{Spectrum}{}` `\refvar{FresnelConductor}{}`::`\initvar[FresnelConductor::Evaluate]{Evaluate}{}`(`\refvar{Float}{}` cosThetaI) const {
    return `\refvar{FrConductor}{}`(std::abs(cosThetaI), `\refvar[FresnelConductor::etaI]{etaI}{}`, `\refvar[FresnelConductor::etaT]{etaT}{}`, `\refvar[FresnelConductor::k]{k}{}`);
}
\end{lstlisting}

\subsubsection*{菲涅尔介电质}
\refvar{FresnelDielectric}{}类似地为介电质材料实现了\refvar{Fresnel}{}接口。
\begin{lstlisting}
`\refcode{BxDF Declarations}{+=}\lastnext{BxDFDeclarations}`
class `\initvar{FresnelDielectric}{}` : public `\refvar{Fresnel}{}` {
public:
    `\refcode{FresnelDielectric Public Methods}{}`
private:
    `\refvar{Float}{}` `\initvar[FresnelDielectric::etaI]{etaI}{}`, `\initvar[FresnelDielectric::etaT]{etaT}{}`;
};
\end{lstlisting}

其构造函数存有表面内外侧的折射率。
\begin{lstlisting}
`\initcode{FresnelDielectric Public Methods}{=}`
`\refvar{FresnelDielectric}{}`(`\refvar{Float}{}` etaI, `\refvar{Float}{}` etaT) : `\refvar[FresnelDielectric::etaI]{etaI}{}`(etaI), `\refvar[FresnelDielectric::etaT]{etaT}{}`(etaT) { }
\end{lstlisting}

\refvar{FresnelDielectric}{}的求值例程类似地调用\refvar{FrDielectric}{()}。
\begin{lstlisting}
`\refcode{BxDF Method Definitions}{+=}\lastnext{BxDFMethodDefinitions}`
`\refvar{Spectrum}{}` `\refvar{FresnelDielectric}{}`::`\initvar[FresnelDielectric::Evaluate]{Evaluate}{}`(`\refvar{Float}{}` cosThetaI) const {
    return `\refvar{FrDielectric}{}`(cosThetaI, `\refvar[FresnelDielectric::etaI]{etaI}{}`, `\refvar[FresnelDielectric::etaT]{etaT}{}`);
}
\end{lstlisting}

\subsubsection*{特殊菲涅尔接口}
\refvar{Fresnel}{}接口的实现\refvar{FresnelNoOp}{}对所有入射方向返回100\%反射率。
尽管这在物理上不可实现,但这是可用的方便能力。
\begin{lstlisting}
`\refcode{BxDF Declarations}{+=}\lastnext{BxDFDeclarations}`
class `\initvar{FresnelNoOp}{}` : public `\refvar{Fresnel}{}` {
public:
    `\refvar{Spectrum}{}` `\initvar[FresnelNoOp::Evaluate]{Evaluate}{}`(`\refvar{Float}{}`) const { return `\refvar{Spectrum}{}`(1.); }
};
\end{lstlisting}

\subsection{镜面反射}\label{sub:镜面反射}
我们现在可以实现类\refvar{SpecularReflection}{},
它用菲涅尔接口计算反射光的占比,描述了物理可实现的镜面反射。
首先,我们将推导描述镜面反射的BRDF。
既然菲涅尔方程给出了反射光的比例$F_{\mathrm{r}}({\bm\omega})$,
那么我们需要这样的BRDF
\begin{align*}
    L_{\mathrm{o}}({\bm\omega}_{\mathrm{o}})=\int{f_{\mathrm{r}}({\bm\omega}_{\mathrm{o}},{\bm\omega}_{\mathrm{i}})L_{\mathrm{i}}({\bm\omega}_{\mathrm{i}})|\cos\theta_{\mathrm{i}}|\mathrm{d}{\bm\omega}_{\mathrm{i}}}=F_{\mathrm{r}}({\bm\omega}_{\mathrm{r}})L_{\mathrm{i}}({\bm\omega}_{\mathrm{r}})\, ,
\end{align*}
其中${\bm\omega}_{\mathrm{r}}=R({\bm\omega}_{\mathrm{o}},{\bm n})$是由${\bm\omega}_{\mathrm{o}}$关于
曲面法线$\bm n$反射的镜面反射向量(回想对于镜面反射有$\theta_{\mathrm{r}}=\theta_{\mathrm{o}}$,
因此$F_{\mathrm{r}}({\bm\omega}_{\mathrm{o}})=F_{\mathrm{r}}({\bm\omega}_{\mathrm{r}})$)。

此类BRDF可用狄拉克$\delta$分布构造。
回顾\refsec{采样理论}中$\delta$分布有个好用的性质
\begin{align}\label{eq:8.4}
    \int{f(x)\delta(x-x_0)\mathrm{d}x}=f(x_0)\, .
\end{align}
然而相比于标准函数,$\delta$分布需要特殊处理。
特别地,对有$\delta$分布的积分求数值积分必须显式考虑$\delta$分布。
考虑\refeq{8.4}中的积分:如果我们尝试用梯形法则或
其他一些数值积分技术计算它,则按$\delta$分布的定义,
在任意取值点$x_i$处$\delta(x_i)$为非零值的概率都为零。
确切地说,我们必须允许$\delta$分布自己确定取值点。
我们将在来自特殊\refvar{BxDF}{}的光传输积分以及
第\refchap{光源}的一些光源中遇到$\delta$分布。

直觉上,我们想让镜面反射BRDF在完美反射方向以外任何地方都取零,
这暗示了要用$\delta$分布。首先可能想到的是用$\delta$函数
把入射方向限制到镜面反射方向${\bm\omega}_{\mathrm{r}}$.
这样得到BRDF
\begin{align*}
    f_{\mathrm{r}}({\bm\omega}_{\mathrm{o}},{\bm\omega}_{\mathrm{i}})=\delta_{\mathrm{r}}({\bm\omega}_{\mathrm{i}}-{\bm\omega}_{\mathrm{r}})F_{\mathrm{r}}({\bm\omega}_{\mathrm{i}})\, .
\end{align*}

尽管这看起来很诱人,但把它代入散射方程\refeq{5.9}就暴露了问题:
\begin{align*}
    L_{\mathrm{o}}({\bm\omega}_{\mathrm{o}}) & =\int{\delta_{\mathrm{r}}({\bm\omega}_{\mathrm{i}}-{\bm\omega}_{\mathrm{r}})F_{\mathrm{r}}({\bm\omega}_{\mathrm{i}})}L_{\mathrm{i}}({\bm\omega}_{\mathrm{i}})|\cos\theta_{\mathrm{i}}|\mathrm{d}{\bm\omega}_{\mathrm{i}} \\
                                             & =F_{\mathrm{r}}({\bm\omega}_{\mathrm{r}})L_{\mathrm{i}}({\bm\omega}_{\mathrm{r}})|\cos\theta_{\mathrm{r}}|\, .
\end{align*}
这是错的,因为它含有额外因子$\cos\theta_{\mathrm{r}}$.
然而,我们可以把该因子分解以求得完美镜面反射正确的BRDF:
\begin{align*}
    f_\mathrm{r}({\bm p},{\bm \omega}_\mathrm{o},{\bm \omega}_\mathrm{i})=F_{\mathrm{r}}({\bm\omega}_{\mathrm{r}})\frac{\delta_{\mathrm{r}}({\bm\omega}_{\mathrm{i}}-{\bm\omega}_{\mathrm{r}})}{|\cos\theta_{\mathrm{r}|}}\, ,
\end{align*}
\begin{lstlisting}
`\refcode{BxDF Declarations}{+=}\lastnext{BxDFDeclarations}`
class `\initvar{SpecularReflection}{}` : public `\refvar{BxDF}{}` {
public:
    `\refcode{SpecularReflection Public Methods}{}`
private:
    `\refcode{SpecularReflection Private Data}{}`
};
\end{lstlisting}

\refvar{SpecularReflection}{}的构造函数接收用于缩放反射颜色的\refvar{Spectrum}{}和
描述介电质或导体菲涅尔性质的\refvar{Fresnel}{}对象指针。
\begin{lstlisting}
`\initcode{SpecularReflection Public Methods}{=}\initnext{SpecularReflectionPublicMethods}`
`\refvar{SpecularReflection}{}`(const `\refvar{Spectrum}{}` &R, `\refvar{Fresnel}{}` *fresnel) 
    : `\refvar{BxDF}{}`(`\refvar{BxDFType}{}`(`\refvar[BSDFREFLECTION]{BSDF\_REFLECTION}{}` | `\refvar[BSDFSPECULAR]{BSDF\_SPECULAR}{}`)), `\refvar[SpecularReflection::R]{R}{}`(R),
      `\refvar[SpecularReflection::fresnel]{fresnel}{}`(fresnel) { }
\end{lstlisting}
\begin{lstlisting}
`\initcode{SpecularReflection Private Data}{=}`
const `\refvar{Spectrum}{}` `\initvar[SpecularReflection::R]{R}{}`;
const `\refvar{Fresnel}{}` *`\initvar[SpecularReflection::fresnel]{fresnel}{}`;
\end{lstlisting}

剩下的实现就简单了。没有散射从\refvar{SpecularReflection::f}{()}返回,
因为对于任意一对方向,$\delta$函数不返回散射
\footnote{如果调用者碰巧传入一个向量及其完美镜像方向,该函数仍然返回零。
    尽管这些反射函数的接口有点奇怪,我们最终仍然能得到正确结果,
    因为带有$\delta$分布奇点的反射函数将得到光传输例程的特殊处理(见第\refchap{光传输I:表面反射})。}。
\begin{lstlisting}
`\refcode{SpecularReflection Public Methods}{+=}\lastnext{SpecularReflectionPublicMethods}`
`\refvar{Spectrum}{}` `\initvar[SpecularReflection::f]{f}{}`(const `\refvar{Vector3f}{}` &wo, const `\refvar{Vector3f}{}` &wi) const { 
    return `\refvar{Spectrum}{}`(0.f); 
}
\end{lstlisting}
然而,我们确实实现了方法\refvar[SpecularReflection::Samplef]{Sample\_f}{()},
它根据$\delta$分布选择合适的方向。
它把输出变量{\ttfamily wi}设为提供的方向{\ttfamily wo}关于
曲面法线的反射。值{\ttfamily *pdf}设为一;
\refsub{镜面反射与透射}讨论了关于该数值一所表示的数学量的一些细节。
\begin{lstlisting}
`\refcode{BxDF Method Definitions}{+=}\lastnext{BxDFMethodDefinitions}`
`\refvar{Spectrum}{}` `\refvar{SpecularReflection}{}`::`\initvar[SpecularReflection::Samplef]{Sample\_f}{}`(const `\refvar{Vector3f}{}` &wo,
        `\refvar{Vector3f}{}` *wi, const `\refvar{Point2f}{}` &sample, `\refvar{Float}{}` *pdf,
        `\refvar{BxDFType}{}` *sampledType) const {
    `\refcode{Compute perfect specular reflection direction}{}`
    *pdf = 1;
    return `\refvar[SpecularReflection::fresnel]{fresnel}{}`->`\refvar[Fresnel::Evaluate]{Evaluate}{}`(`\refvar{CosTheta}{}`(*wi)) * `\refvar[SpecularReflection::R]{R}{}` / `\refvar{AbsCosTheta}{}`(*wi);
}
\end{lstlisting}

期望的入射方向是${\bm\omega}_{\mathrm{o}}$关于曲面法线的反射$R({\bm\omega}_{\mathrm{o}},{\bm n})$.
用向量几何学可以非常简单地计算该方向。
首先,注意到入射方向、反射方向和曲面法线均在同一平面内。
我们可以把平面内的向量$\bm\omega$分解为两个分量之和:
一个平行于$\bm n$,我们记作${\bm\omega}_{\parallel}$,另一个垂直即${\bm\omega}_{\perp}$.

这些向量很容易计算:如果$\bm n$和$\bm\omega$规范化了,
则${\bm\omega}_{\parallel}$是$(\cos\theta){\bm n}=({\bm n}\cdot{\bm\omega}){\bm n}$(\reffig{8.7})。
因为${\bm\omega}_{\parallel}+{\bm\omega}_{\perp}={\bm\omega}$,
\begin{align*}
    {\bm\omega}_{\perp}={\bm\omega}-{\bm\omega}_{\parallel}={\bm\omega}-({\bm n}\cdot{\bm\omega}){\bm n}\, .
\end{align*}
\begin{figure}[htbp]
    \centering
    \includegraphics[width=0.4\linewidth]{Pictures/chap08/Parallelprojectionomeganormal.eps}
    \caption{向量$\bm\omega$在法线$\bm n$上的平行投影由${\bm\omega}_{\parallel}=(\cos\theta){\bm n}=({\bm n}\cdot{\bm\omega}){\bm n}$给出。
    垂直分量由${\bm\omega}_{\perp}=(\sin\theta){\bm n}$给出但用${\bm\omega}_{\perp}={\bm\omega}-{\bm\omega}_{\parallel}$计算更简单。}
    \label{fig:8.7}
\end{figure}

\reffig{8.8}展示了计算反射方向${\bm\omega}_{\mathrm{r}}$的设置。
我们可以看到两个向量都有相同的${\bm\omega}_{\parallel}$分量,
且${\bm\omega}_{\mathrm{r}\perp}$的值是${\bm\omega}_{\mathrm{o}\perp}$取反。
因此,我们有
\begin{align}
    \label{eq:8.5}
    {\bm\omega}_{\mathrm{r}}={\bm\omega}_{\mathrm{r}\perp}+{\bm\omega}_{\mathrm{r}\parallel} & =-{\bm\omega}_{\mathrm{o}\perp}+{\bm\omega}_{\mathrm{o}\parallel}\nonumber                                                        \\
                                                                                             & =-({\bm\omega}_{\mathrm{o}}-({\bm n}\cdot{\bm\omega}_{\mathrm{o}}){\bm n})+({\bm n}\cdot{\bm\omega}_{\mathrm{o}}){\bm n}\nonumber \\
                                                                                             & =-{\bm\omega}_{\mathrm{o}}+2({\bm n}\cdot{\bm\omega}_{\mathrm{o}}){\bm n}\, .
\end{align}
\begin{figure}[htbp]
    \centering
    \includegraphics[width=0.5\linewidth]{Pictures/chap08/Perfectreflectioncomponents.eps}
    \caption{因为角$\theta_{\mathrm{o}}$和$\theta_{\mathrm{r}}$相等,
    所以完美反射方向的平行分量${\bm\omega}_{\mathrm{r}\parallel}$和
    入射方向的相同:${\bm\omega}_{\mathrm{r}\parallel}={\bm\omega}_{\mathrm{o}\parallel}$.
    其垂直分量就是入射方向垂直分量取反。}
    \label{fig:8.8}
\end{figure}

函数\refvar{Reflect}{()}实现了该计算。
\begin{lstlisting}
`\refcode{BSDF Inline Functions}{+=}\lastnext{BSDFInlineFunctions}`
inline `\refvar{Vector3f}{}` `\initvar{Reflect}{}`(const `\refvar{Vector3f}{}` &wo, const `\refvar{Vector3f}{}` &n) {
    return -wo + 2 * `\refvar{Dot}{}`(wo, n) * n;
}
\end{lstlisting}

在BRDF坐标系中,${\bm n}=(0,0,1)$,该表达式可极大简化。
\begin{lstlisting}
`\initcode{Compute perfect specular reflection direction}{=}`
*wi = `\refvar{Vector3f}{}`(-wo.x, -wo.y, wo.z);
\end{lstlisting}

\subsection{镜面透射}\label{sub:镜面透射}
我们现在推导镜面透射的BTDF。斯涅尔定律是推导的基础。
斯涅尔定律不仅给出了透射光线的方向,它还能用于说明
当光线穿过折射率不同的介质时,沿该光线的辐亮度会变。

考虑入射光线到达两种介质的边界,入射和出射介质的折射率
分别为$\eta_{\mathrm{i}}$和$\eta_{\mathrm{o}}$(\reffig{8.9})
\sidenote{译者注:原图把$\eta_{\mathrm{o}}$写成$\eta_{\mathrm{t}}$,
    与正文不一致,已修改。}。
\begin{figure}[htbp]
    \centering
    \includegraphics[width=0.5\linewidth]{Pictures/chap08/Radiancechangeatrefraction.eps}
    \caption{折射率不同的介质边界处透射辐亮度由这两个折射率之比的平方缩放。
        直观上这可以理解为辐射的微分立体角被压缩或展开作为透射的结果。}
    \label{fig:8.9}
\end{figure}

我们用$\tau$来表示菲涅尔方程给出的入射能量透射到出射方向的比例,
所以$\tau=1-F_{\mathrm{r}}({\bm\omega}_{\mathrm{i}})$.
于是透射的微分通量大小为
\begin{align*}
    \mathrm{d}\varPhi_{\mathrm{o}}=\tau\mathrm{d}\varPhi_{\mathrm{i}}\, .
\end{align*}
如果我们用辐亮度的定义\refeq{5.2},则我们有
\begin{align*}
    L_{\mathrm{o}}\cos\theta_{\mathrm{o}}\mathrm{d}A\mathrm{d}{\bm\omega}_{\mathrm{o}}=\tau(L_{\mathrm{i}}\cos\theta_{\mathrm{i}}\mathrm{d}A\mathrm{d}{\bm\omega}_{\mathrm{i}})\, .
\end{align*}
把立体角展开为球面角,我们有
\begin{align}
    \label{eq:8.6}
    L_{\mathrm{o}}\cos\theta_{\mathrm{o}}\mathrm{d}A\sin\theta_{\mathrm{o}}\mathrm{d}\theta_{\mathrm{o}}\mathrm{d}\varphi_{\mathrm{o}}=\tau L_{\mathrm{i}}\cos\theta_{\mathrm{i}}\mathrm{d}A\sin\theta_{\mathrm{i}}\mathrm{d}\theta_{\mathrm{i}}\mathrm{d}\varphi_{\mathrm{i}}\, .
\end{align}

我们现在可以对斯涅尔定律求$\theta$的微分,得到关系
\begin{align*}
    \eta_{\mathrm{o}}\cos\theta_{\mathrm{o}}\mathrm{d}\theta_{\mathrm{o}}=\eta_{\mathrm{i}}\cos\theta_{\mathrm{i}}\mathrm{d}\theta_{\mathrm{i}}\, .
\end{align*}
整理各项,我们有
\begin{align*}
    \frac{\cos\theta_{\mathrm{o}}\mathrm{d}\theta_{\mathrm{o}}}{\cos\theta_{\mathrm{i}}\mathrm{d}\theta_{\mathrm{i}}}=\frac{\eta_{\mathrm{i}}}{\eta_{\mathrm{o}}}\, .
\end{align*}
把该关系和斯涅尔定律代入\refeq{8.6}并化简,我们有
\begin{align*}
    L_{\mathrm{o}}\eta_{\mathrm{i}}^2\mathrm{d}\varphi_{\mathrm{o}}=\tau L_{\mathrm{i}}\eta_{\mathrm{o}}^2\mathrm{d}\varphi_{\mathrm{i}}\, .
\end{align*}
因为$\varphi_{\mathrm{i}}=\varphi_{\mathrm{o}}+\pi$,所以$\mathrm{d}\varphi_{\mathrm{i}}=\mathrm{d}\varphi_{\mathrm{o}}$,
我们有最终关系:
\begin{align}
    \label{eq:8.7}
    L_{\mathrm{o}}=\tau L_{\mathrm{i}}\frac{\eta_{\mathrm{o}}^2}{\eta_{\mathrm{i}}^2}\, .
\end{align}
对于镜面反射的BRDF,我们需要分离一个$\cos\theta_{\mathrm{i}}$项
来得到正确的镜面透射BTDF:
\begin{align*}
    f_{\mathrm{r}}({\bm\omega}_{\mathrm{o}},{\bm\omega}_{\mathrm{i}})=\frac{\eta_{\mathrm{o}}^2}{\eta_{\mathrm{i}}^2}(1-F_{\mathrm{r}}({\bm\omega}_{\mathrm{i}}))\frac{\delta({\bm\omega}_{\mathrm{i}}-T({\bm\omega}_{\mathrm{o}},{\bm n}))}{|\cos\theta_{\mathrm{i}|}}\, ,
\end{align*}
其中$T({\bm\omega}_{\mathrm{o}},{\bm n})$是
镜面透射${\bm\omega}_{\mathrm{o}}$穿过
曲面法线为$\bm n$的界面所得的镜面透射向量。

该方程中的项$1-F_{\mathrm{r}}({\bm\omega}_{\mathrm{i}})$对应了
容易观察到的效应:接近垂直的角度透射更强。例如,如果你直接朝下看着清澈的湖水,
你会看见水下深处,但角度平视时大部分光都像是镜子反射的。

类\refvar{SpecularTransmission}{}几乎和\refvar{SpecularReflection}{}一样,
除了采样方向是完美镜面透射的方向。\reffig{8.10}展示了使用
镜面反射和透射BRDF与BTDF来模拟玻璃的龙模型图像。
\begin{figure}[htbp]
    \centering
    \includegraphics[width=\linewidth]{Pictures/chap08/dragon-glass.png}
    \caption{当用菲涅尔公式调制了介电质的镜面反射BRDF和镜面透射BTDF,
        反射和透射量随角度的真实变化给出了玻璃在视觉上的准确表现
        (感谢Christian Schüller提供模型)。}
    \label{fig:8.10}
\end{figure}

\begin{lstlisting}
`\refcode{BxDF Declarations}{+=}\lastnext{BxDFDeclarations}`
class `\initvar{SpecularTransmission}{}` : public `\refvar{BxDF}{}` {
public:
    `\refcode{SpecularTransmission Public Methods}{}`
private:
    `\refcode{SpecularTransmission Private Data}{}`
};
\end{lstlisting}

\refvar{SpecularTransmission}{}的构造函数保存了曲面两侧的折射率,
其中{\ttfamily etaA}是曲面上方的折射率(这里曲面法线所在的一侧为“上方”),
{\ttfamily etaB}是曲面下方的折射率,而{\ttfamily T}给出了透射缩放因子。
参数\refvar{TransportMode}{}表示入射光线交点的\refvar{BxDF}{}是
从光源还是相机开始计算的。该区别指出了该怎样计算\refvar{BxDF}{}的作用。
\begin{lstlisting}
`\initcode{SpecularTransmission Public Methods}{=}\initnext{SpecularTransmissionPublicMethods}`
`\refvar{SpecularTransmission}{}`(const `\refvar{Spectrum}{}` &T, `\refvar{Float}{}` etaA, `\refvar{Float}{}` etaB,
        `\refvar{TransportMode}{}` mode) 
    : `\refvar{BxDF}{}`(`\refvar{BxDFType}{}`(`\refvar[BSDFTRANSMISSION]{BSDF\_TRANSMISSION}{}` | `\refvar[BSDFSPECULAR]{BSDF\_SPECULAR}{}`)), `\refvar[SpecularTransmission::T]{T}{}`(T), `\refvar[SpecularTransmission::etaA]{etaA}{}`(etaA),
      `\refvar[SpecularTransmission::etaB]{etaB}{}`(etaB), `\refvar[SpecularTransmission::fresnel]{fresnel}{}`(etaA, etaB), `\refvar[SpecularTransmission::mode]{mode}{}`(mode) {
}
\end{lstlisting}
因为导体不透射光,所以\refvar{FresnelDielectric}{}对象
总是用于菲涅尔计算。

\begin{lstlisting}
`\initcode{SpecularTransmission Private Data}{=}`
const `\refvar{Spectrum}{}` `\initvar[SpecularTransmission::T]{T}{}`;
const `\refvar{Float}{}` `\initvar[SpecularTransmission::etaA]{etaA}{}`, `\initvar[SpecularTransmission::etaB]{etaB}{}`;
const `\refvar{FresnelDielectric}{}` `\initvar[SpecularTransmission::fresnel]{fresnel}{}`;
const `\refvar{TransportMode}{}` `\initvar[SpecularTransmission::mode]{mode}{}`;
\end{lstlisting}

对于\refvar{SpecularReflection}{},因为BTDF是缩放后的$\delta$分布,所以\linebreak
\refvar{SpecularTransmission::f}{()}总是返回零。
\begin{lstlisting}
`\refcode{SpecularTransmission Public Methods}{+=}\lastnext{SpecularTransmissionPublicMethods}`
`\refvar{Spectrum}{}` `\initvar[SpecularTransmission::f]{f}{}`(const `\refvar{Vector3f}{}` &wo, const `\refvar{Vector3f}{}` &wi) const { 
    return `\refvar{Spectrum}{}`(0.f); 
}
\end{lstlisting}

\refeq{8.7}描述了当光线从一种介质进入另一种时辐亮度是怎样变化的。
然而事实证明,尽管该缩放应该应用于从光源开始的光线,
但\emph{绝不}该用于从相机开始的光线。
\refsec{路径-空间测量方程}将更细致地讨论该问题以及到时候定义的运用该缩放的
代码片{\refcode{Account for non-symmetry with transmission to different medium}{}}。
\begin{lstlisting}
`\refcode{BxDF Method Definitions}{+=}\lastnext{BxDFMethodDefinitions}`
`\refvar{Spectrum}{}` `\refvar{SpecularTransmission}{}`::`\initvar[SpecularTransmission::Samplef]{Sample\_f}{}`(const `\refvar{Vector3f}{}` &wo,
        `\refvar{Vector3f}{}` *wi, const `\refvar{Point2f}{}` &sample, `\refvar{Float}{}` *pdf,
        `\refvar{BxDFType}{}` *sampledType) const {
    `\refcode{Figure out which $\eta$ is incident and which is transmitted}{}`
    `\refcode{Compute ray direction for specular transmission}{}`
    *pdf = 1;
    `\refvar{Spectrum}{}` ft = `\refvar[SpecularTransmission::T]{T}{}` * (`\refvar{Spectrum}{}`(1.) - `\refvar[SpecularTransmission::fresnel]{fresnel}{}`.`\refvar[FresnelDielectric::Evaluate]{Evaluate}{}`(`\refvar{CosTheta}{}`(*wi)));
    `\refcode{Account for non-symmetry with transmission to different medium}{}`
    return ft / `\refvar{AbsCosTheta}{}`(*wi);
}
\end{lstlisting}

该方法首先确定入射光线是否正进入或退出折射介质。
pbrt遵循局部反射空间中的曲面法线即$(0,0,1)$方向指向
物体外侧的惯例。因此,如果方向${\bm\omega_{\mathrm{o}}}$的$z$分量
大于零,则入射光线来自物体外侧。
\begin{lstlisting}
`\initcode{Figure out which $\eta$ is incident and which is transmitted}{=}`
bool entering = `\refvar{CosTheta}{}`(wo) > 0;
`\refvar{Float}{}` etaI = entering ? `\refvar[SpecularTransmission::etaA]{etaA}{}` : `\refvar[SpecularTransmission::etaB]{etaB}{}`;
`\refvar{Float}{}` etaT = entering ? `\refvar[SpecularTransmission::etaB]{etaB}{}` : `\refvar[SpecularTransmission::etaA]{etaA}{}`;
\end{lstlisting}
\begin{lstlisting}
`\initcode{Compute ray direction for specular transmission}{=}`
if (!`\refvar{Refract}{}`(wo, `\refvar{Faceforward}{}`(`\refvar{Normal3f}{}`(0, 0, 1), wo), etaI / etaT, wi))
    return 0;
\end{lstlisting}

为了推导给出透射方向向量的表达式,我们可以采用
类似于我们为镜面反射所用的更简单的方法。
\reffig{8.11}展示了该设置。

\begin{figure}[htbp]
    \centering
    \includegraphics[width=0.5\linewidth]{Pictures/chap08/Speculartransmissionprojections.eps}
    \caption{镜面透射的几何结构。给定入射方向${\bm\omega}_{\mathrm{i}}$和
        曲面法线$\bm n$间的夹角$\theta_{\mathrm{i}}$,镜面透射方向与
        曲面法线成角为$\theta_{\mathrm{t}}$. 该方向${\bm\omega}_{\mathrm{t}}$可
        用斯涅尔定律算得,即求其垂直分量${\bm\omega}_{\mathrm{t}\perp}$,
        再算${\bm\omega}_{\mathrm{t}\parallel}$得到规范化结果${\bm\omega}_{\mathrm{t}}$.}
    \label{fig:8.11}
\end{figure}

然而这次,我们将从垂直分量开始:如果入射向量规范化了且有
垂直分量${\bm\omega}_{\mathrm{i}\perp}$,则我们从三角学
以及${\bm\omega}_{\perp}$的定义中知道${\bm\omega}_{\mathrm{i}\perp}$的长度
等于$\sin\theta_{\mathrm{i}}$. 斯涅尔定律告诉
我们$\displaystyle\sin\theta_{\mathrm{t}}=\frac{\eta_{\mathrm{i}}}{\eta_{\mathrm{t}}}\sin\theta_{\mathrm{i}}$.
让方向${\bm\omega}_{\mathrm{i}\perp}$取反并相应调整长度,我们有
\begin{align*}
    {\bm\omega}_{\mathrm{t}\perp}=\frac{\eta_{\mathrm{i}}}{\eta_{\mathrm{t}}}(-{\bm\omega}_{\mathrm{i}\perp})\, .
\end{align*}
等价地,因为${\bm\omega}_{\perp}={\bm\omega}-{\bm\omega}_{\parallel}$,
\begin{align*}
    {\bm\omega}_{\mathrm{t}\perp}=\frac{\eta_{\mathrm{i}}}{\eta_{\mathrm{t}}}(-{\bm\omega}_{\mathrm{i}}+({\bm\omega}_{\mathrm{i}}\cdot{\bm n}){\bm n})\, .
\end{align*}

现在对于${\bm\omega}_{\mathrm{t}\parallel}$:我们
知道${\bm\omega}_{\mathrm{t}\parallel}$平行于$\bm n$但面向相反方向,
且我们还知道${\bm\omega}_{\mathrm{t}}$应该规范化。把这些合起来,
\begin{align*}
    {\bm\omega}_{\mathrm{t}\parallel}=-\left(\sqrt{1-\|{\bm\omega}_{\mathrm{t}\perp}\|^2}\right){\bm n}\, .
\end{align*}
于是完整的向量${\bm\omega}_{\mathrm{t}}$是
\begin{align*}
    {\bm\omega}_{\mathrm{t}}={\bm\omega}_{\mathrm{t}\perp}+{\bm\omega}_{\mathrm{t}\parallel}=\frac{\eta_{\mathrm{i}}}{\eta_{\mathrm{t}}}(-{\bm\omega}_{\mathrm{i}})+\left[\frac{\eta_{\mathrm{i}}}{\eta_{\mathrm{t}}}({\bm\omega}_{\mathrm{i}}\cdot{\bm n})-\sqrt{1-\|{\bm\omega}_{\mathrm{t}\perp}\|^2}\right]{\bm n}\, .
\end{align*}
因为$\|{\bm\omega}_{\mathrm{t}\perp}\|=\sin\theta_{\mathrm{t}}$,
平方根下的项为$1-\sin^2\theta_{\mathrm{t}}=\cos^2\theta_{\mathrm{t}}$,
这给出了最终结果:
\begin{align}
    \label{eq:8.8}
    {\bm\omega}_{\mathrm{t}}=\frac{\eta_{\mathrm{i}}}{\eta_{\mathrm{t}}}(-{\bm\omega}_{\mathrm{i}})+\left[\frac{\eta_{\mathrm{i}}}{\eta_{\mathrm{t}}}({\bm\omega}_{\mathrm{i}}\cdot{\bm n})-\cos\theta_{\mathrm{t}}\right]{\bm n}\, .
\end{align}

分别给定入射方向{\ttfamily wi}、和{\ttfamily wi}在同一半球的
曲面法线{\ttfamily n}以及入射和透射介质折射率之比{\ttfamily eta},
函数\refvar{Refract}{()}计算折射方向{\ttfamily wt}。
布尔返回值表示{\ttfamily *wt}中是否返回了有效的折射光线;
全内反射情况下它取{\ttfamily false}。
\begin{lstlisting}
`\refcode{BSDF Inline Functions}{+=}\lastnext{BSDFInlineFunctions}`
inline bool `\initvar{Refract}{}`(const `\refvar{Vector3f}{}` &wi, const `\refvar{Normal3f}{}` &n, `\refvar{Float}{}` eta,
        `\refvar{Vector3f}{}` *wt) {
    `\refcode{Compute cos $\theta_t$ using Snell's law}{}`
    *wt = eta * -wi + (eta * cosThetaI - cosThetaT) * `\refvar{Vector3f}{}`(n);
    return true;
}
\end{lstlisting}

让斯涅尔定律两边平方能让我们计算$\cos\theta_{\mathrm{t}}$:
\begin{align*}
    \eta_{\mathrm{i}}^2\sin^2\theta_{\mathrm{i}} & =\eta_{\mathrm{t}}^2\sin^2\theta_{\mathrm{t}}\, ,                                      \\
    \sin^2\theta_{\mathrm{t}}                    & =\frac{\eta_{\mathrm{i}}^2}{\eta_{\mathrm{t}}^2}\sin^2\theta_{\mathrm{i}}\, ,          \\
    1-\cos^2\theta_{\mathrm{t}}                  & =\frac{\eta_{\mathrm{i}}^2}{\eta_{\mathrm{t}}^2}\sin^2\theta_{\mathrm{i}}\, ,          \\
    \cos\theta_{\mathrm{t}}                      & =\sqrt{1-\frac{\eta_{\mathrm{i}}^2}{\eta_{\mathrm{t}}^2}\sin^2\theta_{\mathrm{i}}}\, .
\end{align*}

\begin{lstlisting}
`\initcode{Compute cos $\theta_t$ using Snell's law}{=}`
`\refvar{Float}{}` cosThetaI = `\refvar{Dot}{}`(n, wi);
`\refvar{Float}{}` sin2ThetaI = std::max(0.f, 1.f - cosThetaI * cosThetaI);
`\refvar{Float}{}` sin2ThetaT = eta * eta * sin2ThetaI;
`\refcode{Handle total internal reflection for transmission}{}`
`\refvar{Float}{}` cosThetaT = std::sqrt(1 - sin2ThetaT);
\end{lstlisting}

我们这里也需要处理全内反射的情况。
如果$\sin\theta_{\mathrm{t}}$的平方值大于或等于一,
就发生了全内反射,所以返回{\ttfamily false}
\footnote{第一版pbrt在这里测试的“>1”而不是“>=1”。
    尽管两者的区别看起来无害,但该差异使得偶尔会算出not-a-number值,
    因为$\omega$(在曲面切平面上)的$z$分量为零,进而项$\frac{1}{\cos\theta}$为无穷。}。
\begin{lstlisting}
`\initcode{Handle total internal reflection for transmission}{=}`
if (sin2ThetaT >= 1) return false;
\end{lstlisting}

\subsection{菲涅尔调制的镜面反射与透射}\label{sub:菲涅尔调制的镜面反射与透射}
为了后续第\refchap{光传输I:表面反射}、
\refchap{光传输II:体积渲染}、\refchap{光传输III:双向方法}的
一些蒙特卡罗光传输算法有更优的效率,拥有能同时表示
镜面反射与镜面透射的单个\refvar{BxDF}{}很有用,
其中散射类型的相对权重由介电质菲涅尔方程调制。
\refvar{FresnelSpecular}{}中提供了这样的\refvar{BxDF}{}。
\begin{lstlisting}
`\refcode{BxDF Declarations}{+=}\lastnext{BxDFDeclarations}`
class `\initvar{FresnelSpecular}{}` : public `\refvar{BxDF}{}` {
public:
    `\refcode{FresnelSpecular Public Methods}{}`
private:
    `\refcode{FresnelSpecular Private Data}{}`
};
\end{lstlisting}
\begin{lstlisting}
`\initcode{FresnelSpecular Public Methods}{=}\initnext{FresnelSpecularPublicMethods}`
`\refvar{FresnelSpecular}{}`(const `\refvar{Spectrum}{}` &R, const `\refvar{Spectrum}{}` &T, `\refvar{Float}{}` etaA,
        `\refvar{Float}{}` etaB, `\refvar{TransportMode}{}` mode) 
    : `\refvar{BxDF}{}`(`\refvar{BxDFType}{}`(`\refvar[BSDFREFLECTION]{BSDF\_REFLECTION}{}` | `\refvar[BSDFTRANSMISSION]{BSDF\_TRANSMISSION}{}` | `\refvar[BSDFSPECULAR]{BSDF\_SPECULAR}{}`)),
      `\refvar[FresnelSpecular::R]{R}{}`(R), `\refvar[FresnelSpecular::T]{T}{}`(T), `\refvar[FresnelSpecular::etaA]{etaA}{}`(etaA), `\refvar[FresnelSpecular::etaB]{etaB}{}`(etaB), `\refvar[FresnelSpecular::fresnel]{fresnel}{}`(etaA, etaB),
      `\refvar[FresnelSpecular::mode]{mode}{}`(mode) { }
\end{lstlisting}

因为我们只关注介电质情形,所以\refvar{FresnelDielectric}{}对象
总是只用于菲涅尔计算。

\begin{lstlisting}
`\initcode{FresnelSpecular Private Data}{=}`
const `\refvar{Spectrum}{}` `\initvar[FresnelSpecular::R]{R}{}`, `\initvar[FresnelSpecular::T]{T}{}`;
const `\refvar{Float}{}` `\initvar[FresnelSpecular::etaA]{etaA}{}`, `\initvar[FresnelSpecular::etaB]{etaB}{}`;
const `\refvar{FresnelDielectric}{}` `\initvar[FresnelSpecular::fresnel]{fresnel}{}`;
const `\refvar{TransportMode}{}` `\initvar[FresnelSpecular::mode]{mode}{}`
\end{lstlisting}
\begin{lstlisting}
`\refcode{FresnelSpecular Public Methods}{+=}\lastcode{FresnelSpecularPublicMethods}`
`\refvar{Spectrum}{}` `\initvar[FresnelSpecular::f]{f}{}`(const `\refvar{Vector3f}{}` &wo, const `\refvar{Vector3f}{}` &wi) const { 
    return `\refvar{Spectrum}{}`(0.f); 
}
\end{lstlisting}

因为一些实现细节依赖于第\refchap{蒙特卡罗积分}介绍的蒙特卡罗积分原理,
方法\refvar[FresnelSpecular::Samplef]{Sample\_f}{()}
的实现在\refsub{镜面反射与透射}。

\input{content/chap0803.tex}

\section{微面模型}\label{sec:微面模型}

许多建模表面反射和透射的基于几何光学的方法
都基于一个观点即粗糙表面可以建模为
小的\keyindex{微面}{microfacet}{}合集。
由微面构成的曲面经常建模为高度场\sidenote{译者注:原文heightfield。},
其中微面的朝向分布是作统计上的描述。
\reffig{8.12}展示了相对粗糙表面的横截面和光滑得多的微面表面。
当区别不明显时,我们将用术语\keyindex{微曲面}{microsurface}{}来
描述微面曲面,用\keyindex{宏曲面}{macrosurface}{}来描述基本的光滑曲面
(即由\refvar{Shape}{}表示的)。
\begin{figure}[htbp]
    \centering
    \includegraphics[width=0.75\linewidth]{Pictures/chap08/Roughsmoothmicrofacets.eps}
    \caption{微面曲面模型经常由给出微面法线${\bm n}_{\mathrm{f}}$关于
        曲面法线$\bm n$的分布的函数来描述。(a)微面法线变化越大,表面越粗糙。
        (b)光滑表面的微面法线变化相对较小。}
    \label{fig:8.12}
\end{figure}

基于微面的BRDF模型通过统计上对来自大量微面的光的散射建模来奏效。
如果我们假设被照明的微分区域$\mathrm{d}A$比起单个微面的尺寸相对大些,
则有大量微面被照亮且它们合起来的表现决定了观察到的散射。

微面模型的两个主要构成是\keyindex{刻面}{facet}{}
\sidenote{译者注:暂未查到facet在图形学中的专门翻译。}分布的表示
和描述光如何从单个微面散射的BRDF。有了这些,
任务是推导BRDF的解析表达式以描述来自这类曲面的散射。
尽管镜面透射对于建模许多透明材料很有用,
但对于微面BRDF,完美镜面反射是最常用的,
(下节介绍的)Oren-Nayar模型把微面看作朗伯反射体。

为了计算来自这类模型的反射,需要考虑微面级的局部光效应(\reffig{8.13})。
微面可能被另一刻面遮挡,可能处在相邻微面的阴影下,
或者\keyindex{互反射}{interreflection}{}可能
造成微面反射出比直接照明量和低层级微面BRDF预计的还多的光。
基于微面的特定BRDF模型以各种精确程度考虑了这些效应的每一种。
一般方法是做出尽可能最好的近似,且仍得到计算简单的表达式。
\begin{figure}[htbp]
    \centering
    \includegraphics[width=\linewidth]{Pictures/chap08/Maskingshadowinginterreflection.eps}
    \caption{微面反射模型要考虑的三种重要几何效应。(a)遮掩(masking):
        考虑的微面因为另一微面的遮挡而对观察者不可见。(b)阴影(shadowing):
        类似地,光无法到达微面。(3)互反射:光在到达
        观察者之前于微面之间反弹。}
    \label{fig:8.13}
\end{figure}

\subsection{Oren-Nayar漫反射}\label{sub:Oren-Nayar漫反射}
\citet{10.1145/192161.192213}观察到真实世界物体不会展现完美朗伯反射。
具体地,当照明方向接近观察方向时,粗糙曲面通常显得更亮。
他们用含有单个参数$\sigma$即微面朝向角度标准差的球面高斯分布
描述的V形微面构建了描述粗糙曲面的反射模型。

在V形假设下,可以通过只考虑相邻微面来处理互反射;
\citeauthor{10.1145/192161.192213}利用这点
推导出对凹槽\sidenote{译者注:原文groove。}集的整体反射建模的BRDF。

所得模型考虑了微面间的阴影、遮掩和互反射,但没有解析解,
所以他们发现下列近似拟合得很好:
\begin{align*}
    f_{\mathrm{r}}({\bm\omega}_{\mathrm{i}},{\bm\omega}_{\mathrm{o}})=\frac{R}{\pi}(A+B\max(0,\cos(\varphi_{\mathrm{i}}-\varphi_{\mathrm{o}}))\sin\alpha\tan\beta)\, ,
\end{align*}
其中如果$\sigma$单位是弧度,则
\begin{align*}
    A      & =1-\frac{\sigma^2}{2(\sigma^2+0.33)}\, ,           \\
    B      & =\frac{0.45\sigma^2}{\sigma^2+0.09}\, ,            \\
    \alpha & =\max(\theta_{\mathrm{i}},\theta_{\mathrm{o}})\, , \\
    \beta  & =\min(\theta_{\mathrm{i}},\theta_{\mathrm{o}})\, .
\end{align*}

\begin{figure}[htbp]
    \centering
    \subfloat[朗伯]{\includegraphics[width=0.7\linewidth]{Pictures/chap08/f8-14a.png}\label{fig:8.14.1}}\\
    \subfloat[Oren-Nayar]{\includegraphics[width=0.7\linewidth]{Pictures/chap08/f8-14b.png}\label{fig:8.14.2}}
    \caption{(a)来自\refvar{LambertianReflection}{}模型的标准漫反射和
        (b)$\sigma$参数为20度的\refvar{OrenNayar}{}模型渲染的龙模型。
        注意用Oren-Nayar模型时暗色轮廓边缘反射增加了,浅色明暗边缘通常也更干脆
        (感谢Christian Schüller提供的模型)。}
    \label{fig:8.14}
\end{figure}

这里的实现在构造函数中预先计算和保存参数$A$与$B$的值
以节约稍后计算BRDF的工作量。\reffig{8.14}比较了理想漫反射和Oren-Nayar模型渲染的区别。
\begin{lstlisting}
`\initcode{OrenNayar Public Methods}{=}`
`\initvar{OrenNayar}{}`(const `\refvar{Spectrum}{}` &R, `\refvar{Float}{}` sigma) 
    : `\refvar{BxDF}{}`(`\refvar{BxDFType}{}`(`\refvar[BSDFREFLECTION]{BSDF\_REFLECTION}{}` | `\refvar[BSDFDIFFUSE]{BSDF\_DIFFUSE}{}`)), `\refvar[OrenNayar::R]{R}{}`(R) {
    sigma = `\refvar{Radians}{}`(sigma);
    `\refvar{Float}{}` sigma2 = sigma * sigma;
    `\refvar[OrenNayar::A]{A}{}` = 1.f - (sigma2 / (2.f * (sigma2 + 0.33f)));
    `\refvar[OrenNayar::B]{B}{}` = 0.45f * sigma2 / (sigma2 + 0.09f);
}
\end{lstlisting}
\begin{lstlisting}
`\initcode{OrenNayar Private Data}{=}`
const `\refvar{Spectrum}{}` `\initvar[OrenNayar::R]{R}{}`;
`\refvar{Float}{}` `\initvar[OrenNayar::A]{A}{}`, `\initvar[OrenNayar::B]{B}{}`;
\end{lstlisting}

比起直接变换基本方程,三角恒等式的应用可以极大提升求值例程的效率。
实现从计算$\sin\theta_{\mathrm{i}}$和$\sin\theta_{\mathrm{o}}$的值开始。
\begin{lstlisting}
`\refcode{BxDF Method Definitions}{+=}\lastnext{BxDFMethodDefinitions}`
`\refvar{Spectrum}{}` `\refvar{OrenNayar}{}`::`\initvar[OrenNayar::f]{f}{}`(const `\refvar{Vector3f}{}` &wo, const `\refvar{Vector3f}{}` &wi) const {
    `\refvar{Float}{}` sinThetaI = `\refvar{SinTheta}{}`(wi);
    `\refvar{Float}{}` sinThetaO = `\refvar{SinTheta}{}`(wo);
    `\refcode{Compute cosine term of Oren-Nayar model}{}`
    `\refcode{Compute sine and tangent terms of Oren-Nayar model}{}`
    return `\refvar[OrenNayar::R]{R}{}` * `\refvar{InvPi}{}` * (`\refvar[OrenNayar::A]{A}{}` + `\refvar[OrenNayar::B]{B}{}` * maxCos * sinAlpha * tanBeta);
}
\end{lstlisting}

为了计算项$\max(0,\cos(\varphi_{\mathrm{i}}-\varphi_{\mathrm{o}}))$,
我们可以应用三角恒等式
\begin{align*}
    \cos(a-b)=\cos a\cos b+\sin a\sin b\, ,
\end{align*}
这样我们只需要计算$\varphi_{\mathrm{i}}$和$\varphi_{\mathrm{o}}$的正弦和余弦。
\begin{lstlisting}
`\initcode{Compute cosine term of Oren-Nayar model}{=}`
`\refvar{Float}{}` maxCos = 0;
if (sinThetaI > 1e-4 && sinThetaO > 1e-4) {
    `\refvar{Float}{}` sinPhiI = `\refvar{SinPhi}{}`(wi), cosPhiI = `\refvar{CosPhi}{}`(wi);
    `\refvar{Float}{}` sinPhiO = `\refvar{SinPhi}{}`(wo), cosPhiO = `\refvar{CosPhi}{}`(wo);
    `\refvar{Float}{}` dCos = cosPhiI * cosPhiO + sinPhiI * sinPhiO;
    maxCos = std::max((`\refvar{Float}{}`)0, dCos);
}
\end{lstlisting}

最后,求得项$\sin\alpha$和$\tan\beta$.
注意无论${\bm\omega}_{\mathrm{i}}$或${\bm\omega}_{\mathrm{o}}$中的哪一个,
有更大的$\cos\theta$(即更大的$z$分量)就有更小的$\theta$.
我们可以用该方法开头计算的近似正弦值来设定$\sin\alpha$.
然后用恒等式$\displaystyle\tan\theta=\frac{\sin\theta}{\cos\theta}$计算正切。
\begin{lstlisting}
`\initcode{Compute sine and tangent terms of Oren-Nayar model}{=}`
`\refvar{Float}{}` sinAlpha, tanBeta;
if (`\refvar{AbsCosTheta}{}`(wi) > `\refvar{AbsCosTheta}{}`(wo)) {
    sinAlpha = sinThetaO;
    tanBeta = sinThetaI / `\refvar{AbsCosTheta}{}`(wi);
} else {
    sinAlpha = sinThetaI;
    tanBeta = sinThetaO / `\refvar{AbsCosTheta}{}`(wo);
}
\end{lstlisting}

\subsection{微面分布函数}\label{sub:微面分布函数}
反射模型所基于的微面展现出完美镜面反射和透射时,
其对来自各种光泽材料的光散射的建模已经很高效了,包括金属、塑料和磨砂玻璃。
在我们讨论这些模型的辐射度量细节前,
我们将首先介绍概括了其几何属性的抽象。
这里的代码包括了两个广泛使用的微面模型的实现。
这些代码都在文件\href{https://github.com/mmp/pbrt-v3/tree/master/src/core/microfacet.h}{\ttfamily core/microfacet.h}
和\href{https://github.com/mmp/pbrt-v3/tree/master/src/core/microfacet.cpp}{\ttfamily core/microfacet.cpp}中。

\refvar{MicrofacetDistribution}{}定义了
由微面实现提供的接口及其常用功能。
\begin{lstlisting}
`\initcode{MicrofacetDistribution Declarations}{=}\initnext{MicrofacetDistributionDeclarations}`
class `\initvar{MicrofacetDistribution}{}` {
public:
    `\refcode{MicrofacetDistribution Public Methods}{}`
protected:
    `\refcode{MicrofacetDistribution Protected Methods}{}`
    `\refcode{MicrofacetDistribution Protected Data}{}`
};
\end{lstlisting}

微面曲面的一个重要特性由分布函数$D({\bm\omega}_{\mathrm{h}})$表示,
它给出了曲面法线为${\bm\omega}_{\mathrm{h}}$的微面的微分面积
(回想\reffig{8.12}展示了曲面的粗糙度是怎样和微面法线分布函数相联系的)。
在pbrt中,微面分布函数和\refvar{BxDF}{}在相同的BSDF坐标系下定义;
这样,完全光滑的曲面可由仅当${\bm\omega}_{\mathrm{h}}$等于
曲面法线时才取非零值的$\delta$分布描述:$D({\bm\omega}_{\mathrm{h}})=\delta({\bm\omega}_{\mathrm{h}}-(0,0,1))$.

微面分布函数必须规范化以保证它们的物理合理性
\sidenote{译者注:除了下文的归一化约束,$D({\bm\omega}_{\mathrm{h}})$还有非负性。}。
直观上,如果我们考虑微曲面上沿法线方向$\bm n$的入射光线,
则每条光线一定和微面曲面恰好相交一次。更形式化地,
给定微曲面的微分面积$\mathrm{d}A$,则该区域之上微面的投影面积
一定等于$\mathrm{d}A$(\reffig{8.15})
\sidenote{译者注:该公式积分区域取半空间并不严谨,实际上应取全空间才对,否则会漏掉背面朝向的微面面积。
    有关$D({\bm\omega}_{\mathrm{h}})$的详细说明可参见笔者编写的\refsec{译者补充:微面模型相关推导}。}。
数学上,这对应着以下条件
\footnote{规范化微面分布的常见错误是在整个立体角上
    而不是投影立体角上执行该积分(即略去的项$\cos\theta_{\mathrm{h}}$),
    这无法保证具有正确分布的高度场存在。}:
\begin{align*}
    \int\limits_{H^2({\bm n})}D({\bm\omega}_{\mathrm{h}})\cos\theta_{\mathrm{h}}\mathrm{d}{\bm\omega}_{\mathrm{h}}=1\, .
\end{align*}
\begin{figure}[htbp]
    \centering
    \includegraphics[width=0.5\linewidth]{Pictures/chap08/MicrofacetnormalizedA.eps}
    \caption{给定曲面上的微分面积$\mathrm{d}A$,
        则微面法线分布函数$D({\bm\omega}_{\mathrm{h}})$必须规范化使得
        该区域上的微面的投影曲面面积等于$\mathrm{d}A$.}
    \label{fig:8.15}
\end{figure}

方法\refvar{MicrofacetDistribution::D}{()}对应函数$D({\bm\omega}_{\mathrm{h}})$;
实现返回具有给定法线向量$\omega$朝向的微面的微分面积。
\begin{lstlisting}
`\initcode{MicrofacetDistribution Public Methods}{=}\initnext{MicrofacetDistributionPublicMethods}`
virtual `\refvar{Float}{}` `\initvar[MicrofacetDistribution::D]{D}{}`(const `\refvar{Vector3f}{}` &wh) const = 0;
\end{lstlisting}

\citet{1987BeckmannSpizzichino}\sidenote{译者注:
    笔者只查到了1987年同名同作者书籍,原文引用为1963年。}提出了
一种广泛使用的基于微面斜率高斯分布的微面分布函数;
我们的实现在类\refvar{BeckmannDistribution}{}中。
\begin{lstlisting}
`\refcode{MicrofacetDistribution Declarations}{+=}\lastnext{MicrofacetDistributionDeclarations}`
class `\initvar{BeckmannDistribution}{}` : public `\refvar{MicrofacetDistribution}{}` {
public:
    `\refcode{BeckmannDistribution Public Methods}{}`
private:
    `\refcode{BeckmannDistribution Private Methods}{}`
    `\refcode{BeckmannDistribution Private Data}{}`
};
\end{lstlisting}

Beckmann-Spizzichino模型的传统定义是
\begin{align}\label{eq:8.9}
    D({\bm\omega}_{\mathrm{h}})=\frac{\mathrm{e}^{-\frac{\tan^2\theta_{\mathrm{h}}}{\alpha^2}}}{\pi\alpha^2\cos^4\theta_{\mathrm{h}}}\, ,
\end{align}
其中如果$\sigma$是微面斜率的均方根\sidenote{译者注:原文RMS。},则$\alpha=\sqrt{2}\sigma$.

定义各向异性分布很有用,即法线分布也随${\bm\omega}_{\mathrm{h}}$的
方位朝向变化。例如,所朝方向垂直于$x$轴的微面对应记为$\alpha_x$,
而$y$轴的为$\alpha_y$,则中间朝向的$\alpha$值可用通过这些值构造的椭圆来插值。

对应的各向异性微面分布函数为
\sidenote{译者注:\refsec{译者补充:微面模型相关推导},
    给出了本节微面分布模型满足规范性的证明。}
\begin{align}\label{eq:8.10}
    D({\bm\omega}_{\mathrm{h}})=\frac{\mathrm{e}^{-\left(\frac{\cos^2\varphi_{\mathrm{h}}}{\alpha_x^2}+\frac{\sin^2\varphi_{\mathrm{h}}}{\alpha_y^2}\right)\tan^2\theta_{\mathrm{h}}}}{\pi\alpha_x\alpha_y\cos^4\theta_{\mathrm{h}}}\, .
\end{align}
注意当$\alpha_x=\alpha_y$时Beckmann-Spizzichino模型
变为原始的各向同性版本。

成员变量\refvar[BeckmannDistribution::alphax]{alphax}{}和
\refvar[BeckmannDistribution::alphay]{alphay}{}都在
\refvar{BeckmannDistribution}{}的构造函数中设定,
它很简单所以这里不再介绍。
\begin{lstlisting}
`\initcode{BeckmannDistribution Private Data}{=}`
const `\refvar{Float}{}` `\initvar[BeckmannDistribution::alphax]{alphax}{}`, `\initvar[BeckmannDistribution::alphay]{alphay}{}`;
\end{lstlisting}

方法\refvar{BeckmannDistribution::D}{()}直接翻译了\refeq{8.10}。
仅有的额外实现细节是必须特殊处理$\tan^2\theta$的无限值。
该情况实际上是有效的——它在完全扫掠的方向上发生。
该情况下,下面的代码最终企图计算$\displaystyle\frac{0}{0}$,
这会得到“not a number”(NaN)值,最终导致为当前像素样本的辐射亮度返回NaN值。
因此,为该情况明确地返回零,也即当$\tan^2\theta_{\mathrm{h}}$趋向
无穷大时$D({\bm\omega}_{\mathrm{h}})$收敛的值。
\begin{lstlisting}
`\initcode{MicrofacetDistribution Method Definitions}{=}\initnext{MicrofacetDistributionMethodDefinitions}`
`\refvar{Float}{}` `\refvar{BeckmannDistribution}{}`::`\initvar[BeckmannDistribution::D]{D}{}`(const `\refvar{Vector3f}{}` &wh) const {
    `\refvar{Float}{}` tan2Theta = `\refvar{Tan2Theta}{}`(wh);
    if (std::isinf(tan2Theta)) return 0.;
    `\refvar{Float}{}` cos4Theta = `\refvar{Cos2Theta}{}`(wh) * `\refvar{Cos2Theta}{}`(wh);
    return std::exp(-tan2Theta * (`\refvar{Cos2Phi}{}`(wh) / (`\refvar[BeckmannDistribution::alphax]{alphax}{}` * `\refvar[BeckmannDistribution::alphax]{alphax}{}`) +
                                  `\refvar{Sin2Phi}{}`(wh) / (`\refvar[BeckmannDistribution::alphay]{alphay}{}` * `\refvar[BeckmannDistribution::alphay]{alphay}{}`))) /
        (`\refvar{Pi}{}` * `\refvar[BeckmannDistribution::alphax]{alphax}{}` * `\refvar[BeckmannDistribution::alphay]{alphay}{}` * cos4Theta);
}
\end{lstlisting}

\citet{Trowbridge:75}提出了另一个有用的微面分布函数
\footnote{该模型也由\citet{10.5555/2383847.2383874}独立推导出,称作“GGX”。}。
其各向异性变体由下式给出:
\begin{align}\label{eq:8.11}
    D({\bm\omega}_{\mathrm{h}})=\frac{1}{\pi\alpha_x\alpha_y\left(1+\left(\frac{\cos^2\varphi_{\mathrm{h}}}{\alpha_x^2}+\frac{\sin^2\varphi_{\mathrm{h}}}{\alpha_y^2}\right)\tan^2\theta_{\mathrm{h}}\right)^2\cos^4\theta_{\mathrm{h}}}\, .
\end{align}

相比于Beckmann-Spizzichino模型,Trowbridge-Reitz模型有更高的拖尾——
在远离曲面法线的方向上它会更慢地降到零。该特性与许多真实世界表面的性质吻合得很好。
见\reffig{8.16}中这两个微面分布函数的图示。
\begin{figure}[htbp]
    \centering
    \includegraphics[width=0.75\linewidth]{Pictures/chap08/beckmann-vs-tr-tails.eps}
    \caption{当$\alpha=0.5$时各向同性的Beckmann-Spizzichino和Trowbridge-Reitz微面分布函数
        关于$\theta$的函数图像。注意Trowbridge-Reitz对于更大量级的$\theta$有更高的尾部。}
    \label{fig:8.16}
\end{figure}

\begin{lstlisting}
`\refcode{MicrofacetDistribution Declarations}{+=}\lastnext{MicrofacetDistributionDeclarations}`
class `\initvar{TrowbridgeReitzDistribution}{}` : public `\refvar{MicrofacetDistribution}{}` {
public:
    `\refcode{TrowbridgeReitzDistribution Public Methods}{}`
private:
    `\refcode{TrowbridgeReitzDistribution Private Methods}{}`
    `\refcode{TrowbridgeReitzDistribution Private Data}{}`
};
\end{lstlisting}

比起直接指定$\alpha$的值,用一个$[0,1]$中的标量参数来
指定BRDF的粗糙度会很方便,其中接近零的值对应几乎完美的镜面反射。
这里略去的方法\refvar{RoughnessToAlpha}{()}执行从该粗糙度值到$\alpha$值的映射。
\begin{lstlisting}
`\initcode{TrowbridgeReitzDistribution Public Methods}{=}`
static inline `\refvar{Float}{}` `\initvar{RoughnessToAlpha}{}`(`\refvar{Float}{}` roughness);
\end{lstlisting}
\begin{lstlisting}
`\initcode{TrowbridgeReitzDistribution Private Data}{=}`
const `\refvar{Float}{}` `\initvar[TrowbridgeReitzDistribution::alphax]{alphax}{}`, `\initvar[TrowbridgeReitzDistribution::alphay]{alphay}{}`;
\end{lstlisting}

方法\refvar[TrowbridgeReitzDistribution::D]{D}{()}是直接照着\refeq{8.11}写的。
\begin{lstlisting}
`\refcode{MicrofacetDistribution Method Definitions}{+=}\lastnext{MicrofacetDistributionMethodDefinitions}`
`\refvar{Float}{}` `\refvar{TrowbridgeReitzDistribution}{}`::`\initvar[TrowbridgeReitzDistribution::D]{D}{}`(const `\refvar{Vector3f}{}` &wh) const {
    `\refvar{Float}{}` tan2Theta = `\refvar{Tan2Theta}{}`(wh);
    if (std::isinf(tan2Theta)) return 0.;
    const `\refvar{Float}{}` cos4Theta = `\refvar{Cos2Theta}{}`(wh) * `\refvar{Cos2Theta}{}`(wh);
    `\refvar{Float}{}` e = (`\refvar{Cos2Phi}{}`(wh) / (`\refvar[TrowbridgeReitzDistribution::alphax]{alphax}{}` * `\refvar[TrowbridgeReitzDistribution::alphax]{alphax}{}`) +
               `\refvar{Sin2Phi}{}`(wh) / (`\refvar[TrowbridgeReitzDistribution::alphay]{alphay}{}` * `\refvar[TrowbridgeReitzDistribution::alphay]{alphay}{}`)) * tan2Theta;
    return 1 / (`\refvar{Pi}{}` * `\refvar[TrowbridgeReitzDistribution::alphax]{alphax}{}` * `\refvar[TrowbridgeReitzDistribution::alphay]{alphay}{}` * cos4Theta * (1 + e) * (1 + e));
}
\end{lstlisting}

\subsection{掩模和遮挡}\label{sub:掩模和遮挡}
对于渲染,只有微面法线分布不足以完全表征微曲面。
同样很重要的是,要考虑到一些微面从某给定视角或光照方向看去
会因为它们是背面朝向而不可见(因而其他微面在它们的前方),
以及一些正面朝向的微面区域会因为受到背面朝向微面的遮挡而被隐藏。
Smith的\keyindex{掩模遮挡函数}{masking-shadowing function}{}
$G_1({\bm\omega},{\bm\omega}_{\mathrm{h}})$考虑了这些效应,
给出了从方向$\bm\omega$可见且法线为${\bm\omega}_{\mathrm{h}}$的微面比例
(注意$0\le G_1({\bm\omega},{\bm\omega}_{\mathrm{h}})\le 1$)。
通常情况下微面可见的概率独立于其朝向${\bm\omega}_{\mathrm{h}}$,
我们可以把该函数写作$G_1({\bm\omega})$.

如\reffig{8.17}所示,从和曲面法线夹角为$\theta$的方向$\bm\omega$上
去观察曲面上的微分面积$\mathrm{d}A$时看到的面积为$\mathrm{d}A\cos\theta$.
从该方向可见的微面面积也一定等于$\mathrm{d}A\cos\theta$,
由此导出对$G_1$的规范化约束:
\begin{align}
    \label{eq:8.12}
    \cos\theta=\int\limits_{H^2({\bm n})}G_1({\bm\omega},{\bm\omega}_{\mathrm{h}})\max(0,{\bm\omega}\cdot{\bm\omega}_{\mathrm{h}})D({\bm\omega}_{\mathrm{h}})\mathrm{d}{\bm\omega}_{\mathrm{h}}\, .
\end{align}
换句话说,对给定方向$\bm\omega$可见的微面的投影面积一定等于
宏曲面微分面积$\mathrm{d}A$的$({\bm\omega}\cdot{\bm n})=\cos\theta$倍。

\begin{figure}[htbp]
    \centering
    \includegraphics[width=0.5\linewidth]{Pictures/chap08/Microfacetvisiblearea.eps}
    \caption{从观察者或光源处看时,曲面上的微分面积变为$\mathrm{d}A\cos\theta$,
        其中$\cos\theta$是入射方向与曲面法线夹角的余弦。
        可见微面(粗线)的投影曲面面积也一定等于$\mathrm{d}A\cos\theta$;
        掩模遮挡函数$G_1$给出了$\mathrm{d}A$上的微面总面积中在给定方向可见的比例。}
    \label{fig:8.17}
\end{figure}

因为微面构成了高度场,所以每个背向微面遮住的正向微面面积
都等于它在方向$\bm\omega$的投影面积。
\refeq{8.12}中,如果$A^{+}({\bm\omega})$是从方向$\bm\omega$看到的
正向微面投影面积,而$A^{-}({\bm\omega})$是背向微面的投影面积,
则$\cos\theta=A^{+}({\bm\omega})-A^{-}({\bm\omega})$.
因此我们可以把掩模遮挡函数改写为可见微面面积与正向微面面积之比:
\begin{align*}
    G_1({\bm\omega})=\frac{A^{+}({\bm\omega})-A^{-}({\bm\omega})}{A^{+}({\bm\omega})}\, .
\end{align*}

掩模遮挡函数习惯上用一个辅助函数$\Lambda({\bm\omega})$来表示,
后者度量了单位可见微面面积内被遮挡不可见的微面面积。
\begin{align}
    \label{eq:8.13}
    \Lambda({\bm\omega})=\frac{A^{-}({\bm\omega})}{A^{+}({\bm\omega})-A^{-}({\bm\omega})}=\frac{A^{-}({\bm\omega})}{\cos\theta}\, .
\end{align}

方法\refvar[MicrofacetDistribution::Lambda]{Lambda}{()}计算该函数。对于每个微面分布它都有特定实现。
\begin{lstlisting}
`\refcode{MicrofacetDistribution Public Methods}{+=}\lastnext{MicrofacetDistributionPublicMethods}`
virtual `\refvar{Float}{}` `\initvar[MicrofacetDistribution::Lambda]{Lambda}{}`(const `\refvar{Vector3f}{}` &w) const = 0;
\end{lstlisting}

我们通过一些代数变换用$\Lambda({\bm\omega})$表示$G_1({\bm\omega})$:
\begin{align*}
    G_1({\bm\omega})=\frac{1}{1+\Lambda({\bm\omega})}\, ,
\end{align*}
因此我们利用\refvar[MicrofacetDistribution::Lambda]{Lambda}{()}来
提供方法\refvar[MicrofacetDistribution::G1]{G1}{()}。
\begin{lstlisting}
`\refcode{MicrofacetDistribution Public Methods}{+=}\lastnext{MicrofacetDistributionPublicMethods}`
`\refvar{Float}{}` `\initvar[MicrofacetDistribution::G1]{G1}{}`(const `\refvar{Vector3f}{}` &w) const {
    return 1 / (1 + `\refvar[MicrofacetDistribution::Lambda]{Lambda}{}`(w));
}
\end{lstlisting}

只是微面分布还不能施加足够的条件以明确一个特定的$\Lambda({\bm\omega})$函数;
许多函数都能满足\refeq{8.12}中的约束。
例如,如果我们假设微面上邻近点的高度之间没有关联,
则可能为给定的$D({\bm\omega}_{\mathrm{h}})$找到唯一的$\Lambda({\bm\omega})$
(对于许多微面模型来说都能找到解析解)。
尽管该基本假设在现实中不成立——对于实际的微面,
一点的高度通常接近邻近点的高度——但所得函数$\Lambda({\bm\omega})$与
从实际表面测量到的情况相比已经很精确了。

在邻近点高度无关的假设下,各向同性的Beckmann-Spizzichino分布的$\Lambda({\bm\omega})$是
\begin{align}
    \Lambda({\bm\omega})=\frac{1}{2}\left(\mathrm{erf}(a)-1+\frac{\mathrm{e}^{-a^2}}{a\sqrt{\pi}}\right)\, ,
\end{align}
其中$a=\displaystyle\frac{1}{\alpha\tan\theta}$,而$\mathrm{erf}$是
误差函数,$\displaystyle\mathrm{erf}(x)=\frac{2}{\sqrt{\pi}}\int_0^x\mathrm{e}^{-x'^2}\mathrm{d}x'$.


{\noindent\hfil$=========$\hfil{\color{red}{施工分割线}}\hfil$=========$\

\section{译者补充:微面模型相关推导}\label{sec:译者补充:微面模型相关推导}
\begin{remark}
    本节内容不是原书内容,而是译者根据\citet{heitz:hal-01024289}和\citet{10.5555/2383847.2383874}整理
    并推导后补充的,请酌情参考和斧正。
\end{remark}
\subsection{微面分布函数的定义与性质}\label{sub:微面分布函数的定义与性质}
如\reffig{08ex01-macrosurfaceMicrosurface},我们考虑一个足够小的宏曲面$\mathcal{G}$,设它是个绝对光滑的平面,具有法线$\bm n$.
微面模型中,真正粗糙起伏的曲面,即微曲面$\mathcal{M}$,是由许多偏离宏曲面的微面构成的。
或者说,微曲面$\mathcal{M}$上所有的点在方向$\bm n$上投影即得$\mathcal{G}$.
将宏曲面上的点记为${\bm p}_{\mathrm{g}}$,其周围的微分面元为$\mathrm{d}{\bm p}_{\mathrm{g}}$,
则宏曲面的面积为
\begin{align}
    S=\int\limits_{\mathcal{G}}\mathrm{d}{\bm p}_{\mathrm{g}}\, .
\end{align}

将$\mathcal{M}$上的点记作${\bm p}_{\mathrm{h}}$,该点处的法线记作${\bm\omega}_{\mathrm{h}}({\bm p}_{\mathrm{h}})$.
引入三维意义下的狄拉克$\delta$分布,
它满足$\displaystyle\int\limits_{\varOmega}\delta({\bm\omega})\mathrm{d}{\bm\omega}=1$.
并记$\delta_{\bm\omega}({\bm\omega}')=\delta({\bm\omega}'-{\bm\omega})$.
由此定义\keyindex{微面分布函数}{microfacet distribution function}{distribution分布}为
\sidenote{原论文全文假定$S=1$来讨论,不用除以面积,这里笔者的处理有所不同。}
\begin{align}\label{eq:08ex01-MicrosurfaceDistribution}
    D({\bm\omega})=\displaystyle\frac{1}{S}\int\limits_{\mathcal{M}}
    \delta_{\bm\omega}({\bm\omega}_{\mathrm{h}}({\bm p}_{\mathrm{h}}))
    \mathrm{d}{\bm p}_{\mathrm{h}}\, ,
\end{align}
其中狄拉克$\delta$分布和$D({\bm\omega}_{\mathrm{h}})$的量纲均为
球面度的倒数即$\displaystyle\frac{1}{\text{sr}}$,也相当于无量纲。

\begin{figure}[htbp]
    \centering
    \includegraphics[width=0.6\linewidth]{Pictures/chap08/macrosurfaceMicrosurface.eps}
    \caption{宏曲面(黑色)与微曲面(红色)。}
    \label{fig:08ex01-macrosurfaceMicrosurface}
\end{figure}

我们可以把${\bm\omega}_{\mathrm{h}}$视作从$\mathcal{M}$到整个方向空间$\varOmega$的映射。
注意$\varOmega$包含了球心到完整球面上任意一点的所有可能方向(总立体角为$4\pi$,尽管该映射不一定能覆盖全)。
现在考虑$\mathcal{M}$和$\varOmega$各自的子集$\mathcal{M'}$和$\varOmega'$,
并设它们满足以下条件:点${\bm p}_{\mathrm{h}}$属于$\mathcal{M'}$当且仅当
该点处的微面法线${\bm\omega}_{\mathrm{h}}({\bm p}_{\mathrm{h}})$属于$\varOmega'$,即
\begin{align}
    {\bm p}_{\mathrm{h}}\in\mathcal{M'}\Leftrightarrow
    {\bm\omega}_{\mathrm{h}}({\bm p}_{\mathrm{h}})\in\varOmega'\, .
\end{align}
由此利用积分换元可得微面分布函数具有计算指定微曲面面积的能力:
\begin{align}
    \label{eq:08ex01-microsurfaceArea}
    \int\limits_{\mathcal{M}'}\mathrm{d}{\bm p}_{\mathrm{h}}
    =S\int\limits_{\varOmega'}D({\bm\omega}_{\mathrm{h}})\mathrm{d}{\bm\omega}_{\mathrm{h}}\, .
\end{align}
而整个微曲面面积就是
\begin{align}
    \int\limits_{\mathcal{M}}\mathrm{d}{\bm p}_{\mathrm{h}}
    =S\int\limits_{\varOmega}D({\bm\omega}_{\mathrm{h}})\mathrm{d}{\bm\omega}_{\mathrm{h}}\, .
\end{align}

进一步地,对于任意关于微面法线的函数$f({\bm\omega}_{\mathrm{h}})$,
利用$D({\bm\omega}_{\mathrm{h}})$可将空间积分与统计积分相互转化:
\begin{align}
    \int\limits_{\mathcal{M}}f({\bm\omega}_{\mathrm{h}}({\bm p}_{\mathrm{h}}))\mathrm{d}{\bm p}_{\mathrm{h}}
    =S\int\limits_{\varOmega}f({\bm\omega}_{\mathrm{h}})
    D({\bm\omega}_{\mathrm{h}})\mathrm{d}{\bm\omega}_{\mathrm{h}}\, .
\end{align}

反之,对于任意定义在$\mathcal{M}$上的函数$g({\bm p}_{\mathrm{h}})$,
我们可以定义对应的统计函数$g({\bm\omega}_{\mathrm{h}})$为
\sidenote{这里原论文为了在记号上强调二者的联系,仍然使用了符号$g$,
    但读者应明白这是一个新的函数了。后面\refeq{08ex01-StaticMaskFunc}等的情况类似。}:
\begin{align}\label{eq:08ex01-StaticFunc}
    g({\bm\omega})=\frac{\displaystyle\int\limits_{\mathcal{M}}
    \delta_{\bm\omega}({\bm\omega}_{\mathrm{h}}({\bm p}_{\mathrm{h}}))
    g({\bm p}_{\mathrm{h}})\mathrm{d}{\bm p}_{\mathrm{h}}}
    {\displaystyle\int\limits_{\mathcal{M}}
    \delta_{\bm\omega}({\bm\omega}_{\mathrm{h}}({\bm p}_{\mathrm{h}}))
    \mathrm{d}{\bm p}_{\mathrm{h}}}\, .
\end{align}
该函数也可以实现空间积分与统计积分的相互转化:
\begin{align}\label{eq:08ex01-TransferSpaceStatic}
    \int\limits_{\mathcal{M}}g({\bm p}_{\mathrm{h}})\mathrm{d}{\bm p}_{\mathrm{h}}
    =S\int\limits_{\varOmega}g({\bm\omega}_{\mathrm{h}})
    D({\bm\omega}_{\mathrm{h}})\mathrm{d}{\bm\omega}_{\mathrm{h}}\, .
\end{align}

如\reffig{08ex01-ProjectionsMicrofacetArea},理解以上推导后,我们可以计算以下面积。
\begin{figure}[htbp]
    \centering
    \includegraphics[width=0.5\linewidth]{Pictures/chap08/ProjectionsMicrofacet.eps}
    \caption{微面可见部分在${\bm\omega}_{\mathrm{o}}$上的投影面积。}
    \label{fig:08ex01-ProjectionsMicrofacetArea}
\end{figure}

首先,微曲面在宏曲面法线方向${\bm\omega}_{\mathrm{g}}$上的
投影面积(不论是否遮挡,且背向时记负值)为
\begin{align}
    \int\limits_{\mathcal{M}}({\bm\omega}_{\mathrm{h}}({\bm p}_{\mathrm{h}})
    \cdot{\bm\omega}_{\mathrm{g}})\mathrm{d}{\bm p}_{\mathrm{h}}
    =S\int\limits_{\varOmega}({\bm\omega}_{\mathrm{h}}\cdot{\bm\omega}_{\mathrm{g}})
    D({\bm\omega}_{\mathrm{h}})\mathrm{d}{\bm\omega}_{\mathrm{h}}
    =\int\limits_{\mathcal{G}}\mathrm{d}{\bm p}_{\mathrm{g}}=S\, .
\end{align}
注意上式同时给出了一个$D({\bm\omega}_{\mathrm{h}})$应满足的重要约束:
\begin{align}\label{eq:08ex01-McrofacetDistributionNormalization}
    \int\limits_{\varOmega}({\bm\omega}_{\mathrm{h}}\cdot{\bm\omega}_{\mathrm{g}})
    D({\bm\omega}_{\mathrm{h}})\mathrm{d}{\bm\omega}_{\mathrm{h}}=1\, .
\end{align}

接着我们从出射方向${\bm\omega}_{\mathrm{o}}$来观察曲面。此时宏曲面在${\bm\omega}_{\mathrm{o}}$上的投影面积为
\begin{align}
    \label{eq:08ex01-AreaMacrosurface}
    ({\bm\omega}_{\mathrm{o}}\cdot{\bm\omega}_{\mathrm{g}})S=S\cos\theta_{\mathrm{o}}\, ,
\end{align}
其中$\theta_{\mathrm{o}}$为${\bm\omega}_{\mathrm{o}}$与${\bm\omega}_{\mathrm{g}}$的夹角。

最后我们算得微面可见部分在${\bm\omega}_{\mathrm{o}}$上的投影面积为
\begin{align}\label{eq:08ex01-AreaProjectionsMicrofacetVisible}
    \int\limits_{\mathcal{M}}G_1({\bm p}_{\mathrm{h}},{\bm\omega}_{\mathrm{o}})
    \max({\bm\omega}_{\mathrm{h}}({\bm p}_{\mathrm{h}})\cdot{\bm\omega}_{\mathrm{o}},0)
    \mathrm{d}{\bm p}_{\mathrm{h}}\, ,
\end{align}
其中\keyindex{空间掩模函数}{spatial masking function}{masking function掩模函数}
$G_1({\bm p}_{\mathrm{h}},{\bm\omega}_{\mathrm{o}})$在${\bm p}_{\mathrm{h}}$被
遮挡时取0,否则取1. 而$\max$项则过滤了背向不可见的微面。
其对应的\keyindex{统计掩模函数}{statistical masking function}{masking function掩模函数}
$G_1({\bm\omega}_{\mathrm{h}},{\bm\omega}_{\mathrm{o}})$的值域为$[0,1]$,
它给出了从方向${\bm\omega}_{\mathrm{o}}$观察时,法线为${\bm\omega}_{\mathrm{h}}$的微面中可见的比例:
\begin{align}\label{eq:08ex01-StaticMaskFunc}
    G_1({\bm\omega},{\bm\omega}_{\mathrm{o}})
    =\frac{\displaystyle\int\limits_{\mathcal{M}}
    \delta_{\bm\omega}({\bm\omega}_{\mathrm{h}}({\bm p}_{\mathrm{h}}))
    G_1({\bm p}_{\mathrm{h}},{\bm\omega}_{\mathrm{o}})\mathrm{d}{\bm p}_{\mathrm{h}}}
    {\displaystyle\int\limits_{\mathcal{M}}
    \delta_{\bm\omega}({\bm\omega}_{\mathrm{h}}({\bm p}_{\mathrm{h}}))
    \mathrm{d}{\bm p}_{\mathrm{h}}}\, .
\end{align}
由此得到投影面积的另一计算方式:
\begin{align}
    \label{eq:08ex01-AreaMicrosurface}
    S\int\limits_{\varOmega}G_1({\bm\omega}_{\mathrm{h}},{\bm\omega}_{\mathrm{o}})
    \max({\bm\omega}_{\mathrm{h}}\cdot{\bm\omega}_{\mathrm{o}},0)
    D({\bm\omega}_{\mathrm{h}})\mathrm{d}{\bm\omega}_{\mathrm{h}}\, .
\end{align}

由于可见微面投影面积等于宏曲面投影面积,所以
结合\refeq{08ex01-AreaMacrosurface}和\refeq{08ex01-AreaMicrosurface}可得
基于物理的掩模函数$G_1$总是满足如下约束:
\begin{align}\label{eq:08ex01-CosThetaO}
    \cos\theta_{\mathrm{o}}=\int\limits_{\varOmega}
    G_1({\bm\omega}_{\mathrm{h}},{\bm\omega}_{\mathrm{o}})
    \max({\bm\omega}_{\mathrm{h}}\cdot{\bm\omega}_{\mathrm{o}},0)
    D({\bm\omega}_{\mathrm{h}})\mathrm{d}{\bm\omega}_{\mathrm{h}}\, .
\end{align}
注意这并不意味着该约束唯一确定了$G_1$,它常常有无数个解。
还需引入其他约束或假设才能限定为唯一解。
\reffig{08ex01-SameDistributionOfNormalsDifferentBRDFs}给出了这种不唯一性的例子。
\begin{figure}[htbp]
    \centering
    \includegraphics[width=0.75\linewidth]{Pictures/chap08/SameDistributionOfNormalsDifferentBRDFs.eps}
    \caption{具有相同微面分布函数$D({\bm\omega}_{\mathrm{h}})$但BRDF却不同的两种微面。}
    \label{fig:08ex01-SameDistributionOfNormalsDifferentBRDFs}
\end{figure}

此外,译者再补充两个\refeq{08ex01-StaticMaskFunc}可能让人感到困惑的地方:

第一个是记号的问题。$G_1({\bm\omega}_{\mathrm{h}},{\bm\omega}_{\mathrm{o}})$中
第一个自变量是${\bm\omega}_{\mathrm{h}}$,该变量应出现在其定义式中狄拉克$\delta$分布的下标,
但狄拉克$\delta$分布后面的括号内还有一个记号相同的函数${\bm\omega}_{\mathrm{h}}({\bm p}_{\mathrm{h}})$,
所以为了区分它们,\refeq{08ex01-StaticMaskFunc}中临时把第一个自变量改写为${\bm\omega}$.
笔者保留了原论文的这个做法,请读者注意区分。\refeq{08ex01-MicrosurfaceDistribution}、
\refeq{08ex01-StaticFunc}、\refeq{08ex01-AnotherStaticFunc}、\refeq{08-ex01-masking-g1-int}和
\refeq{08ex01-VCavityScatteringNormalDistribution}等也用了类似的临时记号。

第二个是定义的问题。细心的读者可能注意到,
比对\refeq{08ex01-AreaProjectionsMicrofacetVisible}和\refeq{08ex01-TransferSpaceStatic},
我们本该设
\begin{align}\label{eq:08ex01-AnotherSpaceFunc}
    g({\bm p}_{\mathrm{h}})=G_1({\bm p}_{\mathrm{h}},{\bm\omega}_{\mathrm{o}})
    \max({\bm\omega}_{\mathrm{h}}({\bm p}_{\mathrm{h}})\cdot{\bm\omega}_{\mathrm{o}},0)\, ,
\end{align}
此时使得\refeq{08ex01-TransferSpaceStatic}成立的$g({\bm\omega}_{\mathrm{h}})$应该是
把\refeq{08ex01-AnotherSpaceFunc}带入\refeq{08ex01-StaticFunc}得到的
\begin{align}\label{eq:08ex01-AnotherStaticFunc}
    g({\bm\omega})=\frac{\displaystyle\int\limits_{\mathcal{M}}
    \delta_{\bm\omega}({\bm\omega}_{\mathrm{h}}({\bm p}_{\mathrm{h}}))
    G_1({\bm p}_{\mathrm{h}},{\bm\omega}_{\mathrm{o}})
    \max({\bm\omega}_{\mathrm{h}}({\bm p}_{\mathrm{h}})\cdot{\bm\omega}_{\mathrm{o}},0)
    \mathrm{d}{\bm p}_{\mathrm{h}}}
    {\displaystyle\int\limits_{\mathcal{M}}
    \delta_{\bm\omega}({\bm\omega}_{\mathrm{h}}({\bm p}_{\mathrm{h}}))
    \mathrm{d}{\bm p}_{\mathrm{h}}}\, .
\end{align}
将上式回代\refeq{08ex01-TransferSpaceStatic}的右边后,
读者会发现它和\refeq{08ex01-AreaMicrosurface}并不是完全一样的——
后者相当于把前者内层积分中的项$\max({\bm\omega}_{\mathrm{h}}({\bm p}_{\mathrm{h}})\cdot{\bm\omega}_{\mathrm{o}},0)$
(里面的${\bm\omega}_{\mathrm{h}}$是函数)提到外层积分中
变成了$\max({\bm\omega}_{\mathrm{h}}\cdot{\bm\omega}_{\mathrm{o}},0)$
(里面的${\bm\omega}_{\mathrm{h}}$是变量)。
一般来说积分变量并不能随便外提,但因为这里内层积分中含有狄拉克$\delta$分布,
它使得此处外提$\max$项的做法恰好没有改变整个式子的值。
所以原论文中作出简化处理,相当于令$g({\bm p}_{\mathrm{h}})=G_1({\bm p}_{\mathrm{h}},{\bm\omega}_{\mathrm{o}})$,
此时对应的$g({\bm\omega}_{\mathrm{h}})$即为\refeq{08ex01-StaticMaskFunc}中
的$G_1({\bm\omega}_{\mathrm{h}},{\bm\omega}_{\mathrm{o}})$.
不过原作者并未交代这样的细节。后面\refeq{08ex01-RadianceMicrofacet}也用了类似技巧。

\subsection{基于微面模型的BRDF}\label{sub:基于微面模型的BRDF}
本节继承上节的记号。在微面尺度上,我们设微面上一点朝出射方向${\bm\omega}_{\mathrm{o}}$的辐亮度
为$L_{\mathcal{M}}({\bm\omega}_{\mathrm{h}}({\bm p}_{\mathrm{h}}),{\bm\omega}_{\mathrm{o}})$.
则从宏观尺度看,该微面整体朝${\bm\omega}_{\mathrm{o}}$等价的出射
辐亮度$L_{\mathrm{o}}({\bm\omega}_{\mathrm{o}})$即为
微面尺度的辐亮度按出射方向可见投影面积比例的加权:
\begin{align}\label{eq:08ex01-RadianceMicrofacetAverageSum}
    L_{\mathrm{o}}({\bm\omega}_{\mathrm{o}})
    =\frac{\displaystyle\int\limits_{\mathcal{M}}
    {L_{\mathcal{M}}({\bm\omega}_{\mathrm{h}}({\bm p}_{\mathrm{h}}),{\bm\omega}_{\mathrm{o}})
    G_1({\bm p}_{\mathrm{h}},{\bm\omega}_{\mathrm{o}})
    \max({\bm\omega}_{\mathrm{h}}({\bm p}_{\mathrm{h}})\cdot{\bm\omega}_{\mathrm{o}},0)
    \mathrm{d}{\bm p}_{\mathrm{h}}}}
    {\displaystyle\int\limits_{\mathcal{M}}
    {G_1({\bm p}_{\mathrm{h}},{\bm\omega}_{\mathrm{o}})
    \max({\bm\omega}_{\mathrm{h}}({\bm p}_{\mathrm{h}})\cdot{\bm\omega}_{\mathrm{o}},0)
    \mathrm{d}{\bm p}_{\mathrm{h}}}}\, .
\end{align}
上式利用空间掩模函数\refeq{08ex01-StaticMaskFunc}转化为统计积分形式,
并将\refeq{08ex01-AreaProjectionsMicrofacetVisible}带入分母可得:
\begin{align}\label{eq:08ex01-RadianceMicrofacet}
    L_{\mathrm{o}}({\bm\omega}_{\mathrm{o}})
    =\frac{1}{\cos\theta_{\mathrm{o}}}\int\limits_{\varOmega}
    L_{\mathcal{M}}({\bm\omega}_{\mathrm{h}},{\bm\omega}_{\mathrm{o}})
    G_1({\bm\omega}_{\mathrm{h}},{\bm\omega}_{\mathrm{o}})
    \max({\bm\omega}_{\mathrm{h}}\cdot{\bm\omega}_{\mathrm{o}},0)
    D({\bm\omega}_{\mathrm{h}})\mathrm{d}{\bm\omega}_{\mathrm{h}}\, .
\end{align}
观察上式,我们把被积分项中对$L_{\mathcal{M}}({\bm\omega}_{\mathrm{h}},{\bm\omega}_{\mathrm{o}})$加权的系数
定义为\keyindex{可见法线分布}{distribution of visible normals}{distribution分布}:
\begin{align}\label{eq:08ex01-DistributionOfVisibleNormals}
    D_{{\bm\omega}_{\mathrm{o}}}({\bm\omega}_{\mathrm{h}})
    =\frac{G_1({\bm\omega}_{\mathrm{h}},{\bm\omega}_{\mathrm{o}})
        \max({\bm\omega}_{\mathrm{h}}\cdot{\bm\omega}_{\mathrm{o}},0)
        D({\bm\omega}_{\mathrm{h}})}{\cos\theta_{\mathrm{o}}}\, .
\end{align}
其中$\cos\theta_{\mathrm{o}}$如\refeq{08ex01-AreaMacrosurface}所述
即${\bm\omega}_{\mathrm{o}}\cdot{\bm\omega}_{\mathrm{g}}$,
于是\refeq{08ex01-RadianceMicrofacet}可以表示为:
\begin{align}\label{eq:08ex01-RadianceMacroOut}
    L_{\mathrm{o}}({\bm\omega}_{\mathrm{o}})
    =\int\limits_{\varOmega}L_{\mathcal{M}}({\bm\omega}_{\mathrm{h}},{\bm\omega}_{\mathrm{o}})
    D_{{\bm\omega}_{\mathrm{o}}}({\bm\omega}_{\mathrm{h}})\mathrm{d}{\bm\omega}_{\mathrm{h}}\, .
\end{align}
同时应注意到该定义下$D_{{\bm\omega}_{\mathrm{o}}}({\bm\omega}_{\mathrm{h}})$满足规范化性质:
\begin{align}\label{eq:08ex01-VisibleDistributionNormalization}
    \int\limits_{\varOmega}D_{{\bm\omega}_{\mathrm{o}}}({\bm\omega}_{\mathrm{h}})
    \mathrm{d}{\bm\omega}_{\mathrm{h}}=1\, .
\end{align}
若再结合\refeq{08ex01-CosThetaO},则$L_{\mathrm{o}}({\bm\omega}_{\mathrm{o}})$还可以表示为以下形式:
\begin{align}\label{eq:08ex01-RadianceMacroOutV2}
    L_{\mathrm{o}}({\bm\omega}_{\mathrm{o}})
    =\frac{\displaystyle\int\limits_{\varOmega}L_{\mathcal{M}}({\bm\omega}_{\mathrm{h}},{\bm\omega}_{\mathrm{o}})
    G_1({\bm\omega}_{\mathrm{h}},{\bm\omega}_{\mathrm{o}})
    \max({\bm\omega}_{\mathrm{h}}\cdot{\bm\omega}_{\mathrm{o}},0)
    D({\bm\omega}_{\mathrm{h}})\mathrm{d}{\bm\omega}_{\mathrm{h}}}
    {\displaystyle\int\limits_{\varOmega}G_1({\bm\omega}_{\mathrm{h}},{\bm\omega}_{\mathrm{o}})
    \max({\bm\omega}_{\mathrm{h}}\cdot{\bm\omega}_{\mathrm{o}},0)
    D({\bm\omega}_{\mathrm{h}})\mathrm{d}{\bm\omega}_{\mathrm{h}}}\, .
\end{align}

接下来我们利用微面尺度上的BRDF推导微面模型在宏观尺度下的BRDF。
把微面尺度上的BRDF记作$f_{\mathcal{M}}({\bm\omega}_{\mathrm{h}},{\bm\omega}_{\mathrm{o}},{\bm\omega}_{\mathrm{i}})$,
考虑到有效的角度范围,则根据\refeq{5.8}中BRDF的定义有
\begin{align}\label{eq:08ex01-MicrosurfaceBRDF}
    f_{\mathcal{M}}({\bm\omega}_{\mathrm{h}},{\bm\omega}_{\mathrm{o}},{\bm\omega}_{\mathrm{i}})
    =\frac{\mathrm{d} L_{\mathcal{M}}({\bm\omega}_{\mathrm{h}},{\bm\omega}_{\mathrm{o}})}
    {\max({\bm\omega}_{\mathrm{h}}\cdot{\bm\omega}_{\mathrm{i}},0)
    L_{\mathrm{i}}({\bm\omega}_{\mathrm{i}})\mathrm{d}{\bm\omega}_{\mathrm{i}}}\, ,
\end{align}
其中$L_{\mathrm{i}}({\bm\omega}_{\mathrm{i}})$是来自入射方向${\bm\omega}_{\mathrm{i}}$的辐亮度。
由此我们根据定义并结合\refeq{08ex01-RadianceMacroOut}、\refeq{08ex01-MicrosurfaceBRDF}、
\refeq{08ex01-DistributionOfVisibleNormals}得到宏观尺度下的BRDF为:
\begin{align}\label{eq:08ex01-MacroBRDFG1}
    f_{\mathrm{r}}({\bm\omega}_{\mathrm{o}},{\bm\omega}_{\mathrm{i}})
     & = \frac{\mathrm{d} L_{\mathrm{o}}({\bm\omega}_{\mathrm{o}})}
    {|{\bm\omega}_{\mathrm{g}}\cdot{\bm\omega}_{\mathrm{i}}|
    L_{\mathrm{i}}({\bm\omega}_{\mathrm{i}})\mathrm{d}{\bm\omega}_{\mathrm{i}}}
    = \frac{1}{|{\bm\omega}_{\mathrm{g}}\cdot{\bm\omega}_{\mathrm{i}}|}
    \int\limits_{\varOmega}\frac{\mathrm{d}L_{\mathcal{M}}({\bm\omega}_{\mathrm{h}},{\bm\omega}_{\mathrm{o}})}
    {L_{\mathrm{i}}({\bm\omega}_{\mathrm{i}})\mathrm{d}{\bm\omega}_{\mathrm{i}}}
    D_{{\bm\omega}_{\mathrm{o}}}({\bm\omega}_{\mathrm{h}})\mathrm{d}{\bm\omega}_{\mathrm{h}}\nonumber \\
     & = \frac{1}{|{\bm\omega}_{\mathrm{g}}\cdot{\bm\omega}_{\mathrm{i}}|}
    \int\limits_{\varOmega}f_{\mathcal{M}}({\bm\omega}_{\mathrm{h}},{\bm\omega}_{\mathrm{o}},{\bm\omega}_{\mathrm{i}})
    \max({\bm\omega}_{\mathrm{h}}\cdot{\bm\omega}_{\mathrm{i}},0)
    D_{{\bm\omega}_{\mathrm{o}}}({\bm\omega}_{\mathrm{h}})\mathrm{d}{\bm\omega}_{\mathrm{h}}\nonumber \\
     & = \frac{1}{|{\bm\omega}_{\mathrm{g}}\cdot{\bm\omega}_{\mathrm{i}}|
    |{\bm\omega}_{\mathrm{g}}\cdot{\bm\omega}_{\mathrm{o}}|}
    \int\limits_{\varOmega}f_{\mathcal{M}}({\bm\omega}_{\mathrm{h}},{\bm\omega}_{\mathrm{o}},{\bm\omega}_{\mathrm{i}})
    \max({\bm\omega}_{\mathrm{h}}\cdot{\bm\omega}_{\mathrm{i}},0)\nonumber                            \\
     & \qquad\qquad\max({\bm\omega}_{\mathrm{h}}\cdot{\bm\omega}_{\mathrm{o}},0)
    G_1({\bm\omega}_{\mathrm{h}},{\bm\omega}_{\mathrm{o}})
    D({\bm\omega}_{\mathrm{h}})\mathrm{d}{\bm\omega}_{\mathrm{h}}\, .
\end{align}

我们应注意到,上式考虑的其实是入射光在曲面上经历第一次反射后刚刚离开的情形,
并未考虑到反射的光线可能还会再命中曲面一次甚至多次然后从其他方向射出的情况。
然而我们在宏观尺度下的BRDF是定义在单次反射的情形下的,所以应排除掉这些光线。
为此,我们引入\keyindex{遮挡函数}{shadowing function}{}来到达这一效果。
实际中,我们通常直接使用联合的\keyindex{掩模遮挡函数}{masking-shadowing function}{}$G_2$来代替$G_1$,
它计入的微面比例要求在${\bm\omega}_{\mathrm{o}}$和${\bm\omega}_{\mathrm{i}}$上双向可见,此时
\begin{align}\label{eq:08ex01-MacroBRDFG2}
    f_{\mathrm{r}}({\bm\omega}_{\mathrm{o}},{\bm\omega}_{\mathrm{i}})
     & =\frac{1}{|{\bm\omega}_{\mathrm{g}}\cdot{\bm\omega}_{\mathrm{i}}||{\bm\omega}_{\mathrm{g}}\cdot{\bm\omega}_{\mathrm{o}}|}
    \int\limits_{\varOmega}f_{\mathcal{M}}({\bm\omega}_{\mathrm{h}},{\bm\omega}_{\mathrm{o}},{\bm\omega}_{\mathrm{i}})
    \max({\bm\omega}_{\mathrm{h}}\cdot{\bm\omega}_{\mathrm{i}},0)\nonumber                                                       \\
     & \qquad\qquad\max({\bm\omega}_{\mathrm{h}}\cdot{\bm\omega}_{\mathrm{o}},0)
    G_2({\bm\omega}_{\mathrm{h}},{\bm\omega}_{\mathrm{o}},{\bm\omega}_{\mathrm{i}})
    D({\bm\omega}_{\mathrm{h}})\mathrm{d}{\bm\omega}_{\mathrm{h}}\, .
\end{align}

最后,笔者认为对于透射的情况,将\refeq{08ex01-RadianceMicrofacetAverageSum}、
\refeq{08ex01-RadianceMicrofacet}、
\refeq{08ex01-DistributionOfVisibleNormals}、
\refeq{08ex01-RadianceMacroOutV2}、\refeq{08ex01-MicrosurfaceBRDF}、
\refeq{08ex01-MacroBRDFG1}、\refeq{08ex01-MacroBRDFG2}
中的$\max(\cdot,0)$项均改为对应的绝对值$|\cdot|$即可。

\subsection{镜面微面模型的BRDF}\label{sub:镜面微面模型的BRDF}
本节继承上节的记号。为了算出微面模型在宏观尺度下的BRDF,
即\refeq{08ex01-MacroBRDFG1}或\refeq{08ex01-MacroBRDFG2},
我们需要知道微面尺度下的BRDF即$f_{\mathcal{M}}({\bm\omega}_{\mathrm{h}},{\bm\omega}_{\mathrm{o}},{\bm\omega}_{\mathrm{i}})$的具体形式。
本小节假设微面都遵循完美镜面反射,由此给出具体推导。
设微面镜面的菲涅尔反射率为$F_{\mathcal{M}}({\bm\omega}_{\mathrm{h}},{\bm\omega}_{\mathrm{o}})$,
这意味着微面的入射和出射辐亮度满足约束
\begin{align}\label{eq:08ex01-FresnelMicrofacet}
    L_{\mathcal{M}}({\bm\omega}_{\mathrm{h}},{\bm\omega}_{\mathrm{o}})=F_{\mathcal{M}}({\bm\omega}_{\mathrm{h}},{\bm\omega}_{\mathrm{o}})L_{\mathrm{i}}({\bm\omega}_{\mathrm{i}})\, .
\end{align}
考虑到微面遵循完美镜面反射,只有当入射方向、出射方向、法线三者满足
反射定律\ref{theorem:0607-LawOfReflection}描述的情形时,
反射才实际成立,也即上式才成立,其余情况则不成立。
所以$f_{\mathcal{M}}({\bm\omega}_{\mathrm{h}},{\bm\omega}_{\mathrm{o}},{\bm\omega}_{\mathrm{i}})$应
含有狄拉克$\delta$分布来区分这两种情况。因此我们构造出
\begin{align}\label{eq:08ex01-FresnelBRDFMicrofacet}
    f_{\mathcal{M}}({\bm\omega}_{\mathrm{h}},{\bm\omega}_{\mathrm{o}},{\bm\omega}_{\mathrm{i}})
    =F_{\mathcal{M}}({\bm\omega}_{\mathrm{h}},{\bm\omega}_{\mathrm{o}})\frac{\delta_{{\bm\omega}_{\mathrm{r}}}({\bm\omega}_{\mathrm{i}})}{|{\bm\omega}_{\mathrm{h}}\cdot{\bm\omega}_{\mathrm{i}}|}\, ,
\end{align}
其中${\bm\omega}_{\mathrm{r}}$是由${\bm\omega}_{\mathrm{h}}$和${\bm\omega}_{\mathrm{o}}$确定的
满足完美镜面反射的规范化入射方向,即意味着它是
关于${\bm\omega}_{\mathrm{h}}$和${\bm\omega}_{\mathrm{o}}$的函数。
我们可以验证这个构造在完美镜面反射成立
(即${\bm\omega}_{\mathrm{r}}={\bm\omega}_{\mathrm{i}}$)时是符合\refeq{08ex01-FresnelMicrofacet}的:
联立\refeq{08ex01-MicrosurfaceBRDF},在有效的入射方向范围${\varOmega}_{\mathrm{i}}$内积分可得
\begin{align}\label{eq:08ex01-FresnelBRDFValid}
    L_{\mathcal{M}}({\bm\omega}_{\mathrm{h}},{\bm\omega}_{\mathrm{o}})
     & =\int\limits_{{\varOmega}_{\mathrm{i}}}f_{\mathcal{M}}({\bm\omega}_{\mathrm{h}},{\bm\omega}_{\mathrm{o}},{\bm\omega}'_{\mathrm{i}})
    \max({\bm\omega}_{\mathrm{h}}\cdot{\bm\omega}'_{\mathrm{i}},0)
    L_{\mathrm{i}}({\bm\omega}'_{\mathrm{i}})\mathrm{d}{\bm\omega}'_{\mathrm{i}}\nonumber                                                  \\
     & =\int\limits_{{\varOmega}_{\mathrm{i}}}F_{\mathcal{M}}({\bm\omega}_{\mathrm{h}},{\bm\omega}_{\mathrm{o}})
    \delta_{{\bm\omega}_{\mathrm{r}}}({\bm\omega}'_{\mathrm{i}})
    L_{\mathrm{i}}({\bm\omega}'_{\mathrm{i}})\mathrm{d}{\bm\omega}'_{\mathrm{i}}\nonumber                                                  \\
     & =F_{\mathcal{M}}({\bm\omega}_{\mathrm{h}},{\bm\omega}_{\mathrm{o}})L_{\mathrm{i}}({\bm\omega}_{\mathrm{r}})\nonumber                \\
     & =F_{\mathcal{M}}({\bm\omega}_{\mathrm{h}},{\bm\omega}_{\mathrm{o}})L_{\mathrm{i}}({\bm\omega}_{\mathrm{i}})\, .
\end{align}
其他情况下,则等价于$L_{\mathrm{i}}({\bm\omega}_{\mathrm{r}})=0$,
使得$L_{\mathcal{M}}({\bm\omega}_{\mathrm{h}},{\bm\omega}_{\mathrm{o}})=0$,即无反射。

然而\refeq{08ex01-FresnelBRDFMicrofacet}仍然不方便我们推导微面模型的BRDF,
因为\refeq{08ex01-MacroBRDFG1}、\refeq{08ex01-MacroBRDFG2}都需要
对${\bm\omega}_{\mathrm{h}}$积分,但\refeq{08ex01-FresnelBRDFMicrofacet}中
狄拉克$\delta$分布的角标${\bm\omega}_{\mathrm{r}}$是
代入了${\bm\omega}_{\mathrm{h}}$的函数,${\bm\omega}_{\mathrm{h}}$处在这个位置并不便于计算积分。
因此我们需要在保证前面关于积分的推导(尤其是\refeq{08ex01-FresnelBRDFValid})仍然成立的条件下进行换元。
我们设${\bm\omega}_{\mathrm{m}}$是由${\bm\omega}_{\mathrm{i}}$和${\bm\omega}_{\mathrm{o}}$确定的
满足完美镜面反射的规范化微面法线方向,并设满足完美镜面反射的入射方向与微面法线的映射关系为$\bm P$,即:
\begin{align}
    {\bm P}({\bm\omega}_{\mathrm{m}})={\bm\omega}_{\mathrm{i}}\, , \\
    {\bm P}({\bm\omega}_{\mathrm{h}})={\bm\omega}_{\mathrm{r}}\, .
\end{align}
对$\bm P$作一阶近似并利用定理\ref{theorem:7.ex01.symmetry}的结论,可得
\begin{align}\label{eq:08ex01-DeltaChangeVar}
    \delta_{{\bm\omega}_{\mathrm{r}}}({\bm\omega}_{\mathrm{i}})
     & =\delta({\bm\omega}_{\mathrm{i}}-{\bm\omega}_{\mathrm{r}})
    =\delta({\bm P}({\bm\omega}_{\mathrm{m}})-{\bm P}({\bm\omega}_{\mathrm{h}}))
    =\displaystyle\delta\left(({\bm\omega}_{\mathrm{m}}-{\bm\omega}_{\mathrm{h}})
    \frac{\partial{\bm P}({\bm\omega}_{\mathrm{m}})}{\partial{\bm\omega}_{\mathrm{m}}}\right)\nonumber \\
     & =\displaystyle\delta\left(({\bm\omega}_{\mathrm{m}}-{\bm\omega}_{\mathrm{h}})
    \frac{\partial{\bm\omega}_{\mathrm{i}}}{\partial{\bm\omega}_{\mathrm{m}}}\right)
    =\delta({\bm\omega}_{\mathrm{m}}-{\bm\omega}_{\mathrm{h}})
    \left\lVert\frac{\partial{\bm\omega}_{\mathrm{m}}}{\partial{\bm\omega}_{\mathrm{i}}}\right\rVert
    =\delta_{{\bm\omega}_{\mathrm{m}}}({\bm\omega}_{\mathrm{h}})
    \left\lVert\frac{\partial{\bm\omega}_{\mathrm{m}}}{\partial{\bm\omega}_{\mathrm{i}}}\right\rVert\, .
\end{align}
其中$\displaystyle\frac{\partial{\bm\omega}_{\mathrm{m}}}{\partial{\bm\omega}_{\mathrm{i}}}$是
${\bm\omega}_{\mathrm{m}}$对${\bm\omega}_{\mathrm{i}}$的\keyindex{雅可比矩阵}{Jacobian matrix}{matrix矩阵}。
将其套在里面的两层$|\cdot|$表示先求矩阵的\keyindex{行列式}{determinant}{}再取绝对值。
几何意义上,它表示${\bm\omega}_{\mathrm{m}}$和${\bm\omega}_{\mathrm{i}}$发生扰动时
两者各自扰动范围对应的立体角大小之比。
我们在\reffig{08ex01-JacobianRefraction}中推导求解该比值的方法:
\begin{figure}[htbp]
    \centering
    \includegraphics[width=0.6\linewidth]{Pictures/chap08/JacobianRefraction.eps}
    \caption{镜面反射中的角度扰动关系和相应雅可比矩阵行列式绝对值的计算。}
    \label{fig:08ex01-JacobianRefraction}
\end{figure}

将表示入射方向${\bm\omega}_{\mathrm{i}}$和出射方向${\bm\omega}_{\mathrm{o}}$的向量首尾相接。
在完美镜面反射配置下,两者之和必与曲面法线${\bm\omega}_{\mathrm{m}}$共线,因此我们记
\begin{align}
    {\bm\omega}_{\mathrm{M}}=\mathrm{sign}({\bm\omega}_{\mathrm{i}}\cdot{\bm\omega}_{\mathrm{m}})
    \cdot({\bm\omega}_{\mathrm{i}}+{\bm\omega}_{\mathrm{o}})\, ,
\end{align}
其中sign一项是为了将其调整到曲面朝外的方向(和${\bm\omega}_{\mathrm{m}}$同向)
(sign对正数取1,对负数取-1,对零取0);
于是规范化的${\bm\omega}_{\mathrm{m}}$满足
\begin{align}
    {\bm\omega}_{\mathrm{m}}=\frac{{\bm\omega}_{\mathrm{M}}}{|{\bm\omega}_{\mathrm{M}}|}\, .
\end{align}
当${\bm\omega}_{\mathrm{i}}$存在立体角大小为$\Delta{\bm\omega}_{\mathrm{i}}$的微小扰动范围时,
其末端扰动范围的面积是在以${\bm\omega}_{\mathrm{i}}$起点
为球心的单位球表面计算的,大小也为$\Delta{\bm\omega}_{\mathrm{i}}$.
同时它也引发了${\bm\omega}_{\mathrm{m}}$的扰动,
我们即需要计算这块扰动区域对于${\bm\omega}_{\mathrm{m}}$而言是多大的立体角。
回顾定义\ref{definition:SolidAngle},考虑到扰动区域法线
(即${\bm\omega}_{\mathrm{i}}$)和${\bm\omega}_{\mathrm{m}}$存在夹角,
因此它对${\bm\omega}_{\mathrm{m}}$的起点所成的立体角大小$\Delta{\bm\omega}_{\mathrm{m}}$为
\begin{align}
    \Delta{\bm\omega}_{\mathrm{m}}=\frac{|{\bm\omega}_{\mathrm{m}}\cdot{\bm\omega}_{\mathrm{i}}|}
    {|{\bm\omega}_{\mathrm{M}}|^2}\Delta{\bm\omega}_{\mathrm{i}}\, .
\end{align}
同时注意到$|{\bm\omega}_{\mathrm{i}}|=|{\bm\omega}_{\mathrm{o}}|=1$,所以
\begin{align}
    |{\bm\omega}_{\mathrm{M}}|=2|{\bm\omega}_{\mathrm{m}}\cdot{\bm\omega}_{\mathrm{i}}|\, .
\end{align}
于是
\begin{align}\label{eq:08ex01-JacobianRefraction}
    \left\lVert\frac{\partial{\bm\omega}_{\mathrm{m}}}{\partial{\bm\omega}_{\mathrm{i}}}\right\rVert
    =\lim\limits_{\Delta{\bm\omega}_{\mathrm{i}}\to0}\frac{|\Delta{\bm\omega}_{\mathrm{m}}|}{|\Delta{\bm\omega}_{\mathrm{i}}|}
    =\frac{|{\bm\omega}_{\mathrm{m}}\cdot{\bm\omega}_{\mathrm{i}}|}{|{\bm\omega}_{\mathrm{M}}|^2}
    =\frac{1}{4|{\bm\omega}_{\mathrm{m}}\cdot{\bm\omega}_{\mathrm{i}}|}\, .
\end{align}
将\refeq{08ex01-DeltaChangeVar}和\refeq{08ex01-JacobianRefraction}代入\refeq{08ex01-FresnelBRDFMicrofacet},即得
\begin{align}
    f_{\mathcal{M}}({\bm\omega}_{\mathrm{h}},{\bm\omega}_{\mathrm{o}},{\bm\omega}_{\mathrm{i}})
    =\frac{F_{\mathcal{M}}({\bm\omega}_{\mathrm{h}},{\bm\omega}_{\mathrm{o}})
    \delta_{{\bm\omega}_{\mathrm{m}}}({\bm\omega}_{\mathrm{h}})}
    {4|{\bm\omega}_{\mathrm{h}}\cdot{\bm\omega}_{\mathrm{i}}|
    |{\bm\omega}_{\mathrm{m}}\cdot{\bm\omega}_{\mathrm{i}}|}
    =\frac{F_{\mathcal{M}}({\bm\omega}_{\mathrm{h}},{\bm\omega}_{\mathrm{o}})
    \delta_{{\bm\omega}_{\mathrm{m}}}({\bm\omega}_{\mathrm{h}})}
    {4|{\bm\omega}_{\mathrm{m}}\cdot{\bm\omega}_{\mathrm{i}}|^2}\, .
\end{align}
将上式代入\refeq{08ex01-MacroBRDFG2},并注意被积项中因为存在狄拉克$\delta$分布,
所以其只在完美镜面反射成立时,即${\bm\omega}_{\mathrm{h}}={\bm\omega}_{\mathrm{m}}$时才可能取非零值,
且此时必有${\bm\omega}_{\mathrm{h}}\cdot{\bm\omega}_{\mathrm{i}}={\bm\omega}_{\mathrm{h}}\cdot{\bm\omega}_{\mathrm{o}}$,
所以我们最终算出镜面微面模型的BRDF是
\begin{align}\label{eq:08ex01-BRDFMicrofacetFinal}
    f_{\mathrm{r}}({\bm\omega}_{\mathrm{o}},{\bm\omega}_{\mathrm{i}})
     & =\frac{1}{|{\bm\omega}_{\mathrm{g}}\cdot{\bm\omega}_{\mathrm{i}}||{\bm\omega}_{\mathrm{g}}\cdot{\bm\omega}_{\mathrm{o}}|}
    \int\limits_{\varOmega}\frac{F_{\mathcal{M}}({\bm\omega}_{\mathrm{h}},{\bm\omega}_{\mathrm{o}})
    \delta_{{\bm\omega}_{\mathrm{m}}}({\bm\omega}_{\mathrm{h}})}
    {4|{\bm\omega}_{\mathrm{m}}\cdot{\bm\omega}_{\mathrm{i}}|^2}
    \max({\bm\omega}_{\mathrm{h}}\cdot{\bm\omega}_{\mathrm{i}},0)\nonumber                                                       \\
     & \qquad\qquad\max({\bm\omega}_{\mathrm{h}}\cdot{\bm\omega}_{\mathrm{o}},0)
    G_2({\bm\omega}_{\mathrm{h}},{\bm\omega}_{\mathrm{o}},{\bm\omega}_{\mathrm{i}})
    D({\bm\omega}_{\mathrm{h}})\mathrm{d}{\bm\omega}_{\mathrm{h}}\nonumber                                                       \\
     & =\frac{F_{\mathcal{M}}({\bm\omega}_{\mathrm{m}},{\bm\omega}_{\mathrm{o}})
    G_2({\bm\omega}_{\mathrm{m}},{\bm\omega}_{\mathrm{o}},{\bm\omega}_{\mathrm{i}})D({\bm\omega}_{\mathrm{m}})}
    {4|{\bm\omega}_{\mathrm{g}}\cdot{\bm\omega}_{\mathrm{i}}||{\bm\omega}_{\mathrm{g}}\cdot{\bm\omega}_{\mathrm{o}}|}\, .
\end{align}

\subsection{微面模型BRDF的规范化测试}\label{sub:微面模型BRDF的规范化测试}
本节继承上节的记号。假设某材质不吸收任何入射辐射能量,
且菲涅尔反射率$F_{\mathcal{M}}({\bm\omega}_{\mathrm{m}},{\bm\omega}_{\mathrm{o}})$恒为1,
也即完全不透射任何光线,则入射的能量会被无损地反射回去。
此时对应的BRDF应该满足以下规范化约束:
\begin{align}\label{eq:08ex-01-WhiteFurnaceTest}
    \forall {\bm\omega}_{\mathrm{o}}: \quad\int\limits_{{\varOmega}_{\mathrm{i}}}
    f_{\mathrm{r}}({\bm\omega}_{\mathrm{o}},{\bm\omega}_{\mathrm{i}})
    |{\bm\omega}_{\mathrm{g}}\cdot{\bm\omega}_{\mathrm{i}}|\mathrm{d}{\bm\omega}_{\mathrm{i}}=1\, ,
\end{align}
上式称作\keyindex{白炉测试}{White Furnace Test}{}等式。
该式意味着,对于这样的材质,由出射光线反推回去的入射光线,
会在表面上反射一次或多次,最终全部离开表面,且无能量损失。

然而常见的可解析表达的BRDF都没有考虑多次反射的情况,
这些多次反射的光线都被遮挡函数滤除了,
所以此类BRDF在完美镜面微面模型上作参数化时都不满足白炉测试等式。
因此我们换个角度来分析——我们考虑光线刚发生第一次反射之后、离开曲面之前的情况,
即把掩模遮挡函数替换为只有掩模函数函数
(令$G_2({\bm\omega}_{\mathrm{m}},{\bm\omega}_{\mathrm{o}},{\bm\omega}_{\mathrm{i}})
    =G_1({\bm\omega}_{\mathrm{m}},{\bm\omega}_{\mathrm{o}})$),
则规范化约束应重新成立。此时来自\refeq{08ex01-BRDFMicrofacetFinal}的BRDF变为
\begin{align}
    f_{\mathrm{r}}({\bm\omega}_{\mathrm{o}},{\bm\omega}_{\mathrm{i}})
    =\frac{G_1({\bm\omega}_{\mathrm{m}},{\bm\omega}_{\mathrm{o}})D({\bm\omega}_{\mathrm{m}})}
    {4|{\bm\omega}_{\mathrm{g}}\cdot{\bm\omega}_{\mathrm{i}}||{\bm\omega}_{\mathrm{g}}\cdot{\bm\omega}_{\mathrm{o}}|}\, .
\end{align}
将其代入白炉测试\refeq{08ex-01-WhiteFurnaceTest},
便得到\keyindex{弱白炉测试}{Weak White Furnace Test}{White Furnace Test白炉测试}等式:
\begin{align}\label{eq:08ex01-WeakWhiteFurnaceTest}
    \forall {\bm\omega}_{\mathrm{o}}: \quad\int\limits_{{\varOmega}_{\mathrm{i}}}
    \frac{G_1({\bm\omega}_{\mathrm{m}},{\bm\omega}_{\mathrm{o}})D({\bm\omega}_{\mathrm{m}})}
    {4|{\bm\omega}_{\mathrm{g}}\cdot{\bm\omega}_{\mathrm{o}}|}\mathrm{d}{\bm\omega}_{\mathrm{i}}=1\, .
\end{align}
容易看出,上式的成立其实并不依赖菲涅尔反射率恒为1。
它给出了镜面微面模型对$G_1$的又一重要约束。
需要说明的是,弱白炉测试丢弃遮挡函数只是为了
提供一个便捷的方法来验证BRDF的物理合理性,
并不是说BRDF在实际使用中也应丢弃遮挡函数。

在实践中,我们经常会面临这样一个问题:
某个基于镜面微面模型的BRDF是“基于物理的”吗?
回顾这四个小节的内容,我们虽然不能正面给出答案,
但却可以给出一些有效的验证方法——
如果这个BRDF没有同时满足以下四个规范化约束,那它必然不是“基于物理的”:
\begin{enumerate}
    \item 微面分布函数$D({\bm\omega}_{\mathrm{h}})$满足\refeq{08ex01-McrofacetDistributionNormalization};
    \item 掩模函数$G_1$满足\refeq{08ex01-CosThetaO};
    \item 可见法线分布$D_{{\bm\omega}_{\mathrm{o}}}({\bm\omega}_{\mathrm{h}})$满足\refeq{08ex01-VisibleDistributionNormalization};
    \item 弱白炉测试\refeq{08ex01-WeakWhiteFurnaceTest}成立。
\end{enumerate}
不过这并不意味着那些不能同时满足上述约束的没有“基于物理的”的BRDF就不能使用,
完全可以根据实际需要做出相应选择,何况以上结论考虑的还是最简单的情况。
多次反射、多层材料、存在衍射等更复杂的情况还需要进一步探索。

\subsection{常见掩模函数的分析}\label{sub:常见掩模函数的分析}
本节继承上节的记号。本节将分析Smith和V形槽两种微面配置,推导它们的掩模函数并讨论其性质。
其他没有相应微面配置因而并非基于物理的常见掩模函数也会有所讨论。

\subsubsection*{Smith微面}
在Smith微面的配置中,它假设微面是非\keyindex{自相关的}{autocorrelated}{}——
不论微面上的一点和它的临近点有多近,它们的高度(或法线)之间是没有相关性的。
这意味着微面上一点的高度和法线是随机变量,整个曲面是微面的随机集合,而不是通常的连续曲面。
另一方面,法线${\bm\omega}_{\mathrm{h}}$某种意义上是微面的\emph{局部}属性,
而对该点产生遮挡的别处微面则是一种\emph{远距}属性(但仍是微观尺度上的)。
在非自相关假设下,局部属性和远距属性应是独立的,所以掩模函数$G_1$可以拆分为两部分:
\begin{align}\label{eq:08ex01-SeparableMaskingFunction}
    G_1({\bm\omega}_{\mathrm{h}},{\bm\omega}_{\mathrm{o}})
    =G_1^{\mathrm{l}}({\bm\omega}_{\mathrm{h}},{\bm\omega}_{\mathrm{o}})
    G_1^{\mathrm{d}}({\bm\omega}_{\mathrm{o}})\, ,
\end{align}
其中局部掩模函数$G_1^{\mathrm{l}}$就简单地滤除背向的微面:
\begin{align}\label{eq:08ex01-LocalMaskFunction}
    G_1^{\mathrm{l}}({\bm\omega}_{\mathrm{h}},{\bm\omega}_{\mathrm{o}})
    =\chi({\bm\omega}_{\mathrm{h}}\cdot{\bm\omega}_{\mathrm{o}})\, .
\end{align}
这里示性函数$\chi$的定义为
\begin{align}
    \chi(a)=\left\{\begin{array}{l}
        1,\quad\text{若}a>0, \\
        0,\quad\text{其他}.
    \end{array}\right.
\end{align}
而远距掩模函数$G_1^{\mathrm{d}}({\bm\omega}_{\mathrm{o}})$表示
被远处微面遮挡的概率,它独立于局部法线${\bm\omega}_{\mathrm{h}}$.

将\refeq{08ex01-SeparableMaskingFunction}和\refeq{08ex01-LocalMaskFunction}
代入\refeq{08ex01-CosThetaO},可得
\begin{align}
    \cos\theta_{\mathrm{o}}
     & =\int\limits_{\varOmega}G_1({\bm\omega}_{\mathrm{h}},{\bm\omega}_{\mathrm{o}})
    \max({\bm\omega}_{\mathrm{h}}\cdot{\bm\omega}_{\mathrm{o}},0)
    D({\bm\omega}_{\mathrm{h}})\mathrm{d}{\bm\omega}_{\mathrm{h}}\nonumber                         \\
     & =\int\limits_{\varOmega}G_1^{\mathrm{l}}({\bm\omega}_{\mathrm{h}},{\bm\omega}_{\mathrm{o}})
    G_1^{\mathrm{d}}({\bm\omega}_{\mathrm{o}})
    \max({\bm\omega}_{\mathrm{h}}\cdot{\bm\omega}_{\mathrm{o}},0)
    D({\bm\omega}_{\mathrm{h}})\mathrm{d}{\bm\omega}_{\mathrm{h}}\nonumber                         \\
     & =\int\limits_{\varOmega}\chi({\bm\omega}_{\mathrm{h}}\cdot{\bm\omega}_{\mathrm{o}})
    G_1^{\mathrm{d}}({\bm\omega}_{\mathrm{o}})
    \max({\bm\omega}_{\mathrm{h}}\cdot{\bm\omega}_{\mathrm{o}},0)
    D({\bm\omega}_{\mathrm{h}})\mathrm{d}{\bm\omega}_{\mathrm{h}}\nonumber                         \\
     & =G_1^{\mathrm{d}}({\bm\omega}_{\mathrm{o}})\int\limits_{\varOmega}
    \max({\bm\omega}_{\mathrm{h}}\cdot{\bm\omega}_{\mathrm{o}},0)
    D({\bm\omega}_{\mathrm{h}})\mathrm{d}{\bm\omega}_{\mathrm{h}}\, .
\end{align}
于是
\begin{align}\label{eq:08-ex01-g1_distance}
    G_1^{\mathrm{d}}({\bm\omega}_{\mathrm{o}})
    =\frac{\cos\theta_{\mathrm{o}}}
    {\displaystyle\int\limits_{\varOmega}\max({\bm\omega}_{\mathrm{h}}\cdot{\bm\omega}_{\mathrm{o}},0)
    D({\bm\omega}_{\mathrm{h}})\mathrm{d}{\bm\omega}_{\mathrm{h}}}\, .
\end{align}
所以完整的掩模函数为
\begin{align}\label{eq:08-ex01-masking-g1-int}
    G_1({\bm\omega}_{\mathrm{h}},{\bm\omega}_{\mathrm{o}})
    =\frac{\chi({\bm\omega}_{\mathrm{h}}\cdot{\bm\omega}_{\mathrm{o}})\cos\theta_{\mathrm{o}}}
    {\displaystyle\int\limits_{\varOmega}\max({\bm\omega}\cdot{\bm\omega}_{\mathrm{o}},0)
        D({\bm\omega})\mathrm{d}{\bm\omega}}\, .
\end{align}
这恰是\citet{10.1145/344779.344814}在法线和遮挡独立性假设下
得到的掩模函数的精确积分形式。然而其中的积分是在法线分布空间中进行的,
计算起来并不方便。我们可以将其积分域转化到斜率分布空间以简化它,下面给出具体推导。

对于曲面上某处的法线${\bm\omega}_{\mathrm{h}}$,
设它在直角坐标系下和球面坐标系下具体的分量为
\begin{align}
    {\bm\omega}_{\mathrm{h}}=(x_{\mathrm{h}},y_{\mathrm{h}},z_{\mathrm{h}})
    =(\sin\theta\cos\varphi,\sin\theta\sin\varphi,\cos\theta)\, ,
\end{align}
其中$\theta$为天顶角(即${\bm\omega}_{\mathrm{h}}\cdot{\bm\omega}_{\mathrm{g}}=\cos\theta$),
$\varphi$为方位角
\sidenote{参见\reffig{5.ex01}。};则该处附近的面元可以近似为以下平面
\begin{align}
    x_{\mathrm{h}}x+y_{\mathrm{h}}y+z_{\mathrm{h}}z=C\, .
\end{align}
其中$C$为某个常量,于是有
\begin{align}
    z=\left(-\frac{x_{\mathrm{h}}}{z_{\mathrm{h}}},
    -\frac{y_{\mathrm{h}}}{z_{\mathrm{h}}}\right)\cdot(x,y)+C\, ,
\end{align}
我们由此定义
\begin{align}
    {\bm s}({\bm\omega}_{\mathrm{h}})=(x_s,y_s)
    =\left(-\frac{x_{\mathrm{h}}}{z_{\mathrm{h}}},
    -\frac{y_{\mathrm{h}}}{z_{\mathrm{h}}}\right)
    =-\tan\theta(\cos\varphi,\sin\varphi)
\end{align}
为曲面在该处的斜率
\sidenote{原文slope,这里作者想表达的是$z$对$x$和$y$的偏导数,
    它和二维直角坐标系下直线斜率的形式很像,所以借用这个称呼。}。
反之,也可以根据斜率求出相应法线为
\sidenote{注意到${\bm s}({\bm\omega}_{\mathrm{h}})={\bm s}(-{\bm\omega}_{\mathrm{h}})$,
所以这里也可以是反向的结果,我们只是取其中一个。}
\begin{align}\label{eq:08-ex01-normals-by-slope}
    {\bm\omega}_{\mathrm{h}}=\frac{1}{\sqrt{x_s^2+y_s^2+1}}(-x_s,-y_s,1)\, .
\end{align}

接下来我们考虑斜率分布$P_{xy}({\bm s})$与法线分布
(即微面分布函数)$D({\bm\omega}_{\mathrm{h}})$之间的关系。
法线分布方面,根据三维球面坐标转换
公式\sidenote{见第\refsec{球体}。},有
\begin{align}\label{eq:08-ex01-D_sphere}
    D({\bm\omega}_{\mathrm{h}})\mathrm{d}{\bm\omega}_{\mathrm{h}}
    =D({\bm\omega}_{\mathrm{h}})\sin\theta\mathrm{d}\theta\mathrm{d}\varphi\, .
\end{align}
斜率分布方面,其积分也有换元关系
\sidenote{这是微积分中常用的换元方法,此处我们不探究其使用条件,读者可参考相关教材。}
\begin{align}\label{eq:08-ex01-Pxy-Jacobian}
    P_{xy}(x_s,y_s)\mathrm{d}x_s\mathrm{d}y_s
    =P_{xy}(x_s,y_s)|J|\mathrm{d}\theta\mathrm{d}\varphi\, ,
\end{align}
其中$J$为$(x_s,y_s)$对参数$(\theta,\varphi)$的雅可比矩阵
\begin{align}
    J=\displaystyle\frac{\partial(x_s,y_s)}{\partial(\theta,\varphi)}
    =\displaystyle\left[\begin{array}{cc}
            \displaystyle\frac{\partial x_s}{\partial \theta} &
            \displaystyle\frac{\partial x_s}{\partial \varphi}  \\
            \displaystyle\frac{\partial y_s}{\partial \theta} &
            \displaystyle\frac{\partial y_s}{\partial \varphi}
        \end{array}\right]
    =\displaystyle\left[\begin{array}{rr}
            \displaystyle -\frac{\cos\varphi}{\cos^2\theta} &
            \displaystyle \tan\theta\sin\varphi               \\
            \displaystyle -\frac{\sin\varphi}{\cos^2\theta} &
            \displaystyle -\tan\theta\cos\varphi
        \end{array}\right]\, ,
\end{align}
于是雅可比行列式为
\begin{align}\label{eq:08-ex01-Jacobian-slope-normals}
    |J|=\left(\frac{\partial x_s}{\partial \theta}\frac{\partial y_s}{\partial \varphi}
    -\frac{\partial x_s}{\partial \varphi}\frac{\partial y_s}{\partial \theta}\right)
    =\frac{\tan\theta}{\cos^2\theta}\, .
\end{align}
注意到$P_{xy}({\bm s})$应满足规范化约束,
即它在$x_s\in(-\infty,+\infty),y_s\in(-\infty,+\infty)$范围内非负,且有
\begin{align}
    \int_{-\infty}^{+\infty}\int_{-\infty}^{+\infty}
    P_{xy}(x_s,y_s)\mathrm{d}x_s\mathrm{d}y_s=1\, .
\end{align}
将上式和\refeq{08ex01-McrofacetDistributionNormalization}比对,可得
\begin{align}\label{eq:08-ex01-P2D}
    P_{xy}(x_s,y_s)\mathrm{d}x_s\mathrm{d}y_s=
    D({\bm\omega}_{\mathrm{h}})\cos\theta\mathrm{d}{\bm\omega}_{\mathrm{h}}\, .
\end{align}
联立\refeq{08-ex01-D_sphere}、\refeq{08-ex01-Pxy-Jacobian}和\refeq{08-ex01-P2D}可得
\begin{align}
    P_{xy}(x_s,y_s)|J|\mathrm{d}\theta\mathrm{d}\varphi
    =D({\bm\omega}_{\mathrm{h}})\sin\theta\cos\theta\mathrm{d}\theta\mathrm{d}\varphi\, ,
\end{align}
代入\refeq{08-ex01-Jacobian-slope-normals}后可知斜率分布与法线分布的关系为
\begin{align}
    P_{xy}(x_s,y_s)=\frac{1}{|J|}D({\bm\omega}_{\mathrm{h}})\sin\theta\cos\theta
    =D({\bm\omega}_{\mathrm{h}})\cos^4\theta\, .
\end{align}

利用前面的结果,我们尝试将积分域从法线分布空间转化到斜率分布空间。
首先注意到,因为${\bm\omega}_{\mathrm{g}}=(0,0,1)$,
所以结合\refeq{08-ex01-normals-by-slope}有
\begin{align}
    {\bm\omega}_{\mathrm{h}}\cdot{\bm\omega}_{\mathrm{g}}
    =\frac{1}{\sqrt{x_s^2+y_s^2+1}}\, .
\end{align}
类似地,对于出射方向${\bm\omega}_{\mathrm{o}}=(x_{\mathrm{o}},y_{\mathrm{o}},z_{\mathrm{o}})$,则有
\begin{align}
    \max({\bm\omega}_{\mathrm{h}}\cdot{\bm\omega}_{\mathrm{o}},0)
    =\frac{(-x_{\mathrm{o}}x_s-y_{\mathrm{o}}y_s+z_{\mathrm{o}})
    \chi(-x_{\mathrm{o}}x_s-y_{\mathrm{o}}y_s+z_{\mathrm{o}})}{\sqrt{x_s^2+y_s^2+1}}\, .
\end{align}
将上述两式与\refeq{08-ex01-P2D}结合可知
\begin{align}\label{eq:08-ex01-trans-normal-slope}
      & \int\limits_{\varOmega}\max({\bm\omega}_{\mathrm{h}}\cdot{\bm\omega}_{\mathrm{o}},0)
    D({\bm\omega}_{\mathrm{h}})\mathrm{d}{\bm\omega}_{\mathrm{h}}\nonumber                          \\
    = & \int\limits_{\varOmega}\frac{\max({\bm\omega}_{\mathrm{h}}\cdot{\bm\omega}_{\mathrm{o}},0)}
    {\cos\theta}D({\bm\omega}_{\mathrm{h}})\cos\theta\mathrm{d}{\bm\omega}_{\mathrm{h}}\nonumber    \\
    = & \int_{-\infty}^{+\infty}\int_{-\infty}^{+\infty}
    \frac{\max({\bm\omega}_{\mathrm{h}}\cdot{\bm\omega}_{\mathrm{o}},0)}
    {{\bm\omega}_{\mathrm{h}}\cdot{\bm\omega}_{\mathrm{g}}}
    P_{xy}(x_s,y_s)\mathrm{d}x_s\mathrm{d}y_s\nonumber                                              \\
    = & \int_{-\infty}^{+\infty}\int_{-\infty}^{+\infty}
    (-x_sx_{\mathrm{o}}-y_sy_{\mathrm{o}}+z_{\mathrm{o}})
    \chi(-x_sx_{\mathrm{o}}-y_sy_{\mathrm{o}}+z_{\mathrm{o}})
    P_{xy}(x_s,y_s)\mathrm{d}x_s\mathrm{d}y_s\, .
\end{align}
为了简化后续推导,这里我们不妨设出射方向${\bm\omega}_{\mathrm{o}}$的方位角$\varphi=0$,于是
\begin{align}
    {\bm\omega}_{\mathrm{o}}=(\sin\theta_{\mathrm{o}},0,\cos\theta_{\mathrm{o}})\, .
\end{align}
将其代入\refeq{08-ex01-trans-normal-slope}得
\begin{align}\label{eq:08-ex01-trans-1d-slope}
      & \int\limits_{\varOmega}\max({\bm\omega}_{\mathrm{h}}\cdot{\bm\omega}_{\mathrm{o}},0)
    D({\bm\omega}_{\mathrm{h}})\mathrm{d}{\bm\omega}_{\mathrm{h}}\nonumber                   \\
    = & \int_{-\infty}^{+\infty}\int_{-\infty}^{+\infty}
    (-x_s\sin\theta_{\mathrm{o}}+\cos\theta_{\mathrm{o}})
    \chi(-x_s\sin\theta_{\mathrm{o}}+\cos\theta_{\mathrm{o}})
    P_{xy}(x_s,y_s)\mathrm{d}x_s\mathrm{d}y_s\nonumber                                       \\
    = & \displaystyle\int_{-\infty}^{+\infty}
    (-x_s\sin\theta_{\mathrm{o}}+\cos\theta_{\mathrm{o}})
    \chi(-x_s\sin\theta_{\mathrm{o}}+\cos\theta_{\mathrm{o}})
    \left(\int_{-\infty}^{+\infty}P_{xy}(x_s,y_s)\mathrm{d}y_s\right)\mathrm{d}x_s\nonumber  \\
    = & \displaystyle\int_{-\infty}^{+\infty}
    (-x_s\sin\theta_{\mathrm{o}}+\cos\theta_{\mathrm{o}})
    \chi(-x_s\sin\theta_{\mathrm{o}}+\cos\theta_{\mathrm{o}})P_x(x_s)\mathrm{d}x_s\, ,
\end{align}
其中
\begin{align}
    P_x(x_s)=\int_{-\infty}^{+\infty}P_{xy}(x_s,y_s)\mathrm{d}y_s
\end{align}
是斜率沿着出射方向(这里假设其方位角为零)的条件分布。
注意到\refeq{08-ex01-trans-1d-slope}中的示性函数把积分域限定在
\begin{align}
    -x_s\sin\theta_{\mathrm{o}}+\cos\theta_{\mathrm{o}}>0\, ,
\end{align}
也即
\begin{align}
    x_s<\cot\theta_{\mathrm{o}}\, .
\end{align}
于是\refeq{08-ex01-trans-1d-slope}可进一步化简为
\begin{align}
    \int\limits_{\varOmega}\max({\bm\omega}_{\mathrm{h}}\cdot{\bm\omega}_{\mathrm{o}},0)
    D({\bm\omega}_{\mathrm{h}})\mathrm{d}{\bm\omega}_{\mathrm{h}}
    =\int_{-\infty}^{\cot\theta_{\mathrm{o}}}
    (-x_s\sin\theta_{\mathrm{o}}+\cos\theta_{\mathrm{o}})P_x(x_s)\mathrm{d}x_s\, .
\end{align}
将上式代入\refeq{08-ex01-g1_distance},可得
\begin{align}
    \cos\theta_{\mathrm{o}}
    = & G_1^{\mathrm{d}}({\bm\omega}_{\mathrm{o}})
    \int\limits_{\varOmega}\max({\bm\omega}_{\mathrm{h}}\cdot{\bm\omega}_{\mathrm{o}},0)
    D({\bm\omega}_{\mathrm{h}})\mathrm{d}{\bm\omega}_{\mathrm{h}}\nonumber \\
    = & G_1^{\mathrm{d}}({\bm\omega}_{\mathrm{o}})
    \int_{-\infty}^{\cot\theta_{\mathrm{o}}}
    (-x_s\sin\theta_{\mathrm{o}}+\cos\theta_{\mathrm{o}})P_x(x_s)\mathrm{d}x_s\, .
\end{align}
将上式两边同除以$\sin\theta_{\mathrm{o}}$,得到
\begin{align}\label{eq:08-ex01-cot-theta0-g1-distant}
    \cot\theta_{\mathrm{o}}=G_1^{\mathrm{d}}({\bm\omega}_{\mathrm{o}})
    \int_{-\infty}^{\cot\theta_{\mathrm{o}}}
    (-x_s+\cot\theta_{\mathrm{o}})P_x(x_s)\mathrm{d}x_s\, .
\end{align}
我们接着要从上式中\sidenote{笔者认为这里将积分式移项即可,
    但原文作者似乎还希望调整里面的正负号,所以有了下面的推导。}
解出$G_1^{\mathrm{d}}({\bm\omega}_{\mathrm{o}})$.
假设微面法线的分布是中心对称的,即
\begin{align}
    P_{xy}(x_s,y_s)=P_{xy}(-x_s,-y_s)
\end{align}
对任意$x_s$和$y_s$恒成立,则易知条件分布$P_{x}(x_s)$是个偶函数,即
\begin{align}
    P_{x}(x_s)=P_{x}(-x_s)
\end{align}
对任意$x_s$恒成立,于是任意出射方向上斜率$x_s$的均值都为零:
\begin{align}\label{eq:08-ex01-mean-slope-conditional}
    \int_{-\infty}^{+\infty}x_sP_{x}(x_s)\mathrm{d}x_s=0\, .
\end{align}
同时我们注意到斜率的条件分布$P_x(x_s)$满足规范化性质
\begin{align}\label{eq:08-ex01-normal-slope-int}
    \int_{-\infty}^{+\infty}P_{x}(x_s)\mathrm{d}x_s=1\, .
\end{align}
利用\refeq{08-ex01-mean-slope-conditional}和\refeq{08-ex01-normal-slope-int}可以
代替常数0和1的特点,我们将\refeq{08-ex01-cot-theta0-g1-distant}两边
同时减去$G_1^{\mathrm{d}}({\bm\omega}_{\mathrm{o}})\cot\theta_{\mathrm{o}}$可得
\begin{align}
    (1-G_1^{\mathrm{d}}({\bm\omega}_{\mathrm{o}}))\cot\theta_{\mathrm{o}}
     & =G_1^{\mathrm{d}}({\bm\omega}_{\mathrm{o}})
    \int_{-\infty}^{\cot\theta_{\mathrm{o}}}
    (-x_s+\cot\theta_{\mathrm{o}})P_x(x_s)\mathrm{d}x_s\nonumber                          \\
     & \qquad -G_1^{\mathrm{d}}({\bm\omega}_{\mathrm{o}})\cot\theta_{\mathrm{o}}
    \int_{-\infty}^{+\infty}P_{x}(x_s)\mathrm{d}x_s\nonumber                              \\
     & \qquad +G_1^{\mathrm{d}}({\bm\omega}_{\mathrm{o}})
    \int_{-\infty}^{+\infty}x_sP_{x}(x_s)\mathrm{d}x_s\nonumber                           \\
     & \displaystyle =G_1^{\mathrm{d}}({\bm\omega}_{\mathrm{o}})
    \left(\int_{-\infty}^{+\infty}x_sP_{x}(x_s)\mathrm{d}x_s
    -\int_{-\infty}^{\cot\theta_{\mathrm{o}}}x_sP_{x}(x_s)\mathrm{d}x_s\right)\nonumber   \\
     & \qquad\displaystyle +G_1^{\mathrm{d}}({\bm\omega}_{\mathrm{o}})
    \left(\int_{-\infty}^{\cot\theta_{\mathrm{o}}}P_x(x_s)\cot\theta_{\mathrm{o}}\mathrm{d}x_s
    -\int_{-\infty}^{+\infty}P_x(x_s)\cot\theta_{\mathrm{o}}\mathrm{d}x_s\right)\nonumber \\
     & =G_1^{\mathrm{d}}({\bm\omega}_{\mathrm{o}})
    \int_{\cot\theta_{\mathrm{o}}}^{+\infty}(x_s-\cot\theta_{\mathrm{o}})P_x(x_s)\mathrm{d}x_s\, .
\end{align}
整理上式,我们最终得到$G_1^{\mathrm{d}}({\bm\omega}_{\mathrm{o}})$的解析解为
\begin{align}\label{eq:08-ex01-g1-d-solution}
    G_1^{\mathrm{d}}({\bm\omega}_{\mathrm{o}})=\frac{1}{1+\Lambda({\bm\omega}_{\mathrm{o}})}\, ,
\end{align}
其中
\begin{align}
    \Lambda({\bm\omega}_{\mathrm{o}})=\frac{1}{\cot\theta_{\mathrm{o}}}
    \int_{\cot\theta_{\mathrm{o}}}^{+\infty}(x_s-\cot\theta_{\mathrm{o}})P_x(x_s)\mathrm{d}x_s\, .
\end{align}
将\refeq{08-ex01-g1-d-solution}代回\refeq{08ex01-SeparableMaskingFunction}即得
\begin{align}\label{eq:08-ex01-Smith-masking-function}
    G_1({\bm\omega}_{\mathrm{h}},{\bm\omega}_{\mathrm{o}})
    =\frac{\chi({\bm\omega}_{\mathrm{h}}\cdot{\bm\omega}_{\mathrm{o}})}
    {1+\Lambda({\bm\omega}_{\mathrm{o}})}\, .
\end{align}
它是对Smith掩模函数的推广,适用于许多随机曲面。

由于前面的推导都不包含近似处理,这说明在非自相关的假设下,Smith掩模函数是准确的。
在实践中,该模型预测的结果和实测数据非常接近,但仍有偏差,
这是由描述曲面的统计模型和非自相关假设引起的。
现实中的连续曲面往往都有范围很宽的自相关函数。
但\citet{841905}表明忽略自相关性引发的误差仅在观察角度
满足$\tan\theta>\frac{\sqrt{2}}{2}\sigma$时
才足够明显($\sigma^2$为斜率的方差),所以通常可以认为
Smith掩模函数可以较为准确地适用于自相关的曲面,
但具有重复性或结构化纹理的材料(例如布料)则不应在建模时忽视这种自相关性。

\subsubsection*{V形槽微面}
如\reffig{08ex01-V-cavityScatteringModel},V形槽微面模型也是常用模型之一。
它不再是对具有特定法线分布的微面的散射情况进行建模,
而是计算每个独立镜面微面的散射再作平均。每块V形槽都有两个具备对称性的法线,
即${\bm\omega}_{\mathrm{h}}=(x_{\mathrm{h}},y_{\mathrm{h}},z_{\mathrm{h}})$和
${\bm\omega}'_{\mathrm{h}}=(-x_{\mathrm{h}},-y_{\mathrm{h}},z_{\mathrm{h}})$,
它们最后以$\max({\bm\omega}_{\mathrm{h}}\cdot{\bm\omega}_{\mathrm{g}},0)D({\bm\omega}_{\mathrm{h}})$
作为权重合成最终的BRDF.
根据这些特性,我们构造出相应的微面分布函数
\begin{align}\label{eq:08ex01-VCavityScatteringNormalDistribution}
    D({\bm\omega})=\frac{1}{2}\left(
    \frac{\delta_{{\bm\omega}_{\mathrm{h}}}({\bm\omega})}
    {{\bm\omega}_{\mathrm{h}}\cdot{\bm\omega}_{\mathrm{g}}}
    +\frac{\delta_{{\bm\omega}'_{\mathrm{h}}}({\bm\omega})}
    {{\bm\omega}'_{\mathrm{h}}\cdot{\bm\omega}_{\mathrm{g}}}\right)\, .
\end{align}
可以验证\refeq{08ex01-VCavityScatteringNormalDistribution}满足
规范化条件\refeq{08ex01-McrofacetDistributionNormalization}。

\begin{figure}[htbp]
    \centering
    \includegraphics[width=0.8\linewidth]{Pictures/chap08/VCavityScatteringModel.eps}
    \caption{V形槽微面模型。}
    \label{fig:08ex01-V-cavityScatteringModel}
\end{figure}

将\refeq{08ex01-VCavityScatteringNormalDistribution}代入约束条件\refeq{08ex01-CosThetaO},可得
\begin{align}\label{eq:08ex01-V-Cavity-configurations}
    \cos\theta_{\mathrm{o}}=\frac{1}{2}\left(
    G_1({\bm\omega}_{\mathrm{h}},{\bm\omega}_{\mathrm{o}})
    \frac{\max({\bm\omega}_{\mathrm{h}}\cdot{\bm\omega}_{\mathrm{o}},0)}
    {{\bm\omega}_{\mathrm{h}}\cdot{\bm\omega}_{\mathrm{g}}}
    +G_1({\bm\omega}'_{\mathrm{h}},{\bm\omega}_{\mathrm{o}})
    \frac{\max({\bm\omega}'_{\mathrm{h}}\cdot{\bm\omega}_{\mathrm{o}},0)}
    {{\bm\omega}'_{\mathrm{h}}\cdot{\bm\omega}_{\mathrm{g}}}\right)\, .
\end{align}
如\reffig{08ex01-V-cavityScattering-Mask},微面的可见性有两种情况。
一种是其中一个${\bm\omega}'_{\mathrm{h}}$因背向而不可见,
即$G_1({\bm\omega}'_{\mathrm{h}},{\bm\omega}_{\mathrm{o}})=0$,
此时\refeq{08ex01-V-Cavity-configurations}可以简化为
\begin{align}
    \cos\theta_{\mathrm{o}}=\frac{1}{2}G_1({\bm\omega}_{\mathrm{h}},{\bm\omega}_{\mathrm{o}})
    \frac{\max({\bm\omega}_{\mathrm{h}}\cdot{\bm\omega}_{\mathrm{o}},0)}
    {{\bm\omega}_{\mathrm{h}}\cdot{\bm\omega}_{\mathrm{g}}}\, .
\end{align}
由此得到
\begin{align}
    G_1({\bm\omega}_{\mathrm{h}},{\bm\omega}_{\mathrm{o}})
    =\frac{2({\bm\omega}_{\mathrm{h}}\cdot{\bm\omega}_{\mathrm{g}})\cos\theta_{\mathrm{o}}}
    {\max({\bm\omega}_{\mathrm{h}}\cdot{\bm\omega}_{\mathrm{o}},0)}
    =\frac{2({\bm\omega}_{\mathrm{h}}\cdot{\bm\omega}_{\mathrm{g}})
    ({\bm\omega}_{\mathrm{o}}\cdot{\bm\omega}_{\mathrm{g}})}
    {\max({\bm\omega}_{\mathrm{h}}\cdot{\bm\omega}_{\mathrm{o}},0)}\, .
\end{align}
另一种情况是${\bm\omega}_{\mathrm{h}}$和${\bm\omega}'_{\mathrm{h}}$均完全可见,
即$G_1({\bm\omega}_{\mathrm{h}},{\bm\omega}_{\mathrm{o}})=G_1({\bm\omega}'_{\mathrm{h}},{\bm\omega}_{\mathrm{o}})=1$.
我们以单个表达式概括上述两种情况:
\begin{align}\label{eq:08ex01-V-Cavity-MaskingFunction}
    G_1({\bm\omega}_{\mathrm{h}},{\bm\omega}_{\mathrm{o}})
    =\min\left(1, \frac{2({\bm\omega}_{\mathrm{h}}\cdot{\bm\omega}_{\mathrm{g}})
    ({\bm\omega}_{\mathrm{o}}\cdot{\bm\omega}_{\mathrm{g}})}
    {\max({\bm\omega}_{\mathrm{h}}\cdot{\bm\omega}_{\mathrm{o}},0)}\right)\, .
\end{align}
这便是\citet{10.1145/357290.357293}所用的著名的V形槽掩模函数。

\begin{figure}[htbp]
    \centering
    \includegraphics[width=\linewidth]{Pictures/chap08/VCavityMicrosurfaceMask.eps}
    \caption{V形槽微面模型的掩模。(a)一个面完全不可见,另一个被部分遮挡;
        (b)两个面都完全可见,此时掩模函数值为1.}
    \label{fig:08ex01-V-cavityScattering-Mask}
\end{figure}

现在我们验证下\refeq{08ex01-V-Cavity-MaskingFunction}是否满足

\subsection{典型微面分布函数的规范性证明}\label{sub:典型微面分布函数的规范性证明}
本节补充了\refeq{8.10}和\refeq{8.11}所给的
微面分布函数$D({\bm\omega}_{\mathrm{h}})$满足规范性要求的证明,即证明
\begin{align}\label{eq:8.ex-01}
    \int\limits_{H^2({\bm n})}D({\bm\omega}_{\mathrm{h}})\cos\theta_{\mathrm{h}}\mathrm{d}{\bm\omega}_{\mathrm{h}}=1\, .
\end{align}

为了简化证明过程,我们先证明以下积分式(其中$\alpha_x,\alpha_y>0$):
\begin{align}\label{eq:8.ex-02}
    \int_{\varphi_{\mathrm{h}}=0}^{2\pi}\frac{1}{2\pi\alpha_x\alpha_y\left(\frac{\cos^2\varphi_{\mathrm{h}}}{\alpha_x^2}+\frac{\sin^2\varphi_{\mathrm{h}}}{\alpha_y^2}\right)}\mathrm{d}\varphi_{\mathrm{h}}=1\, .
\end{align}
\begin{prove}
    \begin{align}
                                                         & \int_{\varphi_{\mathrm{h}}=0}^{2\pi}\frac{1}{2\pi\alpha_x\alpha_y\left(\frac{\cos^2\varphi_{\mathrm{h}}}{\alpha_x^2}+\frac{\sin^2\varphi_{\mathrm{h}}}{\alpha_y^2}\right)}\mathrm{d}\varphi_{\mathrm{h}}\nonumber \\
        =                                                & \int_{\varphi_{\mathrm{h}}=0}^{2\pi}\frac{\alpha_x\alpha_y}{2\pi(\alpha_x^2\sin^2\varphi_{\mathrm{h}}+\alpha_y^2\cos^2\varphi_{\mathrm{h}})}\mathrm{d}\varphi_{\mathrm{h}}\nonumber                               \\
        =                                                & \frac{\alpha_x\alpha_y}{2\pi}\int_{\varphi_{\mathrm{h}}=0}^{2\pi}\frac{1}{(\alpha_x^2\tan^2\varphi_{\mathrm{h}}+\alpha_y^2)\cos^2\varphi_{\mathrm{h}}}\mathrm{d}\varphi_{\mathrm{h}}\nonumber                     \\
        =                                                & \frac{\alpha_x\alpha_y}{\pi}\int_{\varphi_{\mathrm{h}}=0}^{\pi}\frac{1}{\alpha_x^2\tan^2\varphi_{\mathrm{h}}+\alpha_y^2}\mathrm{d}\tan\varphi_{\mathrm{h}}\nonumber                                               \\
        =                                                & \frac{\alpha_x\alpha_y}{\pi}\int_{\varphi_{\mathrm{h}}=-\frac{\pi}{2}}^{\frac{\pi}{2}}\frac{1}{\alpha_x^2\tan^2\varphi_{\mathrm{h}}+\alpha_y^2}\mathrm{d}\tan\varphi_{\mathrm{h}}\nonumber                        \\
        \xlongequal{\text{令}t=\tan\varphi_{\mathrm{h}}} & \frac{\alpha_x\alpha_y}{\pi}\int_{t=-\infty}^{+\infty}\frac{1}{\alpha_x^2t^2+\alpha_y^2}\mathrm{d}t\nonumber                                                                                                      \\
        =                                                & \frac{1}{\pi}\int_{t=-\infty}^{+\infty}\frac{1}{\left(\frac{\alpha_xt}{\alpha_y}\right)^2+1}\mathrm{d}\frac{\alpha_xt}{\alpha_y}\nonumber                                                                         \\
        =                                                & \frac{1}{\pi}\arctan\frac{\alpha_xt}{\alpha_y}\bigg|_{t=-\infty}^{+\infty}\nonumber                                                                                                                               \\
        =                                                & 1\, .
    \end{align}
\end{prove}

接下来证明各向异性的Beckmann-Spizzichino模型即\refeq{8.10}满足规范性。
\begin{prove}
    设
    \begin{align}\label{eq:8.ex-03}
        \beta=\frac{\cos^2\varphi_{\mathrm{h}}}{\alpha_x^2}+\frac{\sin^2\varphi_{\mathrm{h}}}{\alpha_y^2}>0\quad(\alpha_x,\alpha_y>0)\, .
    \end{align}
    利用上述变量简化积分并结合\refeq{8.ex-02},可得
    \begin{align}
          & \int\limits_{H^2({\bm n})}D({\bm\omega}_{\mathrm{h}})\cos\theta_{\mathrm{h}}\mathrm{d}{\bm\omega}_{\mathrm{h}}\nonumber                                                                                                                                                                                                                                                         \\
        = & \int\limits_{H^2({\bm n})}\frac{\mathrm{e}^{-\left(\frac{\cos^2\varphi_{\mathrm{h}}}{\alpha_x^2}+\frac{\sin^2\varphi_{\mathrm{h}}}{\alpha_y^2}\right)\tan^2\theta_{\mathrm{h}}}}{\pi\alpha_x\alpha_y\cos^4\theta_{\mathrm{h}}}\cos\theta_{\mathrm{h}}\mathrm{d}{\bm\omega}_{\mathrm{h}}\nonumber                                                                                \\
        = & \int_{\varphi_{\mathrm{h}}=0}^{2\pi}\int_{\theta_{\mathrm{h}}=0}^{\frac{\pi}{2}}\frac{\mathrm{e}^{-\left(\frac{\cos^2\varphi_{\mathrm{h}}}{\alpha_x^2}+\frac{\sin^2\varphi_{\mathrm{h}}}{\alpha_y^2}\right)\tan^2\theta_{\mathrm{h}}}}{\pi\alpha_x\alpha_y\cos^3\theta_{\mathrm{h}}}\sin\theta_{\mathrm{h}}\mathrm{d}\theta_{\mathrm{h}}\mathrm{d}\varphi_{\mathrm{h}}\nonumber \\
        = & \int_{\varphi_{\mathrm{h}}=0}^{2\pi}\int_{\theta_{\mathrm{h}}=0}^{\frac{\pi}{2}}\frac{\tan\theta_{\mathrm{h}}}{\pi\alpha_x\alpha_y\cos^2\theta_{\mathrm{h}}}\mathrm{e}^{-\beta \tan^2\theta_{\mathrm{h}}}\mathrm{d}\theta_{\mathrm{h}}\mathrm{d}\varphi_{\mathrm{h}}\nonumber                                                                                                   \\
        = & \int_{\varphi_{\mathrm{h}}=0}^{2\pi}\int_{\theta_{\mathrm{h}}=0}^{\frac{\pi}{2}}\frac{-1}{2\pi\alpha_x\alpha_y\beta}\mathrm{d}\mathrm{e}^{-\beta \tan^2\theta_{\mathrm{h}}}\mathrm{d}\varphi_{\mathrm{h}}\nonumber                                                                                                                                                              \\
        = & \int_{\varphi_{\mathrm{h}}=0}^{2\pi}\frac{-1}{2\pi\alpha_x\alpha_y\beta}\left(\mathrm{e}^{-\beta \tan^2\theta_{\mathrm{h}}}\bigg|_{\theta_{\mathrm{h}}=0}^{\frac{\pi}{2}}\right)\mathrm{d}\varphi_{\mathrm{h}}\nonumber                                                                                                                                                         \\
        = & \int_{\varphi_{\mathrm{h}}=0}^{2\pi}\frac{1}{2\pi\alpha_x\alpha_y\beta}\mathrm{d}\varphi_{\mathrm{h}}\nonumber                                                                                                                                                                                                                                                                  \\
        = & 1\, .
    \end{align}
\end{prove}

Trowbridge-Reitz模型即\refeq{8.11}的证明是类似的。
\begin{prove}
    同样按\refeq{8.ex-03}设好$\beta$,结合\refeq{8.ex-02},可得
    \begin{align}
                                                                 & \int\limits_{H^2({\bm n})}D({\bm\omega}_{\mathrm{h}})\cos\theta_{\mathrm{h}}\mathrm{d}{\bm\omega}_{\mathrm{h}}\nonumber                                                                                                                                                                                                                                                            \\
        =                                                        & \int\limits_{H^2({\bm n})}\frac{1}{\pi\alpha_x\alpha_y\left(1+\left(\frac{\cos^2\varphi_{\mathrm{h}}}{\alpha_x^2}+\frac{\sin^2\varphi_{\mathrm{h}}}{\alpha_y^2}\right)\tan^2\theta_{\mathrm{h}}\right)^2\cos^4\theta_{\mathrm{h}}}\cos\theta_{\mathrm{h}}\mathrm{d}{\bm\omega}_{\mathrm{h}}\nonumber                                                                               \\
        =                                                        & \int_{\varphi_{\mathrm{h}}=0}^{2\pi}\int_{\theta_{\mathrm{h}}=0}^{\frac{\pi}{2}}\frac{\sin\theta_{\mathrm{h}}}{\pi\alpha_x\alpha_y\left(1+\left(\frac{\cos^2\varphi_{\mathrm{h}}}{\alpha_x^2}+\frac{\sin^2\varphi_{\mathrm{h}}}{\alpha_y^2}\right)\tan^2\theta_{\mathrm{h}}\right)^2\cos^3\theta_{\mathrm{h}}}\mathrm{d}\theta_{\mathrm{h}}\mathrm{d}\varphi_{\mathrm{h}}\nonumber \\
        =                                                        & \int_{\varphi_{\mathrm{h}}=0}^{2\pi}\int_{\theta_{\mathrm{h}}=0}^{\frac{\pi}{2}}\frac{\tan\theta_{\mathrm{h}}}{\pi\alpha_x\alpha_y(1+\beta\tan^2\theta_{\mathrm{h}})^2\cos^2\theta_{\mathrm{h}}}\mathrm{d}\theta_{\mathrm{h}}\mathrm{d}\varphi_{\mathrm{h}}\nonumber                                                                                                               \\
        =                                                        & \int_{\varphi_{\mathrm{h}}=0}^{2\pi}\int_{\theta_{\mathrm{h}}=0}^{\frac{\pi}{2}}\frac{1}{2\pi\alpha_x\alpha_y\beta(1+\beta\tan^2\theta_{\mathrm{h}})^2}\mathrm{d}(\beta\tan^2\theta_{\mathrm{h}})\mathrm{d}\varphi_{\mathrm{h}}\nonumber                                                                                                                                           \\
        \xlongequal{\text{令}u=1+\beta\tan^2\theta_{\mathrm{h}}} & \int_{\varphi_{\mathrm{h}}=0}^{2\pi}\int_{u=1}^{+\infty}\frac{1}{2\pi\alpha_x\alpha_y\beta u^2}\mathrm{d}u\mathrm{d}\varphi_{\mathrm{h}}\nonumber                                                                                                                                                                                                                                  \\
        =                                                        & \int_{\varphi_{\mathrm{h}}=0}^{2\pi}\int_{u=1}^{+\infty}\frac{-1}{2\pi\alpha_x\alpha_y\beta}\mathrm{d}u^{-1}\mathrm{d}\varphi_{\mathrm{h}}\nonumber                                                                                                                                                                                                                                \\
        =                                                        & \int_{\varphi_{\mathrm{h}}=0}^{2\pi}\frac{-1}{2\pi\alpha_x\alpha_y\beta}\left(\frac{1}{u}\bigg|_{u=1}^{+\infty}\right)\mathrm{d}\varphi_{\mathrm{h}}\nonumber                                                                                                                                                                                                                      \\
        =                                                        & \int_{\varphi_{\mathrm{h}}=0}^{2\pi}\frac{1}{2\pi\alpha_x\alpha_y\beta}\mathrm{d}\varphi_{\mathrm{h}}\nonumber                                                                                                                                                                                                                                                                     \\
        =                                                        & 1\, .
    \end{align}
\end{prove}

{\noindent\hfil$=========$\hfil{\color{red}{施工分割线}}\hfil$=========$\


\input{content/chap09.tex}

\input{content/chap10.tex}

\input{content/chap11.tex}

\input{content/chap12.tex}

\part{光传输算法}
\input{content/chap13.tex}

\input{content/chap14.tex}

\input{content/chap15.tex}

\input{content/chap16.tex}

\part{回顾与未来}
\input{content/chap17.tex}

\appendix
\part{附录}
\input{content/chapextraA.tex}

\input{content/chapextraB.tex}

% \chapter{In-text Elements}

% \section{Theorems}\index{Theorems}

% This is an example of theorems.

% \subsection{Several equations}\index{Theorems!Several Equations}
% This is a theorem consisting of several equations.

% \begin{theorem}[Name of the theorem]
%     In $E=\mathbb{R}^n$ all norms are equivalent. It has the properties:
%     \begin{align}
%          & \big| ||\mathbf{x}|| - ||\mathbf{y}|| \big|\leq || \mathbf{x}- \mathbf{y}||                            \\
%          & ||\sum_{i=1}^n\mathbf{x}_i||\leq \sum_{i=1}^n||\mathbf{x}_i||\quad\text{where $n$ is a finite integer}
%     \end{align}
% \end{theorem}

% \subsection{Single Line}\index{Theorems!Single Line}
% This is a theorem consisting of just one line.

% \begin{theorem}
%     A set $\mathcal{D}(G)$ in dense in $L^2(G)$, $|\cdot|_0$.
% \end{theorem}

% %------------------------------------------------

% \section{Definitions}

% This is an example of a definition.

% \begin{definition}[Definition name]
%     Given a vector space $E$, a norm on $E$ is an application, denoted $||\cdot||$, $E$ in $\mathbb{R}^+=[0,+\infty[$ such that:
%     \begin{align}
%          & ||\mathbf{x}||=0\ \Rightarrow\ \mathbf{x}=\mathbf{0}        \\
%          & ||\lambda \mathbf{x}||=|\lambda|\cdot ||\mathbf{x}||        \\
%          & ||\mathbf{x}+\mathbf{y}||\leq ||\mathbf{x}||+||\mathbf{y}||
%     \end{align}
% \end{definition}

% %------------------------------------------------

% \section{Notations}

% \begin{notation}
%     Given an open subset $G$ of $\mathbb{R}^n$, the set of functions $\varphi$ are:
%     \begin{enumerate}
%         \item Bounded support $G$;
%         \item Infinitely differentiable;
%     \end{enumerate}
%     a vector space is denoted by $\mathcal{D}(G)$.
% \end{notation}

% %------------------------------------------------

% \section{Remarks}

% This is an example of a remark.

% \begin{remark}
%     The concepts presented here are now in conventional employment in mathematics. Vector spaces are taken over the field $\mathbb{K}=\mathbb{R}$, however, established properties are easily extended to $\mathbb{K}=\mathbb{C}$.
% \end{remark}

% %------------------------------------------------

% \section{Corollaries}

% This is an example of a corollary.

% \begin{corollary}[Corollary name]
%     The concepts presented here are now in conventional employment in mathematics. Vector spaces are taken over the field $\mathbb{K}=\mathbb{R}$, however, established properties are easily extended to $\mathbb{K}=\mathbb{C}$.
% \end{corollary}

% %------------------------------------------------

% \section{Propositions}

% This is an example of propositions.

% \subsection{Several equations}\index{Propositions!Several Equations}

% \begin{proposition}[Proposition name]
%     It has the properties:
%     \begin{align}
%          & \big| ||\mathbf{x}|| - ||\mathbf{y}|| \big|\leq || \mathbf{x}- \mathbf{y}||                            \\
%          & ||\sum_{i=1}^n\mathbf{x}_i||\leq \sum_{i=1}^n||\mathbf{x}_i||\quad\text{where $n$ is a finite integer}
%     \end{align}
% \end{proposition}

% \subsection{Single Line}\index{Propositions!Single Line}

% \begin{proposition}
%     Let $f,g\in L^2(G)$; if $\forall \varphi\in\mathcal{D}(G)$, $(f,\varphi)_0=(g,\varphi)_0$ then $f = g$.
% \end{proposition}

% %------------------------------------------------

% \section{Examples}

% This is an example of examples.

% \subsection{Equation and Text}

% \begin{example}
%     Let $G=\{x\in\mathbb{R}^2:|x|<3\}$ and denoted by: $x^0=(1,1)$; consider the function:
%     \begin{equation}
%         f(x)=\left\{\begin{aligned}& \mathrm{e}^{|x|} &  & \text{si $|x-x^0|\leq 1/2$} \\ & 0  &  & \text{si $|x-x^0|> 1/2$}\end{aligned}\right.
%     \end{equation}
%     The function $f$ has bounded support, we can take $A=\{x\in\mathbb{R}^2:|x-x^0|\leq 1/2+\epsilon\}$ for all $\epsilon\in\intoo{0}{5/2-\sqrt{2}}$.
% \end{example}

% \subsection{Paragraph of Text}

% \begin{example}[Example name]
%     rrr
% \end{example}

% %------------------------------------------------

% \section{Exercises}

% This is an example of an exercise.

% \begin{exercise}
%     This is a good place to ask a question to test learning progress or further cement ideas into students' minds.
% \end{exercise}

% %------------------------------------------------

% \section{Problems}

% \begin{problem}
% What is the average airspeed velocity of an unladen swallow?
% \end{problem}

% %------------------------------------------------

% \section{Vocabulary}

% Define a word to improve a students' vocabulary.

% \begin{vocabulary}[Word]
%     Definition of word.
% \end{vocabulary}


% \chapterimage{chapter_head_1.pdf} % Chapter heading image

% \chapter{Presenting Information}


% \begin{enumerate}
%     \item The first item
%     \item The second item
%     \item The third item
% \end{enumerate}


% \begin{itemize}
%     \item The first item
%     \item The second item
%     \item The third item
% \end{itemize}

% \begin{description}
%     \item[Name] Description
%     \item[Word] Definition
%     \item[Comment] Elaboration
% \end{description}

% \section{Table}

% \begin{table}[h]
%     \centering
%     \begin{tabular}{l l l}
%         \toprule
%         \textbf{Treatments} & \textbf{Response 1} & \textbf{Response 2} \\
%         \midrule
%         Treatment 1         & 0.0003262           & 0.562               \\
%         Treatment 2         & 0.0015681           & 0.910               \\
%         Treatment 3         & 0.0009271           & 0.296               \\
%         \bottomrule
%     \end{tabular}
%     \caption{Table caption}
%     \label{tab:example} % Unique label used for referencing the table in-text
%     %\addcontentsline{toc}{table}{Table \ref{tab:example}} % Uncomment to add the table to the table of contents
% \end{table}

% Referencing Table \ref{tab:example} in-text automatically.

% %------------------------------------------------

% \section{Figure}

% \begin{figure}[h]
%     \centering\includegraphics[scale=0.5]{placeholder.jpg}
%     \caption{Figure caption}
%     \label{fig:placeholder} % Unique label used for referencing the figure in-text
%     % \addcontentsline{toc}{figure}{Figure \ref{fig:placeholder}} % Uncomment to add the figure to the table of contents
% \end{figure}

% Referencing Figure \ref{fig:placeholder} in-text automatically.

%----------------------------------------------------------------------------------------
%	BIBLIOGRAPHY
%----------------------------------------------------------------------------------------
\renewcommand{\bibname}{参考文献}
\chapter*{参考文献}
\markboth{\sffamily\normalsize\bfseries 参考文献}{\sffamily\normalsize\bfseries 参考文献}
\addcontentsline{toc}{chapter}{\textcolor{ocre}{参考文献}} % Add a Bibliography heading to the table of contents
\printbibliography[heading=bibempty]

%------------------------------------------------

% \section*{论文}
% \addcontentsline{toc}{section}{论文}
% \printbibliography[heading=bibempty,type=article]

%------------------------------------------------

% \section*{书籍}
% \addcontentsline{toc}{section}{书籍}
% \printbibliography[heading=bibempty,type=book]

%----------------------------------------------------------------------------------------
%	INDEX
%----------------------------------------------------------------------------------------
\renewcommand{\indexname}{\sffamily\bfseries 索引}
\cleardoublepage % Make sure the index starts on an odd (right side) page
\phantomsection
\setlength{\columnsep}{0.75cm} % Space between the 2 columns of the index
\addcontentsline{toc}{chapter}{\textcolor{ocre}{索引}} % Add an Index heading to the table of contents
\printindex % Output the index

%----------------------------------------------------------------------------------------

\end{document}
